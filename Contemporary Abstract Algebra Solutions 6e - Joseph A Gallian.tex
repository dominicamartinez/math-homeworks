\documentclass[12pt]{article}

\pagestyle{empty}
\usepackage{wrapfig}
\usepackage{shapepar}
\usepackage{fullpage}
\usepackage{amssymb}
\usepackage{amsthm}
%\usepackage{calligra}
\usepackage{amsmath}
%\usepackage[symbol*]{footmisc}
%\usepackage[enableskew]{youngtab}
\usepackage[all]{xy}
\usepackage{graphicx}

\begin{document}
Contemporary Abstract Algebra (6th ed.) by Joseph Gallian
\begin{enumerate}

\item[0.4] Find integers $s$ and $t$ such that $1 = 7 \cdot s + 11 \cdot t$. Show that $s$ and
$t$ are not unique.
\begin{enumerate}
\item[] We can choose $s$ and $t$ to be $-3$ and $2$, respectively. By the Euclidean algorithm:
\begin{eqnarray*}
11 &=& 7 \cdot 1 + 4 \\
7 &=& 4 \cdot 1 + 3 \\
4 &=& 3 \cdot 1 + 1 \\
3 &=& 1 \cdot 3
\end{eqnarray*}
Since $1$ is the last nonzero remainder, $\gcd(11, 7) = 1$. Thus, we can show that $s = -3$ and
$t = 2$ are two integers that we can find. Hence,
\begin{align*}
1 &= 4 + 3(-1) = 4 + [7 + 4(-1)](-1) \\
&= 7(-1) + 4(2) = 7(-1) + [11 + 7(-1)](2) \\
&= 11(2) + 7(-1) + 7(-2) \\
&= 11(2) + 7(-3)
\end{align*}
$s$ and $t$ cannot be unique because of the fact that $s = -3 + 11k$ and 
$t = 2 - 7k$, for some $k \in \mathbb{Z}$, will also satisfy our requirements. 
\end{enumerate}

\item[0.10] Let $d = $gcd$(a, b)$. If $a = da'$ and $b = db'$, show that gcd$(a', b') = 1$.
\begin{enumerate}
\item[] By theorem 0.2, $\gcd(a, b) = as + bt$ where $s$ and $t$ are integers, 
but $\gcd(a, b) = d$ as well, thus
$d = as + bt$. Since $a = da'$ and $b = db'$, $d = (da')s + (db')t$ which we can divide
by $d$ and end up with $1 = a's + b't$. Since we know that $\gcd(a, b) = d$ and $d \neq 0$,
it follows that $1 = a's + b't = \gcd(a', b')$.
\end{enumerate}

\item[0.11] Let $n$ be a fixed positive integer greater than $1$. If $a$ mod $n = a'$ and 
$b$ mod $n = b'$, prove that $(a + b)$ mod $n = (a' + b')$ mod $n$ and $(ab)$ mod $n = 
(a'b')$ mod $n$.
\begin{enumerate}
\item[] If $a\mod n = a'$ and $b \mod n = b'$, then by definition, $a = a' + rn$ and 
$b = b' + sn$ for some $r, s \in \mathbb{Z}$. Now,
\[ 
a + b = a' + rn + b' + sn = (a' + b') + n(r + s)
\]
Since $n$ divides $n(r + s)$ evenly, we have $(a + b)\mod n = (a' + b')\mod n $.
Also, 
\[
ab = (a' + rn)(b' + sn) = a'b' + sna' + rnb' + rsn^2 = \]
\[
a'b' + n(sa' + rb' + rsn)
\]
Since $n$ divides $n(sa' + rb' + rsn)$ evenly, we have $(ab)\mod n = (a'b')\mod n$.
\end{enumerate}


\item[0.13] Let $n$ and $a$ be positive integers and let $d = $gcd$(a, n)$. Show that the equation
$ax$ mod $n = 1$ has a solution if and only if $d = 1$.
\begin{enumerate}
\item[] Firstly suppose that $ax\mod n = 1$ has a solution. Then $kn + 1 = ax$ by definition.
Also, $ax - kn = 1$, which also happens to be the form of $\gcd(a, n)$ by theorem 0.2. Thus
$\gcd(a, n) = d$ and $\gcd(a, n) = 1$, then $d = 1$ as required.
Conversely, if $d = gcd(a, n) = 1$, then by Theorem 0.2, there exists $x, t$ such that 
$1 = ax + nt$. Now, $[1]_n = [ax]_n + [nt]_n$ and since $n$ divides $nt$ evenly, 
it is clear that $ax \mod n = 1$ and a solution exists.
\end{enumerate}


\item[0.14] Show that $5n + 3$ and $7n + 4$ are relatively prime for all $n$.
\begin{enumerate}
\item[] We can show using the Euclidean algorithm:
\begin{eqnarray*}
7n + 4 &=& (5n + 3) \cdot 1 + (2n + 1) \\
5n + 3 &=& (2n + 1) \cdot 2 + (n + 1) \\
2n + 1 &=& (n + 1) \cdot 1 + n \\
n + 1 &=& n \cdot 1 + 1 \\
n &=& 1 \cdot n
\end{eqnarray*}
Since $1$ is the last nonzero remainder, $\gcd(7n + 4, 5n + 3) = 1$, which shows that
they are relatively prime.
\end{enumerate}


\item[0.16] Use the Euclidean algorithm to find gcd$(34, 126)$ and write it as a linear combination
of $34$ and $126$.
\begin{enumerate}
\item[] Use Euclidean algorithm to find $\gcd(34, 126)$:
\begin{eqnarray*}
126 &=& 34 \cdot 3 + 24 \\
34 &=& 24 \cdot 1 + 10 \\
24 &=& 10 \cdot 2 + 4 \\
10 &=& 4 \cdot 2 + 2 \\
4 &=& 2 \cdot 2
\end{eqnarray*}
Since $2$ is the last nonzero remainder, $\gcd(34, 126) = 2$. \\
Now to express it as a linear combination:
\begin{align*}
2 &= 10 + 4(-2) = 10 + [24 + 10(-2)](-2) = 24(-2) + 10(5) \\
&= 24(-2) + [34 + 24(-1)](5) = 24(-7) + 34(5) \\
&= [126 + 34(-3)](-7) + 34(5) = 126(-7) + 34(26)
\end{align*}
Hence, we can express $2$ as $126(-7) + 34(26)$.
\end{enumerate}

\item[0.26] What is the largest bet that cannot be made with chips worth 7 dollars and 9 dollars?
Verify that your answer is correct with both forms of induction.
\begin{enumerate}
\item[] To gain insight into this problem, we try various combinations of 7 and 9 dollar bets and obtain
7, 9, 14, 16, 18, 21, 23, 25, 27, 28, 30, 32, 34, 35, 36, 37, 39, 41, 42, 43, 44, 45, 46, 48, 49, 50, 
51, 52, 53, 54, 55, 56, 57, 58, 59, 60, 61, 62, 63, 64, 65, 66, 67, 68, 69, 70, 71, 72, 73, 74, 75, 76,
77, 78, 79, 80, 81, 82, 83, 84, 85, 86, 87, 88, 89, 90. It appears that the answer is 47. Well, we need
to prove that every integer greater than 47 can be written in the form $a \cdot 7 + b \cdot 9$, where
$a$ and $b$ are nonnegative integers. This will solve the problem, since $a$ represents the number of 
7 dollar bets and $b$ the number of 9 dollar bets needed to make a bet of $a \cdot 7 + b \cdot 9$. \\
First, we use the First Principle of Mathematical Induction. Let $S$ be the set of all integers of
the form $a \cdot 7 + b \cdot 9$, where $a$ and $b$ are nonnegative. Then $48 = 3 \cdot 7 + 3 \cdot 9$ 
and $48 \in S$. Now assume that some integer $n \in S$, say, $n = a \cdot 7 + b \cdot 9$. We must show
that $n + 1 \in S$. First note that since $n \geq 48$, we cannot have both $a$ and $b$ less than 3.
If $a \geq 3$, then $n + 1 = (a \cdot 7 + b \cdot 9) + (-5 \cdot 7 + 4 \cdot 9) = (a - 5) \cdot 7 + 
(b + 4) \cdot 9$. If $b \geq 3$, then $n + 1 = (a \cdot 7 + b \cdot 9) + (4 \cdot 7 - 3 \cdot 9) =
(a + 4) \cdot 7 + (b - 3) \cdot 9$. This completes the proof. \\
To prove the same statement by the Second Principle, we note that each of the integers 
48, 49, 50, 51, 52, 53, and 54 is in $S$. Now assume that for some integer $n > 54$, 
$S$ contains all integers $k$ with $48 \leq k < n$. We must show that $n \in S$. Since
$n - 7 \in S$, there are nonnegative integers $a$ and $b$ such that $n - 7 = a \cdot 7 + 
b \cdot 9$. But then $n = (a + 1) \cdot 7 + b \cdot 9$. Thus $n$ is in $S$.
\end{enumerate}

\item[0.30] Prove that for every integer $n$, $n^3$ mod $6 = n$ mod $6$.
\begin{enumerate}
\item[] To show that $n^3\mbox{ mod } 6 = n\mbox{ mod } 6$, for all integers $n$, 
we take into account the 6 different equivalence classes on $\mathbb{Z}$.
That is [0], [1], [2], [3], [4], and [5]. \\
$[0] = \{\ldots, -12, -6, 0, 6, 12, \ldots\}$ and are in the form $6k, k \in \mathbb{Z}$.
Now, we show $(6k)^3\mbox{ mod } 6 = 6k\mbox{ mod } 6$. 
\[ (6k)^3\mbox{ mod } 6 = 216k^3\mbox{ mod } 6 = 6(36k^3)\mbox{ mod } 6 = 0 = 6k\mbox{ mod } 6 \]

$[1] = \{\ldots, -11, -5, 1, 7, 13, \ldots\}$ and are in the form $6k + 1, k \in \mathbb{Z}$.
Now, we show $(6k + 1)^3\mbox{ mod } 6 = (6k + 1)\mbox{ mod } 6$. 
\begin{align*}
(6k + 1)^3\mbox{ mod } 6 &= (216k^3 + 108k^2 + 18k + 1)\mbox{ mod } 6 \\
&= 6(36k^3 + 18k^2 + 3k)\mbox{ mod } 6 + 1\mbox{ mod } 6 \\
&= 1 = (6k + 1)\mbox{ mod } 6
\end{align*}

$[2] = \{\ldots, -10, -4, 2, 8, 14, \ldots\}$ and are in the form $6k + 2, k \in \mathbb{Z}$.
Now, we show $(6k + 2)^3\mbox{ mod } 6 = (6k + 2)\mbox{ mod } 6$. 
\begin{align*}
(6k + 2)^3\mbox{ mod } 6 &= (216k^3 + 216k^2 + 72k + 8)\mbox{ mod } 6 \\
&= 6(36k^3 + 18k^2 + 3k)\mbox{ mod } 6 + 8\mbox{ mod } 6 = 0 + 2 \\
&= 2 = (6k + 2)\mbox{ mod } 6
\end{align*}

$[3] = \{\ldots, -9, -3, 3, 9, 15, \ldots\}$ and are in the form $6k + 3, k \in \mathbb{Z}$.
Now, we show $(6k + 3)^3\mbox{ mod } 6 = (6k + 3)\mbox{ mod } 6$. 
\begin{align*}
(6k + 3)^3\mbox{ mod } 6 &= (216k^3 + 324k^2 + 162k + 27)\mbox{ mod } 6 \\
&= 6(36k^3 + 54k^2 + 27k)\mbox{ mod } 6 + 27\mbox{ mod } 6 = 0 + 3 \\
&= 3 = (6k + 3)\mbox{ mod } 6
\end{align*}

$[4] = \{\ldots, -8, -2, 4, 10, 16, \ldots\}$ and are in the form $6k + 4, k \in \mathbb{Z}$.
Now, we show $(6k + 4)^3\mbox{ mod } 6 = (6k + 4)\mbox{ mod } 6$. 
\begin{align*}
(6k + 4)^3\mbox{ mod } 6 &= (216k^3 + 432k^2 + 288k + 64)\mbox{ mod } 6 \\
&= 6(36k^3 + 72k^2 + 48k)\mbox{ mod } 6 + 64\mbox{ mod } 6 = 0 + 4 \\
&= 4 = (6k + 4)\mbox{ mod } 6
\end{align*}

$[5] = \{\ldots, -7, -1, 5, 11, 17, \ldots\}$ and are in the form $6k + 5, k \in \mathbb{Z}$.
Now, we show $(6k + 5)^3\mbox{ mod } 6 = (6k + 5)\mbox{ mod } 6$. 
\begin{align*}
(6k + 5)^3\mbox{ mod } 6 &= (216k^3 + 540k^2 + 450k + 125)\mbox{ mod } 6 \\
&= 6(36k^3 + 90k^2 + 75k)\mbox{ mod } 6 + 125\mbox{ mod } 6 = 0 + 5 \\
&= 5 = (6k + 5)\mbox{ mod } 6
\end{align*}

\end{enumerate}

\item[0.33] Suppose that in one of the noncheck positions of a money order number, the
digit 0 is substituted for the digit 9 or vice versa. Prove that this error will not be detected
by the check digit. Prove that all other errors involving a single position are detected.
\begin{enumerate}
\item[] Let $a_9a_8\cdots a_0$ be the nine digit money order number. Since we get the check number 
by $(a_9a_8\cdots a_0)\mbox{ mod }9$, the nine digit number can be represented as 
$a_9\cdot 10^9 + a_8\cdot 10^8 + \cdots a_0\cdot 10^0$, that is 
$\sum^9_{i = 0}a_i\cdot 10^i$, where $a_i \in \{0, 1, 2, 3, 4, 5, 6, 7, 8, 9\}$. 
For every nonnegative integer, $10^n\mbox{ mod } 9 = 1$ and the above reduces to just 
$a_i \cdot 1 = a_i$ and each $a_i$ is going to belong to it's own distinct equivalence class, 
except for 0 and 9 because $0\mbox{ mod } 9 = 9\mbox{ mod }9$.
Thus it is clear that switching a 0 for a 9 or vice versa for any $a_i$, will yield the 
same check number and go undetected. Now following from above, we can get the check number, \\
$c = (a_9 + a_8 + \cdots + a_i + \cdots + a_0)\mbox{ mod } 9$. Hence, if we swap one $a_i$, 
$0 \leq i \leq 9$, with some $a_j$, where $a_i, a_j \in \{0, 1, 2, 3, 4, 5, 6, 7, 8, 9\}$ 
and $a_i \neq a_j$ and it is not a 0 for 9 or 9 for 0 swap then $a_j = a_i + (a_j - a_i)$ 
and $c' = c + (a_j - a_i) = (a_9 + a_8 + \cdots + a_i + (a_j - a_i) \cdots + a_0)\mbox{ mod } 9$ 
is the new check number. Since $a_j - a_i \neq 0$, $c' \neq c$ and thus a single position 
digit swap will be detected.
\end{enumerate}

\item[0.48] Let $S$ be the set of real numbers. If $a, b \in S$, define $a \sim b$ if $a-b$
is an integer. Show that $\sim$ is an equivalence relation on $S$. Describe the equivalence
classes of $S$.
\begin{enumerate}
\item[] Reflexive: $a - a = 0$ and $0 \in \mathbb{Z}$ \\
Symmetric: $a - b = k$ where $k \in \mathbb{Z}$, implies $b - a = -k$, where $-k \in \mathbb{Z}$ \\
Transitive: $a - b = k$ and $b - c = n$ where $k, n \in \mathbb{Z}$ implies that 
$a - c = a + b - b - c = (a - b) + (b - c) = k + n$, where $(k + n) \in \mathbb{Z}$ \\
Thus $\sim$ is an equivalence relation. Also, for any real number, $a$, \\
$[a] = \{a + n, $ where $n \in \mathbb{Z}\}$. We might also add that the set of all 
equivalence classes for this relation is $\{ [a]\, :\, a \in [0, 1) \}$.
\end{enumerate}

\item[1.5] For $n \geq 3$, describe the elements of $D_n$ (\emph{Hint:} You will need to
consider two cases - $n$ even and $n$ odd.) How many elements does $D_n$ have?
\begin{enumerate}
\item[] For $n \geq 3$ and $n$ is even: The reflections can be obtained by drawing a line through
opposing vertices and by drawing a line through the middle of two opposing edges (see figure 1.1
for example). Each rotation will be $k \cdot (360^\circ /n)$, where $k \in \{0, 1, \ldots, n-1\}$
and we obtain the rotations by 
fixing a center of rotation and rotating. \\
For $n \geq 3$ and $n$ is odd: The reflections can be obtained by drawing a line through a
vertex and it's opposing edge. Rotations act in the same manner as above. \\
$D_n$ has exactly $2n$ elements, that is $n$ rotations and $n$ reflections.
\end{enumerate}

\item[1.6] In $D_n$, explain geometrically why a reflection followed by a reflection must be a rotation.
\begin{enumerate}
\item[] The simplest case is when the same reflection twice. The second reflection just
undoes what the first reflection did and is ultimately considered a rotation, $R_0$, because
after both reflections it appears that $D_n$ did not move. A reflection fixes a line and 
then rotates about it. Thus after two different reflections, consisting of two different
reflection lines the only thing that remained fixed throughout was the intersection of
those two lines, which is as we said in 1.5, is a rotation.
\end{enumerate}

\item[1.7] In $D_n$, explain geometrically why a rotation followed by a rotation must be a rotation.
\begin{enumerate}
\item[] As was said about the rotational elements in 1.5, rotating fixes a point to be the 
center of rotation. When we rotate upon that point and then rotate again, then the only thing
that remained fixed both times was that point, which is clearly a rotation.
\end{enumerate}


\item[1.8] In $D_n$, explain geometrically why a rotation and a reflection taken together in either order
must be a reflection.
\begin{enumerate}
\item[] Geometrically, we'll describe $D_n$ as a plane in 3-space. 
If we allowed the plane to lie on xy-plane while the z-axis runs directly through the center 
of $D_n$ then a reflection would be a 3 dimensional flip about the 2-D line mentioned in 1.5 and 
a rotation would be a 2 dimensional action (spinning) about a center of rotation (in this case,
would be at the origin). Now, since the plane has a top and bottom, we can label the initial
orientation of the plane as looking down from the positive z-axis as 1 and the bottom as 2.
If we do a reflection, then looking down from the positive z-axis, we would be looking at
the bottom of the plane or side 2. After the reflection, doing a rotation, since it is a
2 dimensional action would not change the fact that we would still be looking down on 
side 2 of the plane. The same can be said for a rotation and then a reflection, because
after rotating it, from the positive z-axis, it would still be side 1 but then a reflection
would cause it to be side 2. Thus a reflection and a rotation, taken in either order, 
is a reflection.
\end{enumerate}


\item[2.2] Referring to Example 13, verify the assertion that subtraction is not associative.
\begin{enumerate}
\item[] Proof by counterexample: Consider $a = 1, b = 2$ and $c = 3$, then
\[ (1 - 2) - 3 = -1 - 3 = -1 \neq 2 = 1 - (-1) = 1 - (2 - 3) \]
Thus, subtraction on the set of integers is not associative.
\end{enumerate}


\item[2.4] Show that the group $GL(2, \mathbb{R})$ of Example 9 is non-Abelian by exhibiting
a pair of matrices $A$ and $B$ in $GL(2, \mathbb{R})$ such that $AB \neq BA$.
\begin{enumerate}
\item[] Consider $A = \left[ \begin{array}{cc} 1 & 2 \\ 0 & 1 \end{array} \right]$ and 
$B = \left[ \begin{array}{cc} 1 & 0 \\ 2 & 1 \end{array} \right]$. 
\[ 
AB = \left[ \begin{array}{cc} 5 & 2 \\ 2 & 1 \end{array} \right] \neq 
\left[ \begin{array}{cc} 1 & 2 \\ 2 & 5 \end{array} \right] = BA 
\]
\end{enumerate}

\item[2.5] Find the inverse of the element 
$\left[ \begin{array}{cc} 2 & 6 \\ 3 & 5 \end{array} \right]$ in $GL(2, \mathbb{Z}_{11})$.
\begin{enumerate}
\item[] First, let us find the determinant; $[(2)(5) - (6)(3)]\mbox{ mod } 11 = -8\mbox{ mod } 11 = 3$.
Now the inverse of the determinant is 4 since $(4 \cdot 3) \mbox{ mod } 11 = 1$. Then 
using what was said in class, example 2.9, and example 2.18, we know that the inverse of
$\left[ \begin{array}{cc} 2 & 6 \\ 3 & 5 \end{array} \right]$ is going to be
$\left[ \begin{array}{cc} 5 \cdot 4 & -6 \cdot 4 \\ -3 \cdot 4 & 2 \cdot 4 \end{array} \right] \mbox{ mod } 11 
= \left[ \begin{array}{cc} 20 & -24 \\ -12 & 8 \end{array} \right] \mbox{ mod } 11 = 
\left[ \begin{array}{cc} 9 & 9 \\ 10 & 8 \end{array} \right]$
\end{enumerate}


\item[2.6] Give an example of group elements $a$ and $b$ with the property that $a^{-1}ba \neq b$.
\begin{enumerate}
\item[] We will use $D_4$ and the Cayley table from pg. 33 to give an example of group elements, $a$
and $b$, with the property that $a^{-1}ba \neq b$. Let $a = R_{90}$ and $b = H$, then that means
that $a^{-1} = R_{270}$. Now, $a^{-1}b = R_{270}H = D$ from the Cayley table, then 
$a^{-1}ba = Da = DR_{90} = V$. Thus $a^{-1}ba = V \neq H = b$.
\end{enumerate}


\item[2.15] (Law of Exponents for Abelian Groups) Let $a$ and $b$ be elements of an Abelian
group and let $n$ be any integer. Show that $(ab)^n = a^nb^n$. Is this also true for 
non-Abelian groups?
\begin{enumerate}
\item[] To show that $(ab)^n = a^nb^n$, first let us show the case when $n = 0$; \\
$(ab)^0 = e = ee = a^0b^0$. Now, we can use induction to show all integers $n \geq 1$.
Our base case is $n = 1$, then $(ab)^1 = ab = a^1b^1$. Now, we assume $(ab)^n = a^nb^n$
holds true for $n$. Then we show that 
\[ (ab)^{n + 1} = (ab)^n(ab) = a^nb^nab = a^nab^nb = a^{n + 1}b^{n + 1} \]
Now, we can show for when $n$ is negative by using the definition that 
$g^n = (g^{-1})^{|n|}$, but clearly we don't need the absolute value exponent
since we're only going to be using positive integers for $n$. Hence, we use induction again,
so that our base case is when $n = 1$, $[(ab)^{-1}]^1 = (ab)^{-1} = a^{-1}b^{-1} = (a^{-1}b^{-1})^1$, 
which is true by Exercise 2.16 and the fact that this group is Abelian.
Now, we assume $[(ab)^{-1}]^n = [a^{-1}b^{-1}]^n$ holds true for $n$. Then we show that 
\[ [(ab)^{-1}]^{n + 1} = [(ab)^{-1}]^n(ab)^{-1} = [a^{-1}b^{-1}]^n(ab)^{-1} 
\]
\[
= [a^{-1}b^{-1}]^n[b^{-1}a^{-1}] = [a^{-1}b^{-1}]^{n + 1}
\]
No, $(ab)^n = a^nb^n$ is not true for non-Abelian groups. One might notice that some of the 
equalities in the above induction proofs might not have been possible if the group hadn't
been Abelian. Furthermore, we can show that this doesn't hold true for non-Abelian groups by
counterexample. Take the GL$(2, \mathbb{R})$ group, which is not Abelian and consider $A$ 
and $B$ to be the same 2x2 matrices from Exercise 2.4, then 
$(AB)^2 = \left[ \begin{array}{cc} 29 & 12 \\ 12 & 5 \end{array} \right] \neq 
\left[ \begin{array}{cc} 17 & 4 \\ 4 & 1 \end{array} \right] = A^2B^2$.
\end{enumerate}

\item[2.16] (Socks-Shoes Property) In a group, prove that $(ab)^{-1} = b^{-1}a^{-1}$. Find
an example that shows that it is possible to have $(ab)^{-2} \neq b^{-2}a^{-2}$. Find
distinct nonidentity elements $a$ and $b$ from a non-Abelian group with the property that
$(ab)^{-1} = a^{-1}b^{-1}$. Draw an analogy between the statement $(ab)^{-1} = b^{-1}a^{-1}$
and the act of putting on and taking off your socks and shoes.
\begin{enumerate}
\item[] Let $e$ be the identity of a group $G$ and $a, b \in G$ be distinct. We want to show
that $(ab)^{-1} = b^{-1}a^{-1}$. First, we show by multiplication on the right:
\begin{eqnarray*}
(ab)^{-1}(ab) &=& b^{-1}a^{-1}(ab) \\
e &=& b^{-1}(a^{-1}a)b \hskip .9cm \mbox{by associativity} \\
&=& b^{-1}eb \hskip 1.1cm \mbox{by inverse property} \\
&=& b^{-1}b \hskip 1.1cm \mbox{by identity property} \\
&=& e \hskip 1.7cm \mbox{by inverse property}
\end{eqnarray*}

Now, on the left:

\begin{eqnarray*}
(ab)^{-1}(ab) &=& (ab)b^{-1}a^{-1} \\
e &=& a(bb^{-1})a^{-1} \hskip .9cm \mbox{by associativity} \\
&=& aea^{-1} \hskip 1.1cm \mbox{by inverse property} \\
&=& a^{-1}a \hskip 1.1cm \mbox{by identity property} \\
&=& e \hskip 1.8cm \mbox{by inverse property}
\end{eqnarray*}

To give an example that shows $(ab)^{-2} \neq b^{-2}a^{-2}$, we will use the Cayley table
from $D_4$ on pg. 33 as a guide. If we let $a = R_{270}$ and $b = H$, then $a^{-1} = R_{90}$
and $b^{-1} = H$. Hence $ab = D$ from the table and since $D$ is it's own inverse, then
$(D)^{-2} = D^2 = R_0 = (ab)^{-2}$. It's just as well, that $a^{-2} = (R_{90})^2 = R_{180}$ and
$b^{-2} = $H$^{2} = R_0$. Thus $R_0R_{180} = R_{180} = b^{-2}a^{-2} \neq (ab)^{-2}$. \\
We can use the same example, $D_4$, from which we can pick distinct nonidentity elements
$a$ and $b$ from the non-Abelian group such that property $(ab)^{-1} = a^{-1}b^{-1}$ is satisfied.
If we let $a = D$ and $b = D'$, then since $D$ and $D'$ are their own inverses, $a^{-1} = D$ and
$b^{-1} = D'$. From the Cayley table, we get that $ab = R_{180}$ and the inverse of $R_{180}$ is
itself, so $(ab)^{-1} = R_{180}$. Now, $a^{-1}b^{-1} = DD'$, which by the Cayley table is $R_{180}$.
Hence, $(ab)^{-1} = R_{180} = a^{-1}b^{-1}$. \\
The process of taking on and off your socks and shoes is supposed to follow in a particular order, 
that is to say, you put on your socks and then shoes, whereby you then take off your shoes and then
socks. Therefore, we can let $a$ be putting on your socks and $b$ be putting on your shoes, which means
that the inverse, $a^{-1}$ is taking off your socks while $b^{-1}$ is taking off your shoes. 
We can see that $ab$ is the whole process of putting on your socks and shoes which means that 
$(ab)^{-1}$ is the entire process of taking off your socks and shoes. From the first thing we proved in 
this exercise, $(ab)^{-1} = b^{-1}a^{-1}$, which tells us that the entire process of taking off
your socks and shoes can be broken down into two actions, that is taking off your shoes, $b^{-1}$ and
then taking off your socks, $a^{-1}$.
\end{enumerate}


\item[2.18] In a group, prove that $(a^{-1})^{-1} = a$ for all $a$.
\begin{enumerate}
\item[] By theorem 2.3, for each element $a$ in a group $G$, there is a unique element 
$a^{-1}$ in $G$ such that $aa^{-1} = a^{-1}a = e$. By this theorem, it also means that
for any $a^{-1}$ in $G$, there is a unique element $(a^{-1})^{-1}$ in $G$ such that 
$a^{-1}[(a^{-1})^{-1}] = [(a^{-1})^{-1}]a^{-1} = e$. Thus by the uniqueness of 
$a^{-1}$ in $G$, $(a^{-1})^{-1}$ must be $a$ for all $a$.
\end{enumerate}

\item[2.19] For any elements $a$ and $b$ from a group and any integer $n$, prove
that $(a^{-1}ba)^n = a^{-1}b^na$.
\begin{enumerate}
\item[] To show that $(a^{-1}ba)^n = a^{-1}b^na$, first let us show the case when $n = 0$; \\
$(a^{-1}ba)^0 = e = a^{-1}a = a^{-1}ea = a^{-1}b^0a$. 
Now, we can use induction to show all integers $n \geq 1$. Our base case is when
$n = 1$, then $(a^{-1}ba)^1 = a^{-1}ba = a^{-1}b^1a$. Now, we assume that 
$(a^{-1}ba)^n = a^{-1}b^na$ holds true for $n$. Then we show 
\[ (a^{-1}ba)^{n + 1} = (a^{-1}ba)^n(a^{-1}ba) = (a^{-1}b^na)(a^{-1}ba) = \] 
\[ a^{-1}b^n(aa^{-1})ba = a^{-1}b^neba = a^{-1}b^nba = a^{-1}b^{n + 1}a \]
Now, we can show for when $n$ is negative by using the definition that 
$g^n = (g^{-1})^{|n|}$, but clearly we don't need the absolute value exponent
since we're only going to be using positive integers for $n$. Hence, we use induction again,
so that our base case is when $n = 1$, then $[(a^{-1}ba)^{-1}]^1 = (a^{-1}ba)^{-1} = a^{-1}b^{-1}a$.
Now, we assume $[(a^{-1}ba)^{-1}]^n = a^{-1}b^na$ holds true for $n$. Then, we show
\[ 
[(a^{-1}ba)^{-1}]^{n + 1} = [(a^{-1}ba)^{-1}]^n \cdot (a^{-1}ba)^{-1} = 
\]
\[
a^{-1}(b^{-1})^na \cdot (a^{-1}ba)^{-1} = a^{-1}(b^{-1})^na \cdot a^{-1}b^{-1}a 
\]
\[
= a^{-1}(b^{-1})^n(aa^{-1})b^{-1}a = a^{-1}(b^{-1})^neb^{-1}a = 
\]
\[ 
a^{-1}[(b^{-1})^nb^{-1}]a = a^{-1}(b^{-1})^{n + 1}a 
\]

\end{enumerate}


\item[3.1] For each group in the following list, find the order of the group and
the order of each element in the group. What relation do you see between the orders
of the elements of a group and the order of a  group? 
$\mathbb{Z}_{12}, U(10), U(12), U(20), D_4$.
\begin{enumerate}
\item[]
\vskip .9cm
\begin{tabular}{|l|l|l|l|l|}
\hline
$\mathbb{Z}_{12}$ & $U(10)$ & $U(12)$ & $U(20)$ & $D_4$\\
\hline
$|0|$ = 1   & $|1|$ = 1 & $|1|$ = 1  & $|1|$ = 1  & $|R_0|$ = 1 \\
$|1|$ = 12  & $|3|$ = 4 & $|5|$ = 2  & $|3|$ = 4  & $|R_{90}|$ = 4 \\
$|2|$ = 6   & $|7|$ = 4 & $|7|$ = 2  & $|7|$ = 4  & $|R_{180}|$ = 2 \\
$|3|$ = 4   & $|9|$ = 2 & $|11|$ = 2 & $|9|$ = 2  & $|R_{270}|$ = 4 \\
$|4|$ = 3   &           &            & $|11|$ = 2 & $|H|$ = 2 \\
$|5|$ = 12  &           &            & $|13|$ = 4 & $|V|$ = 2 \\
$|6|$ = 2   &           &            & $|17|$ = 4 & $|D|$ = 2 \\
$|7|$ = 12  &           &            & $|19|$ = 2 & $|D'|$ = 2 \\
$|8|$ = 3   &           &  &  &  \\
$|9|$ = 4   &           &  &  &  \\
$|10|$ = 6  & &  & &  \\
$|11|$ = 12 & &  & &  \\
\hline
\end{tabular} \\
Now the order of the groups: $|\mathbb{Z}_{12}| = 12$, $|U(10)| = 4$, $|U(12)| = 4$,
$|U(20)| = 8$, and $|D_4| = 8$. The order of each element in the group is a divisor of
the order of the group. 
\end{enumerate}


\item[3.4] Prove that in any group, an element and its inverse have the same order.
\begin{enumerate}
\item[] Suppose that the order of an element $a$ in $G$ is $n$ and $n$ is finite. Then we know
that $a^n = e$, where $e$ is the identity element in $G$. Multiply both sides (left or right)
by the inverse of $a^n$ which is $(a^{-1})^n$. Thus 
\[ a^n = e \Leftrightarrow (a^{-1})^n \cdot a^n = (a^{-1})^n \cdot e \Leftrightarrow e 
= (a^{-1})^n \]
Clearly, $a^{-1}$ has the same order as element $a$ and is also the inverse of $a$. Now, 
if the order of $a$ happens to be infinite, then $a^{-1}$ must also be of infinite order
otherwise it would contradict what was just proven above.
\end{enumerate}

\item[3.14] If $H$ and $K$ are subgroups of $G$, show that $H \cap K$ is a subgroup of $G$.
\begin{enumerate}
\item[] Suppose that $H$ and $K$ are both subgroups of $G$. Then by definition, both
are non-empty and both contain the identity, $e$. Since, $e \in H$ and $e \in K$, then
$e \in H \cap K$, which makes $H \cap K$ non-empty. Now, to prove that $H \cap K$ is a 
subgroup, by theorem 3.1, we show that when $a, b \in H \cap K$ then $ab^{-1} \in H \cap K$.
So, let $a, b \in H \cap K$, which means $a, b \in H$ and $a, b \in K$. But we already know
that $H$ and $K$ are subgroups, so we know that $ab^{-1} \in H$ and $ab^{-1} \in K$. Hence
we have $ab^{-1} \in H \cap K$.
\end{enumerate}

\item[3.19] Prove theorem 3.6.
\begin{enumerate}
\item[] If $C(a) = G$, then we are done. So let us assume that $C(a) \neq G$. Showing
that $e \in C(a)$ is trivial, since $e \cdot a = a \cdot e = a$. Next, suppose that
$b, c \in C(a)$. Then, $(bc)a = b(ca) = b(ac) = (ba)c = (ab)c = a(bc)$, which shows
that $bc \in C(a)$.
Now, we take some 
$b \in C(a)$ and show that $b^{-1}$ is also in $C(a)$. So, 
\begin{eqnarray*}
b \cdot a &=& a \cdot b \\
a &=& b^{-1} \cdot a \cdot b \hskip 1.1cm \mbox{multiply on left by inverse of b}\\
a \cdot b^{-1} &=& b^{-1} \cdot a \hskip 1.6cm \mbox{multiply on right by inverse of b}
\end{eqnarray*}
Thus $b^{-1} \in C(a)$ whenever $b$ is.
\end{enumerate}

\item[3.27] Suppose a group contains elements $a$ and $b$ such that $|a| = 4, |b| = 2,$ and
$a^3b = ba$. Find $|ab|$.
\begin{enumerate}
\item[] 
\[
e = ee = a^4b^2 = aa^3bb = a(a^3b)b = a(ba)b = (ab)(ab) = (ab)^2
\]
First thing to note is that $|a| \neq |b|$, which means by Exercise 3.4
that one is not an inverse to the other. We get the second inequality from the order of $a$
and $b$, then we use associativity, followed by the given equality, followed by more 
associativity, to conclude that $|ab| = 2$.
\end{enumerate}

\item[3.28] Consider the elements $A = \left[ \begin{array}{cc} 0 & -1 \\ 1 & 0 \end{array}
\right]$ and $B = \left[ \begin{array}{cc} 0 & 1 \\ -1 & -1 \end{array} \right]$ from
$SL(2, \mathbb{R})$. Find $|A|, |B|,$ and $|AB|$. Does your answer surprise you?
\begin{enumerate}
\item[] So, $A = \left[ \begin{array}{cc} 0 & -1 \\ 1 & 0 \end{array} \right]$, then
$A^2 = \left[ \begin{array}{cc} -1 & 0 \\ 0 & -1 \end{array} \right]$, 
$A^3 = \left[ \begin{array}{cc} 0 & 1 \\ -1 & 0 \end{array} \right]$, and
$A^4 = \left[ \begin{array}{cc} 1 & 0 \\ 0 & 1 \end{array} \right]$. So $|A| = 4$. 
Now, $B = \left[ \begin{array}{cc} 0 & 1 \\ -1 & -1 \end{array} \right]$, then
$B^2 = \left[ \begin{array}{cc} -1 & -1 \\ 1 & 0 \end{array} \right]$, and
$B^3$ is the identity matrix. Hence, $|B| = 3$. To try to find the order of $AB$, see that 
$AB = \left[ \begin{array}{cc} 1 & 1 \\ 0 & 1 \end{array} \right]$, then 
$(AB)^2 = \left[ \begin{array}{cc} 1 & 2 \\ 0 & 1 \end{array} \right]$, and
$(AB)^3 = \left[ \begin{array}{cc} 1 & 3 \\ 0 & 1 \end{array} \right]$. In fact,
we can find $(AB)^n$ by $\left[ \begin{array}{cc} 1 & n \\ 0 & 1 \end{array} \right]$,
thus we can conclude that the order of $AB$ is infinite. The answer does not surprise me.
\end{enumerate}

\item[3.30] For any positive integer $n$ and any angle $\theta$, show that in the group
$SL(2, \mathbb{R})$,
\[
\left[ \begin{array}{cc} \cos \theta & -\sin \theta \\ \sin \theta & \cos \theta \end{array} \right]^n =
\left[ \begin{array}{cc} \cos n\theta & -\sin n\theta \\ \sin n\theta & \cos n\theta \end{array} \right].
\]
Use this formula to find the order of 
$\left[ \begin{array}{cc} \cos 60^\circ & -\sin 60^\circ \\
\sin 60^\circ & \cos 60^\circ \end{array} \right]$ and

$\left[ \begin{array}{cc} \cos (\sqrt{2})^\circ & -\sin (\sqrt{2})^\circ \\
\sin (\sqrt{2})^\circ & \cos (\sqrt{2})^\circ \end{array} \right]$.
\begin{enumerate}
\item[] Firstly, the trigonometric identities to be used. \\
Sum-difference formulas
\begin{eqnarray}
\cos (u + v) &=& \cos u \cos v - \sin u \sin v \\
\sin (u + v) &=& \sin u \cos v + \cos u \sin v
\end{eqnarray}

We will use induction. Our base case is when $n = 1$; \\
$\left[ \begin{array}{cc} \cos \theta & -\sin \theta \\ \sin \theta & \cos \theta \end{array} \right]^1 =
\left[ \begin{array}{cc} \cos (1\cdot\theta) & -\sin (1\cdot\theta) \\
\sin (1\cdot\theta) & \cos (1\cdot\theta) \end{array} \right]$ is clearly true. \\
Now assume that
$\left[ \begin{array}{cc} \cos \theta & -\sin \theta \\ \sin \theta & \cos \theta \end{array} \right]^n =
\left[ \begin{array}{cc} \cos n\theta & -\sin n\theta \\ \sin n\theta & \cos n\theta \end{array} \right]$ 
is true for $n$. Then, \\
$ \left[ \begin{array}{cc} \cos \theta & -\sin \theta \\ \sin \theta & \cos \theta 
\end{array} \right]^{n + 1} = \left[ \begin{array}{cc} \cos \theta & -\sin \theta \\ \sin \theta & \cos \theta 
\end{array} \right]^n 
\left[ \begin{array}{cc} \cos \theta & -\sin \theta \\ \sin \theta & \cos \theta \end{array} \right] = $
\[ \left[ \begin{array}{cc} \cos n\theta & -\sin n\theta \\ \sin n\theta & \cos n\theta \end{array} \right]
\left[ \begin{array}{cc} \cos \theta & -\sin \theta \\ \sin \theta & \cos \theta \end{array} \right] = \]
\[
\left[ \begin{array}{cc} (\cos n\theta)(\cos \theta) - (\sin n\theta)(\sin \theta)
& (\cos n\theta)(-\sin \theta) - (\sin n\theta)(\cos \theta) \\
(\sin n\theta)(\cos \theta) + (\cos n\theta)(\sin \theta)
& (-\sin \theta)(\sin n\theta) + (\cos n\theta)(\cos \theta)
\end{array} \right]
\]
\[ 
= \left[ \begin{array}{cc} \cos (n + 1)\theta & -\sin (n + 1)\theta \\
\sin (n + 1)\theta & \cos (n + 1)\theta \end{array} \right]
\]
The last equality comes from row 1:column 1 and row 2:column 2 both using (1), while
row 1:column 2 and row 2:column 1 use (2). \\
Using the just proven formula, we can see that the order of \\
$\left[ \begin{array}{cc} \cos 60^\circ & -\sin 60^\circ \\
\sin 60^\circ & \cos 60^\circ \end{array} \right]$ is going to be 6, since 
$\cos 360^\circ = 1$ and $\pm \sin 360^\circ = 0$. The order of \\
$\left[ \begin{array}{cc} \cos (\sqrt{2})^\circ & -\sin (\sqrt{2})^\circ \\
\sin (\sqrt{2})^\circ & \cos (\sqrt{2})^\circ \end{array} \right]$ is infinite
because in order to get the identity matrix, $\theta$ must equal $0, 360^\circ, 720^\circ, \ldots$, 
this is only possible if $n$ were an irrational number, but since $n$ is a positive integer
then the identity can never be reached.
\end{enumerate}


\item[3.34] $U(15)$ has six cyclic subgroups. List them.
\begin{enumerate}
\item[] The six cyclic subgroups of $U(15)$ are as follows: \\
$\langle 2 \rangle = \{ 1, 2, 4, 8 \}$ \\
$\langle 4\rangle = \{ 1, 4 \}$ \\
$\langle 8\rangle = \{ 1, 8 \}$ \\
$\langle 7\rangle = \langle 13\rangle = \{ 1, 7, 4, 13 \}$ \\
$\langle 11\rangle = \{ 1, 11 \}$ \\
$\langle 14\rangle = \{ 1, 14 \}$
\end{enumerate}


\item[3.44] Let $H = \{ A \in GL(2, \mathbb{R}) \mid $ det $A$ is a power of $2 \}$.
Show that $H$ is a subgroup of $GL(2, \mathbb{R})$.
\begin{enumerate}
\item[] First, we check if the identity is in $H$. Well, the determinant of the identity
matrix is 1 which is equal to $2^0$, so $e \in H$ and $H$ is nonempty. Next, we let
$A, B \in H$ and show that $AB \in H$ as well. Well, the determinant of $A$ is $2^k$ and
the determinant of $B$ is $2^n$ for some $k, n \in \mathbb{Z}$, so by Example 2.9 on page
45, $\det(AB) = \det(A) \cdot \det(B) = 2^k \cdot 2^n = 2^{k + n}$. Since $k + n$ must be
an integer, then the determinant of $AB$ must be a power of 2 and we get that $AB \in H$.
Lastly, we need to show that $A^{-1} \in H$ whenever $A \in H$ to satisfy Theorem 3.2.
Clearly, for any $A \in GL(2, \mathbb{R})$ there exists an inverse, 
$A^{-1} \in GL(2, \mathbb{R})$ otherwise this would contradict $GL(2, \mathbb{R})$ 
being a group. So let $A \in H$ as well, then we know that $\det(AA^{-1}) = 
\det(A) \cdot \det(A^{-1}) = 2^k \cdot \det(A^{-1}) = e$, for some $k \in \mathbb{Z}$.
Since this is true in $GL(2, \mathbb{R})$, then $\det(A^{-1})$ must be $2^{-k}$, which is a 
power of 2 and hence $A^{-1} \in H$ as well.
\end{enumerate}

\item[3.45] Let $H$ be a subgroup of $\mathbb{R}$ under addition. Let $K = \{ 2^a \mid a \in H
\}$. Prove that $K$ is a subgroup of $\mathbb{R}^*$ under multiplication.
\begin{enumerate}
\item[] Firstly $0 \in H$ because $H$ is a subgroup of $\mathbb{R}$ with addition. Here
we know that $2^0 = 1 \in K$. So $K$ is nonempty. Now if $a, b \in K$, then $a = 2^k$ and
$b = 2^n$ for some $k, n \in H$. Furthermore $ab = 2^k2^n = 2^{k + n}$ and with $
k, n \in H$ we know that $k + n \in H$ so $ab \in K$. Now if $c \in K$, then $\exists z
\in H$ such that $c = 2^z$. With $z \in H$, then $-z \in H$ because $H$ is a group under
addition. Thus $2^{-z} \in K$. Thus $2^{-z} \cdot c = 2^{-z} \cdot 2^z = 1$ and
$2^{-z} = -c$, so $c$ has an inverse in $K$. By the two-step subgroup test, we are done.
\end{enumerate}

\item[3.52] Let $G$ be a finite group with more than one element. Show that $G$ has an element of prime order.
\begin{enumerate}
\item[] Let $g \in G, g \neq e$. We have shown that with $G$ finite, then $g$ must also be finite.
Hence let $|g| = n$. Since $g \neq e$, then $n > 1$ and can be expressed as a product of prime
numbers by theorem 0.3. This means that $\exists p \in \mathbb{Z}$ where $p$ is prime such that
$n = p \cdot m$, where $m \in \mathbb{Z}$. Now we have
\[
a^n = a^{mp} = (a^m)^p = e
\]
\end{enumerate}


\item[4.6] What do Exercises 3, 4, and 5 have in common? Try to make a generalization
that includes these three cases.
\begin{enumerate}
\item[] Exercises 4.3 - 4.5 have in common the fact that the two cyclic subgroups given, while
generated by a different element of the set, are in the fact the same cyclic subgroup, as 
satisfying theorem 4.2, corollary 1. A generalization we can make about the two 
generators of the cyclic subgroup given in each exercise is that they are in fact inverses 
to each other in their own respective group, e.g., 10 is the inverse of 20 in $\mathbb{Z}_{30}$.
To show this, we let $a$ be an element from these groups, then $\langle a\rangle = 
\{e, a, a^2, \ldots, a^{n-1}\}$, for $|a| = n$. We know that the $\gcd(n, n-1) = 1$, so
that $\langle a\rangle = \langle a^{n-1}\rangle$ and $a^{n-1} = a^{-1}$, thus
$\langle a\rangle = \langle a^{-1}\rangle$.
\end{enumerate}

\item[4.7] Find an example of a noncyclic group, all of whose proper subgroups are cyclic.
\begin{enumerate}
\item[] $U(8)$ satisfies our requirements. Page 74 shows all the cyclic subgroups of $U(8)$, 
yet none of them produce $U(8)$.
\end{enumerate}

\item[4.10] Let $G = \langle a \rangle$ and let $|a| = 24$. List all generators for the subgroup
of order 8.
\begin{enumerate}
\item[] Let's find the $a^k$ in $G$ such that $| \langle a^k\rangle| = 8$. Since we know that 
$|a^k| = | \langle a^k\rangle|$ from Theorem 4.1, Corollary 1, then 
$| \langle a^k\rangle| = 8 = n/\gcd(n, k) = |a^k|$ from Theorem 4.2.
Now, solving for $\gcd(n, k)$ and letting $n = | \langle a\rangle| = |G| = 24$, 
we get $\gcd(n, k) = 3$. Again utilizing theorem 4.2, $\langle a^k \rangle = 
\langle a^{\gcd(n, k)}\rangle$, thus $a^3$ is a generator for the subgroup of
order 8 in $G$. Now we let $n = |a^3| = 8$ in order find all the generators 
to $\langle a^3\rangle$. From theorem 4.2, Corollary 2, $\langle (a^3)^k\rangle = 
\langle a^3\rangle$ if and only if $\gcd(n, k) = 1$. Finding all such $k$'s reduces
to using all the elements from $U(8)$, in which case $(a^3)^1 = a^3$, the 
generator we already found, $(a^3)^3, (a^3)^5,$ and $(a^3)^7$ are all generators.
\end{enumerate}

\item[4.13] Suppose that $|a| = 24$. Find a generator for $\langle a^{21}\rangle \cap
\langle a^{10}\rangle$. In general, what is a generator for the subgroup 
$\langle a^m \rangle \cap \langle a^n \rangle$?
\begin{enumerate}
\item[] We can say that $\langle a^3\rangle \cap \langle a^2\rangle = \langle a^{21}\rangle 
\cap \langle a^{10}\rangle$, by Theorem 4.2, since $\langle a^{21}\rangle = \langle 
a^{\gcd(24, 21)}\rangle = \langle a^3\rangle$ and $\langle a^{21}\rangle = 
\langle a^{\gcd(24, 10)}\rangle = \langle a^2\rangle$. So looking at some of the elements
of $\langle a^3\rangle = \{\ldots, a^3, a^6, a^9, \ldots \}$ and looking at some 
of the elements of $\langle a^2\rangle = \{\ldots, a^2, a^4, a^6,\ldots \}$, we can 
see that $a^6 \in \langle a^3\rangle \cap \langle a^2\rangle$. Now, we can deduce that
$a^6$ is actually a generator of $\langle a^3\rangle \cap \langle a^2\rangle$ 
since all elements of $\langle a^3\rangle$ are some multiple of 3 while all elements of 
$\langle a^2\rangle$ are multiples of 2 thereby making the all the elements in common multiples of 6 
and $\langle a^6\rangle = \langle a^{21}\rangle \cap \langle a^{10}\rangle$. 
Furthermore, 6 happens to be the lowest common multiple of 2 and 3, which leads
us to a more general form, by the same argument as above,
of finding the generator of some subgroup when $|a| = d$, 
$\langle a^m\rangle \cap \langle a^n\rangle = \langle a^{\gcd(m, d)}\rangle 
\cap \langle a^{\gcd(n, d)}\rangle$, by theorem 4.2, then $\langle a^s\rangle 
\cap \langle a^r\rangle = \langle a^t\rangle$, where $s = \gcd(m, d)$, $r = \gcd(n, d)$, 
and $t = $ lcm$(s, r)$.
\end{enumerate}

\item[T1]
\begin{enumerate}
\item[] Any common multiple of 2 integers is divisible by the lcm of those 2 numbers. \\
{\bf Proof:} Let $m, n \in \mathbb{Z}$. Given any $c \in \mathbb{Z}$, $c$ is a common 
multiple of $m$ and $n$, we need to show lcm$(m, n) | \, c$. So, let $d = \gcd(m, n)$. 
Then $m = dm'$ and $n = dn'$. Given $c$, a common multiple of $m$ and $n$, then 
$c = m \cdot k = dm' \cdot k$ and $c = n \cdot l = dn' \cdot l$. Immediately we see
that $d | \, c$. We want to show that $n' |\, k$. If $n' = 1$ then the result follows.
If not, then we know that the $\gcd(m', n') = 1$ (from our first hw. assignment), thus
$n'$ is the product of primes, none of which divide $m'$. But $n' | \, (c/d) = m'k$ and 
$n'$ does not divide $m'$, which implies that $n' |\, k$. Thus $d \cdot m' \cdot n' | \, c$.
With that fact, we see that $d \cdot m' \cdot n' \leq $ all common multiples of $m$ and
$n$. Moreover, $dm'n' = m \cdot n' = n \cdot m'$, so $dm'n'$ is a common multiple, so it 
is the lcm$(m, n)$.
\end{enumerate}

\item[4.15] Let $G$ be an Abelian group and let $H = \{ g \in G \mid |g| $ divides $12 \}$.
Prove that $H$ is a subgroup of $G$. Is there anything special about 12 here? Would your
proof be valid if 12 were replaced by some other positive integer? State the general result.
\begin{enumerate}
\item[] Let's prove the general result first. Let $H = \{ g \in G : |g| \mbox{ divides } m \}$ where
$m \in \mathbb{N}$. First we note that $e \in G$ and $|e| = 1$ and 1 divides any $m$, so that
$e \in H$ and $H$ is nonempty. Next, to show that $a \in H$ has an inverse in $H$, we know that 
$|a| = |a^{-1}|$ by Exercise 3.4, which means that $|a^{-1}|$ divides $m$ as well. Now if we show 
$ab \in H$ then we can conclude that $H$ is a subgroup of $G$ by theorem 3.2 so let $a, b \in H$ 
and $n = |a|$ and $k = |b|$. Then we know that the lcm$(n, k) |\, m$ by T1. It follows then that 
since $G$ is Abelian that $(ab)^t = a^tb^t = e$, where $t = $ lcm$(n, k)$. Furthermore, 
$|ab| = r \leq t$ but $r$ must divide $t$ and $t | m$ then $r | t$ by transitivity, concluding 
that $ab$ must be in $H$. Now going back and answering the first questions. $H$ must subgroup 
of $G$ when $m = 12$ since we proved it for all positive integers. There is nothing special 
about 12 since 12 is only one integer from the entire set of natural numbers. Yes the proof 
would be valid if we replaced any positive integer.
\end{enumerate}

\item[4.18] If a cyclic group has an element of infinite order, how many elements of finite order does it have?
\begin{enumerate}
\item[] Let $G$ be the cyclic group with $b \in G$ and $b$ is infinite. Furthermore
because $G$ is cyclic we also know that there exists an $a \in G$ such that $\langle a
\rangle = G$. Suppose that $\exists c \in G, c \neq e$ and $|c|$ is finite. Then
$\exists k \in \mathbb{Z}$ such that $c^k = e$. Here we see that $c^{2k} = (c^k)^2 = e^2 = e$ as well. Because $G$ is cyclic with a generator, $a$, $\exists n \in \mathbb{N}$ with $a^n = c$.
We see that $c^k = (a^n)^k = e$ and $c^2k = (a^n)^{2k}$. With $nk < 2nk$, we have contradicted
theorem 4.1. Thus there does not exist a finite element in $G$ other than the identity. We conclude
that there is only 1.
\end{enumerate}

\item[T2]
\begin{enumerate}
\item[] Given a group $G$ and $a, b \in G$, $|a| = |b| = p$, $p$ is a prime number,
then if any $c \in \langle a\rangle, c \neq e$ and $c \in \langle b\rangle$, then
$\langle a\rangle = \langle b\rangle$, otherwise 
$\langle a\rangle \cap \langle b\rangle = \{ e \}$. \\
{\bf Proof:} Let $a, b \in G$ and $|a| = |b| = p$, where $p$ is prime.
Further let $c \in \langle a\rangle, c \neq e$ and $c \in \langle b\rangle$. Thus
$c = a^k$ for some $k$, $1 \leq k \leq n-1$, where $|a| = n$. With $n$ prime, 
$\gcd(n, k) = 1$, thus $\langle c\rangle = \langle a\rangle$ by theorem 4.2. 
By the same reasoning (by symmetry) $\langle b\rangle = \langle c\rangle$.
Thus $\langle b\rangle = \langle a\rangle$. If there doesn't exist a $c \in \langle a
\rangle$ with $c \in \langle b\rangle$, $c \neq e$, then with 
$\langle a\rangle$ and $\langle b\rangle$ groups, it follows that 
$\langle a\rangle \cap \langle b\rangle = \{ e \}$.
\end{enumerate}

\item[4.20] Suppose that $G$ is an Abelian group of order 35 and every element of $G$ satisfies
the equation $x^{35} = e$. Prove that $G$ is cyclic. Does your argument work if 35 is replaced
by 33?
\begin{enumerate}
\item[] First, since $G$ is an Abelian group, we know that $(ab)^n = a^nb^n$ from
Exercise 2.15. Since every element of $G$ satisfies $x^{35} = e$, we also know
by theorem 4.1, corollary 2 that every element $a \in G, a \neq e$ must have an
order of 5, 7, or 35. If we know there is some element in $G$ that is order 35, then we
are done and know that $G$ is cyclic. Now, we can show that there must exist an element in $G$ in which
the order is 35 if there are two elements $a, b \in G, a \neq e \neq b$ such that $|a| = 5$ and $|b| = 7$.
We know that $|ab|$ must be a divisor of 35 and $(ab)^5 = a^5b^5 = eb^5 = b^5 \neq e$ 
and that $(ab)^7 = a^7b^7 = a^7e = a^7 \neq e$. Clearly, $|ab| \neq 5$ and $|ab| \neq 7$, thus
it must be that $|ab| = 35$. Now, to show that all the 
elements of $G$ cannot have either order 5 or order 7, let's assume for the sake of contradiction that 
$\forall a \in G, a \neq e$, the order of $a$ is 5. Then since $G$ is finite and the order is 
exactly 35, we can break $G$ down into its subgroups each of order 5, each one containing
exactly 4 elements (not the identity) plus the identity. Then we have 8 groups containing 4 
nonidentity elements with the identity and 2 nonidentity elements leftover from $G$ which 
have no group to belong to of order 5, hence a contradiction by T2. Likewise, if we have 
$\forall a \in G, a \neq e$, the order of $a$ is 7, then we will have 5 groups of 
6 nonidentity elements with the identity and 2 nonidentity elemenst of $G$ leftover,
contradicting that all the elements of $G$ have order 7. \\
No, the argument above would not work if 35 was replaced with 33. Since, we would
break down $G$ into its subgroups, each with order 3 (2 nonidentity elements and
the identity) and since we have 32 nonidentity elements in $G$, then these subgroups 
would break down evenly since 32 is a multiple of 2.
\end{enumerate}

\item[4.22] Prove that a group of order 3 must be cyclic.
\begin{enumerate}
\item[] Theorem 3.4 tells us that if $G = \{e, a, b \}$ is a group with all elements distinct,
then $\langle a\rangle \subseteq G$. Moreover, theorem 4.1 shows that $|a| = |\langle a\rangle |$.
If we let $e$ be the identity of $G$, it suffices to show that $|a| = |\langle a\rangle | = |G|$
to show that $\langle a\rangle = G$. So, suppose that $\langle a\rangle \neq G$, then $|a| < 3$.
Also since $a \neq e$, then $|a| \neq 1 \neq |\langle a\rangle|$, so $|a| = 2 = |\langle a\rangle|$. Because $G$ is a group we have that $ab \in G$ and with $a \neq e$ then $ab \neq b$ and 
with $b \neq e$ then $ab \neq a$ either. Thus $ab = e$, hence $a(ab) = ae$ which implies that
$a^2b = a$, yet $a^2 = e$ so $b = a$. Contradiction since $a$ and $b$ were distinct. Thus
$|\langle a\rangle| = 3 \Rightarrow \langle a\rangle = G$.
\end{enumerate}

\item[4.28] Let $a$ be a group element that has infinite order. Prove that $\langle a^i\rangle
= \langle a^j\rangle$ if and only if $i = \pm j$.
\begin{enumerate}
\item[] Suppose that $i = \pm j$. Well, then we can break it down to two cases. When 
$i = j$ and when $i = -j$. Firstly, when $i = j$, then it is quite obvious that
$a^i = a^j$, which means by definition that $\langle a^i\rangle = \langle a^j\rangle$.
Now, when $i = -j$, we know that $i = -i$, which we know by definition that
$a^{-i} = (a^{-1})^{|i|}$, which then shows us that $(a^{-1})^{|i|} \in \langle a^i\rangle$.
There is no need to show that $a^i \in \langle a^i\rangle$ for all nonnegative integers $i$, thus
$\langle a^i\rangle = \langle a^{-j}\rangle$. 
\end{enumerate}

\item[4.33] Determine the subgroup lattice for $\mathbb{Z}_{p^2q}$, where $p$ and $q$ are
distinct primes.
\begin{enumerate}
\item[]
\begin{displaymath}
\xymatrix{
& \langle 1 \rangle \ar@{-}[dl] \ar@{-}[dr] &    \\
\langle p\rangle \ar@{-}[d] \ar@{-}[drr] &  & \langle q\rangle \ar@{-}[d] \\
\langle p^2 \rangle \ar@{-}[dr] &  & \langle pq \rangle \ar@{-}[dl] \\
& \langle p^2q\rangle &  \\
}
\end{displaymath}
\end{enumerate}

\item[4.36] Prove that a finite group is the union of proper subgroups if and only if
the group is not cyclic.
\begin{enumerate}
\item[] Suppose that a finite group, $G$, is the union of proper subgroups, \\
that is $G = G_1 \cup G_2 \cup \cdots \cup G_k$. Now let's assume for the sake of contradiction 
that $G$ is cyclic. This means that there is some $a \in G$ such that $\langle a\rangle = G$. 
But then since $G$ is comprised of a finite number of $G_i$, then $a \in G_i$. Clearly, whatever 
$G_i$ that contains $a$, must also contain $G$, which contradicts $G_i$ being a proper
subset. Conversely, suppose that a finite group $G$ is not cyclic. This means that there is no
$a \in G$ such that $\langle a\rangle = G$, thus for every $a \in G$, $\langle a\rangle 
< G$, by theorem 3.4. Then it must follow that $G = \bigcup_{a \in G}\langle a\rangle$
and we're done.
\end{enumerate}

\item[4.42] Prove that an infinite group must have an infinite number of subgroups.
\begin{enumerate}
\item[] Suppose that $G$ is an infinite group. For the sake of contradiction, let's assume 
that $G$ has a finite number of subgroups. Let $C$ be a finite subset of $G$ such that
for every $a, b \in C$, then $\langle a\rangle \cap \langle b\rangle
= \emptyset$ and $\bigcup_{a \in C}\langle a\rangle = G$. Clearly, there has to be
one infinite $\langle a\rangle$ otherwise our finite union of subgroups would
fall short of being equal to $G$. Now, since $\langle a\rangle$ is infinite, then we know
by theorem 4.1 that all distinct powers of $a$ are distinct group elements. Furthermore,
$\langle a^i\rangle$ is a subgroup of $\langle a\rangle$ for all $a^i \in \langle a\rangle$. 
Thus a contradiction that $G$ has a finite amount of subgroups.
\end{enumerate}

\item[4.60] Suppose that $|x| = n$. Find a necessary and sufficient condition on $r$ and
$s$ such that $\langle x^r\rangle \subseteq \langle x^s \rangle$.
\begin{enumerate}
\item[] Just by inspecting some of the examples on page 79 and 80, we can build a conjecture
that our necessary and sufficient condition is going to be $\langle x^r\rangle \subseteq 
\langle x^s\rangle$ if and only if $s|r$, where $s$ and $r$ have to be divisors of $n$. 
Now, let us see if this holds true. Suppose that $s|r$, then for some nonzero integer $k$ and since
$r|n$, $x^n = x^{kr}$ by theorem 0.2, but this is the identity and clearly the identity
is in $\langle x^s\rangle$. So, if we let $0 < k < n/r$, then
$x^{kr} = (x^k)^r = (x^{st})^r = (x^s)^{tr} \in \langle x^s\rangle$, for some nonzero integer $t$ and 
by theorem 0.2 again. Thus, we can conclude that $\langle x^r\rangle \subseteq 
\langle x^s\rangle$. Conversely, suppose that $\langle x^r\rangle \subseteq 
\langle x^s\rangle$, then working backwards from the proof of Theorem 4.2 on page 76, we know that
$a^r = (a^s)^t$ for some integer $t$. By what was said on page 75 about Theorem 4.1 in the
finite case, this is essentially, addition modulo $n$, so $r = st$, which by theorem 0.2 is a 
linear combination of $r$ with remainder 0, thus $s$ must divide $r$.
\end{enumerate}

\item[4.64] Suppose that $G$ is a finite group with the property that every nonidentity
element has prime order (for example, $D_3$ and $D_5$). If $Z(G)$ is not trivial, prove
that every nonidentity element of $G$ has the same order.
\begin{enumerate}
\item[] Since $Z(G)$ is not trivial, then that means there exists an element, $a \in Z(G), a \neq e$.
Since, $a \in Z(G)$, then we know that for all $b \in G$, $ab = ba$ and $(ab)^n = a^nb^n$. Now, let's
assume for the sake of contradiction that $m = |a| \neq |b| = n$, yet both orders are prime. Then 
it must be that the lcm$(m, n) = mn$ and $(ab)^{mn} = e$. So let $k = |ab|$ and by theorem 4.1, 
corollary 2, $k$ must divide $mn$. In order for $k$ to divide $mn$ because $m$ and $n$ are distinct
primes, $k$ must be $n$ or $m$ or $mn$. But then, 
\[
(ab)^m = a^mb^m = eb^m = b^m \neq e \] \[
(ab)^n = a^nb^n = a^ne = a^n \neq e
\]
which means that $k$ must be $mn$, since neither $m |\, n$ nor $n |\, m$. Clearly $mn$ is not prime
since $m | \, mn$ and $n | \, mn$ which means that $ab \not\in G$, which is a contradiction. Therefore,
$|a| = |b|$.
\end{enumerate}

\item[5.1] Find the order of each of the following permutations.
\begin{enumerate}
\item[a.] $(12)$ \\
Since there is only one cycle of length 2, then the order must be
2 as any other cycle must be a one cycle (or fixed) and the lcm of all the cycles is 
going to be 2.
\item[b.] $(147)$ \\
For the same reasoning as part (a); the order is 3.
\item[c.] $(14762)$ \\
For the same reasoning as part (a) and (b); the order is 5. 
\end{enumerate}

\item[5.2] What is the order of a $k$-cycle $(a_1a_2\cdots a_k)$?
\begin{enumerate}
\item[] We let 
$(a_1a_2 \cdots a_k)$ be our $k$-cycle and we'll denote the mapping
of all $a_i$ in the $k$-cycle to $a_{i+1}$ , with the exception of $a_k$ which
maps to $a_1$, by $\sigma$. Since we have $a_k$ going to $a_1$, we have actually
have $\sigma(a_i) = a_{i + 1 \mod k}$. 
Then we know inductively that if $\sigma(a_i) = a_{i + 1 \mod k}$, then
$\sigma^2(a_i) = \sigma(\sigma(a_i)) = \sigma(a_{i+1 \mod k}) = a_{i + 1 + 1 \mod k} = 
a_{i + 2 \mod k}$ and so on and so forth until we get $\sigma^k(a_i) = a_{i + k \mod k} = a_i$.
Thus if $1 \leq m < k$, then $\sigma^m(a_i) = a_{i + m \mod k} \neq a_{i + k \mod k} = a_i$ which
let's us know that $\sigma^k$ is the mapping which sends an element back to itself, concluding
that the order of a $k$-cycle must be $k$.
\end{enumerate}

\item[5.3] What is the order of each of the following permutations?
\begin{enumerate}
\item[a)] $(124)(357)$ \\
By theorem 5.3, the lcm of these disjoint cycles is 3. Thus the order is 3.
\item[b)] $(124)(3567)$ \\
Following part (a), lcm(3, 4) = 12
\item[c)] $(124)(35)$ \\
" ", lcm(2, 3) = 6
\item[d)] $(124)(357869)$ \\
" ", lcm(3, 6) = 6
\item[e)] $(1235)(24567)$ \\
Need to write in disjoint cycle form, so we have (241)(5673) and
the order is the lcm(3, 4) which is 12.
\item[f)] $(345)(245)$ \\
Need to write in disjoint cycle form, so we have (25)(43) and
the order is the lcm(2, 2) which is 2.
\end{enumerate}

\item[5.4] What is the order of each of the following permutations. 
\begin{enumerate}
\item[a.] When we write 
\[
\left[ 
\begin{array}{cccccc} 
1 & 2 & 3 & 4 & 5 & 6 \\
2 & 1 & 5 & 4 & 6 & 3 
\end{array} 
\right]
\]
in cycle notation, we have $(12)(356)$, which by Theorem 5.3, 
the lcm$(2, 3) = 6$, which means that 6 is the order.
\item[b.] 
\[
\left[ 
\begin{array}{ccccccc} 
1 & 2 & 3 & 4 & 5 & 6 & 7 \\
7 & 6 & 1 & 2 & 3 & 4 & 5
\end{array} 
\right]
\]
in cycle notation, we have $(1753)(264)$, which by Theorem 5.3, 
the lcm$(4, 3) = 12$, which means that 12 is the order.
\end{enumerate}

\item[5.8] What is the maximum order of any element in $A_{10}$?
\begin{enumerate}
\item[] Without enumerating all the possibilities, we can see that any odd $k$-cycle, $k < 10$, 
with fixed points is going to be in $A_{10}$, so we know that we have elements of order 1, 3, 
5, 7, 9. But then so are the combinations of two disjoint cycles of 8 and 2, 7 and 3, 6 and 4, 5 
and 5. It turns out that 7 and 3, in this case is going to yield the highest order since 
lcm$(7, 3) = 21$. Listing the rest is negligible since the only one that comes close to 21 is
two cycles of length 5 and 3 with 2 fixed points, which has an lcm of 15.
\end{enumerate}

\item[5.13] Prove Theorem 5.6.
\begin{enumerate}
\item[] Let's use the two-step subgroup test. The set of even permutations in $S_n$ is nonempty
since the identity can always be expressed as an even number of transpositions. Also for any even
permutation, $\sigma$, in $S_n$, then the inverse is just the reversed transpositions which made up the 
even permutation in the first place. So the number of transpositions doesn't change, thus
the $\sigma^{-1}$ is an even permutation. Now, we let $\sigma$ and $\tau$ both be even
permutations in $S_n$, then we know that $\sigma$ can be expressed as $n$ transpositions
and $\tau$ can be expressed as $m$ transpositions, where $n$ and $m$ are even. It follows 
that $\sigma\tau$ can be expressed as $n + m$ transpositions, which is even.
\end{enumerate}

\item[5.17] Let 
\[
\alpha = 
\left[
\begin{array}{cccccc} 
1 & 2 & 3 & 4 & 5 & 6 \\
2 & 1 & 3 & 5 & 4 & 6
\end{array}
\right]
\mbox{ and } 
\beta = 
\left[
\begin{array}{cccccc} 
1 & 2 & 3 & 4 & 5 & 6 \\
6 & 1 & 2 & 4 & 3 & 5
\end{array}
\right].
\]
Compute each of the following.
\begin{enumerate}
\item[a.] $\alpha^{-1}$ \\
Mentioned in class and turns out that $\alpha = \alpha^{-1}$.
\item[b.] $\beta\alpha$ \\
Trace out a few, so as to show \emph{some} underlying work; 
$\beta(\alpha(1)) = \beta(2) = 1$ and $\beta(\alpha(4)) = \beta(5) = 3$. 
\[ 
\left[ 
\begin{array}{cccccc} 
1 & 2 & 3 & 4 & 5 & 6 \\
1 & 6 & 2 & 3 & 4 & 5 
\end{array} 
\right]
\]
\item[c.] $\alpha\beta$ \\
In the same fashion as part (b);
$\alpha(\beta(1)) = \alpha(6) = 6$ and $\alpha(\beta(4)) = \alpha(4) = 5$.
\[ 
\left[ 
\begin{array}{cccccc} 
1 & 2 & 3 & 4 & 5 & 6 \\
6 & 2 & 1 & 5 & 3 & 4 
\end{array} 
\right]
\]
\end{enumerate}

\item[5.18] Let 
\[
\alpha = 
\left[
\begin{array}{cccccccc} 
1 & 2 & 3 & 4 & 5 & 6 & 7 & 8\\
2 & 3 & 4 & 5 & 1 & 7 & 8 & 6
\end{array}
\right]
\mbox{ and } 
\beta = 
\left[
\begin{array}{cccccccc} 
1 & 2 & 3 & 4 & 5 & 6 & 7 & 8\\
1 & 3 & 8 & 7 & 6 & 5 & 2 & 4
\end{array}
\right].
\]
Write $\alpha, \beta,$ and $\alpha \beta$ as
\begin{enumerate}
\item[a)] products of disjoint cycles,\\
$\alpha = (12345)(678)$, $\beta = (23847)(56)$ but we
have to find $\alpha \beta$. So, 
\[
\alpha \beta = 
\left[
\begin{array}{ccccccc}
1 & 2 & 3 & 4 & 5 & 6 & 7 \\
2 & 4 & 6 & 8 & 7 & 1 & 3
\end{array}
\right]
\]
which is $\alpha \beta = (12485736)$.
\item[b)] products of 2-cycles.
so using Example 5.4, we see that 
(12345) = (15)(14)(13)(12), then (678) = (68)(67), so
$\alpha = (15)(14)(13)(12)(68)(67)$, \\
$\beta = (27)(24)(28)(23)(56)$ and 
$\alpha \beta = (16)(13)(17)(15)(18)(14)(12)$.
\end{enumerate}

\item[5.19] Show that if $H$ is a subgroup of $S_n$, then either every member of $H$
is an even permutation or exactly half of the members are even.
\begin{enumerate}
\item[] Following what was said in class: If every member of $H$ is an even permutation, then
we are done. So let us suppose that not every member of $H$ is even. Then there is some 
$\tau$ which is an odd permutation and we let $f$ be the function which maps an 
even permutation, $\sigma$, to an odd by composing with $\tau$. So, 
$f(\sigma) = \tau\sigma$ and this function is one-to-one, which we can show by 
$f(\sigma) = f(\sigma') \Leftrightarrow \tau\sigma = \tau\sigma'$ which implies 
$\sigma = \sigma'$ because of cancellation. Now by the same argument, $f^{-1}$, which maps an odd 
permutation to an even one by composing it with $\tau$, is also one-to-one. Thus, $f$ is
one-to-one correspondence, since we have shown that it is one-to-one and invertible, so that for every even 
permutation there is exactly one corresponding odd permutation from which we can conclude that 
exactly half of the members of $H$ are even and other half are odd.
\end{enumerate}

\item[5.23] Use Table 5.1 to compute the following.
\begin{enumerate}
\item[a)] The centralizer of $\alpha_3 = (13)(24)$. \\
Using table 5.1, to find the centralizer of $\alpha_3$, we compare the entry 
in row $\alpha_3$ and column $\alpha_i$ to row $\alpha_i$ and column $\alpha_3$ for all $\alpha_i$ 
in $A_4$ and when the two are equal $\alpha_i \in C(\alpha_3)$. Thus we get 
$C(\alpha_3) = \{ \alpha_1, \alpha_2, \alpha_3, \alpha_4 \}$.
\item[b)] The centralizer of $\alpha_{12} = (124)$.
Following the same procedure as part (a), except it being a little more easier and 
thus quicker to compare the entries as they are on the very tip of the table, we get
$C(\alpha_{12}) = \{ \alpha_1, \alpha_7, \alpha_{12} \}$.
\end{enumerate}

\item[5.27] Let $\beta \in S_7$ and suppose $\beta^4 = (2143567)$. Find $\beta$.
\begin{enumerate}
\item[] Since we can clearly see that $|\beta^4| = 7$, then we know that
$|\beta^{28}| = \varepsilon$ and that $|\beta|$ must divide 28. We know
the divisors of 28 are 1, 2, 4, 7, 14, and 28. So, since
$|\beta^4| \neq \varepsilon$, then we can eliminate 1, 2, and 4 as being the
order of $\beta$ right away. Now to show that the order of $\beta$ cannot be 14,
we can use theorem 5.3, which means that the lcm of the disjoint cycles that
make up $\beta$ must be 14. Thus there must be at least one 14-cycle or at least
one 7-cycle and 2-cycle, but both are impossible since we are in $S_7$ and the
former requires 14 symbols while the latter needs 9. By the same argument (different
numbers, same result), 28 cannot be the order of $\beta$ either, thus it must be 7. 
Now we can use the same solution to finding $\beta$: \\
First, let's observe that \\
$(a_1, a_2, a_3, a_4, a_5, a_6, a_7)^2 = (a_1, a_3, a_5, a_7, a_2, a_4, a_6)$.
Now, we can work backwards, since $\beta^4 = (1435672) = (\beta^2)^2 = 
(1647325)^2 = [(\beta)^2]^2 = [(1362457)^2]^2$. Thus, $\beta =
(1362457)$.
\end{enumerate}


\item[5.28] Let $\beta = (123)(145)$. Write $\beta^{99}$ in disjoint cycle form.
\begin{enumerate}
\item[] Firstly, $\beta = (123)(145) = (14523)$. Then $\beta^{99} = \beta^{-1} = (32541)$.
[Damian, I know you'll probably have questions as to why that is...well, we know the $\beta$
is a 5-cycle, thus the order of it is 5, since any multiple of 5 is going to give us 
$\varepsilon$, we know that this reduces to $\varepsilon \cdot \beta^4$, but $\beta^4 =
\beta^{-1}$ \ldots probably wonder why that is as well? well $\beta^4 \cdot \beta^1 = 
\beta^5 = \varepsilon$ and $\beta^{-1} \cdot \beta = \beta^0 = \varepsilon$ and we're done, 
woohoo!]
\end{enumerate}

\item[5.32] Let $\beta = (1, 3, 5, 7, 9, 8, 6)(2, 4, 10)$. What is the smallest positive
integer $n$ for which $\beta^n = \beta^{-5}$?
\begin{enumerate}
\item[] By theorem 5.3, we know that the order of $\beta$ is 21. Thus, $\beta^{20} = \beta^{-1}$,
which helps us to get $\beta^{100} = \beta^{-5}$. But, we want the smallest, so we can use the 
order of $\beta$, then $\beta^{100} = (\beta^{21})^4\beta^{16} = (\varepsilon)^4\beta^{16} = \beta^{16}$.
So our smallest $n$ is going to be 16.
\end{enumerate}

\item[5.36] In $S_4$, find a cyclic subgroup of order 4 and a noncyclic subgroup of order 4.
\begin{enumerate}
\item[] The cyclic subgroup can be achieved by $\langle (1234)\rangle$.
So, \\ $\{ (1), (1234), (1234)^2 = (13)(24), (1234)^3 = (1432)\}$ is our
cyclic subgroup of $S_4$. Now for our noncyclic subgroup; \\
$\{ (1), (14), (23), (14)(23) \}$.
\end{enumerate}

\item[5.37] Suppose that $\beta$ is a 10-cycle. For which integers $i$ between 2 and
10 is $\beta^i$ also a 10-cycle?
\begin{enumerate}
\item[] We can enumerate the powers of $\beta$ to see exactly which $i$'s we want: \\
\begin{tabular}{cc}
$\beta = (a_0a_1a_2a_3a_4a_5a_6a_7a_8a_9)$ & 
$\beta^2 = (a_0a_2a_4a_6a_8)(a_1a_3a_5a_7a_9)$ \\
$\beta^3 = (a_0a_3a_6a_9a_2a_5a_8a_1a_4a_7)$ & 
$\beta^4 = (a_0a_4a_8a_2a_6)(a_1a_5a_9a_3a_7)$ \\
$\beta^5 = (a_0a_5)(a_1a_6)(a_2a_7)(a_3a_8)(a_4a_9)$ & 
$\beta^6 = (a_0a_6a_2a_8a_4)(a_1a_7a_3a_9a_5)$ \\
$\beta^7 = (a_0a_7a_4a_1a_8a_5a_2a_9a_6a_3)$ &
$\beta^8 = (a_0a_8a_6a_4a_2)(a_1a_9a_7a_5a_3)$ \\
$\beta^9 = (a_0a_9a_8a_7a_6a_5a_4a_3a_2a_1)$ &
$\beta^{10} = (a_0)$
\end{tabular}
\\
Clearly, we can see that $\beta^i$ is a 10-cycle when $i = 3, 7, 9$.
\end{enumerate}

\item[5.38] In $S_3$, find elements of $\alpha$ and $\beta$ such that $|\alpha | = 2,
|\beta | = 2,$ and $|\alpha \beta | = 3$.
\begin{enumerate}
\item[] Let $\alpha = (12)$ and $\beta = (13)$. Then $\alpha \beta = (132)$.
So, $|\alpha| = 2 = |\beta|$ and $|\alpha \beta| = 3$.
\end{enumerate}

\item[5.41] Prove that $S_n$ is non-Abelian for all $n \geq 3$.
\begin{enumerate}
\item[] Example 5.1 shows that $S_3$ is non-Abelian because there exists two
permutations which do not commute, namely $\alpha = (123)$ and $\beta = (23)$.
It follows that these permutations are also in $S_n, n > 3$ just that the
other elements are fixed. Thus $S_n, n \geq 3$ is non-Abelian.
\end{enumerate}

\item[5.42] Let $\alpha$ and $\beta$ belong to $S_n$. Prove that $\beta \alpha \beta^{-1}$ and
$\alpha$ are both even or both odd.
\begin{enumerate}
\item[] From lemma 2 of the conjugacy class handout, we know that if $\alpha$ is a $k$-cycle, then 
$\beta\alpha\beta^{-1}$ is also a $k$-cycle and equivalently even or odd, depending on whether
$\alpha$ was even or odd to begin with because both $k$-cyles will have the same number of 
transpositions. Otherwise, $\alpha$ is comprised of multiple disjoint cycles and
must have a cycle type. Hence by theorem 3 of the conjugacy class handout, $\beta\alpha\beta^{-1}$ must
also have the same cycle type and is equivalently odd or even depending on what $\alpha$ was to begin
with. This is true because there is corresponding cycle of $\beta\alpha\beta^{-1}$ such that a cycle 
of $\alpha$ has the same length and since this holds true for all cycles of both $\alpha$ and 
$\beta\alpha\beta^{-1}$ then both $\beta\alpha\beta^{-1}$ and $\alpha$ must be comprised of
the same amount of transpositions.
\end{enumerate}

\item[5.44] If $\beta \in S_7$ and $|\beta^3| = 7$, prove that $|\beta | = 7$.
\begin{enumerate}
\item[] Since $|\beta^3| = 7$, then we know that
$\beta^{21} = \varepsilon$ and that $|\beta|$ must divide 21. We know
the divisors of 21 are 1, 3, 7, and 21. So, since
$\beta^3 \neq \varepsilon$, then we can eliminate 1 and 3 as being the
order of $\beta$ right away. Now to show that the order of $\beta$ cannot be 21,
we can use theorem 5.3, which means that the lcm of the disjoint cycles that
make up $\beta$ must be 21. Thus there must be at least one 21-cycle or at least
one 7-cycle and 3-cycle, but both are impossible since we are in $S_7$ and the
former requires 21 symbols while the latter needs 10. Thus $|\beta|$ must be 7. 
\end{enumerate}

\item[5.45] Show that every element in $A_n$ for $n \geq 3$ can be expressed as a 3-cycle
or a product of three cycles.
\begin{enumerate}
\item[] Since every permutation in $A_n, n \geq 3$ is even and can be expressed as a product of 2-cycles
then it suffices to show that any 2-cycle can be rewritten as the product of 3-cycles.
Now, $(ac)(ab) = (abc)$ and $(ab)(cd) = (cbd)(acb)$, for any $a, b, c, d$ in $A$, where
$A$ is the finite set in which the permutations act upon.
\end{enumerate}

\item[5.46] Show that for $n \geq 3, Z(S_n) = \{ \epsilon \}$.
\begin{enumerate}
\item[] With $Z(S_n)$ a group from chapter 3, we know that $\varepsilon \in Z(S_n)$. Now let 
$\alpha \in S_n, \alpha \neq \varepsilon$. Then we can write $\alpha$ as disjoint cycles from 
theorem 5.1. Now, we consider the cases for $\alpha$ with $S_n : A \rightarrow A$, and 
$A = \{ 1, \ldots, n \}$. \\
\emph{Case: i)} $\alpha$ is a $k$-cycle for $2 \leq k < n$. Then $\alpha = 
(a_1a_2\ldots a_k)$ and there exists $b \in A$ with $b \not\in \{ a_1, \ldots a_k \}$.
Notice that $\alpha(a_1b) = (a_1ba_2 \ldots a_k)$ and $(a_1b)\alpha = (a_1a_2\ldots a_kb)$
so $\alpha \not\in Z(S_n)$. \\
\emph{Case: ii)} $\alpha$ is an $n$-cycle. Then $\alpha = (a_1a_2\ldots a_n)$ with $n \geq 3$, 
we have $(a_1a_2) \in S_n$. Now consider $\alpha (a_1a_2) = (a_1a_3\ldots a_n)$ and
$(a_1a_2)\alpha = (a_2a_3\ldots a_n)$ so $\alpha \not\in Z(S_n)$. \\
\emph{Case: iii)} $\alpha$ is the composition of at least 2 disjoint cycles of lengths $k$
and $l \geq 2$. Then $\alpha = \beta_1 \beta_2 \ldots \beta_r$, where each $\beta_i$ is a
disjoint cycle. Now, $\beta_1 = (a_1a_2\ldots a_k)$ and $\beta_2 = (b_1b_2\ldots b_l)$ and
$\alpha = (a_1a_2\ldots a_k)(b_1b_2\ldots b_l)\ldots \beta_r$. Then $(a_1b_1) \in S_n$ and
is disjoint from all $\beta_i$ for $2 < i \leq r$. So, consider
\begin{eqnarray*}
\alpha (a_1b_1) &=& \beta_1\beta_2(a_1b_1)\ldots \beta_r \\
&=& (a_1b_2\ldots b_lb_1a_2\ldots a_k)\ldots \beta_r
\end{eqnarray*}
and
\begin{eqnarray*}
(a_1b_1)\alpha &=& (a_1b_1)\beta_1\beta_2\ldots \beta_r \\
&=& (b_1b_2\ldots b_la_1a_2\ldots a_k)\ldots \beta_r
\end{eqnarray*}
so $\alpha \not\in Z(S_n)$ with $(a_1b_1)\alpha \neq \alpha (a_1b_1)$. Thus there does
not exist $\alpha \neq \varepsilon$ with $\alpha \in Z(S_n)$.
\end{enumerate}

\item[6.2] Find Aut$(\mathbb{Z})$.
\begin{enumerate}
\item[] Since the only generators of $\mathbb{Z}$ under addition are 1 and -1, 
any possible element of Aut$(\mathbb{Z})$ must map $1 \rightarrow -1$, $1 \rightarrow 1$,
$-1 \rightarrow 1$, or $-1 \rightarrow -1$, by theorem 6.2, property 4. 
We can achieve all these mapping by $x \rightarrow x$ (the identity map) and $x \rightarrow -x$, 
thus finding the only two elements of Aut$(\mathbb{Z})$. 
\end{enumerate}

\item[6.4] Show that $U(8)$ is not isomorphic to $U(10)$.
\begin{enumerate}
\item[] For one, we know that $U(8)$ is not cyclic while $U(10)$ is.
\end{enumerate}

\item[6.6] Prove that the notion of group isomorphism is transitive. That is, if $G, H,$ and
$K$ are groups and $G \cong H$ and $H \cong K$, then $G \cong K$.
\begin{enumerate}
\item[ ] Suppose that $G \cong H$ and $H \cong K$. Since 
$\phi: G \rightarrow H$ is bijective and $\psi: H \rightarrow K$ is bijective, 
then the composition $\psi\phi: G \rightarrow K$ is bijective. Furthermore,  for any $a, b \in G$, 
\[
\psi\phi(ab) = \psi(\phi(ab)) = \psi[\phi(a)\phi(b)] = \psi[\phi(a)]\psi[\phi(b)]
= \psi\phi(a)\psi\phi(b)
\]
with the second equality from the operation preservation of $\phi$ and the third equality
from the operation preservation of $\psi$. Hence $G \cong K$.
\end{enumerate}

\item[6.12] Find two groups $G$ and $H$ such that $G \not\cong H$, but Aut$(G) \cong $ Aut$(H)$.
\begin{enumerate}
\item[] Consider $G = \mathbb{Z}_8$ and $H = \mathbb{Z}_{12}$. Clearly 
$G \not\cong H$ since $|G| \neq |H|$, yet Aut$(G) \cong U(8)$ and Aut$(H) \cong U(12)$. Now define
$\phi: U(8) \rightarrow U(12)$ with $\phi(1) = 1, \phi(3) = 5, \phi(5) = 7,$ and $\phi(7) = 11$, check
for isomorphism and we have our result.
\end{enumerate}

\item[6.18] Suppose that $\phi: \mathbb{Z}_{50} \rightarrow \mathbb{Z}_{50}$ is a group
isomorphism with $\phi(7) = 13$. Determine $\phi(x)$.
\begin{enumerate}
\item[] Given $\phi : \mathbb{Z}_{50} \to \mathbb{Z}_{50}$ an isomorphism, then $\phi(1^{k}) = \phi(1)^{k}$
from theorem 6.2.
Hence, with $\phi(7) = \phi(1^7) = \phi(1)^{7} = 13$ we find that $\phi(1)$ is congruent to 9mod(50). 
Moreover, we see simply then, that the mapping $\phi(x) = \phi(1)^{x} = 9^{x}$. Of course,
in $\mathbb{Z}_{50}$, $9^{x} = 9x$.
\end{enumerate}

\item[6.20] Prove Property 4 of Theorem 6.3.
\begin{enumerate}
\item[] By theorem 6.2, property 1 and since $e \in K$, then $\bar{e} \in \phi(K)$, so
$\phi(K)$ is nonempty. Now since $K$ is a subgroup of $G$, then we know that $ab^{-1} \in K$
whenever $a, b \in K$ and since $\phi$ is an isomorphism then $\phi(ab^{-1}) \in \phi(K)$. 
So let $\bar{a}, \bar{b} \in \phi(K)$ such that $\phi(a) = \bar{a}$ and
$\phi(b) = \bar{b}$. Then 
\[ 
\phi(ab^{-1}) = \phi(a)\phi(b^{-1}) = \phi(a)[\phi(b)]^{-1} = \bar{a}\bar{b}^{-1}
\]
where the second equality comes from theorem 6.2, property 2. 
Hence $\bar{a}\bar{b}^{-1} \in \phi(K)$ and we are done by the one-step subgroup test.
\end{enumerate}

\item[6.22] Prove or disprove that $U(20)$ and $U(24)$ are isomorphic.
\begin{enumerate}
\item[] Consider 3, an element from $U(20)$ which has order 4. Since, every nonidentity
element in $U(24)$ has order 2, $U(20)$ and $U(24)$ cannot possibly be isomorphic to each
other as they would violate theorem 6.2, property 5.
\end{enumerate}

\item[6.24] Let 
\[
G = \{ a + b\sqrt{2} \mid a, b \mbox{ rational } \}
\]
and
\[
H = \{ \left[ \begin{array}{cc} a & 2b \\ b & a \end{array} \right] \mid a, b \mbox{ rational } \}.
\]
Show that $G$ and $H$ are isomorphic under addition. Prove that $G$ and $H$ are closed under
multiplication. Does your isomorphism preserve multiplication as well as addition?
\begin{enumerate}
\item[] Let $\phi : G \to H$ be defined as 
$\phi(a+\sqrt{2}b) = 
\left[ \begin{array}{ccc}
a & 2b \\
b & a \end{array} \right] $.
\newline
To show that $\phi$ is bijective, let $\phi(x) = \phi(y)$. Then there are rational numbers 
$a, b, c, d$ for which $x = a + \sqrt{2}b$ and $y = c + \sqrt{2}d$. Now, from the mapping
we find that 
$\left[ \begin{array}{ccc}
a & 2b \\
b & a \end{array} \right]
=
\left[ \begin{array}{ccc}
c & 2d \\
d & c \end{array} \right] $ from which it is clear that $a = c$ and $b = d$, thus $x = y$. 
Further, take $h \in H$ then $h = 
\left[ \begin{array}{ccc}
\alpha & 2\beta \\
\beta & \alpha \end{array}\right]$
for some $\alpha, \beta \in \mathbb{Q}$. Then we can find $g \in G$ with 
$g = \alpha + \sqrt{2}\beta$ where $\phi(g) = h$. 
\newline
Finally, to show that $\phi$ is homomorphic, take $i, j \in G$. Then there are
rational numbers $k, l, m, n$ with $i = k + \sqrt{2}l$ and $j = m + \sqrt{2}n$. So, 
$\phi(i+j) = \phi(k+m + \sqrt{2}(l+n)) = 
\left[\begin{array}{ccc}
k+m & 2(l+n)\\
l+n & k+m \end{array} \right]
=
\left[ \begin{array}{ccc}
k & 2l\\
l & k \end{array} \right] 
+
\left[ \begin{array}{ccc}
m & 2n\\
n & m \end{array} \right] 
=
\phi(i) + \phi(j)$.
\newline
To find if $G$ and $H$ to be closed under multiplication, take rational
numbers $a,b,c,d$ and look at $(a + \sqrt{2}b)(c + \sqrt{2}d) = ac + 2bd + \sqrt{2}(bc + ad) 
\in G$, and
$\left[ \begin{array}{ccc}
a & 2b \\
b & a \end{array} \right]
\left[ \begin{array}{ccc}
c & 2d \\
d & c \end{array} \right]
=
\left[ \begin{array}{ccc}
ac+2bd & 2(bc+ad)\\
bc+ad & ac+2bd \end{array}\right]
\in H$. 
Thus, $G$ and $H$ are closed under multiplication.
It is easy to see then that $\phi$ is also an isomorphism under multiplication
as the bijective property holds from before, and the homomorphism follows trivially
from the proof that $G$ and $H$ are closed under multiplication.
\end{enumerate}

\item[6.25] Prove that $\mathbb{Z}$ under addition is not isomorphic to $\mathbb{Q}$ under
addition.
\begin{enumerate}
\item[] To prove that $\mathbb{Z}$ under addition is not isomorphic to $\mathbb{Q}$ under 
addition, let us suppose for the sake of contradiction that they are isomorphic and $\phi : 
\mathbb{Z} \rightarrow \mathbb{Q}$. Now, we know by theorem 6.2, property 4, that the mapping 
of a generator of $\mathbb{Z}$ under addition, let's say 1, will generate $\mathbb{Q}$ under addition. 
Hence $\phi(1)$ will generate $\mathbb{Q}$ under addition. Furthemore, $\phi(1)$ must be a rational 
number and can be expressed as $p/q$. Hence $\langle p/q\rangle = \mathbb{Q}$ under addition, yet we
know that any element in $\langle p/q\rangle$ is of the form $k(p/q)$ for some $k \in \mathbb{Z}$.
Thus there is no $k$ such that $k(p/q) = p/2q$, yet $p/2q \in \mathbb{Q}$ under addition. Contradiction.
\end{enumerate}

\item[6.26] Prove that the quaternion group is not isomorphic to the dihedral group $D_4$.
\begin{enumerate}
\item[] We know there are four elements in $D_4$ with order 2 and since the quaternions 
do not have four elements of order 2, theorem 6.2, property 5 cannot be satisfied. Thus
the two groups are not isomorphic.
\end{enumerate}

\item[6.32] Show that the mapping $a \rightarrow \log_{10} a$ is an isomorphism from 
$\mathbb{R}^+$ under multiplication to $\mathbb{R}$ under addition.
\begin{enumerate}
\item[] First, we must note an algebraic property of logs, namely that $\log_{10}(xy) = \log_{10}(x) + \log_{10}(y)$ and
the fact that $\mbox{exp}_{10}(\log_{10}(x)) = x$, where exp is defined as the exponential function from 
which logarithms (which are the inverse of the exponential functions) are derived from. Now
we let $\phi : \mathbb{R}^+ \rightarrow \mathbb{R}$ such that $\phi(a) = \log_{10}(a)$.
Showing that $\phi$ is one-to-one, suppose that $\phi (a) = \phi (b)$, then $\log_{10}(a) = \log_{10}(b)$.
Using exp on both sides yields $a = b$. To show $\phi$ is onto, we take $a \in \mathbb{R}$ and since
$10^a \in \mathbb{R}^+$, then it happens that $\phi (10^a) = \log_{10}(10^a) = a$. Operation preservation
can be shown by  \[ \phi (ab) = \log_{10}(ab) = \log_{10}(a) + \log_{10}(b) = \phi (a) + \phi (b) \] 
for any $a, b \in \mathbb{R}^+$, where the second equality comes from the aforementioned property of logs.
Thus $\phi$ is an isomorphism.
\end{enumerate}

\item[6.33] Suppose that $g$ and $h$ induce the same inner automorphism of a group $G$.
Prove that $h^{-1}g \in Z(G)$.
\begin{enumerate}
\item[] Suppose that $g$ and $h$ induce the same inner automorphism of $G$, then $\phi_g = \phi_h$.
By definition, for all $x \in G$, $gxg^{-1} = hxh^{-1}$. Now multiply on the right by $g$ and
multiply on the left by $h^{-1}$, so that we obtain $h^{-1}gx = xh^{-1}g$. Since $x$ is any
element in $G$, it must follow that $h^{-1}g \in Z(G)$.
\end{enumerate}

\item[6.36] Let $G = \{ 0, \pm 2, \pm 4, \pm 6, \ldots \}$ and $H = \{ 0, \pm 3, \pm 6, \pm 9, \ldots \}$. Show that $G$ and $H$ are isomorphic groups under addition. Does your isomorphism preserve multiplication?
\begin{enumerate}
\item[] First, it is important to note that $\langle 2\rangle = G$ and $\langle 3\rangle = H$. Now, define $\phi : G \to H$
by $\phi(x) = \frac{3}{2}x$. Clearly, with $\phi$ strictly monotone, it is 1-1. Further, if
we can show that $\phi$ is a homomorphism, then with $2 \to 3$ we have that $\phi(\langle 2\rangle) = \langle \phi(2) \rangle = \langle 3\rangle = H$, and thus onto. To show that $\phi$ is a homomorphism, take $a,b \in G$.Then $a = 2k$ and $b = 2l$ for some $k,l \in \mathbb{Z}$. 
Hence, $\phi(a+b) = \phi(2k+2l) = \phi(2(k+l)) = \frac{3}{2}(2(k+l)) = 3(k+l) = 3k + 3l = \phi(2k)+\phi(2l)=\phi(a)+\phi(b)$. And we're done. Under multiplication, $G$ and $H$ aren't even groups, being that there are no inverses.
\end{enumerate}

\item[6.37] Prove that $\mathbb{Q}$ under addition is not isomorphic to $\mathbb{R}^+$ under
multiplication.
\begin{enumerate}
\item[] Suppose that $\phi: \mathbb{Q} \rightarrow \mathbb{R}^+$ is any isomorphism. Then from
theorem 6.2, notice if $\phi(1) = z$, we have $\phi(1) = \phi((1/n)^n) = \phi(1/n)^n = z 
\implies \phi(1/n) = z^{1/n}$ for any $n \in \mathbb{Z}$. Furthermore, if $m, p \in \mathbb{Z}$
then $\phi(m/p) = \phi(1/p)^m = (z^{1/p})^m = z^{m/p}$. Hence for all $q \in \mathbb{Q}, 
\phi(q) = z^q$ so with $i \in \mathbb{R}\setminus \mathbb{Q}$ and $z^i \in \mathbb{R}^+$, we
see the mapping is not onto, contradicting that it is a isomorphism. Thus no isomorphism exists.
\end{enumerate}

\item[6.40] Show that every automorphism $\phi$ of the rational numbers $\mathbb{Q}$ under addition
to itself has the form $\phi(x) = x\phi(1)$.
\begin{enumerate}
\item[] Consider $\mathbb{Q}$ under addition and let $\phi: \mathbb{Q} \rightarrow \mathbb{Q}$
be an automorphism. Then $\exists a \in \mathbb{Q}$ where $\phi(1) = a$, which is
$\phi(n/n) = a$ for some $n \in \mathbb{Z}$. Since $\phi$ is an automorphism $\phi(n/n) = 
\phi(1/n)^n = a$ which implies that $\phi(1/n) = a^{1/n} = (1/n)a = (1/n)\phi(1)$. Now,
let $m/n \in \mathbb{Q}$, then
\[
\phi(m/n) = \phi((1/n)^m) = \phi(1/n)^m = [(1/n)\phi(1)]^m = (m/n)\phi(1)
\]
\end{enumerate}

\item[6.42] Prove that $\mathbb{Q}$, the group of rational numbers under addition, is not isomorphic to a proper subgroup of itself.
\begin{enumerate}
\item[] Suppose that $H$ is a proper subgroup of $\mathbb{Q}$. Then we know that
$\exists z \in \mathbb{Q}\setminus H$. Also let $\phi: \mathbb{Q} \rightarrow H$ be any
isomorphism, then there does not exist any $x \in \mathbb{Q}$ such that $\phi(x) = z$.
Now let $\phi(1) = a \in H$ and $x = z/a \in \mathbb{Q}$. So $\phi(x) = x\phi(1)$ by 
problem 6.40. Finally $x\phi(1) = (z/a)\phi(1) = (z/a)a = z$ is a contradiction. Thus
no isomorphisms exist between $\mathbb{Q}$ and any of it's proper subgroups.
\end{enumerate}

\item[7.2] Let $H$ be as in Exercise 1. How many left cosets of $H$ in $S_4$ are there?
(Determine this without listing them.)
\begin{enumerate}
\item[] $|S_4| = 24$ and $|H| = 4$, then by Lagrange's theorem, there are 24/4 = 6 left cosets of
$H$ in $S_4$.
\end{enumerate}

\item[7.8] Suppose that $a$ has order 15. Find all of the left cosets of $\langle a^5\rangle$
in $\langle a \rangle$.
\begin{enumerate}
\item[] $\langle a\rangle = \{ e, a, a^2, \ldots, a^{14} \}$ and $\langle a^5\rangle = 
\{ e, a^5, a^{10} \}$. So the left cosets of $\langle a^5\rangle$ in $\langle a\rangle$ are \\
$e\langle a^5\rangle =   \{e, a^5, a^{10} \} = \langle a^5\rangle = a^5\langle a^5\rangle = a^{10}\langle a^5\rangle$ \\
$a\langle a^5\rangle =   \{a, a^6, a^{11} \} = a^6\langle a^5\rangle = a^{11}\langle a^5\rangle$ \\
$a^2\langle a^5\rangle = \{ a^2, a^7, a^{12} \} = a^7\langle a^5\rangle = a^{12}\langle a^5\rangle$\\
$a^3\langle a^5\rangle = \{ a^3, a^8, a^{13} \} = a^8\langle a^5\rangle = a^{13}\langle a^5\rangle$\\
$a^4\langle a^5\rangle = \{ a^4, a^9, a^{14} \} = a^9 \langle a^5\rangle = a^{14}\langle a^5\rangle$
\end{enumerate}

\item[7.14] Suppose that $K$ is a proper subgroup of $H$ and $H$ is a proper subgroup of $G$.
If $|K| = 42$ and $|G| = 420$, what are the possible orders of $H$?
\begin{enumerate}
\item[] Firstly, since $K$ is a proper subgroup of $H$ and $H$ is a proper subgroup of $G$, then 
$|K| < |H| < |G|$. Now by Lagrange's theorem, $|K| = 42\, |\,\, |H|$, which means that $|H| = 42k$, for some
$k \in \mathbb{N}$, and again by Lagrange's theorem $42k\, |\,\,420 = |G|$. This implies that $1 < k < 10$, to 
which we can conclude that $k$ must be $2$ or $5$. Hence the two possible orders of $H$ are $42 \cdot 2 = 84$
or $42 \cdot 5 = 210$.
\end{enumerate}

\item[7.16] Recall that, for any integer $n$ greater than 1, $\phi(n)$ denotes the number of
integers less than $n$ and relatively prime to $n$. Prove that if $a$ is any integer relatively
prime to $n$, then $a^{\phi(n)}$ mod $n = 1$.
\begin{enumerate}
\item[] Firstly, $a$ is relatively prime to $n$, so $a \in U(n)$. Furthermore, 
$|U(n)| = \phi(n)$ and by corollary 4, $a^{|U(n)|}\mbox{ mod } n = a^{\phi(n)}
\mbox{ mod } n = e$. Since $e = 1$ in $U(n)$, then $a^{\phi(n)} \mbox{ mod } n = 1$ as required.
\end{enumerate}

\item[7.17] Compute $5^{15}$ mod $7$ and $7^{13}$ mod $11$.
\begin{enumerate}
\item[]
\begin{eqnarray*}
5^{15} \mbox{ mod } 7 &=& (5^7 \mbox{ mod } 7) \cdot (5^7 \mbox{ mod } 7) \cdot (5 \mbox{ mod } 7) \\
&=& (5 \mbox{ mod } 7) \cdot (5 \mbox{ mod } 7) \cdot (5 \mbox{ mod } 7) \\
&=& (5^3 \mbox{ mod } 7) \\
&=& 126 \mbox{ mod } 7 \\
&=& 6
\end{eqnarray*}
where the second equality (above and below) is from Fermat's Little Theorem.
\begin{eqnarray*}
7^{13} \mbox{ mod } 11 &=& (7^{11} \mbox{ mod } 11) \cdot (7^2 \mbox{ mod } 11) \\
&=& (7 \mbox{ mod } 11) \cdot (49 \mbox{ mod } 11) \\
&=& (7 \cdot 5 \mbox{ mod } 11) \\
&=& 35 \mbox{ mod } 11 \\
&=& 2
\end{eqnarray*}
\end{enumerate}

\item[7.22] Suppose that $G$ is a group with more than one element and $G$ has no
proper, nontrivial subgroups. Prove that $|G|$ is prime. (Do not assume at the outset that
$G$ is finite.)
\begin{enumerate}
\item[] Suppose that $G$ is a group with more than one element and $G$ has no proper, nontrivial subgroups.
Now, let $g$ be any nonidentity element in $G$. Then $\langle g\rangle$ is a subgroup of 
$G$. But since $G$ has no proper subgroups, it must be that $\langle g\rangle = G$, which means
that $G$ is cyclic. Hence, it must follow that $G$ can only be finite, 
otherwise $\langle g^2\rangle$ is a proper subgroup and a contradiction. So by theorem 4.3, 
we know if $k \mid n = |G|$, then there is a subgroup of order $k$, yet this means that $k$ cannot 
be anything other than 1 or $n$ (otherwise we would have a proper subgroup), 
so it must be that the order of $G$ is prime.
\end{enumerate}

\item[7.26] Let $|G| = 8$. Show that $G$ must have an element of order 2.
\begin{enumerate}
\item[] Let $g$ be any nonidentity element in $G$. Then by Corollary 2, $|g|$ divides $|G|$.
Since $|G| = 8$, $|g|$ must be 2, 4, or 8. The case where $|g| = 2$ is clear and we're done.
Otherwise, $|g| = 4$ or $|g| = 8$, then $|g^2| = 2$ or $|g^4| = 2$, respectively and again
we're done.
\end{enumerate}

\item[7.28] Show that $\mathbb{Q}$, the group of rational numbers under addition, has no
proper subgroup of finite index.
\begin{enumerate}
\item[] For the sake of contradiction, let us suppose that $H$ is a proper subgroup of 
$\mathbb{Q}$ under addition with a finite index. Then there exist an 
element, $z \in \mathbb{Q}\backslash H$, which also means that $\frac{z}{2} \not\in H$. Notice now
for every $n \in \mathbb{N}$, either $(\frac{z}{2} + \frac{z}{n}) \not\in H$ or $(\frac{z}{2} - 
\frac{z}{n}) \not\in H$, otherwise $[(\frac{z}{2} + \frac{z}{n}) + (\frac{z}{2} - \frac{z}{n})] \in H$
and that implies that $z \in H$. Thus there exists a $k \in \mathbb{N}$ with
$(\frac{z}{2} \pm \frac{z}{n}) \not\in H$ such that $\frac{z}{2} + H = \frac{z}{2} \pm \frac{z}{k} + H$, which
we know must be true since $H$ has a finite index. 
Now for the positive case, 
\[ 
\left(\frac{z}{2} + \frac{z}{k}\right) - \frac{z}{2} = \frac{z}{k} \in H 
\Rightarrow \left\langle \frac{z}{k} \right\rangle \subseteq H \Rightarrow z \in H
\]
or for the negative case,
\[ 
\frac{z}{2} - \left(\frac{z}{2} - \frac{z}{k}\right) = \frac{z}{k} \in H 
\Rightarrow \left\langle \frac{z}{k} \right\rangle \subseteq H \Rightarrow z \in H
\]
Contradiction since there does not exist a $k \in \mathbb{N}$ such that \\
$\frac{z}{2} + H = \frac{z}{2} \pm \frac{z}{k} + H$. Thus there is no proper subgroups of 
$\mathbb{Q}$ under addition with finite index.
\end{enumerate}

\item[7.33] Let $G$ be a group of order $p^n$ where $p$ is prime. Prove that the center
of $G$ cannot have order $p^{n-1}$.
\begin{enumerate}
\item[] For the sake of contradiction, let $|Z(G)| = p^{n-1}$. Then, there are $p$ elements
in $G$ that are not in $Z(G)$. So, take one of those elements, let's say $g \in G\setminus Z(G)$,
and look at the centralizer of $g$. Since $g$ always commutes with itself, it follows that 
$g \in C(g)$. Furthermore, $g$ commutes with all the elements from the center, so
$Z(G) \subseteq C(g)$. Now, we know by theorem 3.6 that $C(g) \leq G$ and from Lagrange's 
theorem that $|C(g)|$ must divide $G$. But from what we've said, $|C(g)|$ must be strictly larger
than $|Z(G)|$, thus it can only be that $|C(g)|$ is $p^n$. This means that $G = C(g)$, which
implies that $g$ commutes with every element of $G$ and must be in the center. Contradiction since
$g$ was chosen such that $g \not\in Z(G)$.
\end{enumerate}

\item[7.47] Determine all finite subgroups of $\mathbb{C}^*$, the group of nonzero complex
numbers under multiplication.
\begin{enumerate}
\item[] The $n^{\mbox{th}}$ roots of unity are those complex numbers, $x$, such that $x^n = 1$. In fact,
by the fundamental theorem of algebra, there are $n$ such roots for any $n \in \mathbb{N}$.
Furthermore, these roots form a cyclic subgroup in $\mathbb{C}^*$ under multiplication and 
we can find them by $e^{(2\pi ik)/n}$ for any $n \in \mathbb{N}$, $k = 0, 1, \ldots, n - 1$. Thus,
$\{ e^{(2\pi ik)/n} \mid k = 0, 1, \ldots, n - 1 \}$ is a cyclic subgroup of order $n$ in 
$\mathbb{C}^*$ under multiplication for any $n \in \mathbb{N}$ and, moreover, 
this finds all the finite subgroups of $\mathbb{C}^*$. 
\end{enumerate}

\item[8.2] Show that $\mathbb{Z}_2 \oplus \mathbb{Z}_2 \oplus \mathbb{Z}_2$ has
seven subgroups of order 2.
\begin{enumerate}
\item[] $\mathbb{Z}_2 \oplus \mathbb{Z}_2 \oplus \mathbb{Z}_2 = \{ (0, 0, 0), (0, 1, 0),
(0, 0, 1), (0, 1, 1), (1, 0, 0), (1, 0, 1),$\\$ (1, 1, 0), (1, 1, 1) \}$. So, 
(0, 0, 0) is the identity and with any nonidentity element, there being 7 in total, 
forms a subgroup. Note that this is possible since any nonidentity element has an order of
2 and thus must be its own inverse.
\end{enumerate}

\item[8.10] How many elements of order 9 does $\mathbb{Z}_3 \oplus \mathbb{Z}_9$ have?
\begin{enumerate}
\item[] By theorem 8.1, an element in $\mathbb{Z}_3 \oplus \mathbb{Z}_9$ will have an order
of 9, when lcm$(|g|, |h|) = 9$, where $g \in \mathbb{Z}_3$ and $h \in \mathbb{Z}_9$. This
is only possible in two ways, that is, $|g| = 9$ and $|h| = 1$ or $|g| = 1$ and $|h| = 9$.
Since there is no element in $\mathbb{Z}_3$ that has order 9, we can discard the former
case. Now, we know that any element in $\mathbb{Z}_9$ that is going to have order 9 will
be relatively prime to 9, thus the number of elements in $\mathbb{Z}_9$ of order 9 is 
just $\phi(9)$. Since the only element in $\mathbb{Z}$ with order 1 is the identity, 
we are going to have exactly $\phi(9) = 6$ elements of order 9 in $\mathbb{Z}_3 \oplus \mathbb{Z}_9$.
\end{enumerate}

\item[8.12] The dihedral group $D_n$ of order $2n$ $(n \geq 3)$ has a subgroup of $n$ rotations
and a subgroup of order 2. Explain why $D_n$ cannot be isomorphic to the external direct
product of two such groups.
\begin{enumerate}
\item[] If $n$ is odd and $D_n \cong G_n \oplus G_2$, then there exists an $r \in D_n$ such that
$|r| = n$. From theorem 8.1, it follows that with $n$ odd, there exists $t \in G_n$ with 
$|t| = n$ and the element $|(t, e_2)| = |t| = n$. But then $h \in G_2, h \neq e_2$ and
$|(t, h)| = $ lcm$(|t|, |h|) = 2|t| = 2n$. With no element of order $2n$ in $D_n$ we arrive
at a contradiction. \\
If $n$ is even, then notice that there exists an odd number of generators of the rotation 
subgroup. But with $n > 3$, we know $\exists g \in G_n$ with $|g| = n$ so that $|(g, h)| = n$.
But $|(g, h)| = |(g, e_2)|$, so they always come in pairs. Thus, if $D_n \cong G_n \oplus G_2$, 
then there exists an even number of rotations with order $n$, which contradicts that there
are an odd number of generators, so $D_n \not\cong G_n \oplus G_2$.
\end{enumerate}

\item[8.15] If $G \oplus H$ is cyclic, prove that $G$ and $H$ are cyclic.
\begin{enumerate}
\item[] \emph{Proof 1:} Suppose that $G \oplus H$ is cyclic. Then, there is an element, 
$(g, h) \in G\, \oplus H$, where $g \in G$ and $h \in H$, such that $\langle (g, h)\rangle = G 
\oplus H$ and $|\langle(g, h)\rangle| = |G \oplus H| = |G||H| = |(g, h)|$ by theorem 4.1, corollary 1
and since we know that $|G\, \oplus H| = |G||H|$. But by theorem 8.1, $|(g, h)| = $ lcm$(|g|, |h|) = 
|G||H|$, so it must follow that $|g| = |G|$ and $|h| = |H|$ by theorem 8.2. Thus $G$ and $H$ must be cyclic.
\[ \] 
\emph{Proof 2:} Suppose that $G \oplus H$ is cyclic. Now for the sake of contradiction, assume that 
$G$ is not cyclic or $H$ is not cyclic. Since, $G \oplus H$ is cyclic, there exists an element, 
$(g, h) \in G\, \oplus H$, where $g \in G$ and $h \in H$, such that $\langle (g, h)\rangle = G 
\oplus H$. Then $|\langle(g, h)\rangle| = |G \oplus H| = |G||H| = |(g, h)|$ by theorem 4.1, corollary 1
and since we know that $|G\, \oplus H| = |G||H|$. Now, by theorem 8.1, $|(g, h)| = $ lcm$(|g|, |h|) = 
|G||H|$. But this impossible since either $|g| \neq |G|$ or $|h| \neq |H|$. Contradiction.
\end{enumerate}

\item[8.30] Let
\[
H = 
\left\{
\left[
\begin{array}{ccc}
1 & a & b \\
0 & 1 & 0 \\
0 & 0 & 1
\end{array}
\right]
\mid a, b \in \mathbb{Z}_3 
\right\}.
\]
(See Exercise 34 in Chapter 2 for the definition of multiplication.) Show that $H$ is an 
Abelian group of order 9. Is $H$ isomorphic to $\mathbb{Z}_9$ or to $\mathbb{Z}_3 \oplus 
\mathbb{Z}_3$?
\begin{enumerate}
\item[] If we take a look at exercise 2.34, we see that $c = 0$ in this case, so that the defined
multiplication is really just standard matrix multiplication. With this in mind, we show that
$H$ is Abelian, by taking any $a', a, b, b' \in \mathbb{Z}_3$ and concluding that
\[
\left[
\begin{array}{ccc}
1 & a & b \\
0 & 1 & 0 \\
0 & 0 & 1
\end{array}
\right]
\left[
\begin{array}{ccc}
1 & a' & b' \\
0 & 1 & 0 \\
0 & 0 & 1
\end{array}
\right] =
\left[
\begin{array}{ccc}
1 & (a' + a) & (b' + b) \\
0 & 1 & 0 \\
0 & 0 & 1
\end{array}
\right] =
\]
\[
\left[
\begin{array}{ccc}
1 & (a + a') & (b + b') \\
0 & 1 & 0 \\
0 & 0 & 1
\end{array}
\right] =
\left[
\begin{array}{ccc}
1 & a' & b' \\
0 & 1 & 0 \\
0 & 0 & 1
\end{array}
\right]
\left[
\begin{array}{ccc}
1 & a & b \\
0 & 1 & 0 \\
0 & 0 & 1
\end{array}
\right]
\] 
since $\mathbb{Z}_3$ is Abelian. Furthermore, because we have 3 possiblities for $a$
and 3 possibilities for $b$, which is $3 \cdot 3 = 9$ possibilities in total. Hence,
$|H| = 9$. \\
Now, let us note that $\mathbb{Z}_9 \not\cong \mathbb{Z}_3 \oplus \mathbb{Z}_3$ since 
$\mathbb{Z}_3 \oplus \mathbb{Z}_3$ does not contain an element of order 9. So, for this same
reason, $H \not\cong \mathbb{Z}_9$. Thus $H$ must be isomorphic to $\mathbb{Z}_3 \oplus 
\mathbb{Z}_3$. We can further prove this by letting $\phi : H \rightarrow \mathbb{Z}_3 \oplus 
\mathbb{Z}_3$ such that $\phi \left( \left[
\begin{array}{ccc}
1 & a & b \\
0 & 1 & 0 \\
0 & 0 & 1
\end{array}
\right] \right) = (a, b)$. It is not hard to see that $\phi$ is bijective, so all that is left is 
operation preservation, which easily follows from the argument above where we showed that 
$H$ is Abelian.
\end{enumerate}

\item[8.38] Determine Aut$(\mathbb{Z}_2 \oplus \mathbb{Z}_2)$.
\begin{enumerate}
\item[] There are 6 of them, just enumerate the mappings.
\end{enumerate}

\item[extra]
\begin{enumerate}
\item[] Because rotations commute we know that given any 2 rotations, $R_1$ and $R_2$, then
$R_1R_2R_1^{-1} = R_2$. Also with any reflection, $F$, and rotation, $R$, then 
$FRF = FRF^{-1} = R^{-1}$, so that we find $[R] = \{ R, R^{-1} \}$. Notice that if $n$ is even, 
$R_\pi \in D_n$, so $[R_\pi ] = \{ R_\pi , R^{-1}_\pi \} = \{ R_\pi \}$. Now for the reflections. We need 
only consider $f$ and $rf$. If we take any rotation, $r^p$ for $0 \leq p \leq n - 1$, we find 
that 
\[ r^pfr^{-p} = r^pr^pf = r^{2p}f \]
and 
\[
r^p(rf)r^{-p} = r^{p+1}fr^{-p} = r^{p+1}r^pf = r^{2p + 1}f
\]
Also given any reflection, $r^mf$ for $0 \leq l \leq n - 1$, we find that
\[
r^mff(r^mf)^{-1} = r^mfff^{-1}r^{-m} = r^mfr^{-m} = r^{2m}f
\]
and
\[
r^mf(rf)(r^mf)^{-1} = r^mfrff^{-1}r^{-m} = r^mfr^{1 - m} = r^{2m - 1}f
\]
Thus, $[f] = \{ r^xf |\,\,\, x $ is even $\}$ and $[rf] = \{ r^yf |\,\,\, y$ is odd $\}$. \\
So, if $n$ is even, we find that $[f] = \{ r^xf |\,\,\, x \in \{0, 2, 4, \ldots, n-2\}\}$ and
$[rf] = \{ r^yf |\,\,\, y \in \{1, 3, 5, \ldots, n - 1\}\}$. If $n$ is odd, then 
$[f] = \{ r^xf |\,\,\, x \in \{0, 1, \ldots, n - 1\}\}$. \\
With any reflection $F \in [f]$ for odd $n$ or $F \in [f] \cup [rf]$ and knowing that \\
$F \in [f] \Rightarrow [F] = [f]$ and $[f] \in [f] \cup [rf] \Rightarrow [F] = [f]$ or 
$[F] = [rf]$, then we have described the conjugacy classes of $D_n$.
\end{enumerate}

\item[9.7] Prove that if $H$ has index 2 in $G$, then $H$ is normal in $G$.
\begin{enumerate}
\item[] Suppose $|G:H| = 2$. Now, we can divide all the elements of $G$ into two sets 
covered by the following two cases; \\
\emph{Case 1:} $g \in G$ and $g \in H$. Then $gH = H = Hg$ by property 2 of the lemma on page 138. \\
\emph{Case 2:} $g \in G$ and $g \not\in H$. Then $gH$ are the elements in $G$ not covered in the first
case, but then again, so is $Hg$. Hence $gH = Hg$. \\
Since, we have $gH = Hg$ for any $g \in G$, it follows that $H$ is normal in $G$.
\end{enumerate}

\item[9.8] Let $H = \{ (1), (12)(34) \}$ in $A_4$.
\begin{enumerate}
\item[a)] Show that $H$ is not normal in $A_4$. \\
Let us consider the fact that for any $x \in A_4$ and $h \in H$ there is a $h'$ in 
$H$ such that $xhx^{-1} = h'$. Now consider $x = (123)$ and $h = (12)(34)$. Then $x^{-1} = (132)$, 
so that $(123)(12)(34)(132) = (123)(143)(2) = (14)(23) = h'$ and $h' \not\in H$, thus $H$ cannot
be normal in $A_4$. \\
Now, using Table 5.1, we see that $\alpha_6H = \{ \alpha_6, \alpha_7 \} = \alpha_7H$ and that 
$\alpha_9H = \{ \alpha_9, \alpha_{11} \} = \alpha_{11}$, but $\alpha_6\alpha_9 = \alpha_2$ and
$\alpha_7\alpha_{11} = \alpha_4$, so that $\alpha_2H = \{ \alpha_2, \alpha_1 \} \neq \{
\alpha_4, \alpha_3 \} = \alpha_4H$. The left cosets of $H$ cannot be a group under coset multiplication
because $H$ wasn't a normal subgroup to begin with.
\end{enumerate}

\item[9.22] Determine the order of $(\mathbb{Z} \oplus \mathbb{Z})/\langle (2, 2)\rangle$.
Is the group cyclic?
\begin{enumerate}
\item[] To see that the order of $\mathbb{Z} \oplus \mathbb{Z}/\langle (2, 2)\rangle$ infinite, we just
have to observe that $(0, n) + \langle (2, 2)\rangle, n \in \mathbb{Z}$ is infinite since 
$(0, i) + \langle (2, 2)\rangle \neq (0, j) + \langle (2, 2)\rangle$ when $i, j \in \mathbb{Z}, i \neq j$.
No, it is not cyclic because $(1, 1) + \langle (2, 2)\rangle$ has order 2.
\end{enumerate}

\item[9.24] The group $(\mathbb{Z}_4 \oplus \mathbb{Z}_{12})/\langle (2, 2)\rangle$ is isomorphic
to one of $\mathbb{Z}_8, \mathbb{Z}_4 \oplus \mathbb{Z}_2$, or $\mathbb{Z}_2 \oplus \mathbb{Z}_2
\oplus \mathbb{Z}_2$. Determine which one by elimination.
\begin{enumerate}
\item[] Let's observe the 8 cosets of $(\mathbb{Z}_4 \oplus \mathbb{Z}_{12})$/$\langle(2, 2)\rangle$. \\
$\langle(2, 2)\rangle = \{ (2, 2), (0, 4), (2, 6), (0, 8), (2, 10), (0, 0) \}$\\
$(0, 1) + \langle(2, 2)\rangle = \{ (2, 3), (0, 5), (2, 7), (0, 9), (2, 10), (0, 1) \}$\\
$(0, 2) + \langle(2, 2)\rangle = \{ (2, 4), (0, 6), (0, 8), (2, 10), (2, 0), (0, 2) \}$\\
$(1, 1) + \langle(2, 2)\rangle = \{ (3, 3), (1, 5), (3, 7), (1, 9), (3, 10), (1, 1) \}$\\
$(1, 2) + \langle(2, 2)\rangle = \{ (3, 4), (1, 6), (3, 8), (1, 10), (3, 0), (1, 2) \}$\\
$(2, 1) + \langle(2, 2)\rangle = \{ (0, 3), (2, 5), (0, 7), (2, 9), (0, 11), (2, 1) \}$\\
$(3, 2) + \langle(2, 2)\rangle = \{ (1, 4), (3, 6), (1, 8), (3, 10), (1, 0), (3, 2) \}$\\
$(3, 1) + \langle(2, 2)\rangle = \{ (1, 3), (3, 5), (1, 7), (3, 9), (1, 11), (3, 1) \}$ \\
So, let's take $[(0, 1) + \langle(2, 2)\rangle]^2 = (0, 2) + \langle(2, 2)\rangle$, so that 
$|(0, 1) + \langle(2, 2)\rangle| \geq 4$. Thus $(\mathbb{Z}_4 \oplus \mathbb{Z}_{12})$/$\langle(2, 2)\rangle$
cannot be isomorphic to $\mathbb{Z}_2 \oplus \mathbb{Z}_2 \oplus \mathbb{Z}_2$. Direct computation shows
that $(0, 2) + \langle(2, 2)\rangle$ and $(3, 1) + \langle(2, 2)\rangle$ both have order 2, so that
$(\mathbb{Z}_4 \oplus \mathbb{Z}_{12})$/$\langle(2, 2)\rangle$ cannot be isomorphic to $\mathbb{Z}_8$ 
either, since a cylic group of even order has exactly one element of order 2 (Theorem 4.4). This
proves that $(\mathbb{Z}_4 \oplus \mathbb{Z}_{12})$/$\langle(2, 2)\rangle \cong \mathbb{Z}_4 
\oplus \mathbb{Z}_2$.
\end{enumerate}

\item[9.37] Let $G$ be a finite group and let $H$ be a normal subgroup of $G$. Prove that
the order of the element $gH$ in $G/H$ must divide the order of $g$ in $G$.
\begin{enumerate}
\item[] Suppose that $G$ is a finite group and $H \lhd G$. So, let $|g| = n$. Then
\[
(gH)^n = \overbrace{(gH)(gH)\cdots(gH)}^{\mbox{$n$ times}} = 
\overbrace{ggg\cdots gg}^{\mbox{$n$ times}}H = g^nH = eH = H
\]
where the second equality comes from theorem 9.2. Hence, by theorem 4.1, corollary 2, 
what we have shown implies that the order of $gH$ divides $n$.
\end{enumerate}

\item[9.39] If $H$ is a normal subgroup of a group $G$, prove that $C(H)$, the centralizer
of $H$ in $G$, is a normal subgroup of $G$.
\begin{enumerate}
\item[] Suppose that $H$ is a normal subgroup of a group $G$. Then, we need to prove exercise 3.20, 
so that we know $C(H) \leq G$. We will use the one-step subgroup test and make use of the fact that
for any element, $a \in C(H), aha^{-1} = h$ for all $h \in H$, which is by definition.
Now, we know that $eh = he$ for all $h \in H$, so $e \in C(H)$ and $C(H)$ is nonempty. 
Now, let $a, b \in C(H)$, then this means that $aha^{-1} = h$ for all $h \in H$ and 
$bhb^{-1} = h$ for all $h \in H$, but more importantly multiply on the left by $b^{-1}$ and on the right
by $b$, yields $h = b^{-1}hb$. Hence,
\[
(ab^{-1})h(ab^{-1})^{-1} = (ab^{-1})h(ba^{-1}) = a(b^{-1}hb)a^{-1} = aha^{-1} = h
\]
for all $h \in H$, where we got the first equality comes from the socks-shoes property. Thus
$ab^{-1} \in C(H)$ and $C(H) \leq H \leq G$. So, now we need to show that \\
$gcg^{-1} \in C(H)$ for any $c \in C(H)$ and $g \in G$, to show that $C(H) \lhd G$. 
Furthermore, because $H \lhd G$, then we know that $g^{-1}hg \in H$ for any $g \in G$, but this 
also means that $c(g^{-1}hg) = (g^{-1}hg)c$ or better yet, $c(g^{-1}hg)c^{-1} = g^{-1}hg$ for
any $c \in C(H)$. Hence, we show that
\begin{eqnarray*}
(gcg^{-1})h(gcg^{-1})^{-1} &=& (gcg^{-1})h(gc^{-1}g^{-1}) \\
&=& g(cg^{-1}hgc^{-1})g^{-1} \\
&=& g(g^{-1}hg)g^{-1} \\
&=& h
\end{eqnarray*}
and from this we know that $gcg^{-1} \in C(H)$, thereby making $C(H)$ a normal subgroup of $G$. 
\end{enumerate}

\item[9.45] Let $N$ be a normal subgroup of $G$ and let $H$ be a subgroup of $G$. If $N$ is a
subgroup of $H$, prove that $H/N$ is a normal subgroup of $G/N$ if and only if $H$ is a normal
subgroup of $G$.
\begin{enumerate}
\item[] Let $N \lhd G$ and $N \leq H \leq G$. Now suppose that $H/N \lhd G/N$. This means that
for any $g \in G, h \in H, gNhN(gN)^{-1} = ghg^{-1}N \in H/N$. 
Furthermore, it means that $ghg^{-1}N = h'N$, for some
$h' \in H$ and from this we can conclude since $N \leq H$ that $ghg^{-1} = h'n$ for some $n \in N$.
Thus $H \lhd G$. Conversely, suppose that $H \lhd G$. Then $ghg^{-1}N \in H/N$, so $H/N \lhd G/N$.
\end{enumerate}

\item[9.50] Show that the intersection of two normal subgroups of $G$ is a normal subgroup of
$G$.
\begin{enumerate}
\item[] Suppose that $H \lhd G$ and $K \lhd G$. By exercise 3.14, we know that $H \cap K \leq G$.
Now to show that $H \cap K$ is normal in $G$, let $a \in H \cap K$. Then for all $g \in G$, 
$gag^{-1} \in H$ and $gag^{-1} \in K$ since $H$ and $K$ are both normal in $G$. But this also
means that $gag^{-1} \in H \cap K$ for all $g \in G$. Thus, $H \cap K \lhd G$ by theorem 9.1.
\end{enumerate}

\item[9.51] Let $N$ be a normal subgroup of $G$ and let $H$ be any subgroup of $G$. Prove
that $NH$ is a subgroup of $G$. Give an example to show that $NH$ need not be a subgroup
of $G$ if neither $N$ nor $H$ is normal.
\begin{enumerate}
\item[] We will use the two-step subgroup test. So, suppose that $N \lhd G$ and $H \leq G$. We let
$nh \in NH$, when $n \in N$ and $h \in H$. Since $e \in N$ and $e \in H$, then $ee = e \in NH$ and
$NH$ is nonempty. We need to show that $(nh)^{-1} \in NH$ when $nh \in NH$. Since $N$ is normal in
$G$ then $n = g^{-1}ng$ for all $g \in G$. Thus, $(nh)^{-1} = h^{-1}n^{-1} = h^{-1}(g^{-1}ng)^{-1} =
h^{-1}(g^{-1}n^{-1}g) = (g^{-1}n^{-1}g)h^{-1} \in NH$ due to the fact that $h^{-1} \in H$ because it's a subgroup, 
$g^{-1}n^{-1}g \in N$ because it's a normal subgroup, and $h^{-1} \in G$ so that it commutes with any element
of $N$, namely $n^{-1}$ in this case. Now let $(nh)(n'h') \in NH$, then $(nh)(n'h') = 
(nh)n'eh' = nhn'(h^{-1}h)h' = n(hn'h^{-1})(hh') \in NH$ since $n(hn'h^{-1}) \in N$ because it's 
normal in $G$ and $hh' \in H$ because it's a subgroup. Hence $NH \leq G$.
\end{enumerate}

\item[9.52] If $N$ and $M$ are normal subgroups of $G$, prove that $NM$ is also a normal 
subgroup of $G$.
\begin{enumerate}
\item[] Suppose that $N \lhd G$ and $M \lhd G$. Then by exercise 9.51, $NM \leq G$. So, let
$nm \in NM$, when $n \in N$ and $m \in M$. Now for all $g \in G$, \\
$(gng^{-1})(gmg^{-1}) = gn(gg^{-1})mg^{-1} = gn(e)mg^{-1} = g(nm)g^{-1} \in NM$, when
$gng^{-1} \in N$ and $gmg^{-1} \in M$. But this is clearly true since $N$ and $M$ are 
normal in $G$. Hence $NM \lhd G$ by theorem 9.1.
\end{enumerate}

\item[9.58] Suppose that a group $G$ has a subgroup of order $n$. Prove that the intersection
of all subgroups of $G$ of order $n$ is a normal subgroup of $G$.
\begin{enumerate}
\item[] If $G$ has one subgroup of order $n$, then this subgroup is characteristic and by
9.62, normal. So assume that $G$ has more than one subgroup of order $n$. Then since automorphisms 
(inner, as well) preserve order, it must follow that any of the subgroups of order $n$ will be mapped 
to some subgroup of order $n$. Clearly, the intersection of these subgroups of order $n$ will map 
consistly throughout all the automorphisms to the image of the intersection of these subgroups of 
order $n$. Thus the intersection of all the subgroups of order $n$ is characteristic by definition 
(found on page 174, supplementary problem 1). Hence by 9.62, the intersection of all the subgroups
of order $n$ is normal as well.
\end{enumerate}

\item[9.62] Recall that a subgroup $N$ of a group $G$ is called characteristic if 
$\phi(N) = N$ for all automorphisms $\phi$ of $G$. If $N$ is a characteristic subgroup of $G$,
show that $N$ is a normal subgroup of $G$.
\begin{enumerate}
\item[] Suppose $N$ is a characteristic subgroup of $G$. Then we know that for any automorphism, 
$\phi$, of $G$, $\phi(N) = N$. Yet, this also means that the inner automorphisms also map $N$ to $N$,
which means that $\forall g \in G, gNg^{-1} = N$. Thus $N \lhd G$. 
\end{enumerate}

\item[10.6] Let $G$ be the group of all polynomials with real coefficients under addition. For
each $f$ in $G$, let $\int f$ denote the antiderivative of $f$ that passes through the point $(0, 0)$. Show that the mapping $f \rightarrow \int f$ from $G$ to $G$ is a homomorphism. What is the kernel of this mapping? Is this mapping a homomorphism if $\int f$ denotes the antiderivative of $f$ that passes through $(0, 1)$?
\begin{enumerate}
\item[] Let $H \subset G$ be the set of functions in $G$ with $f \in H \implies f(0) = 0$. Clearly, 
$H$ is a subgroup of $G$ with the zero polynomial (the identity) in $H$,
$f^{-1} = -f \in H$ any time $f \in H$ from the fact that $f(0) = -f(0) = 0$, and with 
$f, g \in H$ then $(f+g)(0) = 0 \implies f+g \in H$.
To show that the mapping is homomorphic from $G \to H$, let $f, k \in G$ with $f(x) =
f_nx^n + \cdots + f_0$ and $k(x) = k_mx^m + \cdots + k_0$ with $n \geq m$. Now
\[
\int (f + k) = \sum^n_{i = 1} \frac{f_i + k_i}{i + 1}x^{i+1} = 
\sum^n_{i = 1} \frac{f_i}{i + 1}x^{i+1}  + \sum^n_{i = 1} \frac{k_i}{i + 1}x^{i+1}
= \int f + \int k
\]
where $k_i = 0$ for $m < i \leq n$. Thus the mapping is homomorphic. Further, any
non-trivial element in $G$ is mapped to a non-trivial element in $H$, thus the kernel 
of the homomorphism is soley the zero polynomial. Finally, to see that if the mapping
sent functions to (0,1) that it would not be a homomorphism, we can simply note that the 
identity element in $G$ would be mapped to a non-identity element in $H$, namely
$\int \mathbf{0} \to 1$. 
\end{enumerate}

\item[10.7] If $\phi$ is a homomorphism from $G$ to $H$ and $\sigma$ is a homomorphism from 
$H$ to $K$, show that $\sigma \phi$ is a homomorphism from $G$ to $K$.
\begin{enumerate}
\item[] Suppose that $\phi$ is a homomorphism from $G$ to $H$ and $\sigma$ is a homomorphism from
$H$ to $K$. Now let $a, b \in G$, then 
\[
(\sigma \phi )(ab) = \sigma [\phi (ab)] = \sigma [\phi (a) \phi (b)] =
\sigma [\phi (a)] \sigma [\phi (b)] = (\sigma \phi )(a) (\sigma \phi )(b)
\]
where the second equality comes from the fact that $\phi$ is a homomorphism and 
the third equality is because $\sigma$ is a homomorphism. Thus $\sigma \phi$ is a homomorphism
from $G$ to $K$.
\end{enumerate}

\item[10.8] Let $G$ be a group of permutations. For each $\sigma$ in $G$, define
\begin{enumerate}
\item[] We know from chapter 5 that any permutation is either even or odd, but not 
both. Thus it follows that sgn is a well-defined function when $\sigma \in S_n$, 
since sgn$(\sigma ) = +1$ when $\sigma$ is even and sgn$(\sigma ) = -1$ when $\sigma$ is odd.
We can now show that sgn$(\sigma \tau) = $ sgn$(\sigma)$sgn$(\tau)$
for any $\sigma, \tau \in G$. \\
\emph{Case: i)} $\sigma$ is even and $\tau$ is even, then $\sigma \tau$ is even, so \\
sgn$(\sigma \tau) = +1 = (+1)(+1) = $ sgn$(\sigma)$sgn$(\tau)$. \\
\emph{Case: ii)} $\sigma$ is even and $\tau$ is odd, then $\sigma \tau$ is odd, so \\
sgn$(\sigma \tau) = -1 = (+1)(-1) = $ sgn$(\sigma)$sgn$(\tau)$. \\
\emph{Case: iii)} $\sigma$ is odd and $\tau$ is even, then $\sigma \tau$ is odd, so \\
sgn$(\sigma \tau) = -1 = (-1)(+1) = $ sgn$(\sigma)$sgn$(\tau)$. \\
\emph{Case: iv)} $\sigma$ is odd and $\tau$ is odd, then $\sigma \tau$ is even, so \\
sgn$(\sigma \tau) = +1 = (-1)(-1) = $ sgn$(\sigma)$sgn$(\tau)$. \\
Having showed that sgn is a homomorphism, we can clearly see that the kernel of sgn
is the set of all even permutations since 1 is the identity in the multiplicative 
group $\{-1, 1 \}$.
\end{enumerate}

\item[10.14] Explain why the correspondence $x \rightarrow 3x$ from $\mathbb{Z}_{12}$
to $\mathbb{Z}_{10}$ is not a homomorphism.
\begin{enumerate}
\item[] If the correspondence $x \rightarrow 3x$ from $Z_{12}$ to $Z_{10}$ was a homomorphism
then $\phi(5 + 7) = \phi(5) + \phi(7)$...yet \\
$\phi(5 + 7) = \phi(12 \mbox{ mod }12) = 0 \neq 6 = 15 \mbox{ mod }10 + 21 \mbox{ mod }10 = \phi(5) + \phi(7)$
. So it cannot be a homomorphism.
\end{enumerate}

\item[10.16] Prove that there is no homomorphism from $\mathbb{Z}_8 \oplus \mathbb{Z}_2$ onto
$\mathbb{Z}_4 \oplus \mathbb{Z}_4$.
\begin{enumerate}
\item[] First, we observe that $|\mathbb{Z}_8 \oplus \mathbb{Z}_2| = |\mathbb{Z}_4
\oplus \mathbb{Z}_4|$. Then any homomorphism that is surjective must also be injective due
to our observation. Thus any homomorphism between these two external direct products will be an
isomorphism. But isomorphisms preserve order and since $\mathbb{Z}_8 \oplus \mathbb{Z}_2$ has
an element of order 8, which $\mathbb{Z}_4 \oplus \mathbb{Z}_4$ cannot possibly have, there 
cannot exist a homomorphism between $\mathbb{Z}_8 \oplus \mathbb{Z}_2$ and $\mathbb{Z}_4
\oplus \mathbb{Z}_4$.
\end{enumerate}

\item[10.19] Suppose that there is a homomorphism $\phi$ from $\mathbb{Z}_{17}$ to some
group and that $\phi$ is not one-to-one. Determine $\phi$.
\begin{enumerate}
\item[] Suppose that $\phi$ is a homomorphism from $\mathbb{Z}_{17}$ to some group, $G$, and 
$\phi$ is not one-to-one. Then we know by theorem 10.1, property 4 that Ker $\phi$ is a 
subgroup of $\mathbb{Z}_{17}$ and by Lagrange's theorem the order of Ker $\phi$ must divide the order 
of $\mathbb{Z}_{17}$. Since $|\mathbb{Z}_{17}| = 17$ is prime, Ker $\phi$ can either be 
of size 1 or 17. By theorem 10.2, property 5, $|$Ker $\phi | \neq 1$, so it must 17. Thus
$phi$ is the trivial map, that is it takes all elements of $\mathbb{Z}_{17}$ and maps them to
the identity of $G$.
\end{enumerate}

\item[10.22] Suppose that $\phi$ is a homomorphism from a finite group $G$ onto $\overline{G}$ and
that $\overline{G}$ has an element of order 8. Prove that $G$ has an element of order $8$. Generalize.
\begin{enumerate}
\item[]
Let $\phi: G \rightarrow \overline{G}$ be a homomorphism where $G$ is finite, 
and $\overline{g} \in \overline{G}$ where $|\overline{g}| = 8$. 
Let $g \in G$ where $\phi(g) = \overline{g}$, then from prop 3 theorem 10.1 $|\overline{g}|$
divides $|g|$ and $|g| = 8 · n, n \in N$. Furthermore $g^{8·n} = e$ and
$(g^n)^8 = e$ where $g^n \in G$ and $|g^n| = 8$.
Note, $|g^n|$ cannot be less than 8. That would imply $(g^n)^k = e, k < 8 \Rightarrow 
g^{k·n} = e$ where $k \cdot n < 8 \cdot n$ contradicting the fact that $|g| = 8 \cdot n$.
Therefore $G$ has an element of order 8.
\end{enumerate}

\item[10.31] Suppose that $\phi$ is a homomorphism from $U(30)$ to $U(30)$ and that Ker $\phi =
\{1, 11\}$. If $\phi(7) = 7$, find all elements of $U(30)$ that map to 7.
\begin{enumerate}
\item[] We can create the mapping $\phi : U(30) \to U(30)$ explicitly. Knowing that 
$\phi(1) = \phi(11) = 1$ and $\phi(\langle 7\rangle) = \langle \phi(7)\rangle = \langle 7\rangle$ then we have
$\phi(7) = 7, \phi(7^2) = \phi(19) = 19$, and $\phi(7^3) = \phi(13) = 13$. Now, using
the homomorphic properties of $\phi$ we can find all the other elements. So:
$\newline$
$\phi(17) = \phi(7\cdot 11) = \phi(7)\phi(11) = 7$
$\newline$
$\phi(29) = \phi(7\cdot 17) = \phi(7)\phi(17) = 19$
$\newline$
$\phi(23) = \phi(7\cdot 29) = \phi(7)\phi(29) = 13$
$\newline$
Thus, the elements in $U(30)$ that map to 7 are 7, and 17 (that is $\phi(7) = \phi(17) = 7$).
\end{enumerate}

\item[10.36] Suppose that there is a homomorphism $\phi$ from $\mathbb{Z} \oplus \mathbb{Z}$
to a group $G$ such that $\phi((3, 2)) = a$ and $\phi((2, 1)) = b$. Determine $\phi((4, 4))$
in terms of $a$ and $b$. Assume that the operation of $G$ is addition.
\begin{enumerate}
\item[] The following first equality is from theorem 10.1, property 2 while the rest follows from what
was given:
\[
\phi((4, 4)) = 4\phi((1, 1)) = 4\phi((3, 2) - (2, 1)) = 4[\phi(3, 2) - \phi(2, 1)]
\]
\[ = 4(a - b) = 4a - 4b
\]
\end{enumerate}

\item[10.37] Prove that the mapping $x \rightarrow x^6$ from $\mathbb{C}^*$ to 
$\mathbb{C}^*$ is a homomorphism. What is the kernel?
\begin{enumerate}
\item[] Following example 10.8, we consider the mapping $\phi$ from $\mathbb{C}^*$ to 
$\mathbb{C}^*$ given by $\phi(x) = x^6$. Since $(xy)^6 = x^6y^6$, $\phi$ is a homomorphism.
Clearly, Ker$\phi = \{ x |\,\, x^6 = 1 \}$. But this is just the group of unity of order 6,
hence Ker $\phi = \{ e^{(i\pi k)/3} |\,\, 0 \leq k < 6, k \in \mathbb{Z}\}$.
\end{enumerate}

\item[10.44] Let $N$ be a normal subgroup of a finite group $G$. use the theorems of
this chapter to prove that the order of the group element $gN$ in $G/N$ divides the order
of $g$.
\begin{enumerate}
\item[] Suppose that $G$ is a finite group and $N \lhd G$. Then by theorem 10.4, $N$ is 
the kernel of a homomorphism, let's say $\phi: G \rightarrow G/N$, of $G$, where the mapping 
is defined as $g \rightarrow gN$. Thus, by theorem 10.1, property 3, the order of any
$\phi(g) = gN$ must divide $|g|$.
\end{enumerate}

\item[11.6] Show that there are two Abelian groups of order 108 that have exactly one
subgroup of order 3.
\begin{enumerate}
\item[] Possible Abelian groups of order 108: \\
\begin{tabular}{|r|r|}
\hline
Abelian groups of order 108 & number of subgroups of order 3 \\
\hline
$\mathbb{Z}_6 \oplus \mathbb{Z}_6 \oplus \mathbb{Z}_3$ & 13 \\
$\mathbb{Z}_3 \oplus \mathbb{Z}_3 \oplus \mathbb{Z}_3 \oplus \mathbb{Z}_4$ & 13 \\
$\mathbb{Z}_9 \oplus \mathbb{Z}_3 \oplus \mathbb{Z}_4$ & 4 \\
$\mathbb{Z}_9 \oplus \mathbb{Z}_3 \oplus \mathbb{Z}_2 \oplus \mathbb{Z}_2$ & 4 \\
$\mathbb{Z}_{27} \oplus \mathbb{Z}_2 \oplus \mathbb{Z}_2$ & 1 \\
$\mathbb{Z}_{27} \oplus \mathbb{Z}_4$ & 1 \\
\hline
\end{tabular} \\
We did the first two Abelian groups in class and for the rest we can quickly find the elements which are of order 3. 
Yet we'll only concern ourselves with $\mathbb{Z}_{27} \oplus \mathbb{Z}_2 \oplus \mathbb{Z}_2$ in which 
we have $\langle (9, 0, 0)\rangle$ and $\langle (9, 0)\rangle$ for $\mathbb{Z}_{27} \oplus \mathbb{Z}_4$ and we're done.
\end{enumerate}

\item[11.10] Find all Abelian groups (up to isomorphism) of order 360.
\begin{enumerate}
\item[] Prime factorization of $360: 5 \cdot 3^2 \cdot 2^3$ from which we get all the Abelian groups.\\
$\mathbb{Z}_5 \oplus \mathbb{Z}_9 \oplus \mathbb{Z}_8$ \\
$\mathbb{Z}_5 \oplus \mathbb{Z}_9 \oplus \mathbb{Z}_4 \oplus \mathbb{Z}_2$ \\
$\mathbb{Z}_5 \oplus \mathbb{Z}_9 \oplus \mathbb{Z}_2 \oplus \mathbb{Z}_2 \oplus \mathbb{Z}_2$ \\
$\mathbb{Z}_5 \oplus \mathbb{Z}_3 \oplus \mathbb{Z}_3 \oplus \mathbb{Z}_8$ \\
$\mathbb{Z}_5 \oplus \mathbb{Z}_3 \oplus \mathbb{Z}_3 \oplus \mathbb{Z}_4 \oplus \mathbb{Z}_2$ \\
$\mathbb{Z}_5 \oplus \mathbb{Z}_3 \oplus \mathbb{Z}_3 \oplus \mathbb{Z}_2 \oplus \mathbb{Z}_2 \oplus \mathbb{Z}_2$
\end{enumerate}

\item[11.11] Prove that every finite Abelian group can be expressed as the (external) directo product of
cyclic groups of orders $n_1, n_2, \ldots, n_t$, where $n_{i + 1}$ divides $n_i$ for $i = 1, 2, \ldots, t - 1$.
\begin{enumerate}
\item[] First off, by the fundamental theorem of finite abelian groups, any finite abelian group, $G$,
is isomorphic to a group of the form 
$\mathbb{Z}/p_1^{m_1}\mathbb{Z} \oplus \mathbb{Z}/p_2^{m_2}\mathbb{Z} \oplus \cdots \oplus 
\mathbb{Z}/p_k^{m_k}\mathbb{Z}$. Then by theorem 8.2, corollary 2 we know that we can 
represent all those $\mathbb{Z}/p_i^{m_i}\mathbb{Z}$'s and $\mathbb{Z}/p_j^{m_j}\mathbb{Z}$'s as
$\mathbb{Z}/n\mathbb{Z}$, where $n$ is the product of all the $p_i^{m_i}$'s and $p_j^{m_j}$'s as 
long as the gcd$(p_i^{m_i}, p_j^{m_j}) = 1$ for all $p_i^{n^i}$ and $p_j^{n^j}$. So if any $G$ is indeed
in fact isomorphic to some group, $\mathbb{Z}/p_1^{m_1}\mathbb{Z} \oplus \mathbb{Z}/p_2^{m_2}\mathbb{Z} 
\oplus \cdots \oplus \mathbb{Z}/p_k^{m_k}\mathbb{Z}$ then we can select the maximal $p_i^{n_i}$ and 
using theorem 8.2, corollary 2 combine all the other $p_j^{n_j}$'s with it, labeling it as 
$\mathbb{Z}/n_1\mathbb{Z}$. Now we know for a fact that every other $p_i^{n_i}$ left over divides $n_1$
because otherwise it would have been a part of the product that comprises $n_1$. Thus, by repeating the
process and labeling the newly created products $n_2, n_3, \ldots, n_l$, then due to the fact 
that everything left over after "creating" $n_i$ divides $n_i$ and since $n_{i + 1}$ is made up of 
some or all of those leftovers then $n_{i + 1}$ will clearly divide $n_i$, which is exactly
what we want.
\end{enumerate}

\item[11.15] How many Abelian groups (up to isomorphism) are there
\begin{enumerate}
\item[a)] of order 6? \\ 
1 Abelian group of order 6: $\mathbb{Z}_2 \oplus \mathbb{Z}_3$.
\item[b)] of order 15? \\
1 Abelian group of order 15: $\mathbb{Z}_5 \oplus \mathbb{Z}_3$.
\item[c)] of order 42? \\
1 Abelian group of order 42: $\mathbb{Z}_2 \oplus \mathbb{Z}_3 \oplus \mathbb{Z}_7$.
\item[d)] of order $pq$, where $p$ and $q$ are distinct primes? \\
1 Abelian group of order $pq$: $\mathbb{Z}_p \oplus \mathbb{Z}_q$.
\item[e)] of order $pqr$, where $p, q$, and $r$ are distinct primes? \\
1 Abelian group of order $pqr$: $\mathbb{Z}_p \oplus \mathbb{Z}_q \oplus \mathbb{Z}_r$.
\item[f)] Generalize parts d and e. \\
There will be only 1 Abelian group of order $n$ as long as $n = p_1^{n_1}p_2^{n_1} \cdots
p_k^{n_k}$, where any $p_i \neq p_j$, and $1 = n_1 = n_2 = \ldots = n_k$.
\end{enumerate}

\item[11.16] How does the number (up to isomorphism) of Abelian groups of order $n$ compare with
the number (up to isomorphism) of Abelian groups of order $m$ where
\begin{enumerate}
\item[a)] $n = 3^2$ and $m = 5^2$? \\
The Abelian groups of order $n$ are $\mathbb{Z}_3 \oplus \mathbb{Z}_3$ and $\mathbb{Z}_9$. The
Abelian groups of order $m$ are $\mathbb{Z}_5 \oplus \mathbb{Z}_5$ and $\mathbb{Z}_{25}$. Hence $n$ and $m$
have the same number of Abelian groups.
\item[b)] $n = 2^4$ and $m = 5^4$? \\
Instead of showing all the Abelian groups, we can refer to the claim below. [part (c) of this question], 
from which we can conclude that the amount of Abelian groups of order $n$ and $m$ are the same.
\item[c)] $n = p^r$ and $m = q^r$, where $p$ and $q$ are prime? \\
Referring to page 218, we can see that $p$ is any prime, thus replacing it by some other prime, $q$, 
results in the same amount of possible Abelian groups.
\item[d)] $n = p^r$ and $m = p^rq$, where $p$ and $q$ are distinct prime? \\
The amount of Abelian groups of order $n$ and order $m$ are the same due to the fact that any Abelian group
of order $n$ can have a $\mathbb{Z}_q$ tacked onto the external direct product to yield a "corresponding" Abelian group
of order $m$.
\item[e)] $n = p^r$ and $m = p^rq^2$, where $p$ and $q$ are distinct prime? \\
We have two possibilites for $q^2$, namely $\mathbb{Z}_q \oplus \mathbb{Z}_q$ and $\mathbb{Z}_{q^2}$. Appending the
first possible Abelian group of $q^2$ to the Abelian groups of order $n$ yields the amount of order $n$ that are of order $m$, 
but since you can also append the second possible Abelian group of $q^2$ to those of order $n$, we have an amount of 
Abelian groups of order $m$ twice that than the number of groups of order $n$.
\end{enumerate}


\item[11.22] Characterize those integers $n$ such that any Abelian group of order $n$ belongs
to one of exactly four isomorphism classes.
\begin{enumerate}
\item[] We can characterize $n$ as any integer such that it's prime factorization is $p^2 \cdot q^2$, where
$p$ and $q$ are distinct primes. Furthermore, we can tack on any $\mathbb{Z}_r$ where $r$ is some prime
not equal to $p$ and $q$. Thus we will always have four Abelian groups of order $n$, namely \\
$\mathbb{Z}_{p^2} \oplus \mathbb{Z}_{q^2} \oplus \mathbb{Z}_{r_1} \oplus \cdots \oplus \mathbb{Z}_{r_n}$ \\
$\mathbb{Z}_p \oplus \mathbb{Z}_p \oplus \mathbb{Z}_{q^2} \oplus \mathbb{Z}_{r_1} \oplus \cdots \oplus \mathbb{Z}_{r_n}$ \\
$\mathbb{Z}_q \oplus \mathbb{Z}_q \oplus \mathbb{Z}_{p^2} \oplus \mathbb{Z}_{r_1} \oplus \cdots \oplus \mathbb{Z}_{r_n}$ \\
$\mathbb{Z}_p \oplus \mathbb{Z}_p \oplus \mathbb{Z}_q \oplus \mathbb{Z}_q \oplus \mathbb{Z}_{r_1} \oplus \cdots \oplus \mathbb{Z}_{r_n}$ \\
and any $\mathbb{Z}_{r_i} \neq \mathbb{Z}_{r_j}$.
\end{enumerate}

\item[11.29] Let $G$ be an Abelian group of order 16. Suppose that there are elements $a$ and $b$ in $G$ such that
$|a| = |b| = 4$ and $a^2 \neq b^2$. Determine the isomorphism class of $G$.
\begin{enumerate}
\item[] Since $16 = 2^4$, we have up to 5 possible direct products for $G$, namely \\
$\mathbb{Z}/16\mathbb{Z}$ \\
$\mathbb{Z}/8\mathbb{Z} \oplus \mathbb{Z}/2\mathbb{Z}$ \\
$\mathbb{Z}/4\mathbb{Z} \oplus \mathbb{Z}/4\mathbb{Z}$ \\
$\mathbb{Z}/4\mathbb{Z} \oplus \mathbb{Z}/2\mathbb{Z} \oplus \mathbb{Z}/2\mathbb{Z}$ \\
$\mathbb{Z}/2\mathbb{Z} \oplus \mathbb{Z}/2\mathbb{Z} \oplus \mathbb{Z}/2\mathbb{Z} \oplus 
\mathbb{Z}/2\mathbb{Z}$ \\
We can quickly eliminate the all but the third possibility because $\mathbb{Z}/16\mathbb{Z}$, \\
$\mathbb{Z}/8\mathbb{Z} \oplus \mathbb{Z}/2\mathbb{Z}$, and $\mathbb{Z}/4\mathbb{Z} 
\oplus \mathbb{Z}/2\mathbb{Z} \oplus \mathbb{Z}/2\mathbb{Z}$ 
have only 1 element of order 4 while 
$\mathbb{Z}/2\mathbb{Z} \oplus \mathbb{Z}/2\mathbb{Z} \oplus \mathbb{Z}/2\mathbb{Z} \oplus 
\mathbb{Z}/2\mathbb{Z}$ has no elements of order 4 at all. So the answer must be 
$\mathbb{Z}/4\mathbb{Z} \oplus \mathbb{Z}/4\mathbb{Z}$.
\end{enumerate}

\item[11.30] Prove that an Abelian group of order $2^n (n \geq 1)$ must have an odd number of elements of order 2.
\begin{enumerate}
\item[] First thing to remark is that any finite Abelian group, $G$, of order $2^n$, where $n \geq 1$, will
have an even amount of elements. We can now refer to exercise 3.35 as proof of existence of an element of 
order 2. Knowing this, we can now match each element with it's inverse, noting the fact that
any element with order 1 or 2 will have no matching. Then we separate all the elements into two separate sets, 
$A$ and $B$, where we throw the matching elements in set $A$, while all others (the elements of order 1 or 2) 
go into set $B$. Since each matching contibutes exactly 2 elements to set $A$, it must be that $A$ has an even
amount of elements. But we have an even amount of elements in $G$, which indicates that it must be that 
there are an even amount of elements that are in set $B$. Yet, we forgot about the identity, which also happens 
to be in set $B$ and is the only element of order 1. So taking that from $B$, we must conclude that
the elements left are those of order 2 and, in fact, $B$ must contain an odd number of them.
\end{enumerate}

\item[12.8] Show that a ring is commutative if it has the property that $ab = ca$ implies $b = c$ when $a \neq 0$.
\begin{enumerate}
\item[] Suppose that a ring, $R$, has the property such that $ab = ca$ implies that $b = c$ when $a \neq 0$.
Now we let $a, b \in R$, then we know by definition that $a(ba) = (ab)a$, but this implies $ba = ab$.
Hence the ring is commutative.
\end{enumerate}

\item[12.14] Let $a$ and $b$ belong to a ring $R$ and let $m$ be an integer. Prove that 
$m \cdot (ab) = (m \cdot a)b = a(m \cdot b)$.
\begin{enumerate}
\item[] Let $a$ and $b$ belong to a ring $R$ and let $m$ be an integer. Then
\[
m \cdot (ab) = \overbrace{ab + ab + \cdots + ab}^{\mbox{$m$ times}} = 
\overbrace{a + a + \cdots + a}^{\mbox{$m$ times}}b
= (m \cdot a)b
\]
and
\[
m \cdot (ab) = \overbrace{ab + ab + \cdots + ab}^{\mbox{$m$ times}} = 
a\overbrace{b + b + \cdots + b}^{\mbox{$m$ times}}
= a(m \cdot b)
\]
Hence $m \cdot (ab) = (m \cdot a)b = a(m \cdot b)$.
\end{enumerate}

\item[12.17] Show that a ring that is cyclic under addition is commutative.
\begin{enumerate}
\item[] Suppose that a ring, $R$, is cyclic under addition. Then that means that any element $a \in R$, can
be expressed as $n \cdot x$, where $x \in R$ and $n \in \mathbb{N}$. Now let $a, b \in R$. Clearly, 
$a = n \cdot x$ and $b = m \cdot x$ for some $n, m \in \mathbb{N}$. Hence, we have
\[
ab = (n \cdot x)b = (\overbrace{x + x + \cdots + x}^{\mbox{$n$ times}})b = \overbrace{xb + xb + \cdots + xb}^{\mbox{$n$ times}}
= n \cdot xb
\]
yet we also have
\[
ba = (m \cdot x)a = (\overbrace{x + x + \cdots + x}^{\mbox{$m$ times}})b = \overbrace{xa + xa + \cdots + xa}^{\mbox{$m$ times}}
= m \cdot xa
\]
Furthermore, 
\[
n \cdot xb = n \cdot x(m \cdot x) = n \cdot x(\overbrace{x + x + \cdots + x}^{\mbox{$m$ times}}) = 
n \cdot (\overbrace{x^2 + x^2 + \cdots + x^2}^{\mbox{$m$ times}})
\]
\[
= \overbrace{\overbrace{x^2 + x^2 + \cdots + x^2}^{\mbox{$m$ times}} + \overbrace{x^2 + x^2 + \cdots + x^2}^{\mbox{$m$ times}}
+ \cdots \overbrace{x^2 + x^2 + \cdots + x^2}^{\mbox{$m$ times}}}^{\mbox{$n$ times}} = nm \cdot x^2
\]
and
\[
m \cdot xa = m \cdot x(n \cdot x) = m \cdot x(\overbrace{x + x + \cdots + x}^{\mbox{$n$ times}}) = 
m \cdot (\overbrace{x^2 + x^2 + \cdots + x^2}^{\mbox{$n$ times}})
\]
\[
= \overbrace{\overbrace{x^2 + x^2 + \cdots + x^2}^{\mbox{$n$ times}} + \overbrace{x^2 + x^2 + \cdots + x^2}^{\mbox{$n$ times}}
+ \cdots \overbrace{x^2 + x^2 + \cdots + x^2}^{\mbox{$n$ times}}}^{\mbox{$m$ times}} = mn \cdot x^2
\]
Thus we can conclude that $ab = n \cdot xb = nm \cdot x^2 = mn \cdot x^2 = m \cdot xa = ba$, where the
third equality comes from the fact that $m$ and $n$ were chosen to be natural numbers, which commute.
\end{enumerate}

\item[12.18] Let $a$ belong to a ring $R$. Let $S = \{ x \in R \mid ax = 0 \}$. Show that $S$ is a subring of
$R$.
\begin{enumerate}
\item[] Suppose that $S = \{ x \in R \mid ax = 0 \}$ where $a \in R$ and $R$ is some ring. Let us assume
that $S$ is nonempty. So take $s, s' \in S$, then $a(s - s') = as - as' = 0 - 0 = 0$, where the second
equality comes from theorem 12.1, rule 4, so $(s - s') \in S$. Furthermore, $a(ss') = (as)s' = 0s = 0$, 
so $(ss') \in S$. Thus $S$ is a subring of $R$ by theorem 12.3.
\end{enumerate}

\item[12.22] Let $R$ be a commutative ring with unity and let $U(R)$ denote the set of units of $R$. Prove that 
$U(R)$ is a group under the multiplication of $R$. (This group is called the group of units of $R$.)
\begin{enumerate}
\item[] Firstly, we note that the unity, let's say $e$, is it's own inverse and must be in $U(R)$, so $U(R)$ has an identity. 
Furthermore, every element in $U(R)$ by definition has an inverse. We know $U(R)$ is associative by definition number
5 on pg 235. Lastly, let's check that $U(R)$ is closed under multiplication of $R$. So, take two elements $a, b \in U(R)$.
Then this means that there exists $c, d \in R$ such that $ac = e = ca$ and $bd = e = db$. Hence 
$(ab)(dc) = a(bd)c = aec = ac = e$ and since $dc \in R$, then $ab \in U(R)$. So, $U(R)$ satisfies all properties that
would make it a group.
\end{enumerate}

\item[12.26] Determine $U(\mathbb{R}[x])$.
\begin{enumerate}
\item[] We already know that $0$ is the only real number that has no multiplicative inverse, thus
$U(\mathbb{R}[x]) = \{ f(x) = a \mid a \in \mathbb{R}^* \}$.
\end{enumerate}

\item[12.28] In $\mathbb{Z}_6$, show that $4 \mid 2$; in $\mathbb{Z}_8$, show that $3 \mid 7$;
in $\mathbb{Z}_{15}$, show that $9 \mid 12$.
\begin{enumerate}
\item[] To show that $4 \mid 2$ in $\mathbb{Z}/6\mathbb{Z}$, we need to find an element, $c \in 
\mathbb{Z}/6\mathbb{Z}$ such that $4c = 2$. So, consider $c = 5$, then $(4 \cdot 5) $ mod $6 = 20$ mod $6 = 2$.
Now $3 \mid 7$ in $\mathbb{Z}/8\mathbb{Z}$ due to the fact that $3 \cdot 5 = 7$ and $5 \in \mathbb{Z}/8\mathbb{Z}$.
Lastly, $9 \mid 12$ in $\mathbb{Z}/15\mathbb{Z}$ because $9 \cdot 8 = 12$ and $8 \in 
\mathbb{Z}/15\mathbb{Z}$.
\end{enumerate}

\item[12.29] Suppose that $a$ and $b$ belong to a commutative ring $R$. If $a$ is a unit of $R$ and $b^2 = 0$,
show that $a + b$ is a unit of $R$.
\begin{enumerate}
\item[] Let $a$ and $b$ belong to a commutative ring $R$ with $a$ being a unit of $R$ and 
$b^2 = 0$. We want to show that $a + b$ is a unit, so let's take an element of $R$, namely
$a^{-1} - a^{-2}b$, that will give us the unity, $e$, when we multiply $a + b$ by it. Hence,
\begin{eqnarray*}
(a + b)(a^{-1} - a^{-2}b) &=& aa^{-1} - aa^{-2}b + ba^{-1} - ba^{-2}b \\
&=& e - e(a^{-1}b) + ba^{-1} - a^{-2}b^2 \\
&=& e - a^{-1}b + a^{-1}b - (a^{-2})(0) \\
&=& e + 0 - 0 \\
&=& e
\end{eqnarray*}
\end{enumerate}

\item[12.40] Let $M_2(\mathbb{Z})$ be the ring of all $2 \times 2$ matrices over the integers and let \\ 
$R = \left\{ \left[ 
\begin{array}{cc}
a & a + b \\
a + b & b
\end{array} \right] \mid a, b \in \mathbb{Z} \right\}$. Prove or disprove that $R$ is a subring of $M_2(\mathbb{Z})$.
\begin{enumerate}
\item[] Since it is clear that $R$ is nonempty, let us take two elements in $R$, namely 
$\left[ \begin{array}{cc}
a & a + b \\
a + b & b
\end{array} \right]$ and $\left[ \begin{array}{cc}
c & c + d \\
c + d & d
\end{array} \right]$. Then
\[
\left[ \begin{array}{cc}
a & a + b \\
a + b & b
\end{array} \right] - \left[ \begin{array}{cc}
c & c + d \\
c + d & d
\end{array} \right] =
\left[ \begin{array}{cc}
a - c & a + b - (c + d) \\
a + b - (c + d) & b - d
\end{array} \right]
\]
and
\[
\left[ \begin{array}{cc}
a & a + b \\
a + b & b
\end{array} \right] \left[ \begin{array}{cc}
c & c + d \\
c + d & d
\end{array} \right] =
\left[ \begin{array}{cc}
ac + (a+b)(c+d) & ac + 2ad + bd \\
ac + 2bc + bd & bd + (a+b)(c+d)
\end{array} \right].
\] 
Hence it is not closed under multiplication and thus not a subring of $M_2(\mathbb{Z})$.
\end{enumerate}

\item[12.42] Let $M_2(\mathbb{Z})$ be the ring of all $2 \times 2$ matrices over the integers and let \\ 
$R = \left\{ \left[ 
\begin{array}{cc}
a & a - b \\
a - b & b
\end{array} \right] \mid a, b \in \mathbb{Z} \right\}$. Prove or disprove that $R$ is a subring of $M_2(\mathbb{Z})$.
\begin{enumerate}
\item[] Again $R$ is clearly nonempty. So, let us take two elements in $R$, namely 
$\left[ \begin{array}{cc}
a & a \\
b & b
\end{array} \right]$ and $\left[ \begin{array}{cc}
c & c \\
d & d
\end{array} \right]$. Then
\[
\left[ \begin{array}{cc}
a & a \\
b & b
\end{array} \right] - \left[ \begin{array}{cc}
c & c \\
d & d
\end{array} \right] =
\left[ \begin{array}{cc}
a - c & a - c \\
b - d & b - d
\end{array} \right]
\]
and
\[
\left[ \begin{array}{cc}
a & a \\
b & b
\end{array} \right] \left[ \begin{array}{cc}
c & c \\
d & d
\end{array} \right] =
\left[ \begin{array}{cc}
ac + ad & ac + ad \\
bc + bd & bc + bd
\end{array} \right].
\] 
Both matrices on the right side of the equality are elements of $R$ due to the fact that
they satisfy the form and $a - c, b - d, ac + ad,$
and $bc + bd$ are all elements of $\mathbb{Z}$ whenever $a, b, c, d \in \mathbb{Z}$.
Thus by theorem 12.3, $R$ is a subring of $M_2(\mathbb{Z})$.
\end{enumerate}

\item[12.44] Suppose that there is a positive even integer $n$ such that $a^n = a$ for all elements $a$ of
some ring. Show that $-a = a$ for all $a$ in the ring.
\begin{enumerate}
\item[] Since $n$ is an even integer, we can express it as $2k$ for some $k \in \mathbb{Z}$. Now
suppose $a$ is an element in some ring, $R$, and that $a^n = a$ for all elements in $R$. We want to 
show that $-a = a$ for all $a \in R$. Hence we observe that 
\[ (-a)^{2k} = \overbrace{\overbrace{(-a)(-a)}^{\mbox{$2$ times}}\cdots 
\overbrace{(-a)(-a)}^{\mbox{$2$ times}}}^{\mbox{$k$ times}} = \overbrace{\overbrace{(a)(a)}^{\mbox{$2$ times}}\cdots
\overbrace{(a)(a)}^{\mbox{$2$ times}}}^{\mbox{$k$ times}} = a^{2k}
\]
where the second equality comes from theorem 12.1, rule 3. Yet, $a^n = (-a)^n$ and $a^n = a, \forall a \in R$ 
implies that $-a = a$ and we are done.
\end{enumerate}

\item[12.49] Let $R$ be a ring. Prove that $a^2 - b^2 = (a + b)(a - b)$ for all $a, b$ in $R$
if and only if $R$ is commutative.
\begin{enumerate}
\item[] Let $R$ be a ring and suppose that $a^2 - b^2 = (a + b)(a - b)$ for all $a, b$ in $R$. Then
$(a + b)(a - b) = a^2 - ab + ba - b^2$, but since $(a + b)(a - b)$ is equal to $a^2 - b^2$, it must
be that $-ab + ba = 0$. Hence $ba = ab$ for all $a, b$ in $R$, thus $R$ is commutative. Conversely,
suppose that $R$ is commutative. Then $(a + b)(a - b) = a^2 - ab + ba - b^2$ for any $a, b$ in $R$.
Since $R$ is commutative, $ab = ba$, so $a^2 - ab + ba - b^2 = a^2 - ba + ba - b^2$. With 
$-ba + ba = 0$, we can conclude that $(a + b)(a - b) = a^2 - b^2$.
\end{enumerate}

\item[13.13] Let $a$ belong to a ring $R$ with unity and suppose that $a^n = 0$ for some positive integer $n$.
(Such an element is called \emph{nilpotent}.) Prove that $1 - a$ has a multiplicative inverse in $R$.
\begin{enumerate}
\item[] We have a ring $R$ with unity, $1$. Let $a \in R$ and $a^n = 0$. We want to show that $1 - a$ is a unit of $R$.
The hint suggests considering $(1 - a)(1 + a + a^2 + \cdots + a^{n-1})$ so:
\begin{eqnarray*}
(1 - a)(1 + a + a^2 + \cdots + a^{n-1})&=& 1 + a + a^2 + \cdots + a^{n-1} - a - a^2 - \cdots - a^n \\
&=& 1 + [(a - a) + (a^2 - a^2) + \cdots + (a^{n-1} - a^{n-1})] - a^n \\
&=& 1 + 0 - a^n \\
&=& 1 - a^n \\
&=& 1 - 0 \\
&=& 1
\end{eqnarray*}
with $1 + a + a^2 + \cdots a^{n-1} \in R$, then $1 - a$ does have a multiplicative inverse.
\end{enumerate}

\item[13.22] Let $R$ be the set of all real-valued functions defined for all real numbers under function
addition and multiplication.
\begin{enumerate}
\item[a)] Determine all zero-divisors of $R$.\\
Any real-valued function, $f$, is a zero-divisor as long as there exists an $x \in \mathbb{R}$ such
that $f(x) = 0$.
\item[b)] Determine all nilpotent elements of $R$.\\
The only possible nilpotent element of $R$ must be 0, otherwise we would have a function, where
$f(x) = a$ and $a^n = 0$.
\item[c)] Show that every nonzero element is a zero-divisor or a unit.\\
Clearly any invertible function is going to be a unit. But this just means that for any function, 
$f$, that there does not exist an $x \in \mathbb{R}$ such that $f(x) = 0$. Any other function, therefore,
must have at least one real root, which makes it a zero divisor.
\end{enumerate}

\item[13.26] Let $R = \{ 0, 2, 4, 6, 8 \}$ under addition and multiplication modulo 10. Prove that $R$ is a field.
\begin{enumerate}
\item[] Let's take all the nonzero elements of $R$, that is $2, 4, 6,$ and $8$. Then show that
no two numbers equals $0$ under multiplication modulo $10$. \\
\begin{tabular}{|c|c|}
\hline
$(2)(4) \equiv 8$ (mod 10) & $(2)(6) \equiv 2$ (mod 10) \\
\hline 
$(2)(8) \equiv 6$ (mod 10) & $(4)(6) \equiv 4$ (mod 10) \\
\hline
$(4)(8) \equiv 2$ (mod 10) & $(8)(6) \equiv 8$ (mod 10) \\
\hline
\end{tabular} \\
With $R$ being clearly commutative, $6$ being the unity, and no zero divisors, then $R$ is an integral domain.
Furthermore, since $R$ is obviously finite ($|R| = 5$), then by theorem 13.2, $R$ is also a field.
\end{enumerate}

\item[13.39] Suppose that $R$ is a commutative ring without zero-divisors. Show that all the nonzero elements of
$R$ have the same additive order.
\begin{enumerate}
\item[] Let $x$ and $y$ be any nonzero elements of $R$, with the additive order of $x$ being $n$ and the
additive order of $y$ being $m$. Then $\overbrace{xy + xy + \cdots + xy}^{\mbox{$n$ times}} = n \cdot (xy) = (n \cdot x)y 
= x(n \cdot y)$, where we've obtained some of the equalities from exercise 12.14, and since $n \cdot x = 0, (n \cdot x)y = 0 = x(n \cdot y)$. With no zero divisors, $n \cdot y = 0$ which
implies that $m \leq n$. Furthermore, $\overbrace{xy + xy + \cdots + xy}^{\mbox{$m$ times}} = m \cdot (xy) = (m \cdot x)y 
= x(m \cdot y)$,where we've obtained some of the equalities from exercise 12.14, 
and since $m \cdot y = 0, (m \cdot x)y = 0 = x(m \cdot y)$. With no zero divisors, $m \cdot x = 0$ which
implies that $n \leq m$. Hence with $n \leq m$ and $m \leq n$, we can only conclude that $n = m$.
\end{enumerate}

\item[13.41] Let $x$ and $y$ belong to a commutative ring $R$ with char $R = p \neq 0$.
\begin{enumerate}
\item[a)] Show that $(x + y)^p = x^p + y^p$.\\
Let $x$ and $y$ belong to a commutative ring $R$ with char $R = p \neq 0$.
We know by the binomial theorem that 
\[ (x + y)^p = \sum_{i = 0}^p {p \choose i} x^iy^{p - i}.
\]
First thing we note is that ${p \choose 0} = 1$ and ${p \choose p} = 1$. So, now we have
\[ 
(x + y)^p = x^0y^p + \sum_{i = 1}^{p-1} {p \choose i} x^iy^{p - i} + x^py^0.
\] The next thing to note is that for any ${p \choose i}$, where $1 \leq i \leq p-1$, we have
\[
{p \choose i} = \frac{p!}{(p-i)!i!} = p \cdot \frac{(p-1)!}{(p-i)!i!}
\]
which we know we can do since nothing will divide $p$ being that $p$ is a prime and that anything
in the denominator will divide it. Now, we can see that for any $1 \leq i \leq p-1$, 
\[
{p \choose i}x^iy^{p-i} = \frac{p!}{(p-i)!i!}x^iy^{p-i} = \frac{(p-1)!}{(p-i)!i!}px^iy^{p-i} = 0
\]
because of the $px^iy^{p-i} = 0$ because $p$ is the character of $R$. Thus,
\[
(x + y)^p = x^0y^p + \sum_{i = 1}^{p-1} {p \choose i} x^iy^{p - i} + x^py^0 = x^0y^p + 0 + x^py^0
= x^p + y^p. 
\]
\item[b)] Show that, for all positive integers $n, (x + y)^{p^n} = x^{p^n} + y^{p^n}$.\\
Using part a as our base case, when $n = 1$, then we can see that it holds true. So, let us
assume that $(x + y)^{p^n} = x^{p^n} + y^{p^n}$.  Then
\[
(x + y)^{p^{n+1}} = [(x + y)^{p^{n}}]^p = (x^{p^n} + y^{p^n})^p = (x^{p^n})^p + (y^{p^n})^p =
x^{p^{n+1}} + y^{p^{n+1}}
\]
where we get the second equality by the inductive hypothesis, the third equality by the base case when
$x^{p^n}, y^{p^n} \in R$. Thus $(x + y)^{p^n} = x^{p^n} + y^{p^n}$ for all $n \in \mathbb{N}$.
\end{enumerate}

\item[13.42] Let $R$ be a commutative ring with unity 1 and prime characteristic. If $a \in R$ is nilpotent,
prove that there is a positive integer $k$ such that $(1 + a)^k = 1$.
\begin{enumerate}
\item[] Let $R$ be a commutative ring with unit 1 and prime characteristic. If $a \in R$ is nilpotent, then
this means that for some positive integer $n$, $a^n = 0$. The next thing to note is that 
$a^m = 0, \forall m \in \mathbb{Z}, m > n$ since $a^m = a^{n + (m - n)} = a^na^{m - n} = 0a^{m - n} = 0$. 
Now all we need to do is pick a $p^t$, where $p$ is a prime and $t$ is a positive integer, such that
$p^t > n$, in order to apply exercise 13.41(b). Hence, $(1 + a)^{p^t} = 1^{p^t} + a^{p^t} = 1 + 0 = 1$.
\end{enumerate}

\item[13.43] Show that any finite field has order $p^n$, where $p$ is a prime. 
\begin{enumerate}
\item[] Suppose we have a finite field, $\mathbb{F}$. Then by theorem 13.3 we know that our characteristic is some $n$, 
but more importantly from theorem 13.4, our $n = p$, where $p$ is some prime. Since $p$ is the least positive integer 
such that $p \cdot 1 = 0$ then we know that $p$ is the additive order of all nonzero elements in $\mathbb{F}$
by exercise 13.39. Thus with $\mathbb{F}$ being a finite Abelian group under addition, 
the fact that all the orders of the nonzero elements are $p$, and theorem 7.1, corollary 2 implies that 
the order of $\mathbb{F}$ is $p^n$.
\end{enumerate}

\item[13.49] Consider the equation $x^2 - 5x + 6 = 0$.
\begin{enumerate}
\item[a)] How many solutions does this equation have in $\mathbb{Z}_7$?\\
We can factor the equation down to $(x - 3)(x - 2) = 0$. Now we want to find how many solutions
this has in $\mathbb{Z}/7\mathbb{Z}$. Well, clearly $3$ and $2$ are solutions, but are there any more.
So we plug $1$ in for $x$ and get $(-2)(-1) \equiv 2$ (mod 7) but $2 \neq 0$ so 1 is not a solution. If we
do this to the rest of the elements of $\mathbb{Z}/7\mathbb{Z}$, then we come to find that only 3 and 2 are.
Hence there are only 2 solutions.
\item[b)] Find all solutions of this equation in $\mathbb{Z}_7$.\\
Again, 2 and 3 are clearly solutions in $\mathbb{Z}/8\mathbb{Z}$. To show some work, let's try 5:
$(2)(3) \equiv 6$ (mod 8) and clearly $6 \neq 0$, so 5 is not a solution. In fact, only 2 and 3 are.
\item[c)] Find all solutions of this equation in $\mathbb{Z}_{12}$.\\
\begin{tabular}{|c|c|}
\hline
x & $(x - 3)(x - 2)$ \\
\hline
0 & $(-3)(-2) \equiv 6$ (mod 12) \\ 
\hline
1 & $(-2)(-1) \equiv 2$ (mod 12) \\
\hline
2 & $(-1)(0) \equiv 0$ (mod 12) \\
\hline
3 & $(0)(1) \equiv 0$ (mod 12) \\
\hline
4 & $(1)(2) \equiv 2$ (mod 12) \\ 
\hline
5 & $(2)(3) \equiv 6$ (mod 12) \\
\hline
6 & $(3)(4) \equiv 0$ (mod 12) \\
\hline
7 & $(4)(5) \equiv 8$ (mod 12) \\
\hline
8 & $(5)(6) \equiv 6$ (mod 12) \\ 
\hline
9 & $(6)(7) \equiv 6$ (mod 12) \\
\hline
10 & $(7)(8) \equiv 8$ (mod 12) \\
\hline
11 & $(8)(9) \equiv 0$ (mod 12) \\
\hline 
\end{tabular}
\item[d)] Find all solutions of this equation in $\mathbb{Z}_{14}$.\\
Without enumerating all elements of $\mathbb{Z}/14\mathbb{Z}$ and by direct computation, we find
2, 3, 9 and 10 as the solutions to the equation $(x - 3)(x - 2) = 0$.
\end{enumerate}

\item[13.54] Let $F$ be a finite field with $n$ elements. Prove that $x^{n-1} = 1$ for all nonzero $x$ in $F$.
\begin{enumerate}
\item[] Let $\mathbb{F}$ be a finite field with $n$ elements. Then there are $n - 1$ elements that are nonzero and
they form a group under multiplication. Thus with the group (all nonzero elements of $\mathbb{F}$) 
under multiplication being of order $n - 1$ and theorem 7.1, corollary 4 we know that $x^{n - 1} = 1$ for all
nonzero elements of the group, but more importantly, for all the nonzero elements of the finite field $\mathbb{F}$.
\end{enumerate}

\item[14.14] Let $A$ and $B$ be ideals of a ring. Prove that $AB \subseteq A \cap B$.
\begin{enumerate}
\item[] We define $AB = \{ a_1b_1 + a_2b_2 + \cdots + a_nb_n \mid a \in A, b \in B, n \in \mathbb{N} \}$. 
So, let $c \in AB$. Then $c = a_1b_1 + a_2b_2 + \cdots + a_nb_n$ but for all $a_ib_i$ that comprise $c$,
$a_ib_i \in A$ and $a_ib_i \in B$ because $A$ and $B$ are both ideal. 
Thus since both $B$ and $A$ are closed under addition, then $c \in A$ and $c \in B$. Hence $AB \subseteq A \cap B$.
\end{enumerate}

\item[14.15] If $A$ is an ideal of a ring $R$ and 1 belongs to $A$, prove that $A = R$.
\begin{enumerate}
\item[] Since $A$ is an ideal of $R$, then we know that $A$ is a subring of $R$ and by definition, $A \subseteq R$.
Hence, we only have $R \subseteq A$ to show equality. So, let us take any element $r \in R$, then $r \in A$ because
$r = 1r$ and $1r \in A$ since $A$ is an ideal. Hence $R \subseteq A$ but more importantly $R = A$.
\end{enumerate}

\item[14.20] Let $I = \langle 2\rangle$. Prove that $I[x]$ is not a maximal ideal of 
$\mathbb{Z}[x]$ even though $I$ is a maximal ideal of $\mathbb{Z}$.
\begin{enumerate}
\item[] Let $I[x] = \{ f(x) \in \mathbb{Z}[x] \mid $ all the coefficients of $f(x)$ are even $ \}$ and 
$\langle x, 2\rangle = \{ f(x) \in \mathbb{Z}[x] \mid f(0)$ is an even integer $\}$. With $I[x]$
clearly a proper subset of $\langle x, 2\rangle$ and $\langle x, 2\rangle \subset \mathbb{Z}[x]$ by
example 17, we can conclude that $I[x]$ is not maximal.
\end{enumerate}

\item[14.28] Show that $A = \{ (3x, y) \mid x, y \in \mathbb{Z} \}$ is a maximal ideal of 
$\mathbb{Z} \oplus \mathbb{Z}$.
\begin{enumerate}
\item[] We let $B = \{ (x, y) \mid x, y \in \mathbb{Z} \}$ and it is quite clear that $A \subset B$ (and it
can only be $B$ since $\langle 3\rangle$ is a maximal ideal of $\mathbb{Z}$). But
$B = \mathbb{Z} \oplus \mathbb{Z}$ by definition, hence $A$ is maximal by definition.
\end{enumerate}

\item[14.38] Let $R$ be a ring and let $p$ be a fixed prime. Show that $I_p = \{ r \in R \mid
$ additive order of $r$ is a power of $p \}$ is an ideal of $R$.
\begin{enumerate}
\item[] To show $I_p$ is an ideal of $R$, we can use theorem 14.1. So, let $a, b \in I_p$, $|a| = p^n$ and
$|b| = p^m$, where $n, m \in \mathbb{N}, n < m$. Then 
$p^{m} \cdot (a - b) = \overbrace{(a - b) + \cdots + (a - b)}^{\mbox{$p^{m}$ times}} = 
(p^{m} \cdot a) - (p^{m} \cdot b) = p^{m - n} \cdot (p^n \cdot a) - 0 = p^{m - n} \cdot 0 = 0$, so 
$a - b \in I_p$. Now, we let $a \in R, |a| = p^n, n \in \mathbb{N}$ and $r \in R$, then because 
$p^n \cdot ra = r(p^n \cdot a) = r0 = 0$ and $p^n \cdot ar = (p^n \cdot a)r = 0r = 0$, both by exercise 12.14, 
we can conclude the additive order of $ra$ and $ar$ must divide
$p^n$, but, of course, the only possibilities for the order of $ra$ and $ar$ are those that are some power
of $p$, thus $ar, ra \in I_p$.
\end{enumerate}


\item[14.49] Show that $\mathbb{Z}_3[x]/\langle x^2 + x + 1\rangle$ is not a field.
\begin{enumerate}
\item[] It suffices to show that if $\mathbb{Z}_3[x]/\langle x^2 + x + 1\rangle$ contains an element is a zero divisor, 
then $\mathbb{Z}_3[x]/\langle x^2 + x + 1\rangle$ cannot be field since it cannot be a finite integral domain. 
Thus, consider $x + 2 + \langle x^2 + x + 1\rangle$. $(x + 2 + \langle x^2 + x + 1\rangle)^2 = 
(x + 2)^2 +  \langle x^2 + x + 1\rangle = x^2 + 4x + 4 + \langle x^2 + x + 1\rangle$ which is
equal to $x^2 + x + 1 + \langle x^2 + x + 1\rangle$ and since $\langle x^2 + x + 1\rangle$ "absorbs" 
$x^2 + x + 1$, then we are left with the zero element of $\mathbb{Z}_3[x]/\langle x^2 + x + 1\rangle$.
\end{enumerate}

\item[14.56] Let $R$ be a commutative ring with unity and let $I$ be a proper ideal with the
property that every element of $R$ that is not in $I$ is a unit of $R$. Prove that $I$ is the
unique maximal ideal of $R$.
\begin{enumerate}
\item[] First off, $I$ must be a maximal ideal since if there was an ideal, $B$, that properly contained
$I$ then any element of $B\setminus I$ is a unit from which we can conclude that $B = R$. Now we can assume
that $B$ is a proper ideal of $R$. Then, $B \subseteq I$ since $B$ has no units (otherwise $B = R$). Thus
$B$ can only maximal if it equal to $I$ whereby proving that $I$ is unique. 
\end{enumerate}

\item[15.13] Consider the mapping from $M_2(\mathbb{Z})$ into $\mathbb{Z}$ given by
$\left[ \begin{array}{cc} a & b \\ c & d \end{array} \right] \rightarrow a$. Prove
or disprove that this is a ring homomorphism.
\begin{enumerate}
\item[] Let $\phi$ be the mapping from $M_2(\mathbb{Z})$ into $\mathbb{Z}$ by
$\left[ \begin{array}{cc} a & b \\ c & d \end{array} \right] \rightarrow a$ and let 
$\left[ \begin{array}{cc} a & b \\ c & d \end{array} \right] \in M_2(\mathbb{Z})$. Then 
$\phi(\left[ \begin{array}{cc} a & b \\ c & d \end{array} \right]^2) = a^2 + bc \neq a^2 = 
(\phi(\left[ \begin{array}{cc} a & b \\ c & d \end{array} \right]))^2$. This violates
theorem 15.1, property 1, hence the mapping is not a ring homomorphism.
\end{enumerate}

\item[15.26] Show that $(\mathbb{Z} \oplus \mathbb{Z})/(\langle a\rangle \oplus \langle b
\rangle )$ is ring-isomorphic to $\mathbb{Z}_a \oplus \mathbb{Z}_b$.
\begin{enumerate}
\item[] Fist it is quite clear $\forall r \in \mathbb{Z} \oplus \mathbb{Z}$, where $r$ is an element
in the form of $(x, y)$ with $x, y \in \mathbb{Z}$, that $r(a, b), 
(a, b)r \in \langle a\rangle \oplus \langle b\rangle$, so we note
that when $ra, rb, r'a, r'b \in \langle a\rangle \oplus \langle b\rangle$ then 
$(ra, rb) - (r'a, r'b) = (ra - r'a, rb - r'b) = ((r - r')a, (r - r')b) \in \langle a\rangle \oplus \langle b\rangle$ and
hence we have $\langle a\rangle \oplus \langle b\rangle$ is an ideal. Now 
let $\phi: \mathbb{Z} \oplus \mathbb{Z}/\langle a\rangle \oplus \langle b\rangle \rightarrow \mathbb{Z}/a\mathbb{Z} 
\oplus \mathbb{Z}/b\mathbb{Z}$ by the mapping $s + \langle a\rangle \oplus \langle b\rangle = s$. Then both 
operation preservations follow from theorem 14.2, that is $\phi((s + \langle a\rangle \oplus \langle b\rangle) + 
(t + \langle a\rangle \oplus \langle b\rangle)) = \phi(s + t + \langle a\rangle \oplus \langle b\rangle) =
s + t = \phi(s + \langle a\rangle \oplus \langle b\rangle) + \phi(t + \langle a\rangle \oplus \langle b\rangle)$ and
$\phi((s + \langle a\rangle \oplus \langle b\rangle)(t + \langle a\rangle \oplus \langle b\rangle)) = 
\phi(st + \langle a\rangle \oplus \langle b\rangle) = st = \phi(s + \langle a\rangle \oplus \langle b\rangle)
\phi(t + \langle a\rangle \oplus \langle b\rangle)$. Let $s + \langle a\rangle \oplus \langle b\rangle, t +
\langle a\rangle \oplus \langle b\rangle \in \mathbb{Z} \oplus \mathbb{Z}/\langle a\rangle \oplus \langle b\rangle$, then
$\phi(s + \langle a\rangle \oplus \langle b\rangle) = \phi(t + \langle a\rangle \oplus \langle b\rangle)$
implies that $s = t$, so $\phi$ is clearly one-to-one and with it being obviously onto, $\phi$ is a ring isomorphism.
\end{enumerate}

\item[15.38] Let $R$ be a commutative ring of prime characteristic $p$. Show that the
\emph{Frobenius} map $x \rightarrow x^p$ is a ring homomorphism from $R$ to $R$.
\begin{enumerate}
\item[] We let $a, b \in R$ and $\phi$ be the Frobenius map $x \rightarrow x^p$. Then 
\[ 
\phi(a + b) = (a + b)^p = a^p + b^p = \phi(a) + \phi(b)
\]
where the second equality comes from exercise 13.41(a) and
\[
\phi(ab) = (ab)^p = a^pb^p = \phi(a)\phi(b)
\]
since $R$ is commutative. Thus the Frobenius map is a ring homomorphism from $R$ to $R$.
\end{enumerate}


\item[15.42] A \emph{principle ideal ring} is a ring with the property that every ideal has the
form $\langle a\rangle$. Show that the homomorphic image of a principal ideal ring is a principal
ideal ring.
\begin{enumerate}
\item[] Let $\phi: R \rightarrow \phi(R)$. Then we know that $\phi^{-1}(B)$ is an 
ideal in $R$ when $B$ is an ideal in $\phi(R)$ by theorem 15.1, property 4. Furthermore, 
for some $b \in R, \phi^{-1}(B) = \langle b\rangle$. Thus $\langle \phi(b)\rangle$ must
be a principal ideal ring as well.
\end{enumerate}

\item[15.46] Show that the only ring automorphism of the real numbers is the identity mapping.
\begin{enumerate}
\item[] Let $\phi$ be an automorphism from $\mathbb{R}$ to $\mathbb{R}$. With the unity of $\mathbb{R}$ being 1,
then it must be that $\phi(1) = 1$. Then for any element $n/m \in \mathbb{Q}, \phi(n/m) = n/m$ by 
$m \cdot \phi(n/m) = \phi(m \cdot n/m) = \phi(n) = n$, which we know we can do since 
$\phi(n) = n \cdot \phi(1) = n \cdot 1 = n$. Furthermore, $\phi(n) \leq \phi(m)$ whenever $n, m \in \mathbb{R}, n \leq m$,
since for any $r \in \mathbb{R}^+, \phi(r) = \phi(\sqrt{r}^2) = \phi(\sqrt{r})^2$ which is greater than 0. Hence,
if $\phi(x) > x$ then we know that there is some $q \in \mathbb{Q}$ such that $\phi(x) > q > x$. But $\phi(q) = q$, which
means that $\phi(x) < \phi(q)$, a contradiction. It's just as well that if $\phi(x) < x$ then we know that 
there is some $q \in \mathbb{Q}$ such that $\phi(x) < q < x$ and with $\phi(q) = q$, we have $\phi(x) > \phi(q)$. Thus the
only automorphism from $\mathbb{R}$ to itself is the identity mapping.
\end{enumerate}

\item[15.51] Let $\mathbb{Z}[i] = \{ a + bi \mid a, b \in \mathbb{Z} \}$. Show that field of
quotients of $\mathbb{Z}[i]$ is ring-isomorphic to $\mathbb{Q}[i] = \{ r + si \mid r, s
\in \mathbb{Q} \}$.
\begin{enumerate}
\item[] The field of quotients of $\mathbb{Z}[i]$ is $\mathbb{Z}(i) = \{ (a + bi)/(c + di) \mid (a + bi), (c + di) \in \mathbb{Z}[i], 
c + di \neq 0 \}$, $(a + bi)/(c + di) \in \mathbb{Q}[i]$. So let $\phi: \mathbb{Q}[i] \rightarrow \mathbb{Z}(i)$ by the 
mapping that $(a/b) + (c/d)i \rightarrow (ad + bci)/(bd)$. Now
\[
\frac{a + bi}{c + di} = \frac{(a + bi)(c - di)}{(c + di)(c - di)} =
\frac{ac + db + (bc - ad)i}{c^2 + d^2} = \frac{ac + db}{c^2 + d^2} + \frac{bc - da}{c^2 + d^2}i
\]

So if $(a + bi)/(c + di) \in \mathbb{Z}(i)$, then 
$\phi(\frac{ac + db}{c^2 + d^2} + \frac{bc - da}{c^2 + d^2}i) = (a + bi)/(c + di)$, so $\phi$ is onto. Also, 
let
\[
\phi(\frac{a}{b} + \frac{c}{d}i) = \phi(\frac{a'}{b'} + \frac{c'}{d'}i)
\]
\[
\frac{ad + bci}{bd} = \frac{a'd' + b'c'i}{b'd'}
\]
\[
b'd'ad + b'd'bci = bda'd' + bdb'c'i
\]
Then
\[
\frac{a}{b} = \frac{a'}{b'} \mbox{ and } \frac{c}{d} = \frac{c'}{d'}
\]
by splitting up the parts and dividing out. Hence we have
\[
\frac{a}{b} + \frac{c}{d}i = \frac{a'}{b'} + \frac{c'}{d'}i
\]
which shows one-to-one. Furthermore,
\[
\phi(\frac{a}{b}+ \frac{c}{d}i + \frac{e}{f} + \frac{g}{h}i) = \phi(\frac{af + eb}{bf} + \frac{ch + dg}{dh}i)
\]
\[
= \frac{afdh + ebdhi + bfch + bfdgi}{dhbf} = \frac{a}{b} + \frac{c}{d}i + \frac{e}{f} + \frac{g}{h}i = 
\phi(\frac{a}{b}+ \frac{c}{d}i) + \phi(\frac{e}{f}+ \frac{g}{h}i)
\]
and with the second operation preservation (not shown) being just as tedious, then we have $\phi$ is a ring isomorphism.
\end{enumerate}

\item[15.60] Let $R = 
\left\{ \left[
\begin{array}{cc} a & b \\ b & a \end{array}
\right] \mid a, b \in \mathbb{Z}
\right\}$, and let $\phi$ be the mapping that takes 
$\left[ \begin{array}{cc} a & b \\ b & a \end{array} \right]$ to $a - b$.
\begin{enumerate}
\item[a)] Show that $\phi$ is a homomorphism.\\
So let $\left[ \begin{array}{cc} a & b \\ b & a \end{array} \right], 
\left[ \begin{array}{cc} c & d \\ d & c \end{array} \right] \in R$. Then
\[
\phi(\left[ \begin{array}{cc} a & b \\ b & a \end{array} \right] + 
\left[ \begin{array}{cc} c & d \\ d & c \end{array} \right]) = 
\phi(\left[ \begin{array}{cc} a + c & b + d\\ b + d & a + c \end{array} \right]) = (a + c) - (b + d) 
\]
\[ 
= (a - b) + (c - d) = \phi(\left[ \begin{array}{cc} a & b \\ b & a \end{array} \right]) + 
\phi(\left[ \begin{array}{cc} c & d \\ d & c \end{array} \right])
\]
and 
\[
\phi(\left[ \begin{array}{cc} a & b \\ b & a \end{array} \right]\left[ \begin{array}{cc} c & d \\ d & c \end{array} \right])
= \phi(\left[ \begin{array}{cc} ac + bd & ad + bc \\ ad + bc & ac + bd \end{array} \right]) = (ac + bd) - (ad + bc)
\]
\[
= (ac - ad) + (bd - bc) = a(c - d) - b(c - d) = (a - b)(c - d) = 
\phi(\left[ \begin{array}{cc} a & b \\ b & a \end{array} \right])\phi(\left[ \begin{array}{cc} c & d \\ d & c \end{array} \right])
\]
and we're done.
\item[b)] Determine the kernel of $\phi$.\\
Since we can only get $a - b = 0$ when $a = b$. Thus, the kernel can be represented as 
$\langle \left[ \begin{array}{cc} 1 & 1 \\ 1 & 1 \end{array} \right] \rangle = 
\{ r\left[ \begin{array}{cc} 1 & 1 \\ 1 & 1 \end{array} \right] \mid r \in R\} = 
\{ \left[ \begin{array}{cc} a + b & a + b \\ a + b & a + b \end{array} \right] \mid a, b \in \mathbb{Z} \}$.
\item[c)] Show that $R/\ker \phi$ is isomorphic to $\mathbb{Z}$.\\
By the first isomorphism theorem for rings, we have $R/\ker \phi \cong \phi(R)$.But $\phi(R) \cong \mathbb{Z}$,
since $a - b \in \phi(R)$ and $a - b \in \mathbb{Z}$ whenever $a, b \in \mathbb{Z}$.
\item[d)] Is $\ker \phi$ a prime ideal?\\
$\ker \phi$ must be a prime ideal by theorem 14.3 since $R/\ker \phi$ is an integral domain since 
it's isomorphic to $\mathbb{Z}$.
\item[e)] Is $\ker \phi$ a maximal ideal?\\
By theorem 14.4, $\ker \phi$ is not maximal because $R/\ker \phi$ is not a field, since $\mathbb{Z}$ is
not a field and $R/\ker \phi$ is isomorphic to $\mathbb{Z}$.
\end{enumerate}

\item[16.42] Let $F$ be a field and let $I = \{ f(x) \in F[x] \mid f(a) = 0$ for all $a$ in
$F \}$. Prove that $I$ is an ideal in $F[x]$. Prove that $I$ is infinite when $F$ is finite and 
$I = \{ 0 \}$ when $F$ is infinite.
\begin{enumerate}
\item[] To show that $I$ is an ideal of $F[x]$, we apply the ideal test, so let
$f(x), g(x) \in I$, then for any $a \in F$, we have $f(a) = 0$ and $g(a) = 0$.
So $f(a) - g(a) = 0 - 0 = 0$, thus $f(x) - g(x) \in I$. Now let $f(x) \in F[x]$ and 
$g(x) \in I$ so that for any $a \in F$, $f(a)g(a) = f(a)0 = 0 = 0f(a) = g(a)f(a)$ and $f(x)g(x), g(x)f(x) \in I$.
Hence $I$ is an ideal.
\end{enumerate}

\item[17.8] Construct a field of order 25.
\begin{enumerate}
\item[] By theorem 17.1 and corollary 1 of theorem 27.5, it suffices to find a cubic polynomial over $\mathbb{Z}_3$ that has
no zero in $\mathbb{Z}_3$. By inspection, $x^3 - x^2 + 1$ is what we had hoped to find. Thus 
$\mathbb{Z}_3/\langle x^3 - x^2 + 1\rangle = \{ ax^2 + bx + c + \langle x^3 - x^2 + 1\rangle \mid a, b, c \in \mathbb{Z}_3 \}$
is a field of order 27 (that is 3 possibilities for $a, b$ and $c$, hence $3^3$).
\end{enumerate}

\item[17.10] Determine which of the polynomials below is (are) irreducible over $\mathbb{Q}$.
\begin{enumerate}
\item[a)] $x^5 + 9x^4 + 12x^2 + 6$\\
We will use theorem 17.4, thus we want a $p$ such that it divides all coefficients except the
one in front of $x^5$, which in this case is 1. Letting $p = 3$, we have what we want plus $3^2 = 9$
clearly does not divide $6$. Thus $x^5 + 9x^4 + 12x^2 + 6$ is irreducible in $\mathbb{Q}$.
\item[b)] $x^4 + x + 1$\\
We already know by example 7 that $x^4 + x + 1$ is irreducible over $\mathbb{Z}_2[x]$ and
thus irreducible over $\mathbb{Q}$ by theorem 17.3.
\item[c)] $x^4 + 3x^2 + 3$\\
Again using $p = 3$ for theorem 17.4, we have that $x^4 + 3x^2 + 3$ is irreducible over $\mathbb{Q}$.
\item[d)] $x^5 + 5x^2 + 1$\\
If we take $x^5 + 5x^2 + 1$ through mod 3 of theorem 17.3, we have $x^5 + 2x^2 + 1$ and by plugging
in 0, 1, and 2 come to find that it has no linear factors. So, we check for quadratic factors, which
has the form $x^2 + ax + b$ and just like example 8 can rule out the ones which have a zero over 
$\mathbb{Z}_3$. This leaves only $x^2 + 1, x^2 + x + 2$, and $x^2 + 2x + 2$ to which we do long division 
(on the back of the paper) to see that $x^5 + 2x^2 + 1$ is irreducible over $\mathbb{Z}_3$, hence $x^5 + 5x^2 + 1$ is irreducible over $\mathbb{Q}$.
\item[e)] $(5/2)x^5 + (9/2)x^4 + 15x^3 + (3/7)x^2 + 6x + 3/14$\\
We can do what we did in class and that is multiply the polynomial by the lcm of the denominators, which
for $(5/2)x^5 + (9/2)x^4 + 15x^3 + (3/7)x^2 + x + 3/14$ is 14. So we have $35x^5 + 63x^4 + 210x^3 + 6x^2 + 84x + 3$, 
which once again, applying theorem 17.4 with $p = 3$, we have that $14[(5/2)x^5 + (9/2)x^4 + 15x^3 + (3/7)x^2 + x + 3/14]$
is irreducible and thus, $(5/2)x^5 + (9/2)x^4 + 15x^3 + (3/7)x^2 + x + 3/14$ is irreducible.
\end{enumerate}


\item[17.12] Show that $x^2 + x + 4$ is irreducible over $\mathbb{Z}_{11}$.
\begin{enumerate}
\item[] It suffices to show that there are no linear factors, so \\
\begin{tabular}{|c|c|c|}
\hline
$x$ & $x^2 + x + 4$ & $=$ in mod 11\\
\hline 
$0$ & $0^2 + 0 + 4$ & 4\\
\hline 
$1$ & $1^2 + 1 + 4$ & 6\\
\hline
$2$ & $2^2 + 2 + 4$ & 10\\
\hline
$3$ & $3^2 + 3 + 4$ & 5\\
\hline 
$4$ & $4^2 + 4 + 4$ & 2\\
\hline
$5$ & $5^2 + 5 + 4$ & 1\\
\hline
$6$ & $6^2 + 6 + 4$ & 2\\
\hline 
$7$ & $7^2 + 7 + 4$ & 5\\
\hline
$8$ & $8^2 + 8 + 4$ & 10\\
\hline
$9$ & $9^2 + 9 + 4$ & 6\\
\hline 
$10$ & $10^2 + 10 + 4$ & 4\\
\hline
\end{tabular} \\
Since none of the values on the most right column are 0, then we have shown that $x^2 + x + 4$ is 
irreducible over $\mathbb{Z}_{11}$.
\end{enumerate}

\item[18.1] For the ring $\mathbb{Z}[\sqrt{d}] = \{ a + b\sqrt{d} \mid a, b \in \mathbb{Z} \}$,
where $d \neq 1$ and $d$ is not divisible by the square of a prime, prove that the norm
$N(a + b\sqrt{d}) = |a^2 - db^2 |$ satisfies the four assertions made preceding Example 1.
\begin{enumerate}
\item[]
\item[Prop. 1] $N(x) = 0$ if and only if $x = 0$. \\
Suppose that $N(x) = 0$, then this means that $|a^2 - db^2| = 0$, which implies that 
$a^2 = db^2$. Since the LHS is a square, the RHS must also be a square. It must follow then
that $a = b = 0$, since $d$ cannot be 1 and cannot be divisible by the square of a prime.

\item[Prop. 2] $N(xy) = N(x)N(y)$ for all $x$ and $y$.
\[ 
N((a + b\sqrt{d})(c + e\sqrt{d})) = N((ac + bed) + (ae + bc)\sqrt{d}) =
\]
\[ 
|(ac + bed)^2 - d(ae + bc)^2| = |a^2 - db^2||c^2 - de^2| = N(a + b\sqrt{d})N(c + e\sqrt{d})
\]

\item[Prop. 3] $x$ is a unit if and only if $N(x) = 1$. \\
Suppose that $x$ is a unit, which means that $xy = 1$ for some $y$, 
then $N(xy) = N(1) = N(1) = N(x)N(y)$, and so $N(x) = N(y) = 1$. 
Now suppose that $N(x) = N(a + b\sqrt{d}) = 1$, then $a^2 - db^2 = \pm 1$, but $ a^2 - db^2 = (a + b\sqrt{d})(a - b\sqrt{d})$.
Thus $a + b\sqrt{d} = x$ is a unit.

\item[Prop. 4] If $N(x)$ is prime, then $x$ is irreducible in $\mathbb{Z}[\sqrt{d}]$. \\
Suppose that $N(x)$ is prime. For the sake of contradiction, assume that $x$ is reducible to $ab$.
Then $N(x) = N(ab) = N(a)N(b)$ by Prop. 2. That means that either $N(a)$ or $N(b)$ is a unit (equal to 1 by 
prop. 3) since $N(x)$ is prime. Contradiction. Hence $x$ must be irreducible.
\end{enumerate}

\item[18.10] Let $D$ be a principal ideal domain and let $p \in D$. Prove that $\langle p
\rangle$ is a maximal ideal in $D$ if and only if $p$ is irreducible.
\begin{enumerate}
\item[] Firstly, suppose that $\langle p\rangle$ is a maximal ideal. Now, let's assume
that $p$ is reducible or that $p = ab$ for some $a, b \in D$. Furthermore, $p \in \langle a\rangle$ and
$\langle p\rangle \subseteq \langle a\rangle \subseteq D$. Since, $\langle p\rangle$ is maximal, then
either $\langle p\rangle = \langle a\rangle$ or $\langle a\rangle = D$.
If $\langle p\rangle = \langle a\rangle$, then $a \in \langle p\rangle$, which means that so $a = pc$ for some
$c \in D$. Clearly, $p = p \cdot 1 = ab$ and $ab = pcb$, so $p = pcb$ which is $1 = cb$ by cancellation. Thus $b$ must be 
a unit. Now, if $\langle a\rangle = D$, then $an = 1$ for some $n \in D$. Thus $a$ is a unit. We can now conclude that
when $p$ reduces that one of the factors is a unit, hence $p$ is irreducible.\\
Conversely, suppose that $p$ is irreducible in $D$. Then $\langle p\rangle \subseteq \langle d\rangle \subseteq D$, 
where $\langle d\rangle$ is some ideal (since $D$ is a PID) in $D$ for some $d \in D$. 
Clearly $p = cd$ for  some $c \in D$. But $p$ is irreducible, which means that either $c$ is a unit or $d$ is a unit.
If $d$ is a unit, then $\langle d\rangle = D$ by exercise 14.17 (used on an earlier homework assignment). If $c$ is a unit,
then $c^{-1}p = d$ which means that $d \in \langle p\rangle$ and $\langle d\rangle \subseteq \langle p\rangle$. Hence
$\langle d\rangle = \langle p\rangle$. Thus $\langle p\rangle$ must be maximal.
\end{enumerate}

\item[18.22] In $\mathbb{Z}[\sqrt{5}]$, prove that both 2 and $1 + \sqrt{5}$ are irreducible
but not prime.
\begin{enumerate}
\item[] We first note that both $N(2)$ and $N(1 + \sqrt{5})$ are both equal to 4. Now if we can show that 
$\forall x \in \mathbb{Z}[\sqrt{5}], N(x) = N(a + b\sqrt{5}) \neq 2$, then we'll be done. So, we assume that
$|a^2 + 5b^2| = 2$, then we have $a^2 + 5b^2 = 2$ or $a^2 + 5b^2 = -2$. We can reduce modulo 5 so that
we have $a^2 = 2$ or $a^2 = 3$. But there is no element in $\mathbb{Z}_5$ such that this is true, a contradiction.
So if $y \in \mathbb{Z}[\sqrt{5}]$ and $N(y) = 4$, then $y$ is irreducible. Thus, 2 and $1 + \sqrt{5}$ are both
irreducible. We know that $2 \mid (1 + \sqrt{5})(1 - \sqrt{5}) = -4$, yet $2$ does not divide 
$1 + \sqrt{5}$ and $1 - \sqrt{5}$ (because if they did, then $2(a + b\sqrt{5}) = 1 + \sqrt{5}$ and $2(a + b\sqrt{5}) = 
1 - \sqrt{5}$, which means that $2a = 1$ and $2b = 1$ for both equality, but this can't be). Hence $2$ is not prime.
Furthermore, $1 + \sqrt{5} \mid 4 = 2^2$, so that $(1 + \sqrt{5})(a + b\sqrt{5}) = [(a + 5b) + (a + b)\sqrt{5}] = 2$.
Clearly $a + b = 0$, which means that $a = -b$, so that $a + 5b = -b + 5b = 4b = 2$, but this can't be. Thus
$1 + \sqrt{5}$ is not prime also.
\end{enumerate}

\item[19.2] (Subspace Test) Prove that a nonempty subset $U$ of a vector space $V$ over a field $F$ is a subspace of $V$ if, for every $u$ and $u'$ in $U$ and every $a$ in $F, u + u' \in U$ and $au \in U$.
\begin{enumerate}
\item[] 
\end{enumerate}

\item[19.3] Verify that the set in Example 6 is a subspace. Find a basis for this subspace. Is $\{x^2 + x + 1, x+5, 3\}$ a basis?
\begin{enumerate}
\item[] A simple basis should follow from the definition of the elements within the set, that is $\{1, x, x^2\}$. To see that $\{x^2 + x + 1, x+5, 3\}$ is in fact a basis, let $n, m, k \in \mathbb{R}$ and note that $n(x^2+x+1) + m(x+5) + k(3) = nx^2 + (n+m)x+[n+5m+3k]$. We get three equations from this; $n=0, n+m=0$ and $n+5m+3k=0$. Well $n=0$ and from the second equation we get $m=0$ and finally the last equation tells us that $k=0$, so $\{x^2 + x + 1, x+5, 3\}$ is linearly independent. We note further that $|\{x^2 + x + 1, x+5, 3\}| = 3$ which is the same size as the dimension and thus a basis.
\end{enumerate}

\item[19.5] Determine whether or not the set $\{(2, 1,0), (1,2,5), (7,-1,5)\}$ is linearly independent over $\mathbb{R}$.
\begin{enumerate}
\item[] $\left[ \begin{array}{ccc} 2 & -1 & 0 \\ 1 & 2 & 5 \\ 7 & -1 & 5 \end{array} \right]$
$\sim$ $\left[ \begin{array}{ccc} 1 & 2 & 5 \\ 0 & -5 & -10 \\ 0 & -15 & -30 \end{array} \right]$
$\sim$ $\left[ \begin{array}{ccc} 1 & 2 & 5 \\ 0 & 1 & 2 \\ 0 & 0 & 0 \end{array} \right]$
So not linearly independent.
\end{enumerate}

\item[19.6] Determine whether or not the set \[
\{ \left[ \begin{array}{cc} 2 & 1 \\ 1 & 0 \end{array} \right], 
\left[ \begin{array}{cc} 0 & 1 \\ 1 & 2 \end{array} \right], \left[ \begin{array}{cc} 1 & 1 \\ 1 & 1 \end{array} \right] \}
\]
is linearly independent over $\mathbb{Z}_5$.
\begin{enumerate}
\item[] $a \cdot \left[ \begin{array}{cc} 2 & 1 \\ 1 & 0 \end{array} \right] + 
b \cdot \left[ \begin{array}{cc} 0 & 1 \\ 1 & 2 \end{array} \right] + c \cdot \left[ \begin{array}{cc} 1 & 1 \\ 1 & 1 \end{array} \right] = \left[ \begin{array}{cc} 0 & 0 \\ 0 & 0 \end{array} \right]$ when $a = b = 2$ and $c = 1$, all of which are not zero. Thus not linearly independent over $\mathbb{Z}_5$.
\end{enumerate}

\item[19.7] If $\{u, v, w\}$ is a linearly independent subset of a vector space, show that $\{u, u+v, u+v+w\}$ is also linearly independent.
\begin{enumerate}
\item[] $au + b(u+v) + c(u+v+w) = (a+b+c)u + (b+c)v + cw$. Yet $u, v, w$ are linearly independent so it must be that $a+b+c=0, b+c=0$ and $c=0$. Well this implies that $b=0$ from the second equation and $a=0$ from the first equation so $\{u, u+v, u+v+w\}$ is linearly independent set of vectors.
\end{enumerate}

\item[19.8] If $\{v_1, \ldots, v_n\}$ is a linearly dependent set of vectors, prove that one of these vectors is a linear combination of the others.
\begin{enumerate}
\item[] Let's first note that if $a_1v_1 + \cdots + a_nv_n = 0$ and $\{ v_1, \ldots, v_n \}$ are linearly dependent then at least two coefficients, $a_i$ and $a_j$, must be nonzero since all but one coefficient having zero entries implies that the last coefficient is in fact zero. So then let $a_i \neq 0$, so that $
a_1v_1 + \cdots + a_nv_n = 0$ becomes $a_1v_1 + \cdots + a_nv_n = -a_iv_i$ but then $-a_i^{-1}$ exists since $a_i \neq 0$. Multiply by sides by $-a_i^{-1}$ to yield $(-a_i^{-1}a_1)v_1 + \cdots + (-a_i^{-1}a_n)v_n = v_i$. With at least one $a_j \neq 0$ on the LHS, we conclude that one of the vectors was a linear combination of the others.
\end{enumerate}

\item[19.13] Let $V$ be the set of all polynomials over $\mathbb{Q}$ of degree 2 together with the zero polynomial. Is $V$ a vector space over $\mathbb{Q}$?
\begin{enumerate}
\item[] Take $x^2 + 2, x^2 - 1 \in V$, then $(x^2 + 2) + (x^2 - 1) = 1 \not\in V$.
\end{enumerate}

\item[19.14] Let $V = \mathbb{R}^3$ and $W = \{(a,b,c) \in V \mid a^2+b^2=c^2\}$. Is $W$ a subspace of $V$? If so, what is its dimension?
\begin{enumerate}
\item[] We take two elements $(1,1,1),(3,4,5) \in W$ and note that $(1,1,1)+(3,4,5) = (4, 5, 6) \not\in W$ since $4^2 + 5^2 \neq 6^2$.
\end{enumerate}

\item[19.15] Let $V = \mathbb{R}^3$ and $W = \{(a,b,c) \in V \mid a+b=c\}$. Is $W$ a subspace of $V$? If so, what is its dimension?
\begin{enumerate}
\item[] Since $0 = 0 + 0$, then $(0,0,0) \in W$ and $W$ is nonempty. Now take \\
$(a,b,c), (a',b',c') \in W$ then $(a+a')+(b+b') = (a+b)+(a'+b') = c + c'$ so $(a,b,c)+(a',b',c') \in W$. Now take $x \in F$ and $(a,b,c) \in W$, then $xa + xb = x(a+b) = xc$ so $x(a,b,c) = (xa, xb, xc) \in W$.
\end{enumerate}

\item[19.18] Let $P = \{ (a,b,c) \mid a, b, c \in \mathbb{R}, a = 2b + 3c \}$. Prove that $P$ is a subspace of $\mathbb{R}^3$. Find a basis for $P$. Give a geometric description of $P$.
\begin{enumerate}
\item[] Since $0 = 2(0) + 3(0)$, then $(0,0,0) \in P$ and $P$ is not empty. Now let \\
$(a,b,c), (a',b',c') \in P$, then $2(b + b') + 3(c + c') = 2b + 3c + 2b' + 3c' = a + a'$ thus $(a,b,c)+(a',b',c') \in P$. Let $x \in \mathbb{R}$ and $(a,b,c) \in P$, then $2(xb) + 3(xc) = x(2b) + x(3c) = x(2b + 3c) = xa$ hence $x(a,b,c) = (xa, xb, xc) \in P$. We note that elements in $P$ are of the form $(2b + 3c, b, c)$, so we let $b = 0$ and we get $(3c, 0, c)$, now we let $c = 0$ to get $(2b, b, 0)$. This is a linearly independet set since $(0, 0, 0) = x(2b, b, 0) + y(3c, 0, c)$ implies that $x = y = 0$. A geometric description of this is a line passing through the origin in $\mathbb{R}^3$.
\end{enumerate}

\item[19.20] If $U$ is a proper subspace of a finite-dimensional vector space $V$, show that the dimension of $U$ is less than the dimension of $V$.
\begin{enumerate}
\item[] Let $\{v_1, \ldots, v_n \}$ be a basis for $V$ such that a subset is also a basis for $W$. Then since $W$ is a proper subspace there must be at least one element $v_i \in \{v_1, \ldots, v_n \}$ such that $v_i \not\in W$ otherwise $V = W$. Thus $|\{v_1, \ldots, v_n \}\setminus \{v_i\}| < |\{v_1, \ldots, v_n \}|$ so that the dimension of $W$ must be strictly less than that of dim $V$.
\end{enumerate}

\item[19.22] If $V$ is a vector space of dimension $n$ over the field $\mathbb{Z}_p$, how many elements are in $V$?
\begin{enumerate}
\item[] $n$ slots, $p$ choices. Hence we have $p^n$ elements.
\end{enumerate}

\item[19.24] Let $U$ and $W$ be subspaces of a vector space $V$. Show that $U \cap W$ is a subspace of $V$ and that $U + W = \{ u + w \mid u \in U, w \in W \}$ is a subspace of $V$.
\begin{enumerate}
\item[] Since $U$ and $W$ are both subspaces then both have the zero vector of $V$ and thus $U \cap W$ is nonempty. Let $a, b \in U \cap W$ and $x \in F$. This means that $a, b \in U$ and $a, b \in W$ but since $U$ and $W$ are both subspaces, it must follow that $a+b, xa \in U$ and $a+b, xa \in W$, hence $a+b, xa \in U \cap W$ Now since $0 \in W$ and $0 \in V$, then $0 + 0 = 0 \in W+V$ and $W+V$ is nonempty. We let $u + w, u' + w' \in U + W$, then $(u + w) + (u' + w') = (u + u') + (w + w')$ and we know that since $u, u' \in U$, then $u + u' \in U$ and $w, w' \in W$, then $w+ w' \in W$, so $(u + w) + (u' + w') \in U + W$. Let $x \in F$ and $u + w \in U + W$, then $a(u + w) = au + aw \in U + W$ since $au \in U$ whenever $u \in U$ and $aw \in W$ whenever $w \in W$.
\end{enumerate}

\item[19.25] If a vector space has one basis that contains infinitely many elements, prove that every basis contains infinitely many elements.
\begin{enumerate}
\item[] Suppose a vector space has a basis of infinitely many elements. Then we assume that it also has a basis of finitely many elements. Then by theorem 19.1, all bases have finitely many elements. But this contradicts that it has a basis with infinite elements. Thus, there does not exist a finite basis in this vector space.
\end{enumerate}

\item[19.26] Let $u = (2,3,1), v = (1,3,0)$, and $w = (2, -3, 3)$. Since $\frac{1}{2}u - \frac{2}{3}v - \frac{1}{6}w = (0,0,0)$, can we conclude that the set $\{ u, v, w\}$ is linearly dependent over $\mathbb{Z}_7$?
\begin{enumerate}
\item[] No since $\frac{1}{2}, -\frac{2}{3}, -\frac{1}{6} \not\in \mathbb{Z}_7$.
\end{enumerate}

\item[19.28] Let $T$ be a linear transformation from $V$ to $W$. Prove that the image of $V$ under $T$ is a subspace of $W$.
\begin{enumerate}
\item[] Since $V \neq \emptyset$, then $T(V) \neq \emptyset$. Now take $T(x), T(y) \in T(V)$, then $T(x) +T(y) = T(x+y)$ but since $x, y \in V$ it must also be that $x+y \in V$ and thus $T(x)+T(y) \in T(V)$. Now take $a \in F$, so that $aT(x) = T(ax)$, which we know since $x \in V$ that $ax \in V$, hence $aT(x) \in T(V)$.
\end{enumerate}

\item[19.29] Let $T$ be a linear transformation of a vector space $V$. Prove that $\{ v \in V \mid T(v) = 0\}$, the kernel of $T$, is a subspace of $V$.
\begin{enumerate}
\item[] Let $K = \{ v \in V \mid T(v) = 0\}$. We note that $T(0) = 0$, thus $K$ is nonempty. Now take $v, v' \in K$, then $v + v' \in K$ since $T(v + v') = T(v) +T(v') = 0 + 0 = 0$. Now take $a \in F$ and $v \in K$, so that $av \in K$ since $T(av) = aT(v) = a(0) = 0$.
\end{enumerate}

\item[20.28] For any prime $p$, find a field of characteristic $p$ that is not perfect.
\begin{enumerate}
\item[] Let us refer to example 15.12 where we see that $F = \mathbb{Z}_p(t)$ is an infinite
field of characteristic $p$. We claim that this is our field for any prime characteristic,
$p$, such that it is not a perfect field. We take note that $f(x) = x^p - t \in F[x]$ is irreducible,
which we know is irreducible by same argument as example 20.9 except we replace the 2 by a $p$ and
note that when we have $f(t^p) = tg(t^p)$ that $p$ divides the degree of one but not the other, in 
$\mathbb{Z}_p(t)$ and make use of theorem 20.8 to say that if $\mathbb{Z}_p(t)$ was indeed perfect
then $f(x)$ would not have multiple zeros. Yet, $f(x)$ does have multiple zeros 
since $f'(x) = 0$.
\end{enumerate}

\item[20.30] Show that $x^4 + x + 1$ over $\mathbb{Z}_2$ does not have any multiple zeros in
any field extensions of $\mathbb{Z}_2$.
\begin{enumerate}
\item[] Notice that the derivative of $x^4 + x + 1$ is just $1$, thus $x^4 + x + 1$ and it's
derivative will always be relatively prime. Hence $x^4 + x + 1$ over $\mathbb{Z}_2$ will never
have any multiple zeros in any field extension of $\mathbb{Z}_2$ by theorem 20.5.
\end{enumerate}

\item[20.31] Show that $x^{21} + 2x^8 + 1$ does not have multiple zeros in any extension
of $\mathbb{Z}_3$.
\begin{enumerate}
\item[] To show the fact that $x^{21} + 2x^8 + 1$ has no multiple zeros, it suffices to look 
at the derivative which is $21x^{20} + 16x^7 = x^7$. By theorem 20.5, if there is a factor
in common between $x^{21} + 2x^8 + 1$ and $x^7$, then we have multiple zeros, yet
$x^i$ does not divide $x^{21} + 2x^8 + 1$ for $i = 1, \ldots, 7$, hence there
can be no multiple zeros for $x^{21} + 2x^8 + 1$ in any extension field of $\mathbb{Z}_3$.
\end{enumerate}

\item[20.32] Show that $x^{21} + 2x^9 + 1$ has multiple zeros in some extension of 
$\mathbb{Z}_3$.
\begin{enumerate}
\item[] We take the derivative of $x^{21} + 2x^9 + 1$ and see that it is $21x^{20} + 18x^8 = 0$.
Thus any factor of $x^{21} + 2x^9 + 1$ will clearly divide 0, so $x^{21} + 2x^8 + 1$ has
multiple roots in some extension field of $\mathbb{Z}_3$.
\end{enumerate}

\item[20.33] Let $F$ be a field of characteristic $p \neq 0$. Show that the polynomial
$f(x) = x^{p^n} - x$ over $F$ has distinct zeros.
\begin{enumerate}
\item[] Since the characteristic of $F$ is $p$, then $f'(x) = p^nx^{p^n - 1} - 1 = -1$ and
thus $f(x)$ and $f'(x)$ are relatively prime, which means that there are no multiple zeros
by theorem 20.5.
\end{enumerate}

\item[21.8] Find teh degree and a basis for $\mathbb{Q}(\sqrt{3} + \sqrt{5})$ over $\mathbb{Q}(\sqrt{15})$. 
Find the degree and a basis for $\mathbb{Q}(\sqrt{2}, \sqrt[3]{2}, \sqrt[4]{2})$ over $\mathbb{Q}$.
\begin{enumerate}
\item[] We know by example 21.7 that $[\mathbb{Q}(\sqrt{3} + \sqrt{5}): \mathbb{Q}] = 4$.
With $[\mathbb{Q}(\sqrt{15}): \mathbb{Q}]$ clearly being 2, noting that
$\mathbb{Q}(\sqrt{15})$ is a subfield of $\mathbb{Q}(\sqrt{3} + \sqrt{5})$ then 
it follows by theorem 21.5 that $[\mathbb{Q}(\sqrt{3} + \sqrt{5}): \mathbb{Q}(\sqrt{15})] = 2$.
The basis is then $\{1, \sqrt{3} + \sqrt{5}\}$.\\
Firstly, we note that $\mathbb{Q}(\sqrt{2}, \sqrt[3]{2}, \sqrt[4]{2}) = 
\mathbb{Q}(\sqrt[3]{2}, \sqrt[4]{2})$ since $(\sqrt[4]{2})^2 = \sqrt{2}$. Now by theorem 21.5,
we have both 
\[
[\mathbb{Q}(\sqrt[3]{2}, \sqrt[4]{2}): \mathbb{Q}] = 
[\mathbb{Q}(\sqrt[3]{2}, \sqrt[4]{2}): \mathbb{Q}(\sqrt[3]{2})]
[\mathbb{Q}(\sqrt[3]{2}: \mathbb{Q}]
\] and
\[
[\mathbb{Q}(\sqrt[3]{2}, \sqrt[4]{2}): \mathbb{Q}] =
[\mathbb{Q}(\sqrt[3]{2}, \sqrt[4]{2}): \mathbb{Q}(\sqrt[4]{2})]
[\mathbb{Q}(\sqrt[4]{2}): \mathbb{Q}]
\]
with $[\mathbb{Q}(\sqrt[4]{2}): \mathbb{Q}] = 4$
and $[\mathbb{Q}(\sqrt[3]{2}): \mathbb{Q}] = 3$. Since 4 and 3 are relatively prime and both
must divide $[\mathbb{Q}(\sqrt[3]{2}, \sqrt[4]{2}): \mathbb{Q}]$, then 
$[\mathbb{Q}(\sqrt[3]{2}, \sqrt[4]{2}): \mathbb{Q}] \geq 12$. On the other hand, 
$[\mathbb{Q}(\sqrt[3]{2}, \sqrt[4]{2}): \mathbb{Q}(\sqrt[3]{2})]$ is at most 4 since
$\sqrt[4]{2}$ is a zero of $x^4 - 2 \in \mathbb{Q}(\sqrt[3]{2})[x]$. Therefore
\[
[\mathbb{Q}(\sqrt[3]{2}, \sqrt[4]{2}): \mathbb{Q}] =
[\mathbb{Q}(\sqrt[3]{2}, \sqrt[4]{2}): \mathbb{Q}(\sqrt[3]{2})]
[\mathbb{Q}(\sqrt[3]{2}: \mathbb{Q}] \leq 4 \cdot 3 = 12
\]
So $[\mathbb{Q}(\sqrt[3]{2}, \sqrt[4]{2}): \mathbb{Q}] = 
[\mathbb{Q}(\sqrt{2}, \sqrt[3]{2}, \sqrt[4]{2}): \mathbb{Q}] = 12$. \\
To help get the basis we can show that $\mathbb{Q}(\sqrt[3]{2}, \sqrt[4]{2}) = 
\mathbb{Q}(\sqrt[12]{2})$. So we note that $(\sqrt[12]{2})^4 = \sqrt[3]{2}$ and 
$(\sqrt[12]{2})^3 = \sqrt[4]{2}$ so $\mathbb{Q}(\sqrt[12]{2}) \subseteq 
\mathbb{Q}(\sqrt[3]{2}, \sqrt[4]{2})$, then with $[\mathbb{Q}(\sqrt[3]{2}, \sqrt[4]{2}): 
\mathbb{Q}] = 12$ and $[\mathbb{Q}(\sqrt[12]{2}): \mathbb{Q}] = 12$ we have 
$\mathbb{Q}(\sqrt[3]{2}, \sqrt[4]{2}) = \mathbb{Q}(\sqrt[12]{2})$. 
So the basis we want is 
$\{1, \sqrt[12]{2}, (\sqrt[12]{2})^2, (\sqrt[12]{2})^3, \ldots, (\sqrt[12]{2})^{11}\}$.
\end{enumerate}

\item[21.10] Let $a$ be a complex number that is algebraic over $\mathbb{Q}$ and let $p(x)$ denote
the minimal polynomial for $a$ over $\mathbb{Q}$. Show that $\sqrt{a}$ is algebraic over $\mathbb{Q}$.
\begin{enumerate}
\item[] Since $a$ is a complex number that is algebraic over $\mathbb{Q}$ and 
$p(x)$ the minimal polynomial for $a$ over $\mathbb{Q}$, then it suffices to consider
the polynomial $p(x^2)$ which we know is nonzero since $p(x)$ is nonzero, then we can
see that $p((\sqrt{a})^2) = p(a) = 0$ and thus $\sqrt{a}$ is algebraic over $\mathbb{Q}$.
\end{enumerate}

\item[21.14] Find the minimal polynomial for $\sqrt{-3} + \sqrt{2}$ over $\mathbb{Q}$.
\begin{enumerate}
\item[] Let $x = \sqrt{-3} + \sqrt{2}$, then 
\begin{eqnarray*}
x^2 &=& -3 + 2\sqrt{-6} + 2 \\
x^2 + 1 &=& 2\sqrt{-6} \\
(x^2 + 1)^2 &=& (2\sqrt{-6})^2 \\
x^4 + 2x^2 + 1 &=& 4(-6)
\end{eqnarray*}
So our minimal polynomial for $\sqrt{-3} + \sqrt{2}$ over $\mathbb{Q}$ is $x^4 + 2x^2 + 25$.
\end{enumerate}

\item[21.18] Suppose that $[E:\mathbb{Q}] = 2$. Show that there is an integer $d$ such that
$E = \mathbb{Q}(\sqrt{d})$ and $d$ is not divisible by the square of any prime.
\begin{enumerate}
\item[] Suppose that $[E : \mathbb{Q}] = 2$. Then to find exactly what $E$ is, we can
adjoin to $\mathbb{Q}$ the zeros of the minimal polynomial, $f(x)$, over $\mathbb{Q}$. 
The deg $f(x) = 2$, hence we can express $f(x)$ as $x^2 + bx + c$ 
where $b,c \in \mathbb{Q}$. To find the zeros we can use the quadratic formula, so we have
$(-b \pm \sqrt{b^2 - 4c})/2$. Since the only thing in the roots of $f(x)$ that isn't
in $\mathbb{Q}$, it suffices to adjoin $\sqrt{b^2 - 4c}$. We can further simplify this
by noting that $b^2 - 4c \in \mathbb{Q}$ since $b, c \in \mathbb{Q}$ so we can express
$\sqrt{b^2 - 4c} = \sqrt{(p/q)}$ where $p, q \in \mathbb{Z}, q \neq 0$ and the
gcd$(p, q) = 1$. But $\sqrt{p/q} = (\sqrt{p})/(\sqrt{q})$, in which we can rationalize the
denominator to get $1/q \cdot \sqrt{pq}$, since $1/q \in \mathbb{Q}$ with $\mathbb{Q}$
absorbing it by a previous homework problem and $pq = d \in \mathbb{Z}$, then $E = 
\mathbb{Q}(\sqrt{d})$ for some $d \in \mathbb{Z}$. Clearly, we note that $d$ cannot be
divisible by the square of a prime since otherwise $\mathbb{Q}(\sqrt{d}) = \mathbb{Q} = E$ and thus
contradicts that $[E : \mathbb{Q}] = 2$.
\end{enumerate}

\item[21.20] Let $E$ be a field extension of $F$. Show that $[E:F]$ is finite if and only if
$E = F(a_1, a_2, \ldots, a_n)$, where $a_1, a_2, \ldots, a_n$ are algebraic over $F$.
\begin{enumerate}
\item[] Suppose that $[E: F]$ is finite and thus by theorem 21.4 an algebraic extension over $F$.
Since $E$ is an extension field of $F$, then we know
$\exists a_1 \in E\setminus F$ (note $a_1$ must be algebraic) 
so that by theorem 21.5, we have $[E:F] = [E:F(a_1)][F(a_1):F]
\Rightarrow [F(a_1):F] \leq [E:F]$. If $[F(a_1):F] = [E:F]$ then $E = F(a_1)$ and
we're done. Otherwise we note that $[F(a_1): F] < [E:F]$  and use induction. So assume
true for $n - 1$, $[F(a_1, a_2, \ldots, a_{n-1}): F] < [E:F]$. Then 
$\exists a_n \in E\setminus F(a_1, a_2, \ldots, a_{n-1})$, so that theorem 21.5 implies that
$[F(a_1, a_2, \ldots, a_{n-1}, a_n): F] \leq [E:F]$. Since $[E:F]$ is finite, we know that
the induction must end and hence we have the result that $E = F(a_1, a_2, \ldots, a_n)$.
Conversely, suppose that $E = F(a_1, a_2, \ldots, a_n)$, where $a_1, a_2, \ldots, a_n$ are
algebraic over $F$. Let $E = F(a_1)$, then $[E:F] =$ deg $p_1(x)$, where
$p_1(x)$ is the minimal polynomial for $a_1$ over $F$, and is thus finite. 
Now assume true for $n-1$. Then we have by theorem 21.5, we have
$[F(a_1, a_2, \ldots, a_n): F] = [F(a_1, a_2, \ldots, a_n):F(a_1, a_2, \ldots, a_{n-1})]
[F(a_1, a_2, \ldots, a_{n-1}):F]$. But by the induction hypothesis 
$[F(a_1, a_2, \ldots, a_{n-1}):F]$ is finite and with 
$[F(a_1, a_2, \ldots, a_n):F(a_1, a_2, \ldots, a_{n-1})] =$ deg $p_n(x)$, where
$p_n(x)$ is the minimal polynomial for $a_n$ over $F(a_1, a_2, \ldots, a_{n-1})$, 
then $[F(a_1, a_2, \ldots, a_n): F]$ is finite.
\end{enumerate}

\item[21.21] If $\alpha$ and $\beta$ are real numbers and $\alpha$ and $\beta$ are transcendental
over $\mathbb{Q}$, show that either $\alpha \beta$ or $\alpha + \beta$ is also transcendental
over $\mathbb{Q}$.
\begin{enumerate}
\item[] Let $\alpha$ and $\beta$ be real numbers that are transcendental over $\mathbb{Q}$.
Now let us assume that $\alpha + \beta$ and $\alpha \beta$ are algebraic over $\mathbb{Q}$. 
Then we consider the polynomail $f(x) = x^2 - (\alpha + \beta)x + \alpha \beta$ which is 
algebraic since it has algebraic coefficients. Yet the litte part after the corollary of
theorem 21.7 tells us that the algebraic numbers over $\mathbb{Q}$ 
form an algebraically closed field and this would mean that $\alpha$ and 
$\beta$ are algebraic since they are roots of $f(x)$,
which we can see by $f(x) = (x - \alpha)(x - \beta)$. Hence a contradiction.
\end{enumerate}

\item[21.32] Let $f(x)$ and $g(x)$ be irreducible polynomials over a field $F$ and let $a$ and
$b$ belong to some extension $E$ of $F$. If $a$ is a zero of $f(x)$ and $b$ is a zero of
$g(x)$, show that $f(x)$ is irreducible over $F(b)$ if and only if $g(x)$ is irreducible over
$F(a)$.
\begin{enumerate}
\item[] Let deg $f(x) = n$ and deg $g(x) = m$ and 
suppose that $f(x)$ is irreducible over $F(b)$. Then $[F(a, b):F(b)] = n$ and 
by theorem 21.5, we have $[F(a, b):F] = [F(a, b):F(b)][F(b):F]$, yet
$[F(b):F] = m$. So $[F(a, b):F] = nm$ (in this case, not in general). Furthermore, 
$[F(a, b):F] = [F(a, b):F(a)][F(a):F]$, so we have $nm = [F(a, b):F(a)] \cdot n$ 
or namely that $[F(a, b):F(a)]$ must be $m$, which implies that $g(x)$ is 
irreducible over $F(a)$. The converse is done using exactly the same argument.
\end{enumerate}

\item[21.33] Let $\beta$ be a zero of $f(x) = x^5 + 2x + 4$ (see Example 8 in Chapter 17).
Show that none of $\sqrt{2}, \sqrt[3]{2}, \sqrt[4]{2}$ belongs to $\mathbb{Q}(\beta)$.
\begin{enumerate}
\item[] We know that by example 8 in chapter 17 that $f(x) = x^5 + 2x + 4$ is irreducible
over $\mathbb{Q}$. Let $\beta$ be a zero of $f(x)$ in some extension of $\mathbb{Q}$. For,
if so, then $\mathbb{Q} \subset \mathbb{Q}(\sqrt[3]{2}) \subseteq \mathbb{Q}(\beta)$ and
$5 = [\mathbb{Q}(\beta) : \mathbb{Q}] = 
[\mathbb{Q}(\beta) : \mathbb{Q}(\sqrt[3]{2})][\mathbb{Q}(\sqrt[3]{2}) : \mathbb{Q}]$ implies
that 3 divides 5. Thus $\sqrt[3]{2}$ is not an element of $\mathbb{Q}(\beta)$. By the exact
same reasoning, $\sqrt{2}$ and $\sqrt[4]{2}$ are also not in $\mathbb{Q}(\beta)$ since
5 is prime and cannot be divided by 2 and 4, respectively.
\end{enumerate}

\item[22.8] Determine the possible finite fields whose largest proper subfield is $GF(2^5)$.
\begin{enumerate}
\item[] Since we want the possible finite fields such that GF$(2^5)$ is the largest
possible proper subfield, we look at GF$(2^{5k})$ where $k \in \mathbb{N}$ and 
5 is the largest divisor of $k$. Hence, we get 
GF$(2^{10})$, GF$(2^{15})$, and GF$(2^{25})$ as the only possible finite
fields that satisfy our requirements.
\end{enumerate}

\item[22.15] Prove the uniqueness portoin of Theorem 22.3 using a group-theoretic argument.
\begin{enumerate}
\item[] We prove the uniquenss portion of theorem 22.3 by noticing, by theorem
22.2, that as a group under multiplication, the set of nonzero elements of GF$(p^n)$ is
isomorphic to $\mathbb{Z}_{p^n-1}$ and is thus cylic. Yet by the fundamental theorem of cyclic
groups, we know that any subgroup is unique. 
\end{enumerate}

\item[22.17] Construct a field of order 9 and carry out the analysis as in Example 1, including
the conversion table.
\begin{enumerate}
\item[] To construct the field of 9 elements that we want, we use the irreducible $x^2 + 1$
over $\mathbb{Z}_3$ just like example 17.11 to get $\mathbb{Z}_3[x]/\langle x^2 + 1\rangle$.
So, we can think of GF(9) as the set 
$F = \{ ax + b + \langle x^2 + 1\rangle \mid a, b \in \mathbb{Z}_3\}$, 
where addition is done as in $\mathbb{Z}_2[x]$, but
multiplication is done module $x^2 + 1$. For example, $x^4 + 2x^2 + x + 1 = x$ since the
remainder upon dividing $x^4 + 2x^2 + x + 1$ by $x^2 + 1$ in $\mathbb{Z}_3[x]$ is $x$. An easier
way to perform the same calculation is to observe that in this context $x^2 + 1$ is 0, so 
$x^2 = 2$. Thus $x^4 + 2x^2 + x + 1 = 4 + 4 + x + 1 = 9 + x = x$. Another way to simplify the
multiplication process is to make use of the fact that the nonzero elements of GF(9) form
a cyclic group of order 8. To take advantage of this, we must first find a generator of this group.
After some computation, we find that $x + 1$ is our generator. So we may think of GF(9) as
the set $\{0, 1, x + 1, (x + 1)^2, \ldots, (x + 1)^7 \}$, where $x^8 = 1$. This makes multiplication
in $F$ trivial, but, unfortunately, it makes addition more difficult. Yet, if we write the elements
of $F^*$ in the additive form $ax^3 + bx^2 + cx + d$, then addition is easy and multiplication
is hard. So let's do like the book and get the best of both worlds by using the relation $x^2 = 2$ 
and make a two-way conversion table. \\
\begin{tabular}{cc|cc}
\hline
Multiplicative Form&Additive Form&Additive Form&Multiplicative Form\\
$1$ & $1$ & $1$ & $1$ \\
$x + 1$ & $x + 1$ & $2$ & $(x + 1)^4$ \\
$(x + 1)^2$ & $2x$ & $x$ & $(x + 1)^6$ \\
$(x + 1)^3$ & $2x + 1$ & $x + 1$ & $x + 1$ \\
$(x + 1)^4$ & $2$ & $x + 2$ & $(x + 1)^7$ \\
$(x + 1)^5$ & $2x + 2$ & $2x$ & $(x + 1)^2$ \\
$(x + 1)^6$ & $x$ & $2x + 1$ & $(x + 1)^3$ \\
$(x + 1)^7$ & $x + 2$ & $2x + 2$ & $(x + 1)^5$
\end{tabular}
\end{enumerate}

\item[22.22] Draw the subfield lattices of GF$(3^{18})$ and of GF$(2^{30})$.
\begin{enumerate}
\item[] 
\begin{displaymath}
\xymatrix{
&&&&\\
&&& GF(3^{18}) & \\
&&
GF(3^6) \ar@{-}[d] \ar@{-}[ur] &
GF(3^9) \ar@{-}[u] & \\
&
GF(3^2) \ar@{-}[ur] &
GF(3^3) \ar@{-}[ur] \ar@{-}[u] &
& 
\\
&GF(3) \ar@{-}[u] \ar@{-}[ur]&& &\\
}
\end{displaymath}

\begin{displaymath}
\xymatrix{
&&&&\\
&&GF(2^{30})&& \\
&
GF(2^{10}) \ar@{-}[ur]
&
GF(2^6) \ar@{-}[u] \ar@{-}[dl] \ar@{-}[dr]
&
GF(2^{15}) \ar@{-}[ul] & \\
&
GF(2^2) \ar@{-}[u] &
GF(2^5) \ar@{-}[ul] \ar@{-}[ur]
& GF(2^3) \ar@{-}[u]  & \\
&&GF(2) \ar@{-}[u] \ar@{-}[ul] \ar@{-}[ur]& &\\
}
\end{displaymath}
\end{enumerate}

\item[22.25] Suppose that $p$ is a prime and $p \neq 2$. Let $a$ be a nonsquare in GF$(p)$ 
(that is, $a$ does not have the form $b^2$ for any $b$ in GF$(p)$. Show that $a$ is a 
non-square in GF$(p^n)$ if $n$ is odd and that $a$ is a square in GF$(p^n)$ if $n$ is even.
\begin{enumerate}
\item[] Suppose that $p$ is prime and $p \neq 2$. Let $a$ be a nonsquare in GF($p$). Then
we consider the irreducible polynomial $f(x) = x^2 - a$. Yet, we know that
GF$(p)[x]/\langle f(x)\rangle$ has order $p^2$ and $f(x)$ has a zero in it. 
So, we note that $a$ must be a square in GF$(p^2)$ and apply theorem 22.3 to get the result
that since $2$ divides any even power of $p$, then $a$ will be square when it is in 
GF($p^n$) and $n$ is even. It consequently follows that $a$ cannot be square in 
GF($p^n$) when $n$ is odd, since $2$ does not divide $n$ in this case.
\end{enumerate}

\item[22.26] Let $f(x)$ be a cubic irreducible over $\mathbb{Z}_p$, where $p$ is a prime. Prove
that the splitting field of $f(x)$ over $\mathbb{Z}_p$ has order $p^3$ or $p^6$.
\begin{enumerate}
\item[] So since we want $f(x)$ to split in some extension field of $\mathbb{Z}_p = F$, we can then
express $f(x)$ as $(x - a_1)(x - a_2)(x - a_3)$ in the splitting field. Well, it's clear that
$a_1, a_2, a_3$ must be algebraic over $F$ since they are zeros to $f(x)$. Now by
theorem 21.1, we know that $F(a_1) \cong F[x]/\langle f(x)\rangle$. Yet, we know that $F[x]/\langle
f(x)\rangle$ has order $p^3$, so $F(a_1)$ must also have order $p^3$. If, $f(x)$ splits in this field,
then we're done. Otherwise, we need to adjoin one of the other roots, so without loss of generality,
let us choose $a_2$. Now we note what we actually want to split in the extension field of 
$F(a_1)$ is $g(x)$ such that $g(x) = f(x)/(x - a_1)$ which is clearly a polynomial
of degree 2. Thus, again by theorem 21.1, we get $F(a_1, a_2) = F(a_1)(a_2) \cong
F(a_1)[x]/\langle g(x)\rangle$, which is of order $(p^3)^2 = p^6$. It is quite clear that if the
latter case is needed to find a splitting field for $f(x)$, then the third root is automatically 
in that field since we broke down a quadratic polynomial into two linear factors.
\end{enumerate}

\item[22.30] Show that no finite field is algebraically closed.
\begin{enumerate}
\item[] Suppose that $F$ is a finite field. Now for the sake of contradiction, assume
that $F$ is algebraically closed. Now we consider where $0, 1 \in F$ (which we know must be there
since they are the additive and multiplicative identity) get mapped to when we look at the
function $f: F \rightarrow F$ given by $f(x) = x^2 - x$. Obviously, $f(0) = f(1) = 0$ so that
$f$ is not onto and not one-to-one. Thus, it must be that for any $x \in F, x^2 - x \neq y$ where
$y \in F$. Hence $x^2 - x - y$ has no root in $F$ which gives us our contradiction.
\end{enumerate}

\item[23.4] If $a$ and $b (b \neq 0)$ are constructible numbers, give a geometric proof that
$a/b$ is constructible.
\begin{enumerate}
\item[] 
Suppose that $a$ and $b$ are constructible numbers with $b \neq 0$, then we use the
figure below

\begin{displaymath}
\xymatrix{
&&&&\bullet \ar@/_2pc/[ddddllll]&\\
&a&&& &\\
&&\bullet \ar@{-}[uurr]&& &\\
&&&& &\\
\bullet \ar@/^2pc/[uuuurrrr] \ar@{-}[uurr]^d \ar@{-}[rr]^1 \ar@/_2pc/[rrrr] &&
\bullet \ar@{-}[rr] \ar@{-}[uu] &&\ar@{-}[uuuu]\bullet \ar@/^2pc/[llll] &\\
&&b&& &\\
}
\end{displaymath}
\[ \]
\[ \]
So, we have two similar right triangles; 
the smaller one with a hypotenuse of length $d$ and a 
bottom side length of 1 and the bigger one with a
hypotenuse of length $a$ and a bottom side length of $b$. Then by proportionality,
$1/b = d/a$. Hence $db = 1a = a$, so $d = a/b$ and thus $a/b$ is contructible. 
\end{enumerate}


\item[Extra 1] Let $p$ be prime, let $\mathbb{F}_p = \mathbb{Z}/p\mathbb{Z}$, and let $G$ be the following subgroup of $GL(2, \mathbb{F}_p)$:
\[
G = \left\{
\left(
\begin{array}{cc}
1 & a \\ 0 & 1
\end{array}
\right) \mid a \in \mathbb{F}_p
\right\} .
\]
Let $X = \mathbb{F}^2_p$ be the set of $2 \times 1$ vectors with entries from $\mathbb{F}_p$, that is,
\[
\mathbb{F}^2_p = \left\{
\left(
\begin{array}{c}
x \\ y
\end{array}
\right) \mid x, y \in \mathbb{F}_p
\right\} .
\]
\begin{enumerate}
\item[a)] For $g \in G$ and $v \in X$, define $g \cdot v = gv$, where $gv$ is matrix-vector multiplication. Show that this satisfies definition 2 of a group action. \\
It suffices to show (i) and (ii) of definition 2. So let for any 
$\left( \begin{array}{cc} 1 & a \\ 0 & 1\end{array}\right), \left( \begin{array}{cc} 1 & b \\ 0 & 1\end{array}
\right) \in G$ and $\left( \begin{array}{c} x \\ y\end{array}\right) \in X$, we have
\[
\left[ \left( \begin{array}{cc} 1 & a \\ 0 & 1\end{array}\right) \left( \begin{array}{cc} 1 & b \\ 0 & 1\end{array} \right)
\right] \left( \begin{array}{c} x \\ y\end{array}\right) = 
\left( \begin{array}{cc} 1 & a + b \\ 0 & 1\end{array}\right) \left( \begin{array}{c} x \\ y\end{array}\right)
= \left( \begin{array}{c} x + (a + b)y \\ y\end{array}\right)
\]
and 
\[
\left( \begin{array}{cc} 1 & a \\ 0 & 1\end{array}\right) \left[ \left( \begin{array}{cc} 1 & b \\ 0 & 1\end{array} \right)
\left( \begin{array}{c} x \\ y\end{array}\right) \right] = 
\left( \begin{array}{cc} 1 & a \\ 0 & 1\end{array}\right) \left( \begin{array}{c} x + by \\ y\end{array}\right)
= \left( \begin{array}{c} x + (a + b)y \\ y\end{array}\right)
\]
so (i) is satisfied. Now, we note that the identity of $G$ is $\left( \begin{array}{cc} 1 & 0 \\ 0 & 1\end{array} \right)$.
Then
\[
\left( \begin{array}{cc} 1 & 0 \\ 0 & 1\end{array} \right) \left( \begin{array}{c} x \\ y\end{array}\right) = 
\left( \begin{array}{c} x \\ y\end{array}\right)
\] and we're done.
\item[b)] Show that for every $v \in X$, we have $|\mbox{orb}_G(v)| = 1$ or $p$. \\
By the orbit-stabilizer lemma, we know that $|G| = |\mbox{stab}_G(v)||\mbox{orb}_G(v)|$ for any $v \in X$.
Once we show that $G \cong \mathbb{F}_p$, then we know that $|G| = p$ and our required result following. 
That is that $|\mbox{stab}_G(v)| = 1$ or $p$ and $|\mbox{orb}_G(v)| = 1$ or $p$. So
let $\phi: G \rightarrow \mathbb{F}_p$ by $\left( \begin{array}{cc} 1 & a \\ 0 & 1 \end{array} \right) \rightarrow 
a$. $\phi$ is clearly a bijection and with $\phi(\left( \begin{array}{cc} 1 & a \\ 0 & 1 \end{array} \right)
\left( \begin{array}{cc} 1 & b \\ 0 & 1 \end{array} \right)) = \phi(
\left( \begin{array}{cc} 1 & a + b \\ 0 & 1 \end{array} \right) ) = a + b = 
\phi(\left( \begin{array}{cc} 1 & a \\ 0 & 1 \end{array} \right) ) + \phi(\left( \begin{array}{cc} 1 & b \\ 0 & 1 \end{array} \right)
)$, we are done.
\item[c)] If $v = \left( \begin{array}{c} x \\ y \end{array} \right)$, show that stab$_G(v) = G$
if and only if $y = 0$. \\
Suppose that stab$_G(v) = G$, then $\forall g \in G, g \cdot v = v$. So,
$\left( \begin{array}{cc} 1 & a \\ 0 & 1 \end{array} \right) \cdot \left( \begin{array}{c} x \\ y \end{array} \right) = 
\left( \begin{array}{c} x + ya \\ y \end{array} \right)$ and this must equal 
$\left( \begin{array}{c} x \\ y \end{array} \right)$, where it becomes quite clear that this is only possible when $y = 0$.
Conversely, suppose that $y = 0$. Then for any element, $\left( \begin{array}{cc} 1 & a \\ 0 & 1 \end{array} \right) \in 
G$, $\left( \begin{array}{cc} 1 & a \\ 0 & 1 \end{array} \right) \cdot \left( \begin{array}{c} x \\ 0 \end{array} \right) = 
\left( \begin{array}{c} x \\ 0 \end{array} \right)$, but this is just the definition of the stabilizer and thus
stab$_G(v) = G$. 
\end{enumerate}

\item[Extra 2] Let $G$ be a group which acts on a set $X$, and for every $x, y \in X$, define
$x \sim y$ if there is a $g \in G$ such that $g \cdot x = y$. Prove that $\sim$ is an equivalence relation on $X$ and the equivalence class of $x \in X$ is orb$_G(x)$.
\begin{enumerate}
\item[] Reflexive: $x \sim x$ is true since $e \cdot x = x$ and $e \in G$ because it is a group. \\
Symmetric: If $x \sim y$, then $g \cdot x = y$ for some $g \in G$, yet multiplying on the left to both
sides by $g^{-1}$ yields $x = g^{-1} \cdot y$. So $y \sim x$ since $g^{-1} \in G$ because $G$ is a group. \\
Transitive: If $x \sim y$ and $y \sim z$, then $g \cdot x = y$ and $h \cdot y = z$ for some $g, h \in G$. 
But $y = h^{-1} \cdot z$, which means $g \cdot x = h^{-1} \cdot z$. Thus $hg \cdot x = z$, with $hg \in G$ because
it is a group, and $x \sim z$. 

Hence $\sim$ is an equivalence relation.
We know that orb$_G(x) = \{ g \cdot x \mid g \in G \}$. Yet, we defined the equivalence relation in exactly
the same manner. Thus orb$_G(x) = [x]$.
\end{enumerate}

\item[Extra 3] Let $G$ be a finite group such that $|G| = p^k$ , for some prime $p$ and integer
$k \geq 1$. Use Theorem 4 to show that the center of $G$ is nontrivial, that is,
that $Z(G)$ contains more than just the identity element $e$.
\begin{enumerate}
\item[] Since $|G| = p^k$, then $p \mid |G|$. Thus by the class formula, $p$ must divide 
$|Z(G)| + \sum_{a \in A}[G : C_G(a)]$. With the terms in the summation all being proper divisors of $|G|$, then
it must follow that $p \mid Z(G)$.Thus $|Z(G)| \neq 1$.
\end{enumerate}

\item[Extra 4] Use Problem 3 above and Theorem 9.3 in Gallian to show that any
group of order $p^2$, where $p$ is prime, is abelian
\begin{enumerate}
\item[] Let $G$ be of order $p^2$ where $p$ is prime. Then we know by Problem 3 and the fact that $Z(G)$ is a subgroup
and by Lagrange's, $|Z(G)|$ must divide $|G|$. So the order of $Z(G)$ is either $p$ or $p^2$. If $|Z(G)| = p^2$, then
$|G/Z(G)| = 1$, in which case $G/Z(G)$ must be Abelian. Otherwise, $|G/Z(G)| = p$, in which we know that $G/Z(G)$ must be
cyclic and, by theorem 9.3, Abelian.
\end{enumerate}

\end{enumerate}

\end{document}
