\documentclass{letter}
\usepackage{geometry,amsmath,amssymb,bbm}
\usepackage{fullpage}
\usepackage{calrsfs}
\geometry{letterpaper}

%%%%%%%%%% Start TeXmacs macros
\newcommand{\nin}{\not\in}
\newcommand{\tmem}[1]{{\em #1\/}}
\newcommand{\tmop}[1]{\ensuremath{\operatorname{#1}}}
%%%%%%%%%% End TeXmacs macros

\begin{document}

Algebra by Larry C. Grove

I.12.1 If $G$ is a group and $f : G \rightarrow G$ is defined by $f (x) = x^{-
1}$, all $x \in G$, show that $f$ is a homomorphism if and only if $G$ is
abelian.

{\tmem{Solution:}} Suppose $f$ is a homomorphism. Then $f (x y) = f (x) f
(y)$, for all $x, y \in G$. Furthermore, $x^{- 1} y^{- 1} = f (x) f (y) = f (x
y) = (x y)^{- 1}$. Multiplying on the left by $(y x)$ and then on the right by
$(x y)$ yields $y x = x y$. Thus $G$ is abelian. Conversely, suppose that $G$
is abelian. Then $f (x y) = (x y)^{- 1} = x^{- 1} y^{- 1} = f (x) f (y)$ for
all $x, y \in G$ and $f$ is a homomorphism.

I.12.2: If a group $G$ has a unique element $x$ of order 2 show that $x \in Z
(G)$.

{\tmem{Solution:}} First, let $e$ be the identity in $G$ and $y$ any element
in $G$. We note that if $y^{- 1} x y = e$, then by multiplying on the left by
$y$ and on the right by $y^{- 1}$, $x = e$ which is a contradiction since $x$
has order 2. So $|y^{- 1} x y| > 1$. Now we show that $|y^{- 1} x y| = 2$:

$(y^{- 1} x y)^2 = (y^{- 1} x y) (y^{- 1} x y) = y^{- 1} x (y y^{- 1}) x y =
y^{- 1} x (e) x y = y^{- 1} x^2 y = y^{- 1} e y = y^{- 1} y = e$.

Since $x$ is the only element of order 2, it must be that $x = y^{- 1} x y$
and multiplying on the left by $y$, we see that $y x = x y$ for all $y \in G$
and hence $x \in Z (G)$.

I.12.6: Suppose $G$ is a group, $H \leq G$ and $K \leq G$. Show that $H \cup
K$ is not a group unless $H \leq K$ or $K \leq H$.

{\tmem{Solution:}} If $H \leq K$, then $H \cup K = K$, a subgroup. Also if $K
\leq H$, then $H \cup K = H$, a subgroup. So assume now that $H \nleq K$ and
$K \nleq H$. Then we know there exists an $h \in H \setminus K$, $k \in K
\setminus H$ and thus $h k \nin H$ and $h k \nin K$. Now $h, k \in H \cup K$
but $h k \nin H \cup K$, so we can conclude that $H \cup K$ is not a subgroup
since it's not closed under the binary operation of $G$.

I.12.9: If $A, B \leq G$ and both $[G : A]$ and $[G : B]$ are finite show that
$[G : A \cap B] \leq [G : A] [G : B]$, with equality if and only if $G = A B$.

{\tmem{Solution:}} We let $f$ be the mapping from $G / (A \cap B) \rightarrow
G / A \times G / B$, where $\times$ is the Cartesian product, and $f (g (A
\cap B)) = (g A, g B)$. To show that $f$ is well defined, let $g (A \cap B) =
h (A \cap B)$, then $h^{- 1} g (A \cap B) = A \cap B$ or more importantly
$h^{- 1} g \in A \cap B$, but this also means $h^{- 1} g \in A$ and $h^{- 1} g
\in B$, hence $g A = h A$ and $g B = h B$. So $(g A, g B) = (h A, h B)$ and $f
(g (A \cap B) = f (h (A \cap B))$. Injectivity of $f$ can be shown by simply
reversing the process of showing that $f$ is well defined. So $f$ is a
bijection to $f (G / (A \cap B)) \subseteq G / A \times G / B$ and thus the
$[G : A \cap B] = |G / (A \cap B) | \leq |G / A \times G / B| = |G / A| |G /
B| = [G : A] [G : B]$. We note now that $[G : A \cap B]$ is finite when $[G :
A]$ and $[G : B]$ are both finite, so that $[G : A \cap B] = [G : A] [A : A
\cap B]$. Now the equality holds if and only if $[A : A \cap B] = [G : B]$
which means we have $g \in A B$ for all $g \in G$ and with $A B \subseteq G$,
we have $G = A B$.

I.12.12: If $A, B \leq G$ and $y \in G$ define the $(A, B)$-double coset $A y
B =\{a y b : a \in A, b \in B\}$. Show that $G$ is the disjoint union of its
$(A, B)$-double cosets. Show that $|A y B| = [A^y : A^y \cap B] |B|$ if $A$
and $B$ are finite.

{\tmem{Solution:}} We let $a y b = c x d \in A y B$. Then $x^{- 1} c^{- 1} a y
= d b^{- 1}$, where $c^{- 1} a \in A$ since $A$ is a subgroup and $d b^{- 1}
\in B$ since $B$ is a subgroup. Thus we can say $x \sim y$, $x$ is congruent
to $y$, if and only if $x^{- 1} a y \in B, a \in A.$ We show that this is an
equivalence relation on $G$.

Reflexive: We know $e \in A \leq G$, then $x^{- 1} e x = x^{- 1} x = e \in B
\leq G$. So $x \sim x$.

Symmetric: If $x \sim y$, then $x^{- 1} a y = b$ for some $b \in B$. We have
$y^{- 1} a^{- 1} x = (x^{- 1} a y)^{- 1} = b^{- 1} \in B$ since $B$ is a
subgroup. Moreover $a^{- 1} \in A$ since $A$ is a subgroup and thus $y \sim
x$.

Transitive: If $x \sim y$ and $y \sim z$, then $x^{- 1} a y = b$ and $y^{- 1}
a' z = b'$ for some $b, b' \in B$. Now $b b' = x^{- 1} a y y^{- 1} a' z = x^{-
1} a a' z \in B$. Since $a a' \in A$ whenever $a, a' \in A$ since $A$ is a
subgroup and $b b' \in B$ whenever $b, b' \in B$ since $B$ is a subgroup, we
have $x \sim z$.

With this equivalence relation partitioning $G$ and the equivalence classes
$(A, B)$-double cosets, then $G$ is a disjoint union of its $(A, B)$-double
cosets.

Moreover an $(A, B)$-double coset can easily be seen as a union of left $B$
cosets with the form $a y B$, for all $a \in A$. Suppose that $A$ and $B$ are
finite. This implies that $| A y B| = [A y B : B] |B |$. It's simple to show
that $a^{- 1} H a \leq G$ when $H \leq G$ and $a \in G$, so let $a^{- 1} h a,
a^{- 1} h' a \in a^{- 1} H a$, then $(a^{- 1} h a)^{- 1} = a^{- 1} h^{- 1} a$
and thus $(a^{- 1} h' a) (a^{- 1} h a)^{- 1} = a^{- 1} h' a a^{- 1} h^{- 1} a
= a^{- 1} h' h^{- 1} a \in a^{- 1} H a$ since $h' h^{- 1} \in H$ whenever $h,
h' \in H.$

We now create the mapping $f : A y B \rightarrow y^{- 1} A y B$ by $f (a y b)
= y^{- 1} a y b$. If $f (a y b) = f (a' y b)$, then $y^{- 1} a y b = y^{- 1}
a' y b$ and multiplication by $y$ on the left, we get $a y b = a' y b'$ and
$f$ is one-to-one with well defined going in the other direction. Let $y^{- 1}
a y b \in y^{- 1} A y B$, then $f (a y b) = y^{- 1} a y b$, $a y b \in A y B$,
so $f$ is onto. Thus $f$ is a bijection between $A y B \tmop{and} y^{- 1} A y
B$, so $|A y B| = | y^{- 1} A y B|$. It follows that since $y^{- 1} A y, B
\leq G$, by exercise 2.2, we have $|y^{- 1} A y B| = \frac{|y^{- 1} A y| |
B|}{|y^{- 1} A y \cap B|}$. Hence

$[A y B : B] = \frac{|A y B|}{|B|} = \frac{|y^{- 1} A y B|}{|B|} = \frac{|y^{-
1} A y| |B|}{|y^{- 1} A y \cap B||B|} = \frac{|y^{- 1} A y|}{|y^{- 1} A y \cap
B|} = [A^y : A^y \cap B]$. Using the equalities, we see that $[A^y : A^y \cap
B] = \frac{|A y B|}{|B|}$, our desired result.

I.12.13: Suppose $G$ is a permutation group on a set $S$, with $|S| > 1$. Say
that $G$ is doubly transitive on $S$ if given any $(a, b), (c, d) \in S \times
S$, with $a = b$ if and only if $c = d$, then $x a = c$ and $x b = d$ for some
$x \in G$.

(1) If $G$ is transitive on $S$ show that $G$ is doubly transitive if and only
if $H = \tmop{Stab}_G (s)$ is transitive on $S \setminus \{s\}$ for each $s
\in S$.

{\tmem{Solution:}} We have that $G$ is transitive on $S$. Now suppose that $G$
is doubly transitive on $S$. First off, we know that for all $s \in S$ that
there exists a $g \in G$ such that $s = g t$ for some $t \in S$. Now let $h
\in \tmop{Stab}_G (s)$ so that we have $h s = g t$. But this means that $t =
g^{- 1} h s = g^{- 1} h g t$, so that $g^{- 1} h g \in \tmop{Stab}_G (t)
\Rightarrow h \in g \tmop{Stab}_G (t) g^{- 1}$. Going in the reverse
direction, we conclude that $\tmop{Stab}_G (s) = g \tmop{Stab}_G (t) g^{- 1}$.
Letting $x, y \in S$ and $x \neq s \neq y$ we see that $g^{- 1} x \neq g^{- 1}
s \neq g^{- 1} y$ or better seen as $g^{- 1} x \neq t$ and $g^{- 1} y \neq t$.
Thus there must exist an $h \in \tmop{Stab}_G (t)$ such that $h g^{- 1} x =
g^{- 1} y \Rightarrow g h g^{- 1} x = y$ by our supposition. But $g h g^{- 1}
\in \tmop{Stab}_G (s)$ by above and hence $\tmop{Stab}_G (s)$ acts
transitively on $S \setminus \{s\}$.

Conversely, suppose that $\tmop{Stab}_G (s)$ is transitive on $S \setminus
\{s\}$ for all $s \in S$. Now let $(a, b), (c, d) \in S \times S$. We use $g
\in \tmop{Stab}_G (a)$ for $g (a, b) = (a, d)$ and then $h \in \tmop{Stab}_G
(d)$ so that $h (a, d) = (c, d)$ and $G$ is doubly transitive on $S$ for $h g
\in G$ except when $a = d$ because then method just described won't work. So
we do something different in this case. Let $e \in S$ so that $a \neq e \neq
b$. Then we use $g \in \tmop{Stab}_G (a)$ so that $g (a, b) = (a, e)$,
followed by $h \in \tmop{Stab}_G (e)$ for $h (a, e) = (c, e)$ and lastly $f
\in \tmop{Stab}_G (c)$ to yield $f (c, e) = (c, d)$, hence $f g h \in G$ and
we filled our hole.

(2) If $G$ is doubly transitive on $S$ and $|S| = n$ show that $n (n - 1) |
|G|$.

{\tmem{Solution:}} We first note by letting $s \in S$ that by the
orbit-stabilizer theorem $|S| = [G : \tmop{Stab}_G (s)]$. Let $s \in S$ and $t
\in S \setminus \{s\}$. We know that $\tmop{Stab}_G (s)$ acts transitively on
$S \setminus \{s\}$ by above, so $|S \setminus \{s\}| = [\tmop{Stab}_G (s) :
\tmop{Stab}_G (t) \cap \tmop{Stab}_G (s)] = n - 1$. Hence $[G : \tmop{Stab}_G
(s)] [\tmop{Stab}_G (s) : \tmop{Stab}_G (t) \cap \tmop{Stab}_G (s)] |
\tmop{Stab}_G (t) \cap \tmop{Stab}_G (s) | = n (n - 1) | \tmop{Stab}_G (t)
\cap \tmop{Stab}_G (s) | = |G|$ by Lagrange's theorem and we have $n (n - 1)$
divides $|G|$.

I.12.20: Let $T$ be the set of $n - 1$ successive transpositions $(12), (23),
(34), \ldots, (n - 1 n)$ in $S_n$. Show that $\langle T \rangle = S_n$.

{\tmem{Solution:}} We use Proposition 3.1, to note that any permutation$,
\sigma \in S_n$, can be expressed as a product of disjoint cycles. Let $(a_1
a_2 \cdots a_n)$ be one of those cycles. Then we note that $(a_1 a_2 \cdots
a_n) = (1 a_1) (1 a_n) (1 a_{n - 1}) \cdots (1 a_1)$ by direction computation.
So any transposition expressing a cycle is of the form $(1 k)$, $1 \leq k \leq
n$. Since the transpositions in $T$ are of the form $(k - 1 k)$, then in
general $(k - 2 k - 1) (k - 1 k) (k - 2 k - 1) = (k - 2 k)$, we can express
$(1 k) = (12) \cdots (k - 2 k - 1) (k - 1 k) (k - 2 k - 1) \cdots (12)$ or
more clearly a product of transpositions contained in $T$. Starting with
$\sigma$ and decomposing to a product of transpositions in $T$ shows us that
$S_n \subseteq \langle T \rangle$ and starting with a product of
transpositions in $T$ and building up to a permutation in $S_n$ verifies that
$\langle T \rangle \subseteq S_n$. Hence $\langle T \rangle = S_n$.

I.12.22: Suppose $H \leq S_n$ but $H \nleq A_n$. Show that $[H : A_n \cap H] =
2$.

{\tmem{Solution:}} Since $H \nleq A_n$, then $H$ must contain an odd
permutation $\sigma$. Moreover, $H$ must contain the identity so that $H A_n$
contains both $A_n$ and $\sigma A_n$. But $S_n = A_n \cup \sigma A_n$, so $H
A_n = S_n$. Now we can apply exercise 2.2, since $A_n \tmop{and} H$ are
finite, $|H A_n | = \frac{|H| |A_n |}{|H \cap A_n |} = |S_n | \Leftrightarrow
\frac{|H|}{|H \cap A_n]} = \frac{|S_n |}{|A_n |} = 2 = [H : H \cap A_n]$.

I.12.31 Suppose $G$ is a finite group, $H \vartriangleleft G$, and $P$ is a
$p$-Sylow subgroup of $H$. Show that $G = N_G (P) H$.

{\tmem{Solution:}} With $N_G (P)$ and $N$ both subgroups of $G$, it is clear
that $G \supseteq N_G (P) N =\{x n | x \in N_G (P), n \in N\}$ by closure. Now
let $g \in G$, by the normalcy of $N$ in $G$, we have $g P g^{- 1} \subseteq g
N g^{- 1} = N$. This means that $g P g^{- 1}$ is also a Sylow $p$-subgroup in
$N$. Furthermore by Sylow's second theorem $g P g^{- 1}$ and $P$ are
conjugates and we have for some $n \in N, n g P g^{- 1} n^{- 1} = P$. Hence $n
g \in N_G (P)$ from which we get $g \in N_G (P) n^{- 1}$. Yet $N_G (P) n^{- 1}
\subseteq N_G (P) N$, so $G \subseteq N_G (P) N$. Both inclusions are true,
thus $G = N_G (P) N$.

I.12.35: Show that there are no simple groups of order 104, 176, 182, 312.

{\tmem{Solution:}} Let $|G| = 104 = 2^3 \cdot 13$ then by the third Sylow
theorem, $G$ has a unique Sylow 13-subgroup and by the corollary (pg 21) to
the second Sylow theorem, that Sylow 13-subgroup is normal. Since $G$ has a
nontrivial proper normal subgroup, $G$ cannot be simple. The exact same
argument is used to show that groups of order 176, 182, 312 cannot be simple
once we note that they have a unique Sylow 11-subgroup, 7-subgroup, and
13-subgroup, respectively.

I.12.44: If $G$ is a group and $x \in G$ define the inner automorphisms $f_x$
by setting $f_x (y) = x y x^{- 1}$, all $y \in G$. Write $I (G)$ for the set
of all inner automorphisms of $G$.

(1) Show that $I (G) \leq \tmop{Aut} (G)$.

{\tmem{Solution:}} We first show that $f_x$, $x \in G$, is actually an
automorphism. Clearly, $f_x : G \rightarrow G$ is a well-defined map by
closure of $G$. Let $x, g, h \in G$ so that $f_x (g h) = x g h x^{- 1} = (x g
x^{- 1}) (x h x^{- 1}) = f_x (g) f_x (h)$. Now suppose that $f_x (g) = f_x
(h)$, then $x g x^{- 1} = x h x^{- 1} \Rightarrow g = h$ by cancellation. If
$h \in G$, then $h = f_x (x^{- 1} h x)$ for some $x \in G$. So $f_x$ is indeed
a bijective endomorphism. Thus the set $I (G)$ contains automorphisms and must
be a subset of $\tmop{Aut} (G)$. Now if $f_x, f_y \in I (G)$, then for all $g
\in G$, $f_x \circ f_y (g) = x (y g y^{- 1}) x^{- 1} = x y g (x y)^{- 1} =
f_{x y} (g)$. Furthermore, if $x \in G$, then $x^{- 1} \in G$ and $f_{x^{- 1}}
\in I (G)$, so that $f_x (g) \circ f_{x^{- 1}} (g) = f_{x x^{- 1}} (g) = f_e
(g) = f_{x^{- 1} x} (g) = f_{x^{- 1}} (g) f_x (g)$ and we have shown inverses.
Hence $I (G) \leq \tmop{Aut} (G)$.

(2) Show that $I (G) \cong G / Z (G)$.

{\tmem{Solution:}} We define a map $\phi : G \rightarrow \tmop{Aut} (G)$ by
$\phi (g) = f_g$. If $g, h \in G$, then $\phi (g h) = f_{g h} = f_g \circ f_h
= \phi (g) \phi (h)$, so $\phi$ is a homomorphism. Clearly, $\tmop{Im} (\phi)
= I (G)$ and we have by the fundamental homomorphism theorem that $G / \ker
\phi \cong I (G)$. Yet $f_x (g) = x g x^{- 1} = g x x^{- 1} = g$, for all $g$
when $x \in Z (G)$, hence $\ker \phi = Z (G)$ and we have our desired result.
\

(3) If $I (G)$ is abelian show that $G' \leq Z (G)$. Conclude that $G$ is
nilpotent.

{\tmem{Solution:}} First we show that if $G / N$ is abelian, then $G' \leq N$,
where $N \vartriangleleft G$. Suppose that $G / N$ is abelian, then $(g N) (h
N) = (h N) (g N) \Rightarrow g^{- 1} h^{- 1} g h N = N$, so $[g, h] \in N$ and
$G' \leq N$. Since $I (G)$ is abelian and isomorphic to $G / Z (G)$ and $Z (G)
\vartriangleleft G$, then it follows that $G' \leq Z (G)$. From the ascending
central series, we have $\{1\}= Z_0 \leq Z_1 = Z (G) \leq Z_2$, where $Z_2 /
Z_1 = Z (G / Z_1)$, but $Z (G / Z_1) = G / Z (G)$ which is abelian, so $Z_2 =
G$ and thus $G$ is nilpotent.

I.12.48: If $A \vartriangleleft G$ and $B \vartriangleleft G$ show that $G /
(A \cap B)$ is isomorphic with a subgroup of $G / A \times G / B$.

{\tmem{Solution:}} Let $\phi : G \rightarrow G / A \times G / B$ by $\phi (g)
= (g A, g B)$. We can see that $\phi$ acts like the combination of two
canonical quotient maps and thus $\phi$ must be well defined. Then $\phi (g h)
= (g h A, g h B) = (g A h A, g B h B) = (g A, g B) (h A, h B) = \phi (g) \phi
(h)$, for all $g, h \in G$, hence $\phi$ is a homomorphism. Now by the
fundamental homomorphism theorem, $\phi$ is from $G$ onto $\tmop{Im} (\phi$)
so that $G / \ker \phi \cong \tmop{Im} (\phi) \leq G / A \times G / B$. But
the ker $\phi$ is the set of all $g \in G$ such that $\phi (g) = (g A, g B) =
(A, B)$, so $g \in A$ and $g \in B$, yet this is just $A \cap B$.

I.12.61: If $H$ and $K$ are subgroups of $G$, with $K \vartriangleleft G, K
\cap H = 1, \tmop{and} K H = G$, then $G$ is called a semidirect product of
$K$ by $H$.

(1) If $\sigma = (12) \in S_n, n \geq 2$, show that $S_n$ is a semidirect
product of $A_n$ by $\langle \sigma \rangle$.

{\tmem{Solution:}} This was done on problem 22 from homework 4 but will repeat
for completeness. We note that $\sigma$ is an odd permutation and that
$\langle \sigma \rangle =\{\varepsilon, \sigma\} \leq S_n$. Clearly $A_n \cap
\langle \sigma \rangle =\{\varepsilon\}$. We know that $A_n \vartriangleleft
S_n$ since it's of index 2. Then $\langle \sigma \rangle A_n = A_n \cup \sigma
A_n = S_n$.

(2) Show that the dihedral group $D_n = \langle a, b | a^n = b^2 = 1, b^{- 1}
a b = a^{- 1} \rangle$ is a semidirect product of $A = \langle a \rangle$ by
$B = \langle b \rangle$.

{\tmem{Solution:}} We know that $|D_n | = 2 n$, so with $| \langle a \rangle |
= n$, we have $\langle a \rangle \vartriangleleft D_n$ by the fact that the
index is 2. Clearly $\langle a \rangle \cap \langle b \rangle =\{1\}$ since $b
\neq a$. Once again $\langle b \rangle \langle a \rangle = \langle a \rangle
\cup b \langle a \rangle = D_n$. We can also view this $a$ representing all
the rotational symmetrics of an $n$-gon and $b$ being the reflection: $\langle
a \rangle$ being all the rotational symmetries on one side of the $n$-gon
while $b \langle a \rangle$ being the ``flip'' and then rotate through all the
symmetries on the other side. This gives us all symmetries of the $n$-gon.

(3) Show that the quaternion group $Q_2$ can not be expressed as a semidirect
product (except with one of the subgroups trivial).

{\tmem{Solution:}} Using problem 3 from homework 4, we listed all the
nontrivial normal subgroups, namely $G = Q_2$, $\langle A \rangle =\{I, A,
A^2, A^3 \}, \langle B \rangle =\{I, B, A^2, A^2 B\}, \langle A B \rangle
=\{I, A B, A^2, A^3 B\}$ and $\langle A^2 \rangle =\{I, A^2 \}$. It's clear
now that the intersection of any of these groups is nontrivial since they all
contain $A^2$. So $Q_2$ cannot be a semidirect product of two nontrivial
subgroups.

I.12.62 Suppose $K$ and $H$ are groups and $\phi : H \rightarrow \tmop{Aut}
(K)$ is a homomorphism. Let $G$ be the Cartesian product $K \times H$ as a
set, but with the binary operation $(x, y) (u, v) = (x \cdot \phi (y) u, y
v)$. Show that $G$ is a group; denote it by $G = K \rtimes_{\phi} H$, and call
it the external semidirect product of $K$ by $H$ relative to $\phi$. Show that
$K_1 =\{(x, 1) : x \in K\} \vartriangleleft G, H_1 =\{(1, y) : y \in H\} \leq
G$, and that $G$ is the semidirect product of $K_1$ by $H_1$ as in exercise
61.

{\tmem{Solution:}} Let $(x, y), (u, v) \in G$, this means that $x, u \in K$
and $y, v \in H$. Then $(x, y) (u, v) = (x \cdot \phi (y) u, y v)$, clearly $y
v \in H \tmop{whenever} y, v \in H$ and $f = \phi (y) \in \tmop{Aut} (K)$, so
that $f (u) = x' \in K$, thus $x x' \in K$ whenever $x, x' \in K$, hence $(x,
y) (u, v) \in G$. Associativity comes from

$[(x, y) (u, v)] (w, z) = (x \cdot \phi (y) u, y v) (w, z) = ([x \cdot \phi
(y) u] \cdot \phi (y v) w, y v z) = (x \cdot \phi (y) u \cdot \phi (y) \phi
(v) w, y v z)$

$(x, y) [(u, v) (w, z)] = (x, y) (u \cdot \phi (v) w, v z) = (x \cdot \phi (y)
[u \cdot \phi (v) w], y v z) = (x \cdot \phi (y) u \cdot \phi (y) \phi (v) w,
y v z)$.

$(x, y) (1_K, 1_H) = (x \cdot \phi (y) 1_K, y 1_H) = (x 1_K, y) = (x, y) =
(1_K x, y) = (1_K \cdot \phi (1_H) x, 1_H y) = (1_K, 1_H) (x, y)$ where we get
the first equality by the fact that any $\phi (y) \in \tmop{Aut} (K)$ sends
the identity, $1_K$ to $1_K$, and the fifth equality since $\phi (1_H) \in
\tmop{Aut} (K)$ is the trivial map, so $(1_K, 1_H)$ is the identity. Lastly,
$(\phi (y^{- 1}) x^{- 1}, y^{- 1}) \in G$ is the inverse of $(x, y) \in G$:
$(x, y) (\phi (y^{- 1}) x^{- 1}, y^{- 1}) = (x \cdot \phi (y) \phi (y^{- 1})
x^{- 1}, y y^{- 1}) = (1_K, 1_H) = (\phi (y^{- 1}) x^{- 1} \cdot \phi (y^{-
1}) x, y^{- 1} y) = (\phi (y^{- 1}) x^{- 1}, y^{- 1}) (x, y)$. Thus $G$ is a
group.

Let $(1_K, y), (1_K, h) \in H_1$, then $(1_K, y) (1_K, h) = (1_K \cdot \phi
(y) 1_K, y h) = (1_K, y h) \in H_1$ since $y h \in H$ whenever $y, h \in H$.
Now if $(1_K, y) \in H_1$, then $(\phi (y^{- 1}) 1_K, y^{- 1}) = (1_K, y^{-
1}) \in H_1$ since $y^{- 1} \in H$ whenever $y \in H$. Hence $H_1 \leq G$.

Let $(x, 1_H), (k, 1_H) \in K_1$, then $(x, 1_H) (k, 1_H) = (x \cdot \phi
(1_H) k, 1_H) = (x k, 1_H) \in K_1$ since $x k \in K$ whenever $x, k \in K$.
Now if $(x, 1_H) \in K_1$, then $(\phi (1_H) x^{- 1}, 1_H) = (x^{- 1}, 1_H)
\in K_1$ since $x^{- 1} \in K$ whenever $x \in K$. Hence $K_1 \leq G$. To show
that $K_1 \vartriangleleft G$, let $(x, 1_H) \in K_1$ and $(k, h) \in G$, then
$(k, h) (x, 1_H) (\phi (h^{- 1}) k^{- 1}, h^{- 1}) = (k \cdot \phi (h) x \cdot
\phi (h) \phi (h^{- 1}) k^{- 1}, h h^{- 1}) = (k \phi (h) x k^{- 1}, 1_H) \in
K_1$.

Since $K_1 \vartriangleleft G$ and $H_1 \leq G$ and $K_1 \cap H_1 =\{(1_K,
1_H)\}$, then $K_1 H_1 \leq G$. Likewise, $G \leq K_1 H_1$ since if we let
$(k, h) \in G$, then $(k, h) = (k \phi (1_H) 1_K, h) = (k, 1_H) (1_K, h)$. So
$G = K_1 H_1$ like in the previous exercise.

II.5.1: Let $R =\{a + b \sqrt{- 5} : a, b \in \mathbbm{Z}\} \subseteq
\mathbbm{C}$.

(1) Show that $R$ is an integral domain with 1 (it is a subring of
$\mathbbm{C})$.

{\tmem{Solution:}} We note that $a = 1, b = 0 \in \mathbbm{Z}$ and thus $1 \in
R \neq \emptyset$. Now let $a + b \sqrt{- 5}, c + d \sqrt{- 5} \in R$. Then
$\text{($a + b \sqrt{- 5}) - (c + d \sqrt{- 5})$} = (a - c) + (b - d) \sqrt{-
5} \in R$ since $a - c, b - d \in \mathbbm{Z}$ whenever $a, b, c, d \in
\mathbbm{Z}$. Furthermore, $(a + b \sqrt{- 5}) (c + d \sqrt{- 5}) = (a c - 5 b
d) + (b c + a d) \sqrt{- 5} \in R$ since $a c - 5 b d, b c + a d \in
\mathbbm{Z}$ whenever $a, b, c, d \in \mathbbm{Z}$ and $R$ is a subring of
$\mathbbm{C}$. With $\mathbbm{C}$ having no zero divisors and $R$ a subset of
$\mathbbm{C}$, it must be that $R$ has no zero divisors and conclusively is an
integral domain.

(2) Show that $U (R) =\{\pm 1\}$.

{\tmem{Solution:}} We first say that $U(R) = \{ a + b\sqrt{-5} \in R \mid b = 0\}$ since 
$U(\mathbb{Z}[x])$ are the constants. Thus if $a \in U (R)$, 
then $a (c + d\sqrt{- 5}) = a c + a d \sqrt{- 5} = 1$ 
but with $a \neq 0$ and no zero divisors it must be that $d = 0$. 
Thus $a c = 1$ with $a, c \in \mathbbm{Z}$. So $a \in U (\mathbbm{Z}_{})$, 
hence $a = 1$ or $a = - 1$. \

(3) Show that 3 is irreducible in $R$.

{\tmem{Solution:}} Suppose that $3 = (a + b \sqrt{- 5}) (c + d \sqrt{- 5})$
for $a + b \sqrt{- 5}, c + d \sqrt{- 5} \in R$. By complex conjugates, it's
also true that $3 = (a - b \sqrt{- 5}) (c - d \sqrt{- 5})$. Thus $9 = (a^2 + 5
b^2) (c^2 + 5 d^2)$. With the fact that $x^2 + 5 y^2 = 3$ has no solutions in
$\mathbbm{Z}$, it must be that $a^2 + 5 b^2 = 1$ or $c^2 + 5 d^2 = 1$. Clearly
this forces $a = \pm 1 \tmop{and} b = 0$ or $c = \pm 1 \tmop{and} d = 0$.

(4) Show that $a = 2 + \sqrt{- 5}$ and $b = 2 - \sqrt{- 5}$ are both
irreducible in $R$.

{\tmem{Solution:}} Suppose that $2 + \sqrt{- 5} = (a + b \sqrt{- 5}) (c + d
\sqrt{- 5})$ for $a + b \sqrt{- 5}, c + d \sqrt{- 5} \in R$. By complex
conjugates, it's also true that $2 - \sqrt{- 5} = (a - b \sqrt{- 5}) (c - d
\sqrt{- 5})$. Thus $9 = (a^2 + 5 b^2) (c^2 + 5 d^2)$. With the fact that $x^2
+ 5 y^2 = 2 \pm \sqrt{- 5}$ has no solutions in $\mathbbm{Z}$, it must be that
$a^2 + 5 b^2 = 1$ or $c^2 + 5 d^2 = 1$. Clearly this forces $a = \pm 1
\tmop{and} b = 0$ or $c = \pm 1 \tmop{and} d = 0$.

(5) Conclude that $3 \nmid 2 + \sqrt{- 5}$ and $3 \nmid 2 - \sqrt{- 5}$ in
$R$.

{\tmem{Solution:}} Suppose that $3 | 2 + \sqrt{- 5}$, then $3 (a + b \sqrt{-
5}) = 2 + \sqrt{- 5} \tmop{for} \tmop{some} a + b \sqrt{- 5} \in R \Rightarrow
3 a = 2$ and $3 b = 1$. Thus $a = 2 / 3$ and $b = 1 / 3$, both of which are
not in $\mathbbm{Z}$, hence a contradiction. Similarly, suppose that $3 | 2 -
\sqrt{- 5}$, then $3 (a + b \sqrt{- 5}) = 2 - \sqrt{- 5} \tmop{for}
\tmop{some} a + b \sqrt{- 5} \in R \Rightarrow 3 a = 2$ and $3 b = - 1$. Thus
$a = 2 / 3$ and $b = - 1 / 3$, both of which are not in $\mathbbm{Z}$, hence a
contradiction.

(6) Conclude that $3$ is irreducible but not prime in $R$.

{\tmem{Solution:}} We know that 3 is irreducible by (3) above. To show that 3
is not prime it suffices to remark that $3$ divides $9 = (2 + \sqrt{- 5}) (2 -
\sqrt{- 5})$ but by (5) we see that 3 cannot be prime in $R$.

II.5.2: Show that $\mathbbm{Q}[ \sqrt{m}] =\{r + s \sqrt{m} : r, s \in
\mathbbm{Q}\}$, and that $\mathbbm{Q}[ \sqrt{m}]$ is a field.

{\tmem{Solution:}} We note that $a = 1, b = 0 \in \mathbbm{Q}$ and thus $1
\in \mathbbm{Q}[ \sqrt{m}] \neq \emptyset$. Now let $a + b \sqrt{m}, c + d
\sqrt{m} \in \mathbbm{Q}[ \sqrt{m}]$. Then $\text{($a + b \sqrt{m}) - (c + d
\sqrt{m})$} = (a - c) + (b - d) \sqrt{m} \in \mathbbm{Q}[ \sqrt{m}]$ since $a
- c, b - d \in \mathbbm{Q}$ whenever $a, b, c, d \in \mathbbm{Q}$.
Furthermore, $(a + b \sqrt{m}) (c + d \sqrt{m}) = (a c + m b d) + (b c + a d)
\sqrt{m} \in \mathbbm{Q}[ \sqrt{m}]$ since $a c + m b d, b c + a d \in
\mathbbm{Q}$ whenever $a, b, c, d \in \mathbbm{Q}$ and $\mathbbm{Q}[
\sqrt{m}]$ is a subring of $\mathbbm{C}$. With $\mathbbm{C}$ commutative, it
must be that $\mathbbm{Q}[ \sqrt{m}]$ is commutative as well. Since
$\mathbbm{Q}[ \sqrt{m}] \subseteq \mathbbm{C}$, then for any $a + b \sqrt{m}
\in \mathbbm{Q}[ \sqrt{m}] \setminus \{0\}$, there exists $1 / (a + b
\sqrt{m}) \in \mathbbm{C}$ such that $(a + b \sqrt{m}) (1 / (a + b \sqrt{m}))
= 1$. By rationalizing the denominator of $1 / (a + b \sqrt{m})$, we yield
$\frac{a}{a^2 + m b^2} - \frac{b}{a^2 + m b^2} \sqrt{m}$ and we can clearly
see that $1 / (a + b \sqrt{m}) \in \mathbbm{Q}[ \sqrt{m}] \setminus \{0\}$,
thus $U (\mathbbm{Q}[ \sqrt{m}]) =\mathbbm{Q}[ \sqrt{m}] \setminus \{0\}$ and
$\mathbbm{Q}[ \sqrt{m}]$ is a division ring. With $\mathbbm{Q}[ \sqrt{m}]$ a
commutative division ring, by definition, $\mathbbm{Q}[ \sqrt{m}]$ is a field.

II.5.4: (1) Show that $N (x) \in \mathbbm{Z} \tmop{if} x \in R_m$.

{\tmem{Solution:}} When $m \equiv 2 (\tmop{mod} 4)$ or $m \equiv 3 (\tmop{mod}
4)$, then $x = a + b \sqrt{m} \in R_m$ and $N (x) = a^2 - m b^2 \in
\mathbbm{Z}$ since $a, b, m \in \mathbbm{Z}$. Now when $m \equiv 1 (\tmop{mod}
4)$, then $x = \frac{a + b \sqrt{m}}{2} \in R_m$ and $N (x) = \frac{a^2}{4} -
\frac{m b^2}{4}$. Since $a \equiv b (\tmop{mod} 2)$, then we take the case
where $a \equiv 0 (\tmop{mod} 2)$ and $a = 2 k, b = 2 n$ for some $n, k \in
\mathbbm{Z}$. Hence $N (x) = \frac{(2 k)^2}{4} - \frac{m (2 n)^2}{4} = k^2 - m
n^2 \in \mathbbm{Z}$ since $k, n, m \in \mathbbm{Z}$. Now we take the case
where $a \equiv 1 (\tmop{mod} 2)$, and $a = 2 k + 1, b = 2 n + 1$ for some $n,
k \in \mathbbm{Z}$. Hence $N (x) = \frac{(2 k + 1)^2}{4} - \frac{m (2 n +
1)^2}{4} = \frac{4 (k^2 + k - m n^2 - n m) - m + 1}{4} = k^2 + k - m n^2 - m n
- \frac{m - 1}{4}$ and since $m \equiv 1 (\tmop{mod} 4)$ it must be that $m -
1 \equiv 0 (\tmop{mod} 4)$ and $m = 4 s$, where $s \in \mathbbm{Z}$. Thus $N
(x) = k^2 = k - m n^2 - m n - s \in \mathbbm{Z}$, since $k, m, n, s \in
\mathbbm{Z}$.

(2) Show that $u \in U (R_m)$ if and only if $N (u) = \pm 1$.

{\tmem{Solution:}} Suppose that $u \in U (R_m)$. Then $u r = 1$ and thus $N
(u) N (r) = N (u r) = N (1) = 1$. By (1), $N (u), N (r) \in \mathbbm{Z}
\Rightarrow N (u) = 1 = N (r)$. Conversely, suppose that $N (u) = \pm 1$. Then
when $m \equiv 2 (\tmop{mod} 4)$ or $m \equiv 3 (\tmop{mod} 4)$, $(a + b
\sqrt{m}) (a - b \sqrt{m}) = a^2 - m b^2 = \pm 1$ and $a + b \sqrt{m} \in U
(R_m)$. When $m \equiv 1 (\tmop{mod} 4)$, $( \frac{a + b \sqrt{m}}{2}) (
\frac{a - b \sqrt{m}}{2}) = \frac{a^2}{4} - \frac{m b^2}{4} = \pm 1$ and
$\frac{a + b \sqrt{m}}{2} \in U (R_m)$.

(3) Use (2) to show that $U (R_{- 1}) =\{\pm 1, \pm i\}$, $U (R_{- 3}) =\{\pm
1, \pm (1 \pm \sqrt{- 3}) / 2\}$ and $U (R_m) =\{\pm 1\}$ for all other
negative square-free in $\mathbbm{Z}$.

{\tmem{Solution:}} If $x = a + b \sqrt{- 1} \in U (R_{- 1})$, then by (2), we
have $N (x) = a^2 + b^2 = \pm 1$, where $a, b \in \mathbbm{Z}$. Immediately we
can throw out $- 1$, since $a^2 \geq 0$ and $b^2 \geq 0$. So $a^2 + b^2 = 1$
and we have $a = \pm 1$ while $b = 0$ and $a = 0$ while $b = \pm 1$. So $x =
\pm 1$ or $\pm \sqrt{- 1}$. If $x = (a + b \sqrt{- 1}) / 2 \in U (R_{- 3})$,
then by (2), we have $N (x) = \frac{a^2}{4} + \frac{3 b^2}{4} = \pm 1$, where
$a, b \in \mathbbm{Z}$. Immediately we can throw out $- 1$, since $a^2 \geq 0$
and $b^2 \geq 0$. So $\frac{a^2}{4} + \frac{3 b^2}{4} = 1$ and we have $a =
\pm 1$ while $b = \pm 1$. We also have $a = \pm 1$ while $b = \mp 1$ and $b =
0$ while $b = \pm 2$. So $x = \pm \frac{2}{2} = \pm 1$ or $\pm (1 \pm \sqrt{-
3}) / 2$. Now for all $m \in \mathbbm{Z}$, $m < 0$ and $m$ square-free, we
break into two cases. The first case is when $m \equiv 2 (\tmop{mod} 4)$ or $m
\equiv 3 (\tmop{mod} 4)$ which we notice that $|m| > 1$, If $x = a + b \sqrt{-
m} \in U (R_{- m})$, then by (2), we have $N (x) = a^2 + m b^2 = \pm 1$, where
$a, b \in \mathbbm{Z}$. Immediately we can throw out $- 1$, since $a^2 \geq 0$
and $b^2 \geq 0$. So $a^2 + m b^2 = 1$ and we have $a = \pm 1$ while $b = 0$
as the only possibilities since $|m| > 1 \Rightarrow m b^2 > 1$. So $x = \pm
1$. The other case is when $m \equiv 1 (\tmop{mod} 4)$, which we observe that
$|m| > 4$ since already know $R_{- 3}$. If $x = a + b \sqrt{- m} \in U (R_{-
m})$, then by (2), we have $N (x) = a^2 / 4 + (m b^2) / 4 = \pm 1$, where $a,
b \in \mathbbm{Z}$. Immediately we can throw out $- 1$, since $a^2 \geq 0$ and
$b^2 \geq 0$. So $a^2 / 4 + (m b^2) / 4 = 1$ and we have $a = \pm 1$ while $b
= 0$ as the only possibilities since $|m| > 4$ and $\frac{m b^2}{4} > 1$. So
$x = \pm 1$.

II.8.3 If $R$ is any ring denote by $R_1$ the additive group $R \oplus
\mathbbm{Z}$, with multiplication defined by setting $(r, n) (s, m) = (r s + m
r + n s, n m)$. Show that $R_1$ is a ring with $1$. If $r \in R$ is identified
with $(r, 0) \in R$, show that $R$ is a subring of $R_1$. Conclude that every
ring is a subring of a ring with $1$.

{\tmem{Solution:}} To show that $R_1$ is a ring is simple verification of
definitions. To show that $R_1$ has a 1, we consider $(0, 1)$, then $(r, s)
(0, 1) = (r \cdot 0 + 1 \cdot r + 0 \cdot s, s \cdot 1) = (r, s)$ for all $(r,
s) \in R_1$. Now, let's call $R$ identified in $R_1$ as $R' =\{(r, 0) | r \in
R\}$. If $x, y \in R$, then $x - y, x y \in R$ by definition. So let $(x, 0),
(y, 0) \in R'$, then $(x, 0) + (y, 0) = (x + y, 0) \in R'$ and $(x, 0) (y, 0)
= (x y, 0) \in R'$ so $R'$ is a subring by proposition II.1.2. It's clear that
the construction of $R \oplus \mathbbm{Z}$ for any ring $R$ is itself a ring
with 1 that contains a subring identified as $R$.

II.8.6: If $F$ is a field show that the ring $M_n (F)$ of $n \times n$
matrices over $F$ is a simple ring.

{\tmem{Solution:}} We make a few observations first. Let $E_{i j}$ be the
matrix in $M_n (F)$ such that $e_{i j} = 1$ and everywhere else is 0.
Furthermore, any $A \in M_n (F)$ is $\sum^n_{i = 1} \sum^n_{j = 1} a_{i j}
E_{i j}$, where $a_{i j}$ is the entry corresponding to the $i$th row and
$j$th column in $A$. We also see that it's straightforward from the binary
operations definition on $M_n (F)$, that if $I$ is an ideal of $F$, then $M_n
(I)$ is an ideal of $M_n (F)$. Now, we let $I$ be an ideal of $M_n (F)$. We
define $J =\{a \in F | \exists A \in I \ni a_{1 1} = a\}$. To see that $J$ is
an ideal of $F$, let $x, y \in J$ and $z \in F$, then $x - y \in J$ since $X -
Y \in I$ whenever $X, Y \in I$ and $z x, x z \in J$ since $z E_{1 1} X, X (z
E_{1 1}) \in I$, whenever $X \in I$ and $z E_{11}, E_{11} z \in M_n (F)$. Now
we note that $E_{1 k} A E_{m 1} = E_{1 k} ( \text{$\sum^n_{i = 1} \sum^n_{j =
1} a_{i j} E_{i j})$} E_{m 1} = a_{k m} E_{1 1} \in I$. Hence $a_{i j} \in J$
for all entries in $A$ whenever $A \in J$ and thus $I \subseteq M_n (J)$. To
show the other inclusion, we note that $M_n (J) = (a E_{i j})$ where $a \in
J$. Since $a E_{i j} = E_{i 1} A E_{1 j} \in I$, it must be that $a E_{i j}
\in I$ and $M_n (J) \subseteq I$. Thus $M_n (J) = I$ and it follows by
proposition 1.9, $F$ is simple and has only two ideals: $\{0\}$ and $F$. Thus
$M_n (\{0\})$ and $M_n (F)$ are the only ideals and $M_n (F)$ is simple.

II.8.14: Show that the ideal $I = (2, x)$ is not principle in
$\mathbbm{Z}[x]$.

{\tmem{Solution:}} First off, $2 h + x q \neq \pm 1 \tmop{for} \tmop{any} h,
q \in \mathbbm{Z}[x]$ and thus $\pm 1 \nin I$. Now suppose that $I = (f)$.
Then we know that $2 \in (f)$ which means that $f g = 2$ for some $g \in
\mathbbm{Z}[x]$. This means $f | 2$ and thus must be either $\pm 1$ or $\pm
2$. If $f = \pm 1$, then $\pm 1 \in (2, x)$, a contradiction. If $f = \pm 2$,
then $f g \neq x \in I$ for any $g \in I$, another contradiction.

II.8.26: Suppose $R$ is a Euclidean domain, $a, b \in R$ and $a b \neq 0$.
Write $a = b q_1 + r_1, d (r_1) < d (b)$, $b = r_1 q_2 + r_2, d (r_2) < d
(r_1)$, $r_1 = r_2 q_3 + r_3, d (r_3) < d (r_2), \ldots, r_{k - 2} = r_{k - 1}
q_k + r_k, d (r_k) < d (r_{k - 1})$, $r_{k - 1} = r_k q_k$ with all $r_i, q_j
\in R$. Show that $r_k = (a, b)$, and ``solve'' for $r_k$ in terms of $a$ and
$b$, thereby expressing $(a, b)$ in the form $u a + v b$ with $u, v \in R$.

{\tmem{Solution:}} Note that the explicit use of functional gcd notation,
$\gcd (a, b)$, is needed to avoid confusion with the principal ideal generated
by $a$ and $b$. We first want to show that $g = \gcd (a, b)$ if and only if
$(g) = (a, b)$ in PID. Since $R$ being a Euclidean domain, then it must be
that $R$ is also a PID by proposition 5.6. So, suppose that $g = \gcd (a, b)$.
Now let $(a, b) = (d)$, then $g | a$ and $g| b$ so $g \in (d)$ and thus $(d)
\subseteq (g)$. To show the other inclusion, we note that $d | a$ and $d | b$
since $a, b \in (d)$ and with $g = \gcd (a, b)$, $d | g$ and hence $(g)
\subseteq (d)$. So $(g) = (d) = (a, b)$. Conversely, suppose that $(g) = (a,
b)$. Then $g = r_1 a + r_2 b$ for some $r_1, r_2 \in R$. Thus if $c$ is any
divisor of $a$ and $b$, then it must be that $c | g$, so $g$ must be the $\gcd
(a, b)$.

Now we get to the claim that $r_k = \gcd (a, b)$. For convienence, we let
$r_{- 1} = a$ and $r_0 = b$ and we also see that $r_{k + 2} = 0$. We show that
the $\gcd (r_i, r_{i + 1}) = \gcd (r_{i + 1}, r_{i + 2})$, $- 1 \leq i \leq k
- 1$; we have $s r_i + t r_{i + 1} = (s q_{i + 2} + t) r_{i + 1} + s r_{i +
2}$ since $r_i = r_{i + 1} q_{i + 2} + r_{i + 2}$ and hence $(r_i, r_{i + 1})
\subseteq (r_{i + 1}, r_{i + 2})$. We also have $s r_{i + 1} + t r_{i + 2} = s
r_i + (t - q_{i + 2}) r_{i + 1}$ since $r_i = r_{i + 1} q_{i + 2} + r_{i + 2}$
and hence $(r_{i + 1}, r_{i + 2}) \subseteq (r_i, r_{i + 1})$. Conclusively,
we have $(r_{i + 1}, r_{i + 2}) = (r_i, r_{i + 1})$ and thus $\gcd (r_i, r_{i
+ 1}) = \gcd (r_{i + 1}, r_{i + 2})$ by above . It follows that $\gcd (r_{-
1}, r_0) = \gcd (r_0, r_1) = \cdots = \gcd (r_{k - 1}, r_k)$. With $r_k | r_{k
- 1}$ since $r_{k - 1} = r_k q_k$, we have $\gcd (r_{k - 1}, r_k) = r_k = \gcd
(r_{- 1}, r_0) = \gcd (a, b)$.

II.8.28: Suppose $f (x) = 1 + x + x^2 + \cdots + x^{p - 1}$, where $p$ is
prime in $\mathbbm{Z}$.

(1) Show that $f (x)$ is irreducible in $\mathbbm{Q}[x]$.

{\tmem{Solution:}} First we prove that if $f (x + c)$ is irreducible in $F
[x]$, then $f (x)$ is irreducible in $F [x]$, where $F$ is a field and $c \in
F$. We note that $\deg (g (x + c)) = \deg (g (x))$ and $\deg (h (x + c)) =
\deg (h (x))$. Suppose that $f (x + c)$ is irreducible. Then for the sake of
contradiction suppose that $f (x)$ is reducible. Hence $f (x) = g (x) h (x)$,
where $g (x), h (x) \in F [x]$ and $\deg (g (x)), \deg (h (x)) > 0$. It
follows that $f (x + c) = g (x + c) h (x + c)$, where $\deg (g (x + c)), \deg
(h (x + c)) > 0$ and $g (x + c), h (x + c) \in F [x]$ and consequently $f (x +
c)$ is reducible, a contradiction.

We notice that $f (x) = 1 + x + x^2 + \cdots + x^{p - 1} = \frac{x^p - 1}{x -
1}$. Then we look at $f (x + 1) = \frac{(x + 1)^p - 1}{(x + 1) - 1} = x^{p -
1} + \binom{p}{1} x^{p - 2} + \cdots + \binom{p}{p}$. We see that $p$ divides
all coefficients except for the leading one and $p^2 \nmid 1$ so $f (x + 1)$
is irreducible in $\mathbbm{Z}[x]$ by Eisensteins criterion. Yet we know that
$\mathbbm{Q}$ is the field of fractions for $\mathbbm{Z}$ and by proposition
5.15, $f (x + 1)$ is irreducible in $\mathbbm{Q}[x]$ as well. Conclusively by
above, we have $f (x)$ is irreducible in $\mathbbm{Q}[x]$.

(2) Show that $\binom{p}{k} = \sum^{k + 1}_{i = 1} \binom{p - i}{p - k - 1}$
for all $k < p$.

{\tmem{Solution:}} First, let $n, r \in \mathbbm{Z}$ and $0 \leq r \leq n$,
then we note a well known binomial identity, $\binom{r}{r} + \binom{r + 1}{r}
+ \cdots + \binom{n}{r} = \binom{n + 1}{r + 1}$ and the fact that
$\binom{n}{r} = \binom{n}{n - r}$. Looking at $\sum^{k + 1}_{i = 1} \binom{p -
i}{p - k - 1}$, when $k = 1$, $2, \ldots, p - 1$ , we see $\binom{p - 1}{p -
2} + \binom{p - 2}{p - 2}$, $\binom{p - 1}{p - 3} + \binom{p - 2}{p - 3} +
\binom{p - 3}{p - 3}, \ldots, \binom{p - 1}{0} + \cdots + \binom{1}{0}$. So,
we can actually rewrite $\sum^{k + 1}_{i = 1} \binom{p - i}{p - k - 1} =
\sum^k_{i = 0} \binom{p - k - 1 + i}{p - k - 1}$, yet by the binomial identity
$\sum^k_{i = 0} \binom{p - k - 1 + i}{p - k - 1} = \binom{p}{p - k}$. But this
is just $\binom{p}{p - k} = \binom{p}{k}$. Thus $\binom{p}{k} = \sum^{k +
1}_{i = 1} \binom{p - i}{p - k - 1}$.

II.8.29: If $p \in \mathbbm{Z}$ is prime and $1 < m \in \mathbbm{Z}$ show that
$f (x) = x^m - p$ is irreducible in $\mathbbm{Q}[x]$. Conclude that $p^{1 /
m}$ is irrational.

{\tmem{Solution:}} $f (x)$ is irreducible in $\mathbbm{Z}[x]$by Eisenstein's
criteria when we consider $p$ as our prime. Thus it must be irreducible in
$\mathbbm{Q}[x]$ by proposition 5.15. Suppose that $p^{1 / m} \in
\mathbbm{Q}$, then $f (p^{1 / m}) = (p^{1 / m})^m - p = 0$ and by the
corollary to proposition 3.5, $f (x)$ has a root in $\mathbbm{Q}$. Hence $f
(x)$ is reducible, which is a contradiction, so $p^{1 / m} \nin \mathbbm{Q}$.

II.8.30: Establish the Eisenstein Criterion for a polynomial $f (x)$ over a
UFD.

{\tmem{Solution:}} Let $R$ be a UFD. First let's prove the converse of Gauss's
lemma: If $g (x) h (x) \in R [x]$ is primitive, where $g (x), h (x) \in R
[x]$, then $g (x)$ and $h (x)$ is primitive. We do this by showing the
contrapositive; If $g (x)$ or $h (x)$ is not primitive, where $g (x), h (x)
\in R [x]$, then $g (x) h (x)$ is not primitive.

{\tmem{Proof:}} Without loss of generality, suppose that $g (x)$is not
primitive. Then $g (x) = a b_0 + a b_1 x + \cdots + a b_n x^n$ and $h (x) =
d_0 + d_1 x + \cdots + d_m x^m$, where $0 \neq a \in R$. Hence $g (x) h (x) =
a b_0 d_0 + \cdots + a b_n d_m x^{m + n}$ and $g (x) h (x)$ is not primitive
since we can factor out $a$.

Suppose $R$ is an UFD and $f (x) = a_0 + a_1 x + \cdots + a_n x^n$ is
primitive in $R [x]$. If there is a prime $p \in R$ such that $p$ divides
every coefficient $a_i$, $0 \leq i \leq n - 1$, $p \nmid a_n$, and $p^2 \nmid
a_0$, then $f (x)$ is irreducible in $R [x]$.

{\tmem{Proof:}} Suppose there is a prime $p \in R$ such that $p$ divides
every coefficient $a_i$, $0 \leq i \leq n - 1$, $p \nmid a_n$, and $p^2 \nmid
a_0$ of $f (x)$. Now suppose that $f (x)$ is reducible in $R [x]$, that is, $f
(x) = g (x) h (x)$, where $g (x) = b_0 + b_1 x + \cdots + b_m x^m, h (x) = c_0
+ c_1 x + \cdots + c_k x^k \in R [x]$. But since $f (x)$ is primitive, it
follows from above that $g (x)$ and $h (x)$ are primitive. Furthermore it must
be $p \nmid b_0$ and $p \nmid c_0$ is not true otherwise $p^2 | a_0$, which
would be a contradiction. So $p|b_0$ or $p|c_0$ and, without loss of
generality, let's say $p | b_0$, which means $p \nmid c_0$. It follows that
since $g (x)$ is primitive that $p \nmid b_i$ for some $0 < i \leq m$ and
$p|b_j, 0 \leq j < i$. Yet $p | a_i = b_i c_0 + b_{i - 1} c_1 + \cdots + b_0
c_i$ and since $p \nmid b_i$ and $p \nmid c_0$ with $R$ a UFD, $p \nmid b_i
c_0$. Ultimately we have $p$ dividing all but one summand on the right hand
side except for one, showing that $p$ does not divide $a_i$, a contradiction.

II.8.34: Let $R$ be the set of rational numbers $a / b$ with $b$ odd. Then $R$
is an integral domain.

(1) Find $U (R)$.

{\tmem{Solution:}} Since $R \subseteq \mathbbm{Q}$, it must be that for any $a
/ b \in R$, $(a / b) (b / a) = 1$. Since $b / a$ can only be in $R$ when $a$
is odd as well, hence it follows that $U (R) =\{a / b \in R | a \equiv 1
(\tmop{mod} 2)\}$.

(2) Show that $P = R \setminus U (R)$ is a maximal ideal in $R$.

{\tmem{Solution:}} First, we claim that if an ideal $I$ of $R$ contains a
unit, then $I = R$. To show this let $u \in U (R)$ such that $u \in I$ and $r
\in R$, then $r = r u^{- 1} (u) = r (u^{- 1} u) \in I$ and $R \subseteq I$ but
$I \subseteq R$ thus $R = I$. Now let $M$ be an ideal in $R$ such that $P
\subseteq M$. If $P$ is to be proven maximal, then either $M = P$ or $M = R$.
Well, if $M = P$, then we are done. Otherwise $P \subset M$, which means that
$M \setminus P$ is nonempty and furthermore, since $P$ contains all $a / b$,
where $a \equiv 0 (\tmop{mod} 2)$, then there exists an $c / d \in M$ such
that $c \equiv 1 (\tmop{mod} 2)$. Yet this implies that $c / d \in U (R)$ and
hence $M = R$ concluding that $P$ is maximal.

II.8.37: Solve the congruences $x \equiv i (\tmop{mod} 1 + i), x \equiv 1
(\tmop{mod} 2 - i), x \equiv 1 + i (\tmop{mod} 3 + 4 i)$ simultaneously for
$x$ in the ring $R_{- 1}$ of Gaussian integers.

{\tmem{Solution:}} First we need to check that the ideals generated by $(1 +
i, 2 - i), (2 - i, 3 + 4 i), \tmop{and} (1 + i, 3 + 4 i)$ are $R_{- 1}$. After
quick computation, we see that $(2 - i) + i (i + 1) = (3 + 4 i) - 2 i (2 - i)
= (- i) (3 + 4 i) + 3 i (1 + i) = 1$, with 1 a unit, then the ideals are in
fact $R_{- 1}$. Now taking the product of each choice of two of the three
above equalities, we get $5 - 10 i + (3 + 7 i) (1 + i) = (- 7 - i) + (3 + 2 i)
(2 - i) = (18 + 6 i) + (- 3 + 2 i) (3 + 4 i) = 1$. Thus the solution is $i (5
- 10 i) + 1 (- 7 - i) + (1 + i) (18 + 6 i) = 15 + 28 i$.

III.1.2 Find a splitting field $K \subseteq \mathbbm{C}$ for $f (x) \in
\mathbbm{Q}[x]$ if $f (x) = x^3 - 1$.

{\tmem{Solution:}} We quickly note that $1 \in \mathbbm{Q}$ is a root, so that
$f (x) = (x - 1) (x^2 + x + 1)$. Using the quadratic equation to the
polynomial $x^2 + x + 1$, which is irreducible by a homework problem from last
semester, yields the other two roots, $\frac{- 1 \pm \sqrt{- 3}}{2}$, neither
of which are contained in $\mathbbm{Q}$. Since $\pm \frac{1}{2} \in
\mathbbm{Q}$, $K =\mathbbm{Q}( \sqrt{- 3})$.

III.1.3 Show that $K$ is an algebraic closure of $F$ if and only if $K$ is a
splitting field over $F$ for $\mathcal{F} = F [x]$.

{\tmem{Solution:}} Suppose that $K$ is an algebraic closure of $F$. By
definition, $K$ is an algebraic extension of $F$ and is algebraically closed,
which means that every nonconstant $f (x) \in F [x]$ splits over $K$. With the
constants being identified as $F$ contained in $F [x]$, they certainly split
in $K$ by virtue of being an extension field. Hence $K$ is a splitting field
for $F [x] = \mathcal{F}$ over $F$. Conversely, suppose that $K$ is a
splitting field over $F$ for $\mathcal{F} = F [x]$. By definition, $K = F
(S)$, where $S$ is the set of all roots in $K$ of all $f (x) \in \mathcal{F}$,
and thus $K$ is an algebraic extension. It's clear that $K$ is algebraically
closed since every $f (x) \in \mathcal{F}$ has a root in $K$. Since $K$ is an
algebraic extension of $F$ and is algebraically closed, $K$ is an algebraic
closure of $F$.

III.2.1 If $m \in \mathbbm{Z}$ is square-free and $m \neq 0, 1$ show that $K
=\mathbbm{Q}( \sqrt{m})$ is Galois over $F =\mathbbm{Q}$.

{\tmem{Solution:}} We already know that $\mathbbm{Q}( \sqrt{m}) =\{a + b
\sqrt{m} | a, b \in \mathbbm{Q}\}$. We can further see that $\mathbbm{Q}(
\sqrt{m}) \setminus \mathbbm{Q}=\{a + b \sqrt{m} | a, b \in \mathbbm{Q}, b
\neq 0\}$. Now any $\phi \in G (\mathbbm{Q}( \sqrt{m}) : \mathbbm{Q})$ must
take $\sqrt{m}$ to a root of the minimal polynomial $p (x) = x^2 - m$. Hence
there exists a $\phi \in G (\mathbbm{Q}( \sqrt{m}) : \mathbbm{Q})$ such that
$\phi ( \sqrt{m}) = - \sqrt{m}$, the only other root of $p (x)$. Furthermore,
let $a + b \sqrt{m} \in \mathbbm{Q}( \sqrt{m}) \setminus \mathbbm{Q}$, then
$\phi (a + b \sqrt{m}) = \phi (a) + \phi (b) \phi ( \sqrt{m}) = a + b (-
\sqrt{m}) = a - b \sqrt{m} \in \mathbbm{Q}( \sqrt{m}) \setminus \mathbbm{Q}$.
This means that for each $a + b \sqrt{m} \in \mathbbm{Q}( \sqrt{m}) \setminus
\mathbbm{Q}$, we have a $\phi \in G (\mathbbm{Q}( \sqrt{m}) : \mathbbm{Q})$
such that $\phi (a + b \sqrt{m}) \neq a + b \sqrt{m}$ and thus $\mathbbm{Q}(
\sqrt{m})$ is Galois over $\mathbbm{Q}$.

II.2.2 Prove Proposition 2.6

(1) If $F \subseteq E \subseteq K$, and $E$ is stable, then $\mathcal{G} E
\vartriangleleft G$.

{\tmem{Solution:}} Since $E$ is stable, $\phi (e) = e'$, where $e, e' \in E$,
for all $\phi \in G$. With $E$ an intermediate field, we know that
$\mathcal{G} E \leq G$. Now let $\phi \in G$, $\theta \in \mathcal{G} E$, and
$e \in E$. Then $\phi \theta \phi^{- 1} (e) = \phi \theta (e') = \phi (e') =
e$, where $e' \in E$ and thus $\phi \theta \phi^{- 1} \in \mathcal{G} E$ since
it fixes any $e \in E$. As $\phi$ and $\theta$ were chosen to be arbitrary, we
can conclude that $\mathcal{G} E$ is normal in $G$.

(2) If $H \vartriangleleft G$, then $\mathcal{F} H$ is stable.

{\tmem{Solution:}} Suppose that $H \vartriangleleft G$. Now assume
$\mathcal{F} H$ is not stable. This means that for some $a \in K$, there
exists a $\phi \in G$ such that $\phi (a) = b \in K \setminus \mathcal{F} H$.
Since $H$ is normal in $G$, we can express any $\theta' \in H$ as $\phi \theta
\phi^{- 1}$ for some other $\theta \in H$. Now we consider $\theta' (b) = \phi
\theta \phi^{- 1} (b) = \phi \theta (a) = \phi (a) = b$. Hence all $\theta'
\in H$ fix $b$ and thus $b \in \mathcal{F} H$, a contradiction to how $b$ was
chosen. Thus $\mathcal{F} H$ must be stable.

III.3.3: Find a primitive element over $\mathbbm{Q}$ for a splitting field $K
\subseteq \mathbbm{C}$ for the polynomial $f (x) = x^4 - 5 x^2 + 6$.

{\tmem{Solution:}} $f (x) = (x^2 - 3) (x^2 - 2)$ for which the roots are $\pm
\sqrt{3}$ and $\pm \sqrt{2}$. Hence the splitting field for $f (x)$ is
$\mathbbm{Q}( \sqrt{2}, \sqrt{3})$. We now claim that $\sqrt{2} + \sqrt{3}$ is
the primitive element, which means, we want to show that $\mathbbm{Q}(
\sqrt{2} + \sqrt{3}) =\mathbbm{Q}( \sqrt{2}, \sqrt{3})$. Clearly $\sqrt{2} +
\sqrt{3} \in \mathbbm{Q}( \sqrt{2}, \sqrt{3})$ and thus $\mathbbm{Q}( \sqrt{2}
+ \sqrt{3}) \subseteq \mathbbm{Q}( \sqrt{2}, \sqrt{3})$. Let $\sqrt{2} +
\sqrt{3} \in \mathbbm{Q}( \sqrt{2} + \sqrt{3})$ and since $\mathbbm{Q}(
\sqrt{2} + \sqrt{3})$ is a field, we have $( \sqrt{2} + \sqrt{3})^{- 1} = - (
\sqrt{2} - \sqrt{3}) \Rightarrow \sqrt{2} - \sqrt{3} \in \mathbbm{Q}( \sqrt{2}
+ \sqrt{3})$. Now $\frac{\sqrt{2} + \sqrt{3}}{2} + \frac{\sqrt{2} -
\sqrt{3}}{2} = \sqrt{2} \in \mathbbm{Q}( \sqrt{2} + \sqrt{3})$ and
$\frac{\sqrt{2} + \sqrt{3}}{2} - \frac{\sqrt{2} - \sqrt{3}}{2} = \sqrt{3} \in
\mathbbm{Q}( \sqrt{2} + \sqrt{3})$, so that $\mathbbm{Q}( \sqrt{2} + \sqrt{3})
\supseteq \mathbbm{Q}( \sqrt{2}, \sqrt{3})$. Both inclusions hold and we have
our desired result.

III.4.1: If $f (x) \in F [x]$ and $K$ is a splitting field for $f (x)$ over
$F$, denote by $S$ the set of distinct roots of $f (x)$ in $K$ and let $G = G
(K : F)$. If $f (x)$ is irreducible over $F$ show that $G$ is transitive on
$S$. If $f (x)$ has no repeated roots and $G$ is transitive on $S$ show that
$f (x)$ is irreducible over $F$.

{\tmem{Solution:}} Suppose that $f (x)$ is irreducible over $F$. Let $s_i, s_j
\in S$, then we know by corollary to proposition 1.9 and 1.10 that there is an
$F$-automorphism of $K, \phi,$ such that $\phi (s_i) = s_j$. Yet $\phi \in G$
and with $s_i$ and $s_j$ arbitrary, we have that $G$ acts transitively on $S$.
Conversely, suppose that $G$ is transitive on $S$ and $f (x)$ has no repeated
roots. Let $g \in F [x]$ be a nonconstant irreducible factor of $f$, then we
know that $g (s_i) = 0$ for some $s_i \in S$. Since $G$ is transitive on $S$,
then we know there exists a $\phi \in G$ such that $\phi (s_i) = s_j \in S$.
But this means that $g (s_j) = 0$, by corollary to proposition 1.9, hence
$\deg g \geq |S|$ and it follows that $f$ must be irreducible.

III.8.3: Suppose that $K$ is an algebraic extension of $F$ and $R$ is a ring,
with $F \subseteq R \subseteq K$. Show that $R$ is a field.

{\tmem{Solution:}} It suffices to show that $R$ is closed under multiplicative
inverses since commutativity and unity follow directly from the field
inclusions. So, let $r \in R$, then $r^{- 1} \in K$ since $K$ is a field that
contains $r$. But $K$ is also algebraic over $F$ which means for some $f (x)
\in F [x]$, $f (r) = 0 = a_0 + a_1 r + a_2 r^2 + \cdots + a_n r^n$. Hence $a_0
= - \sum^n_{i = 1} a_i r^i$ and multiplying both sides by $r^{- 1}$ and
$a_0^{- 1}$, which we know exists since $a_0$ is in the field $F$, we achieve
$r^{- 1} = - a^{- 1}_0 \sum^n_{i = 1} a_i r^{i - 1}$ and $r^{- 1}$ is clearly
in $R$.

III.8.11: Determine the Galois groups (over $\mathbbm{Q}$) of the following
polynomials:

(a) $x^3 - 1$

{\tmem{Solution:}} The splitting field for $x^3 - 1$ over $\mathbbm{Q}$ is
$\mathbbm{Q}(\omega)$ and the roots are $1, \omega, \omega^2$, where $\omega$
is the primitive 3rd root of unity. So with $1 \in \mathbbm{Q}$ then any
$\mathbbm{Q}$-automorphism of $\mathbbm{Q}(\omega)$ is going to fix 1. Hence
any $\mathbbm{Q}$-automorphism of $\mathbbm{Q}(\omega)$ is dictated by where
$\omega$ is sent. This leads us to only two possible mappings $\omega \mapsto
\omega$ and $\omega \mapsto \omega^2$. Thus $G (\mathbbm{Q}(\omega) :
\mathbbm{Q}) \cong \mathbbm{Z}_2$.

(b) $x^5 - 1$

{\tmem{Solution:}} From homework 1, we have that $\mathbbm{Q}(\omega)$ is the
splitting field for $x^5 - 1$ over $\mathbbm{Q}$, where $\omega$ is the
primitive 5th root of unity. Similar to part (a), we know for any $\phi \in G
(\mathbbm{Q}(\omega) : \mathbbm{Q})$, $\phi (\omega) = \omega^i$, where $1
\leq i \leq 4$. Once $i$ is chosen then $\phi (\omega^2), \phi (\omega^3),
\tmop{and} \phi (\omega^4)$ are determined. We use a table to illustrate all
possibilities

\begin{tabular}{|c|c|c|c|c|}
  & $\omega$ & $\omega^2$ & $\omega^3$ & $\omega^4$\\
  \hline
  $1$ & $\omega$ & $\omega^2$ & $\omega^3$ & $\omega^4$\\
  \hline
  $\phi$ & $\omega^2$ & $\omega^4$ & $\omega$ & $\omega^3$\\
  \hline
  $\theta$ & $\omega^3$ & $\omega$ & $\omega^4$ & $\omega^2$\\
  \hline
  $\sigma$ & $\omega^4$ & $\omega^3$ & $\omega^2$ & $\omega$\\
  \hline
\end{tabular} but notice that $\phi^2 = \sigma$, $\phi^3 = \theta$, and
$\phi^4 = 1$. Thus $\langle \phi \rangle = G (\mathbbm{Q}(\omega) :
\mathbbm{Q}) \cong \mathbbm{Z}_4$.

(c) $x^6 - 1$

{\tmem{Solution:}} Once again the splitting field is $\mathbbm{Q}(\omega)$,
where $\omega$ is the primitive 6th root of unity. We know that
$[\mathbbm{Q}(\omega) : \mathbbm{Q}] = \phi (6) = 2$ and thus $G
(\mathbbm{Q}(\omega) : \mathbbm{Q}) \cong \mathbbm{Z}_2$ by fundamental
theorem of galois theory. We could also have noted that any $\theta \in G
(\mathbbm{Q}(\omega) : \mathbbm{Q})$ sends $\omega \mapsto \omega^i$, where $1
\leq i \leq 5$. Since 2, 3, and 4 are not relatively prime to 5, it follows
that when $i = 2, 3, 4$ then $\theta (\omega^k) = 1 \neq \omega^j$ for some $1
\leq j, k \leq 5$. Hence only the automorphisms which takes $\omega \mapsto
\omega^5$ and $\omega \mapsto \omega$ will be contained in $\text{$G
(\mathbbm{Q}(\omega) : \mathbbm{Q})$}$.

(d) $x^3 - 2$

{\tmem{Solution:}} The splitting field of $x^3 - 2$ over $\mathbbm{Q}$ is
$\mathbbm{Q}( \sqrt[3]{2}, \omega)$, since $\sqrt[3]{2}, \omega \sqrt[3]{2},
\tmop{and} \omega^2 \sqrt[3]{2}$ are the roots, where $\omega$ is the
primitive 3rd root of unity. With 3 distinct roots, we know that $G
(\mathbbm{Q}( \sqrt[3]{2}, \omega) : \mathbbm{Q})$ can be viewed as a subgroup
of $S_3$ (page 96, statement before Ex. 4.1). Yet, $[\mathbbm{Q}( \sqrt[3]{2})
: \mathbbm{Q}] = 3$ and $[\mathbbm{Q}(\omega) : \mathbbm{Q}( \sqrt[3]{2})] =
2$, so that $[\mathbbm{Q}( \sqrt[3]{2}, \omega) : \mathbbm{Q}] = 6 = |G
(\mathbbm{Q}( \sqrt[3]{2}, \omega) : \mathbbm{Q}) | = |S_3 |$ by proposition
1.1 and corollary to theorem 2.10. Thus $S_3 \cong G (\mathbbm{Q}(
\sqrt[3]{2}, \omega) : \mathbbm{Q})$.

(e) $x^4 - 2$

{\tmem{Solution:}} The splitting field of $x^4 - 2$ over $\mathbbm{Q}$ is
$\mathbbm{Q}( \sqrt[4]{2}, i)$. We have $[\mathbbm{Q}( \sqrt[4]{2}, i) :
\mathbbm{Q}] = [ \text{$\mathbbm{Q}( \sqrt[4]{2}, i)$} : \mathbbm{Q}(
\sqrt[4]{2})] [\mathbbm{Q}( \sqrt[4]{2} : \mathbbm{Q}] = 2 \cdot 4 = 8 = |G
(\mathbbm{Q}( \sqrt[4]{2}, i) : \mathbbm{Q}) |$. Furthermore, any $\phi \in G
( \mathbbm{Q}( \sqrt[4]{2}, i) : \mathbbm{Q})$ is dictated by $\phi (
\sqrt[4]{2}) \in \{ \sqrt[4]{2}, - \sqrt[4]{2}, i, - i\}$ and $\phi (i) \in
\{i, - i\}$. We choose the particular $\sigma, \theta \in G ( \mathbbm{Q}(
\sqrt[4]{2}, i) : \mathbbm{Q})$ such that $\sigma ( \sqrt[4]{2}) =
\sqrt[4]{2}$, $\sigma (i) = - i$ and $\theta ( \sqrt[4]{2}) = i \sqrt[4]{2},
\theta (i) = i$ to analyze. First, direct calculation shows that $\sigma^2 =
1$ and $\theta^4 = 1$, now if we show that $\sigma \theta \sigma = \theta^{-
1}$, then we have an epimorphism from $D_4$ onto $\langle \theta, \sigma
\rangle$ by theorem 10.2 and consequently an isomorphism. So, $\sigma \theta
\sigma ( \sqrt[4]{2}) = \sigma \theta ( \sqrt[4]{2}) = \sigma (i \sqrt[4]{2})
= \sigma (i) \sigma ( \sqrt[4]{2}) = - i \sqrt[4]{2} = \theta^{- 1} (
\sqrt[4]{2})$ and $\sigma \theta \sigma (i) = \sigma \theta (- i) = \sigma (-
i) = i = \theta^{- 1} (i)$. With $\langle \theta, \sigma \rangle \leq G
(\mathbbm{Q}( \sqrt[4]{2}, i) : \mathbbm{Q})$ and $| \langle \theta, \sigma
\rangle | = |G (\mathbbm{Q}( \sqrt[4]{2}, i) : \mathbbm{Q}) |$ then $G
(\mathbbm{Q}( \sqrt[4]{2}, i) : \mathbbm{Q}) \cong \langle \theta, \sigma
\rangle \cong D_4$.

III.8.14 Suppose $L$ is a finite Galois extension of $F$, and $F \subseteq E
\subseteq K$, $F \subseteq L \subseteq K$. Show that the join $E \vee L$ is
finite and Galois over $E$, that $L$ is Galois over $E \cap L$, and that $G (E
\vee L : E) \cong G (L : E \cap L)$.

{\tmem{Solution:}} $L$ is finite over $F$, hence by theorem 1.4, $L$ is also
algebraic over $F$. Now, by theorem 3.4, $L$ is a separable splitting field
for some set $\mathcal{F} \subseteq F [x]$ since it's Galois over $F$.
Consequently since $L \subseteq E \vee L$, then $E \vee L$ is a separable
splitting field for $\mathcal{F} \subseteq F [x] \subseteq E [x]$ and thus $E
\vee L$ is Galois over $E$. The fact that $L$ is Galois over $E \cap L$
follows directly from part (2) of corollary following theorem 2.5, as $F
\subseteq E \cap L \subseteq L$ and with $[L : F]$ finite, then $[E \cap L :
F]$ is finite. To show that $G (E \vee L : E) \cong G (L : E \cap L)$, we
consider the map $f : G (E \vee L : E) \rightarrow G (L : F)$ such that $f
(\phi) = \phi_L$, where $\phi \in G (E \vee L : E)$ and $\phi_L$ is the
restriction to $L$. Straightforwardly, $f$ is a homomorphism and it's well
defined since we have $L$ a Galois extension of $F$. Consider $\phi \in G (E
\vee L : E)$ such that $f (\phi) = 1_L$. Now, $\phi$ must fix $E$, by
definition, but it must also fix $L$ by restriction, hence $\phi$ fixes $E
\vee L$ and is the identity in $G (E \vee L : E)$, so $f$ is injective. This
means that $f (G (E \vee L : E)) = G (L : X) \subseteq G (L : F)$, where $F
\subseteq X \subseteq L$. We observe that any $f (\phi) \in G (L : X)$ fixes
all elements of $E$ and consequently those elements of $E$ which are in $L$ as
well, thus $G (L : X) \subseteq G (L : E \cap L)$. Yet $\theta \in G (L : E
\cap L)$ is an $L$-automorphism that fixes those elements in common with both
$E$ and $L$ in which we can extend to a $E \vee L$-automorphism, $\theta'$,
that continues to fix all elements in $E$, then $\theta' \in G (E \vee L : E)$
in which next we notice that $\theta = \theta'_L = f (\theta') \in G (L : X)$.
Ultimately, $X = E \cap L$ and by the fundamental homomorphism theorem, we
have our result.

III.8.15 Let $F =\mathbbm{C}$, let $K$ be the field $\mathbbm{C}(t)$ of
rational functions, and let $G = G (K : F)$. If $\phi$ and $\theta$ in $G$ are
determined by $\phi (t) = \zeta t$ and $\theta (t) = 1 / t$, where $\zeta \in
\mathbbm{C}$ is a primitive $n$th root of unity, show that $H = \langle \phi,
\theta \rangle \leq G$ is isomorphic with the dihedral group $D_n$. Show that
$\mathcal{F}H =\mathbbm{C}(t^n + t^{- n})$.

{\tmem{Solution:}} By a previous homework, we know that $D_n = \langle a, b |
a^n = b^2 = 1, b^{- 1} a b = a^{- 1} \rangle$. Furthermore, $f : \{a, b\}
\rightarrow \{\phi, \theta\}$ where $f (a) = \phi$ and $f (b) = \theta$ is
clearly a function that is onto. We now observe that $\theta^{- 1} \phi \theta
(t) = \theta^{- 1} \phi (1 / t) = \theta^{- 1} (\zeta / t) = t / \zeta =
\phi^{- 1} (t)$, $\phi^n (t) = \zeta^n t = t$ so $\phi^n = 1$ and $\theta^2
(t) = \frac{1}{1 / t} = t$ so $\theta^2 = 1$. Hence the generators of $H$
satisfy the relations defining $D_n$ and thus by proposition I.10.2, we will
have that there is a homomorphism from $D_n$ onto $H$. We know that since $H$
satisfies those relations that it can be written as $\theta^i \phi^j$, where
$0 \leq i \leq 1, 0 \leq j < n$, and much like the example below proposition
I.10.2, we have that $|H| = 2 n$ and $H \cong D_n$. To find the fixed field of
$H$, we first note that $\mathcal{F} H \subseteq \mathbbm{C}(t)$ and that $[
\mathcal{F} H : \mathbbm{C}] = 2 n = |H|$ by the fundamental theorem of Galois
theory. Noticing that $t^n + t^{- n}$ is fixed under $\phi$ and $\theta$,
since $\phi (t^n + t^{- n}) = [\phi (t)]^n + [\phi (t)]^{- n} = \omega^n t^n +
\omega^{- n} t^{- n} = t^n + t^{- n}$ and $\theta (t^n + t^{- n}) = [\theta
(t)]^n + [\theta (t)]^{- n} = (t^{- 1})^n + (t^{- 1})^{- n} = t^{- n} + t^n =
t^n + t^{- n}$ , it's clear that $\mathbbm{C}(t^n + t^{- n}) \subseteq
\mathcal{F} H \subseteq \mathbbm{C}(t)$. Now we use proposition VI.5.5, $t^n +
t^{- n} = \frac{t^{2 n} + 1}{t^n} \in \mathbbm{C}(t) \setminus \mathbbm{C}$
and $t^n + t^{- n}$ is transcendental over $\mathbbm{C}$ while
$[\mathbbm{C}(t^n + t^{- n}) : \mathbbm{C}] = \max \{\deg (t^{2 n} + 1), \deg
(t^n)\}= 2 n$ since $t^{2 n} + 1$ and $t^n$ have no common divisors other than
1. Thus we have $[\mathbbm{C}(t^n + t^{- n}) : \mathbbm{C}] = [ \mathcal{F} H
: \mathbbm{C}]$ with $\mathbbm{C}(t^n + t^{- n}) \subseteq \mathcal{F} H$, so
it must be that $\mathcal{F} H =\mathbbm{C}(t^n + t^{- n})$.

III.8.20: Let $s$ and $t$ be distinct indeterminates over $\mathbbm{Z}_p$, and
let $F =\mathbbm{Z}_p (s, t)$, the field of rational functions in $s$ and $t$.
Let $a$ and $b$ be roots of $x^p - s$ and $x^p - t$, respectively, in some
extension of $F$, and let $K = F (a, b)$. Show that $K$ is not a simple
extension of $F$.

{\tmem{Solution:}} By corollary 3 to proposition 3.2, we have $x^p - s$ is
irreducible over $F$, so let $E = F (a)$ and $x^p - t$ is irreducible over
$E$, so the $K = F (a, b)$. By proposition 1.1 and 1.3, we have that $[K : F]
= [K : E] [E : F] = p \cdot p = p^2$. We want to show that the number of
intermediate fields is infinite between $K$ and $F$, so that by theorem 3.8,
we have that $K$ is not a simple extension of $F$. We consider $f \in F$ and
the subfield $F (f a + b) \subseteq K$. Yet $(f a + b)^p = f^p s + t \in F$
and $[F (f a + b) : F] \leq p$. Consider $f \neq g \in F$ and suppose for the
sake of contradiction that $F (f a + b) = F (g a + b)$. Then $f a + b, g a + b
\in F (f a + b)$ which implies that $(f a + b) - (g a + b) = (f - g) a$ and $a
\in F (f a + b)$. But then $(f a + b) - f a = b \in F (f a + b)$ and $F (f a +
b) = K$. Contradiction since $[F (f a + b) : F] = [K : F] = p^2 \nleq p$. Thus
with an infinite number of $f \in F$ and $F \subseteq F (f a + b) \subseteq
K$, we have achieved our result.

III.8.23: Suppose $F_q$ and $F_r$ are Galois fields, with $q = p^m$ and $r =
p^n$, $p$ prime. Show that $F_q$ has a subfield (isomorphic with) $F_r$ if and
only if $n | m$.

{\tmem{Solution:}} Suppose that $F_q$ has a subfield isomorphic with $F_r$,
which we will just call $F_r$ for all intents and purposes. Now by proposition
1.1 and theorem 3.11, we have $[F_q : F_p] = n = [F_q : F_r] [F_r : F_p] =
[F_q : F_r] m$. Thus $n | m$. Conversely, suppose that $n | m$. By theorem
3.11, we have that $G (F_q : F_p)$ is cyclic of order $m$. Furthermore, we
know that there exists a unique $H \leq G (F_q : F_p)$ of order $m / n$. Since
$F_p \subseteq \mathcal{F} H \subseteq F_q$, we have by the fundamental
theorem of galois theory that $[G (F_q : F_p) : H] = m / (m / n) = n = [
\mathcal{F} H : F_p]$. Once again by theorem 3.11, we have that $\mathcal{F}
H$ is a unique field isomorphic to $F_{p^n} = F_r$.

III.8.24: List all subfields of $F_q$ if $q = 2^{20}$; if $q = p^{30}$, $p$
prime.

{\tmem{Solution:}} Up to isomorphism and by problem 23 above; if $q = 2^{20}$,
then $F_{2^{}}, F_{2^2}, F_{2^4}, F_{2^5}, F_{2^{10}}, F_{2^{20}}$. If $q =
p^{30}$, then $F_p, F_{p^2}, F_{p^3}, F_{p^5}, F_{p^6}, F_{p^{10}},
F_{p^{15}}, F_{p^{30}}$.

III.8.26: If $0 < k \in \mathbbm{Z}$ show that there exists an irreducible
polynomial of degree $k$ over any Galois field $F_q$.

{\tmem{Solution:}} First we note that any finite field, $F$, is not
algebraically closed since there are no roots in $F$ for the polynomial $[
\prod_{a \in F} (x - a)] + 1 \in F [x]$. So by theorem 1.12, let $A$ be the
algebraic closure of $F_q$ and consider the subfield, $F_{q^k}$. By theorem
3.11, we have $[F_{q^k} : F_q] = k$ and by both theorem 3.8 and Pr. 23, we
have $F_{q^k}$ is simple, which means that for a primitive element $a \in
F_{q^k}$ such that $F_{q^k} = F_q (a)$, $\deg m_{a, F_q} (x) = k$ and we are
done.

III.8.37: Show that $f (x) = x^7 - 7 x^6 - 189 x^4 + 1701 x^2 - 7 \in
\mathbbm{Q}[x]$ is not solvable by radicals.

{\tmem{Solution:}} This problem will follow similar to the example on pg 101
of Grove's book. First, we note that $f$ is irreducible over $\mathbbm{Q}$ by
Eisenstein's by $p = 7$. To simplify things, we use a computer program like
{\tmem{maxima}} to quickly determine the roots of $f$, or more precisely
approximations of the roots, which is good enough for our purposes here as we
only want to count the number of real and complex roots of $f$. For $f$, we
have $- \frac{86917673}{33554432}, - \frac{2153011}{33554432},
\frac{2153011}{33554432}, \frac{92967571}{33554432},
\frac{303765015}{33554432}, - 1.12 - 4.99 i, - 1.12 + 4.99 i$. Thus $f$ has
$5$ real roots and $2$ complex ones. Considering $G (K : \mathbbm{Q})$ as the
Galois group of $f$, we know that $G = G (K : \mathbbm{Q}) \leq S_7$.
Moreover, since $7 | [K : \mathbbm{Q}]$, which we can see since we first
adjoin an element of degree $7$ since $f$ is irreducible, then $7 | |G|$ and
by Cauchy's theorem, $G$ contains an element of order 7. Yet the only elements
in $S_7$ of order 7 are 7-cycles, hence $G$ contains a 7-cycle. Furthermore,
$G$ also contains a 2-cycle induced by complex conjugation by the reasoning
given in the book that follows from theorem 3.5(c). Now, utilizing the
``trick'' mentioned in class and on pg 101, since $G$ contains a 7-cycle and
2-cycle, then $G = S_7$. Note the ``trick'' is a result that follows from
direct computation of conjugating the 2-cycle by the 7-cycle then continuously
taking the product of that and conjugating again by the 7-cycle. Eventually
this yields all transpositions of the form $(n - 1, n)$, which reduces the
problem to an exercise already done on a previous homework. Conclusively, we
know that $S_n$ is not solvable for $n \geq 5$, hence $G$ is not solvable, and
by theorem 4.6 (contrapositive), we have that $f$ is not solvable by radicals.

III.8.44: The discriminant of the general polynomial $f (x) = \prod \{x - x_i
: 1 \leq i \leq n\}$ is $D = \prod \{(x_i - x_j)^2 : 1 \leq i \leq j \leq
n\}$. Write $D$ as a polynomial in $\sigma_1, \ldots, \sigma_n$ in the cases
$n = 2$ and $n = 3$.

{\tmem{Solution:}} When $n = 2$: $f (x) = (x - x_1) (x - x_2) = x^2 - (x_1 +
x_2) x + x_1 x_2$. So $\sigma_1 = x_1 + x_2$ and $\sigma_2 = x_1 x_2$. We note
that $\sigma^2_1 = x_1^2 + x^2_2 + 2 x_1 x_2 \Rightarrow x_1^2 + x_2^2 =
\sigma^2_1 - 2 x_1 x_2 = \sigma_1^2 - 2 \sigma_2$ Now $D = (x_1 - x_2)^2 =
x_1^2 + x^2_2 - 2 x_1 x_2 = \sigma^2_1 - 2 \sigma_2 - 2 \sigma_2 = \sigma_1^2
- 4 \sigma_2$ by substition.

When $n = 3$: $\sigma_1 = x_1 + x_2 + x_3, \sigma_2 = x_1 x_2 + x_1 x_3 + x_2
x_3, \sigma_3 = x_1 x_2 x_3$.

$D = (x_1 - x_2)^2 (x_1 - x_3)^2 (x_2 - x_3)^2 = x_1^4 x_2^2 + x_1^2 x_2^4 +
x_2^4 x_3^2 - 2 x_1^3 x_2^3 + x_1^4 x_3^2 + x_1^2 x_3^4 - 2 x_1^3 x_3^3 +
x_2^2 x_3^4 - 2 x_2^3 x_3^3 - 6 x_1^2 x_2^2 x_3^2 + 2 x_1^3 x_2^2 x_3 - 2 x_1
x_2^4 x_3 + 2 x_1^2 x_2^3 x_3 + 2 x_1 x_2^2 x_3^3 + 2 x_1^3 x_2 x_3^2 - 2 x_1
x_2 x_3^4 + 2 x_1^2 x_2 x_3^3 - 2 x_1^4 x_2 x_3 + 2 x_1 x_2^3 x_3^2
= - 4 x_1^3 x_2^3 - 4 x_2^3 x_3^3 - 4 x_1^3 x_3^3 - 12 x_1^2 x_2^3 x_3 - 12
x_1^3 x_2^2 x_3 - 12 x_1 x_2^3 x_3^2 - 24 x_1^2 x_2^2 x_3^2 - 12 x_1^3 x_2
x_3^2 - 12 x_1 x_2^2 x_3^3 - 12 x_1^2 x_2 x_3^3 - 27 x_1^2 x_2^2 x_3^2 + x_1^4
x_2^2 + x_1^4 x_3^2 + x_1^2 x_2^4 + x_2^4 x_3^2 + 2 x_1^3 x_2^3 + 2 x_2^3
x_3^3 + 2 x_1^3 x_3^3 + x_1^2 x_3^4 + x_2^2 x_3^4 + 8 x_1^3 x_2^2 x_3 + 2
x_1^4 x_2 x_3 + 15 x_1^2 x_2^2 x_3^2 + 8 x_1^3 x_2 x_3^2 + 2 x_1 x_2^4 x_3 + 8
x_1^2 x_2^3 x_3 + 8 x_1 x_2^3 x_3^2  + 8 x_1 x_2^2 x_3^3 + 8 x_1^2 x_2
x_3^3 + 2 x_1 x_2 x_3^4 - 4 x_1^4 x_2 x_3 - 4 x_1 x_2^4 x_3 - 4 x_1 x_2 x_3^4
- 12 x_1^2 x_2^3 x_3 - 12 x_1^2 x_2 x_3^3 - 12 x_1^3 x_2^2 x_3 - 24 x_1^2
x_2^2 x_3^2 - 12 x_1^3 x_2 x_3^2 - 12 x_1 x_2^2 x_3^3 - 12 x_1 x_2^3 x_3^2 +
18 x_1^3 x_2^2 x_3 + 18 x_1^3 x_2 x_3^2 + 18 x_1^2 x_2^3 x_3 + 18 x_1 x_2^3
x_3^2 + 18 x_1^2 x_2 x_3^3 + 18 x_1 x_2^2 x_3^3 + 54 x_1^2 x_2^2 x_3^2 $
$= - 4 \sigma_2^3 - 27 \sigma_3^2 + \sigma_1^2 \sigma_2^2 - 4 \sigma_1^3
\sigma_3 + 18 \sigma_1 \sigma_2 \sigma_3$.

III.8.45: If char $F \neq 3$ show that a monic cubic polynomial $f (x) = a_0
+ a_1 x + a_2 x^2 + x^3 \in F [x]$ can be written as $f (x) = g (y) = y^3 + a
y + b$ by means of the substitution $x = y - a_2 / 3$. Observe that $f (x)$
and $g (y)$ have the same Galois group over $F$.

{\tmem{Solution:}} We note that $x^2 = y^2 - \frac{2 b_2}{3} y +
\frac{a^2_2}{9}$ and $x^3 = y^3 - a_2 y^2 + \frac{a_2^2}{3} y -
\frac{a_2^3}{27}$. Now

$f (x) = a_0 + a_1 x + a_2 x^2 + x^3 = a_0 + a_1 y - \frac{a_1 a_2}{3} + a_2
y^2 - \frac{2 a_2^2}{3} y + \frac{a_2^3}{9} + y^3 - a_2 y^2 + \frac{a_2^2}{3}
y - \frac{a_2^3}{27}$

$= (a_0 - \frac{a_1 a_2}{3} + \frac{2 a_2^3}{27}) + (a_1 - \frac{a_2^2}{3}) y
+ y^3 = a + b y + y^3 = g (y)$ by letting $a = a_0 - \frac{a_1 a_2}{3} +
\frac{2 a_2^3}{27}$ and $b = a_1 - \frac{a_2^2}{3}$. Furthermore, once we have
found the roots, $y_i$, of $g (y)$, then $x_i = y_i - a_2 / 3$ are the roots
of $f (x)$ and it's clear that any permutation of $y_i$ is going to permuate
$x_i$ in the same fashion.

III.8.46: Suppose $F$ is a field and char $F \neq 2$, $f (x) = x^3 + a x + b
\in F [x]$, and $F \subseteq K$ is a splitting field for $f (x)$ over $F$.
Write $D$ for the discriminant of $f (x)$ and let $d$ be a square root for
$D$.

(1) Show that $D = - 4 a^3 - 27 b^2$ and that $d \in K$.

{\tmem{Solution:}} We first show that $d = \sqrt{D} \in K$, where $K$ is the
splitting field for $g (x) = (x - x_i) \in F [x]$, $1 \leq i \leq 3$. It
follows that $x_1, x_2, x_3 \in K$ and $D = (x_1 - x_2)^2 (x_1 - x_3)^2 (x_2 -
x_3)^2$, by above, so that $d = (x_1 - x_2) (x_1 - x_3) (x_2 - x_3)$, which is
clearly in $K$. Since the result holds for the general polynomial, $g (x)$, it
must also hold for $f (x)$. Now, we can use Newton's identities, proposition
VI.9.2, and proposition VI.9.3 to show that $D = - 4 a^3 - 27 b^2$, and this
is exactly the method used in example 2 on page 278. Referring to the
determinant matrix of proposition VI.9.3 and with $n = 3$, we need to compute
up to $s_4$. Since $\sigma_1 = 0, \sigma_2 = a, \sigma_3 = - b$ then: $s_1 =
s_0 \cdot 0 = 0$, $s_2 = s_1 \cdot \sigma_1 - s_0 \cdot \sigma_2 = - 2 a$,
$s_3 = s_2 \cdot \sigma_1 - s_1 \cdot \sigma_2 + s_0 \cdot \sigma_3 = - 3 b$,
and $s_4 = s_3 \cdot \sigma_1 - s_2 \cdot \sigma_2 + s_1 \cdot \sigma_3 = 2
a^2$. Now \ $\left|\begin{array}{ccc}
  3 & 0 & - 2 a\\
  0 & - 2 a & - 3 b\\
  - 2 a & - 3 b & 2 a^2
\end{array}\right| = - 4 a^3 - 27 b^2 = D$.

(2) If $f (x)$ is irreducible show that $G = G (K : F)$ is the alternating
group $A_3$ if and only if $d \in F$.

{\tmem{Solution:}} We consider the sign homomorphism from page 17, $\sigma :
S_3 \rightarrow \{- 1, 1\}$, where $\sigma (\tau) = 1$ when $\tau$ is even and
$\sigma (\tau) = - 1$ when $\tau$ is odd and note that $G \leq S_3$. Now we
want to show that for all $\phi \in G$, that $\phi (d) = \sigma (\phi) \cdot
d$, observing here the small abuse of notation as we use $\phi$ as the field
automorphism on the LHS and for the corresponding element in $S_3$ on the RHS.
Since $\phi \in G$ permutes the $x_i$'s in $(x_1 - x_2) (x_1 - x_3) (x_2 -
x_3)$, we can list all possibilities in a table:

\begin{tabular}{|l|l|l|}
  \hline
  $\phi$ & $\sigma (\phi)$ & $\phi (d)$\\
  \hline
  (1) & $1$ & $d$\\
  \hline
  (12) & $- 1$ & $(x_2 - x_1) (x_2 - x_3) (x_1 - x_3)$\\
  \hline
  (23) & $- 1$ & $(x_1 - x_3) (x_1 - x_2) (x_3 - x_2)$\\
  \hline
  (13) & $- 1$ & $(x_3 - x_2) (x_3 - x_1) (x_2 - x_3)$\\
  \hline
  (123) & $1$ & $(x_2 - x_3) (x_2 - x_1) (x_3 - x_1)$\\
  \hline
  (213) & $1$ & $(x_3 - x_1) (x_3 - x_2) (x_1 - x_2)$\\
  \hline
\end{tabular} quick observation tell us that $\sigma (\phi)$ is exactly the
product of -1's that transform $d$ into $\phi (d)$ and thus we have our
equality. Now suppose that $G \cong A_3$, then for all $\phi \in G$, $\phi (d)
= d$, so $d$ is fixed so $d \in F$. Conversely, suppose that $d \in F$ and let
$\phi \in G$, then $\phi (d) = d = \sigma (\phi) \cdot d$, so $\sigma (\phi) =
1$. But $\sigma (\phi) = 1$ if and only if $\phi \in A_3$.

(3) Conclude, if $f (x)$ is irreducible, that $G$ is $A_3$ if $- 4 a^3 - 27
b^2$ is a square root in $F$, and otherwise $G$ is $S_3$.

{\tmem{Solution:}} Since $f (x)$ is irreducible of degree 3, then we know that
$G$ must be a transitive subgroup of $S_3$ by a previous homework problem. Yet
we know exactly which subgroups of $S_3$ are transitive by a previous homework
problem from last semester. Since $G$ cannot be the trivial group, it must be
either $A_3$ or $S_3$. We have by part (2) a necessary and sufficient
condition for $G$ being $A_3$, namely $d = \sqrt{D} = \sqrt{- 4 a^3 - 27 b^2}
\in F$, otherwise $G$ must be $S_3$.

IV.7.1: If $R$ is a ring with 1 and $M$ is an $R$-module that is not unitary
show that $R m = 0$ for some $0 \neq m \in M$.

{\tmem{Solution:}} Since $M$ is not unitary, then there exists an $0 \neq x
\in M$ such that $1 \cdot x \neq x$. This means that $1 \cdot x - x \neq 0$,
thus $1 \cdot x - x = m \in M$ and $m \neq 0$. Now for any $r \in R$, we have
$r \cdot m = r \cdot (1 \cdot x - x) = r \cdot (1 \cdot x) - r \cdot x = (r
\cdot 1) \cdot x - r \cdot x = r \cdot x - r \cdot x = 0$.

IV.7.2: Give an example of an $R$-module $M$ having $R$-isomorphic submodules
$N_1$ and $N_2$ such that $M / N_1$ and $M / N_2$ are not isomorphic.

{\tmem{Solution:}} Let $M =\mathbbm{Z}$, the group under addition, and $R
=\mathbbm{Z}$ as the commutative ring with 1. We then choose $N_1 =
2\mathbbm{Z}$ and $N_2 = 3\mathbbm{Z}$ as our submodules of $M$, which is
clear since $2\mathbbm{Z}$ and $3\mathbbm{Z}$ are maximal ideals in
$\mathbbm{Z}$ and thus subgroups of $M$ under addition. Once we have shown
that $N_1$ is $R$-isomorphic to $N_2$, then $M / N_1$ and $M / N_2$ cannot
possibly be isomorphic since they have differing finite cardinalities. So let
$\phi : N_1 \rightarrow N_2$ such that $\phi (2 k) = 3 k$. Let $2 k, 2 m \in
N_1$, so that $\phi (2 k + 2 m) = \phi (2 (k + m)) = 3 (k + m) = 3 k + 3 m =
\phi (2 k) + \phi (2 m)$ and $\phi$ is a group homomorphism. Furthermore, let
$\phi (2 k) = \phi (2 m)$, then $3 k = 3 m \Rightarrow k = m$ which means $2 k
= 2 m$ and $\phi$ is injective. Let $3 k \in N_2$, then there exists an
element in $N_1$, namely $2 k$, such that $\phi (2 k) = 3 k$ and $\phi$ is
onto. Now let $r \in R$ and $2 k \in N_1$, then $\phi (r 2 k) = \phi (2 r k) =
3 r k = r (3 k) = r \phi (2 k)$. Hence $N_1$ and $N_2$ are $R$-isomorphic.

IV.7.4: A sequence $K \overset{f}{\rightarrow} M \overset{g}{\rightarrow} K$
of $R$-homomorphisms of $R$-modules is exact at $M$ if $\tmop{Im} (f) = \ker
g$. A short exact sequence $0 \rightarrow K \rightarrow M \rightarrow N
\rightarrow 0$ is exact at $K, M, \tmop{and} N$. If $0 \rightarrow K
\rightarrow M \rightarrow N \rightarrow 0$ is short exact show that $M$ is
Noetherian if and only if $K$ and $N$ are both Noetherian.

{\tmem{Solution:}} Let $0 \overset{i}{\rightarrow} K \overset{f}{\rightarrow}
M \overset{g}{\rightarrow} N \overset{p}{\rightarrow} 0$ be a short exact
sequence. Now look at $0 \overset{i}{\rightarrow} K \overset{f}{\rightarrow}
M$, where $K$ is exact, so it must be that $\{0\}= \tmop{Im} (i) = \ker f$ and
$f$ must be injective. Moreover, $M \overset{g}{\rightarrow} N
\overset{p}{\rightarrow} 0$ where $N$ is exact means that $\tmop{Im} (g) =
\ker p = N$ and so $g$ must be onto. With $f$ an $R$-monomorphism, then we can
view $K$ as a submodule of $M$ and by proposition 1.7, $M$ is Noetherian if
and only if every submodule is finitely generated. Hence $K$ is a finitely
generated submodule of $M$ and any submodule of $K$ is also a submodule of $M$
and thus must be finitely generated, so by proposition 1.7, $K$ is also
Noetherian. With $g$ an $R$-epimorphism, then it follows by the FHTM that $M /
\ker g \cong N$ and it follows by proposition 1.8 that $M$ is Noetherian if
and only if $N$ and $\ker g$ are Noetherian.

IV.7.9: Suppose $R$ is a PID and $M = R \langle a \rangle$ is a cyclic
$R$-module of order $0 \neq r \in R$. Show that if $N$ is a submodule of $M$,
then $N$ is cyclic of order $s$ for some divisor $s$ of $r$. Conversely, $M$
has a cyclic submodule $N$ of order $s$ for each divisor $s$ of $r$ in $R$.

{\tmem{Solution:}} We first prove that $M [s] =\{x \in M : s x = 0\}$, where
$s \in R$, is a submodule of $M$. Let $x, y \in M [s]$ and $r \in R$, then $s
(x - y) = s x - s y = 0$, so $x - y \in M [s]$ and $s (r x) = (s r) x = (r s)
x = r (s x) = 0$ and $r x \in M [s]$. We first show that for each divisor $s$
of $r$ in $R$, we can construct $R / (r, s) \cong M [s]$, a cyclic submodule
of $M$, by proposition 3.5. Now, suppose that $N$ is a submodule of $M$, then
by theorem 2.9, rank$(N) \leq 1 = \tmop{rank} (M)$. Since $M \neq 0$, then
rank$(N) \neq 0$, hence rank$(N) = 1$, which by definition makes $N$ cyclic.
Since $N$ is now a cyclic submodule there exists an $x \in M$ of order $s \in
R$ such that $R \langle x \rangle = N$. By the corollary to proposition 3.5,
if $\gcd (r, s) = 1$, then $M [s] = 0$, so it must be that $0 \neq M [s] \cong
N$, when $s | r$. IV.7.10: Suppose $R$ is a commutative ring and $M$ is an
$R$-module. A submodule $N$ is called pure if $r N = r M \cap N$ for all $r
\in R$.

(i) Show that any direct summand of $M$ is pure.

{\tmem{Solution:}} Let's prove exercise 2.2 first since we will use it in our
solution: Show that $M = M_1 \oplus \cdots \oplus M_k$ if and only if each $x
\in M$ has a unique expression of the form $x = x_1 + x_2 + \cdots x_k$, for
some $k \in \mathbbm{N}$, with $x_i \in M_i$ .

{\tmem{Proof:}} Suppose $M = M_1 \oplus \cdots \oplus M_k .$ \ Then $M = M_1
+ \cdots + M_k$ and hence if $x \in M$, then $x = x_1 + \cdots + x_k, x_i \in
M_i$. Now assume $x = x_1' + \cdots + x_k'$, then $x_i - x_i' = (x_1 - x_1') +
\cdots + (x_{i - 1} - x_{i - 1}') + (x_{i + 1} - x_{i + 1}') + \cdots + (x_k -
x_k')$, so $x_i - x_i' \in M_i$ and $x_i - x_i' \in \sum \{M_j : 1 \leq j \leq
k, j \neq i\}$. Thus $x_i - x_i' \in M_i \cap \sum \{M_j : 1 \leq j \leq k, j
\neq i\}=\{0\}$ by theorem 2.3 and$x_i - x_i' = 0$ so each $x_i$ is uniquely
determined by $x$. \ Conversely, suppose that each $x \in M$ has a unique
expression of the form $x = x_1 + x_2 + \cdots x_k$, for some $k \in
\mathbbm{N}$, with $x_i \in M_i$. Then $x = x_1 + \cdots + x_k$ so $M = M_1 +
\cdots + M_k$. \ Let $x \in \bigcap_{i = 1}^k M_i$, then it holds for all $i$
that $x = 0 + \cdots + 0 + x + 0 + \cdots + 0$, where $x \in M_i$ and $0 \in
M_j$, $1 \leq j \leq k$ and $i \neq j$. But by distinctness $x = 0 \in
\bigcap_{i = 1}^k M_i$ and by theorem 2.3, $M = M_1 \oplus \cdots \oplus M_k$.
\ \ \

Now, let $M = M_1 \oplus \cdots \oplus M_k$ and consider any $M_i$ where $1
\leq i \leq k$ and $r \in R$. We want to show that $r M_i = r M \cap M_i$. So
let $x \in r M_i$, then $x = r x_i$ for some $x_i \in M_i$ and it follows that
since $M_i$ is a submodule then $x \in M_i$. Now $r M = r (M_1 + \cdots + M_k)
= r M_1 + \cdots + r M_k$ and clearly $x \in r M$. Hence $x \in r M$ and $x
\in M_i$ so $x \in r M \cap M_i$ and $r M_i \subseteq r M \cap M_i$. For the
reverse inclusion, let $x \in r M \cap M_i$, which means that $x \in r M$ and
$x \in M_i$. Since $x = r y \in r M$ for some $y \in M$, then $x = r (y_1 +
\cdots + y_k) = r y_1 + \cdots + r y_k$ and conclusively $x = r y_i$ since $y
\in M_i$. Thus $x \in r M_i$ and we have our desired result.

(ii) If $M$ is torsion free and $N$ is a pure submodule show that $M / N$ is
torsion free.

{\tmem{Solution:}} Suppose $r (x + N) = r x + N = N$ in $M / N$, then $r x \in
N$. Yet $r x \in M$ since $N \subseteq M$ and since $N$ is pure, $r x \in r N$
which implies $x \in N$. Thus $M / N$ is torsion free.

(iii) If $M / N$ is torsion free show that $N$ is pure.

{\tmem{Solution:}} Since $r M \cap N \subseteq r N$ is almost trivial by the
fact that if $x = r y \in r M$ and $x \in N$ says that $r y \in N$ and with
$N$ a submodule of $M$, $y \in N$ and $x \in r N$. Consider $r x \in r N$.
Then $r x \in r M$ by definition and $r x + N = r (x + N) = N \in M / N$ since
$M / N$ is torsion free. Hence $r x \in N$ and thus $r N \subseteq r M \cap
N$. Consequently, $r N = r M \cap N$ and $N$ is pure.

IV.7.17: If $p$ and $q$ are distinct primes use elementary divisors to
describe all abelian groups of order

(i) $p^2 q^2$

{\tmem{Solution:}} $\mathbbm{Z}_{p^2 q^2}, \mathbbm{Z}_{p q^2} \oplus
\mathbbm{Z}_p, \mathbbm{Z}_{p^2 q} \oplus \mathbbm{Z}_q, \mathbbm{Z}_{p q}
\oplus \mathbbm{Z}_{p q}$

(ii) $p^4 q$

{\tmem{Solution:}} $\mathbbm{Z}_{p^4 q}, \mathbbm{Z}_{p^3 q} \oplus
\mathbbm{Z}_p, \mathbbm{Z}_{p^2 q} \oplus \mathbbm{Z}_p \oplus \mathbbm{Z}_p,
\mathbbm{Z}_{p q} \oplus \mathbbm{Z}_p \oplus \mathbbm{Z}_p \oplus
\mathbbm{Z}_p, \mathbbm{Z}_{p^2 q} \oplus \mathbbm{Z}_{p^2}$

(iii) $p^n$, $1 \leq n \leq 6$

{\tmem{Solution:}} $p^1 : \mathbbm{Z}_p$ \ \ \ \ \ \ \ \ \ \ \ $p^2 :
\mathbbm{Z}_{p^2}, \mathbbm{Z}_p \oplus \mathbbm{Z}_p$ \ \ \ \ \ \ \ \ \ \ \ \
$p^3 : \mathbbm{Z}_{p^3}, \mathbbm{Z}_{p^2} \oplus \mathbbm{Z}_p,
\mathbbm{Z}_p \oplus \mathbbm{Z}_p \oplus \mathbbm{Z}_p$

$p^4 : \mathbbm{Z}_{p^4}, \mathbbm{Z}_{p^3} \oplus \mathbbm{Z}_p,
\mathbbm{Z}_{p^2} \oplus \mathbbm{Z}_{p^2}, \mathbbm{Z}_{p^2} \oplus
\mathbbm{Z}_p \oplus \mathbbm{Z}_p, \mathbbm{Z}_p \oplus \mathbbm{Z}_p \oplus
\mathbbm{Z}_p \oplus \mathbbm{Z}_p$

$p^5 : \mathbbm{Z}_{p^5}, \mathbbm{Z}_{p^4} \oplus \mathbbm{Z}_p,
\mathbbm{Z}_{p^3} \oplus \mathbbm{Z}_{p^2}, \mathbbm{Z}_{p^3} \oplus
\mathbbm{Z}_p \oplus \mathbbm{Z}_p, \mathbbm{Z}_{p^2} \oplus \mathbbm{Z}_{p^2}
\oplus \mathbbm{Z}_p, \mathbbm{Z}_{p^2} \oplus \mathbbm{Z}_p \oplus
\mathbbm{Z}_p \oplus \mathbbm{Z}_p, \mathbbm{Z}_p \oplus \mathbbm{Z}_p \oplus
\mathbbm{Z}_p \oplus \mathbbm{Z}_p \oplus \mathbbm{Z}_p $

$p^6 : \mathbbm{Z}_{p^6}, \mathbbm{Z}_{p^5} \oplus \mathbbm{Z}_p,
\mathbbm{Z}_{p^4} \oplus \mathbbm{Z}_{p^2}, \mathbbm{Z}_{p^4} \oplus
\mathbbm{Z}_p \oplus \mathbbm{Z}_p, \mathbbm{Z}_{p^3} \oplus
\mathbbm{Z}_{p^3}, \mathbbm{Z}_{p^3} \oplus \mathbbm{Z}_{p^2} \oplus
\mathbbm{Z}_p, \mathbbm{Z}_{p^3} \oplus \mathbbm{Z}_p \oplus \mathbbm{Z}_p
\oplus \mathbbm{Z}_p, \mathbbm{Z}_{p^2} \oplus \mathbbm{Z}_{p^2} \oplus
\mathbbm{Z}_{p^2}, \mathbbm{Z}_{p^2} \oplus \mathbbm{Z}_{p^2} \oplus
\mathbbm{Z}_p \oplus \mathbbm{Z}_p, \mathbbm{Z}_{p^2} \oplus \mathbbm{Z}_p
\oplus \mathbbm{Z}_p \oplus \mathbbm{Z}_p \oplus \mathbbm{Z}_p, \mathbbm{Z}_p
\oplus \mathbbm{Z}_p \oplus \mathbbm{Z}_p \oplus \mathbbm{Z}_p \oplus
\mathbbm{Z}_p \oplus \mathbbm{Z}_p $

IV.7.19: Suppose $R$ is a PID and $M$ is a finitely generated torsion
$R$-module. Show that the elementary divisors of $M$ are just the prime power
factors (with multiplicities) that result when the invariant factors of $M$
are factored into prime powers. Show on the other hand that the invariant
factors can be recovered from the list of elementary divisors as follows: the
last invariant factor is the LCM of all the elementary divisors. If the
elementary divisors whose product is the last invariant factor are removed
from the list, then the LCM of those that remain is the next invariant factor,
etc.

{\tmem{Solution:}} By theorem 3.7, $M = M_1 \oplus \cdots \oplus M_k$, for
some $k \in \mathbbm{N}$, where each $M_i$ is cyclic of order $r_i$ with $r_i
| r_{i - 1}, 2 \leq i \leq k$ and $r_1$ the exponent of $M$. The $r_i$'s are
the invariant factors and, moreover, $r_i = u_i p_1^{e_{i 1}} \cdots p_n^{e_{i
n}}$ by theorem II.5.11 since by proposition II.5.10, $R$ is a UFD. Now by
theorem 3.11, each $M_i = \bigoplus \{M_{i j} : 1 \leq j \leq n\}$, where each
$M_{i j}$ is cyclic of order $p_i^{e_{i j}}$, and the set $\{p_i^{e_{i j}} : 1
\leq i \leq k, 1 \leq j \leq n\}$ are the set of elementary divisors. Working
the other way, assume we have the set of elementary divisors $E =\{p_i^{e_{i
j}} : 1 \leq i \leq k, 1 \leq j \leq n\}$, and let $e_j = \min \{e_{1 j},
\ldots, e_{k j} \}$ where $1 \leq j \leq n$. Then $r_i = p_1^{e_1} \cdots
p_k^{e_k}$ is the LCM of all the elementary divisors and moreover by theorem
3.11, we have just created the exponent of some $M_i$ in $M = \bigoplus \{M_i
: 1 \leq i \leq k\}$. Yet we know by the ordering of the exponents of the
$M_i$'s, that $r_i | r_{i - 1}$ for $2 \leq i \leq k$ and thus $r_i = r_k$
since it is the LCM. Removing the set $\{p_1^{e_1}, \ldots, p_k^{e_k} \}$ from
$E$ and repeating the process, we yield $r_{k - 1}$. Eventually the process
will terminate since $E$ is finite and we will have reconstructed all
invariant factors from the elementary divisors.

IV.7.23: Find all integral solutions to the following systems $A X = B$ of
equations:

(iv) $A = \left(\begin{array}{ccc}
  8 & 19 & 30\\
  6 & 14 & 22
\end{array}\right), B = \left(\begin{array}{c}
  5\\
  7
\end{array}\right)$

{\tmem{Solution:}}

$\left(\begin{array}{ccccc}
  8 & 19 & 30 & | & 5\\
  6 & 14 & 22 & | & 7
\end{array}\right) \sim \left(\begin{array}{ccccc}
  2 & 0 & - 2 & | & 63\\
  0 & - 1 & - 2 & | & 13
\end{array}\right)$ by $R_{12 (- 1)}, R_{21 (- 3)}, R_{125}$ and so $2 x_1 - 2
x_3 = 2 (x_1 - x_3) \neq 63$ with $x_1, x_3 \in \mathbbm{Z}$. Hence there are
no integral solutions to this system.

IV.7.24: What are the invariant factors and the elementary divisors of a
diagonal matrix over a field $F$?

{\tmem{Solution:}} Letting the diagonal matrix, $D$, represent the linear
transformation $T : V \rightarrow V$, where $V$ is a finite dimensional vector
space over $F$, we have by theorem 4.6 that $m_T (x)$ splits as a product of
distinct linear factors in $F [x]$. Furthermore, the set of elementary
divisors is comprised of these linear factors of $m_T (x)$ and since each
linear factor of $m_T (x)$ is the exponent of its respective primary
submodule, then the set of elementary divisors is also the set of invariant
factors.

IV.7.27: Determine whether or not $A = \left(\begin{array}{ccc}
  3 & 0 & 2\\
  0 & 1 & - 1\\
  - 4 & 0 & 3
\end{array}\right)$ and $B = \left(\begin{array}{ccc}
  5 & - 8 & 4\\
  6 & - 11 & 6\\
  6 & - 12 & 7
\end{array}\right)$ are similar over $\mathbbm{Q}$.

{\tmem{Solution:}} Assuming that $A$ is similar to $B$ over $\mathbbm{Q}$,
then we would have $A = P^{- 1} B P$. It is a well known fact that $\det (P^{-
1}) = \det (P)^{- 1}$ and we have that $\det (A - x I) = \det (P^{- 1} B P - x
I) = \det (P^{- 1} (B - x I) P) = \det (B - x I)$ by properties of
determinants. Hence if $A$ is similar to $B$, then characteristic polynomial
of $A$ and $B$ are the same (or contrapositively, if the characteristic
polynomial of $A$ and $B$ are not the same, then $A$ and $B$ are not similar).
Yet $\det (A - x I) = - x^3 + 7 x^2 - 23 x + 17$ and $\det (B - x I) = - x^3 +
x^2 + x - 1$ means that $A$ and $B$ have two different characteristic
polynomials. Thus $A$ and $B$ are not similar.

IV.7.30: Find the characteristic polynomial, invariant factors, elementary
divisors, rational canonical form, and Jordan canonical form (when possible)
over $\mathbbm{Q}$ for each of the following matrices:

(ii) $\left(\begin{array}{cc}
  c + 6 & - 9\\
  4 & c - 6
\end{array}\right), c \in \mathbbm{Q}$

{\tmem{Solution:}} det($\left(\begin{array}{cc}
  c + 6 & - 9\\
  4 & c - 6
\end{array}\right) - x I)) = (x - c)^2$ and we have $(x - c)^2$ as our
characteristic polynomial, $\chi$. It follows that $(x - c)^2$ is also the
minimal polynomial, $m$, since $m| \chi$ and $\left(\begin{array}{cc}
  c + 6 & - 9\\
  4 & c - 6
\end{array}\right) - \left(\begin{array}{cc}
  c & 0\\
  0 & c
\end{array}\right)$ is not the zero matrix. It follows that $m$ is the only
invariant factor, so the rational canonical form is $\left(\begin{array}{cc}
  0 & - c^2\\
  1 & 2 c
\end{array}\right)$. $m$ also happens to be the elementary divisor and thus
the Jordan canonical form is $\left(\begin{array}{cc}
  c & 0\\
  1 & c
\end{array}\right)$.

IV.7.31: An $n \times n$ matrix $A$ over a field $F$ is called idempotent if
$A^2 = A$.

(i) What are the possible minimal polynomials for an idempotent matrix?

{\tmem{Solution:}} Since $A^2 = A \Rightarrow A^2 - A = 0$, then we have $x^2
- x = 0$. Thus the minimal polynomials are either $x, x - 1, \tmop{or} x^2 -
x$. \

(ii) Show that an idempotent matrix is similar over $F$ to a diagonal matrix?

{\tmem{Solution:}} By theorem 4.6, if the minimal polynomial of $A$ splits as
a product of distinct factors, then it can be represented by a diagonal
matrix. By part (i), we see that the minimal polynomial does in fact split
into linear factors.

(iii) Show that idempotent $n \times n$ matrices $A$ and $B$ are similar over
$F$ if and only if they have the same rank.

{\tmem{Solution:}} Matrices with the same rank will have equivalent row and
column space and consequently the same minimal polynomial, so their Jordan
canonical form is are the same, which means they are similar by theorem 4.5.

IV.7.35: Compute the invariants of the abelian groups with generators $a_1,
\ldots, a_n$ subject to the following relations:

(i) $n = 2, a_1 - 5 a_2 = 0, 3 a_1 + 7 a_2 = 0$

{\tmem{Solution:}} $\left(\begin{array}{cc}
  1 & 3\\
  - 5 & 7
\end{array}\right) \sim \left(\begin{array}{cc}
  1 & 0\\
  0 & 22
\end{array}\right)$ by the sequence $R_{215}$ and $C_{21 (- 3)}$ and thus the
invariant factors are $22$. So $G$ is isomorphic to $\mathbbm{Z}_{22}$.

IV.7.37: If $R$ is a commutative ring with 1 and $x_1, x_2$ are distinct
indeterminates show that $R [x_1, x_2]$ and $R [x_1] \otimes_R R [x_2]$ are
isomorphic as $R$-algebras. \

{\tmem{Solution:}} Let $\phi : R [x_1] \times R [x_2] \rightarrow R [x_1,
x_2]$ where $\phi (f, g) = f g$. We check it's $R$-balanced:

$\phi (f + f', g) = (f + f') g = f g + f' g = \phi (f, g) + \phi (f', g), \phi
(f, g + g') = f (g + g') = f g + f g' = \phi (f, g) + \phi (f, g'), \tmop{and}
\phi (f r, g) = (f r) g = f (r g) = \phi (f, r g)$ for all $f, f' \in R [x_1],
g, g' \in R [x_2] \tmop{and} r \in R$. Thus there is a group homomorphism
$\psi : R [x_1] \otimes R [x_2] \rightarrow R [x_1, x_2]$ where $\psi (f
\otimes g) = \phi (f, g) = f g$, by the universal property. Let $r \in R$,
then $\psi (r f \otimes g) = (r f) g = r (f g) = r \psi (f \otimes g)$ and
$\psi$ is a $R$-homomorphism. We check that $\psi$ is a ring homomorphism,
$\psi ((f \otimes g) (f' \otimes g')) = \psi (f f' \otimes g g') = f f' g g' =
f g f' g' = \psi (f \otimes g) \psi (f' \otimes g')$. We let $\sigma : R [x_1,
x_2] \rightarrow R [x_1] \otimes R [x_2]$, where $\sigma (f (x_1) g (x_2)) = f
\otimes g$. By similar arguments above, we have $\sigma$ a homomorphism of
rings and module structure. Furthermore, $\sigma \circ \psi = \psi \circ
\sigma$ is the identity mapping and so $\sigma$ and $\psi$ are inverses of
eachother and $R [x_1, x_2]$ is isomorphic to $R [x_1] \otimes_R R [x_2]$ as
$R$-algebras.

IV.7.41: If $F$ is a field and $K$ is an extension field show that $M_n (K)
\cong K \otimes_F M_n (F)$ (as $F$-algebras).

{\tmem{Solution:}} Let us show that $\phi : K \times M_n (F) \rightarrow M_n
(K)$, where $\phi (k, A) = k A$ is an $F$-balanced map:

$\phi (k + k', A) = (k + k') A = k A + k' A = \phi (k, A) + \phi (k', A)$,
$\phi (k, A + B) = k (A + B) = k A + k B = \phi (k, A) + \phi (k, B)$, and
$\phi (k f, A) = (k f) A = k (f A) = \phi (k, f A)$ for all $k, k' \in K, f
\in F, \tmop{and} A, B \in M_n (F)$. Thus, we have a group homomorphism, $\psi
: K \otimes M_n (F) \rightarrow M_n (K)$, where $\psi (k \otimes A) = \phi (k,
A) = k A$. Next, we show that $\psi$ is a ring homomorphism as well, $\psi ((k
\otimes A) (k' \otimes B)) = \psi (k k' \otimes A B) = k k' A B = k A k' B =
\psi (k \otimes A) \psi (k' \otimes B)$. $\psi$ being an $R$-homomorphism
follows from $\psi (f k \otimes A) = (f k) A = f (k A) = f \psi (k \otimes
A)$. Now we know from linear algebra and tensor products that $\dim_F (M_n
(K)) = [K : F] \dim_F (M_n (F)) = \dim_F (K) \dim_F (M_n (F)) = \dim_F (K
\otimes_F M_n (F))$ and thus vector spaces over the same base field of the
same dimension must be isomorphic. Thus $M_n (K) \cong K \otimes_F M_n (F)$ as
desired.

IV.7.42: Suppose that $F$ is a field, $K$ is a finite Galois extension, and
that $G = G (K : F)$ is a direct product of two subgroups $G_1$ and $G_2$. \
Set $L_1 =\mathcal{F}G_1$ and $L_2 =\mathcal{F}G_2$. \ Show that the ring $L_1
\otimes_F L_2$ is a field, with a subfield isomorphic with $F$, and that if
$L_1 \otimes_F L_2$ is viewed as an extension of $F$, then $K$ and $L_1
\otimes_F L_2$ are $F$-isomorphic.

{\tmem{Solution:}} First, $L_1$ and $L_2$ are vector spaces over $F$ (more
precisely $F$-algebras since $L_1$ and $L_2$ are both extensions of $F$). \
Let $\phi : L_1 \times L_2 \rightarrow K$, where $\phi (l_1, l_2) = l_1 l_2$.
\ We show that $\phi$ is $F$-balanced: $\phi (l_1 + l_1', l_2) = (l_1 + l_1')
l_2 = l_1 l_2 + l_1' l_2 = \phi (l_1 + l'_1, l_2')$, $\phi (l_1, l_2 + l_2') =
l_1 (l_2 + l_2') = l_1 l_2 + l_1 l_2' = \phi (l_1, l_2) + \phi (l_1, l'_2)$,
and $\phi (l_1 f, l_2) = (l_1 f) l_2 = (f l_1) l_2 = f (l_1 l_2) = f \phi
(l_1, l_2)$. Thus, there is a homomorphism $\psi : L_1 \otimes_f L_2
\rightarrow K$ by $\psi (l_1 \otimes l_2) = \psi (l_1, l_2)$. \ Let $f \in F$,
then $\psi (f l_1 \otimes l_2) = (f l_1) l_2 = f (l_1 l_2) = f \psi (l_1
\otimes l_2)$ so then $\psi$ is an $F$ homomorphism. Furthermore, $\psi ((l_1
\otimes l_2) (l_1' \otimes l_2')) = \psi (l_1 l_1' \otimes l_2 l_2') = l_1
l_1' l_2 l_2' = l_1 l_2 l_1' l_2' = \psi (l_1 \otimes l_2) \psi (l_1' \otimes
l_2')$ and so $\psi$ is also a ring homomorphism. Now $\dim_F (L_1 \otimes_F
L_2) = \dim_F (L_1) \dim_F (L_2) = \dim_{L_2} (K) \dim_F (L_2) = [K : L_2]
[L_2 : F] = [K : F] = \dim_F (K)$ and so $L \otimes_F L_2$ is isomorphic to
$K$ as an $F$-algebra. Notice we get the second equality by Galois
correspondence. It now follows directly that there is a subfield within $L_1
\otimes_F L_2$ isomorphic to $F$ since $K$ contains $F$ and an isomorphism of
algebras means there is an $F$-isomorphism between $L_1 \otimes_F L_2$ and
$K$.

IV.7.44: Suppose $R$ is a commutative ring and $L, M, N$ are $R$-modules. \
Show that $\tmop{Hom}_R (L, \tmop{Hom}_R (M, N))$ and $\tmop{Hom}_R (L
\otimes_R M, N)$ are isomorphic $R$-modules. \ Conclude in particular that $(L
\otimes_R M)^{\ast} \cong \tmop{Hom}_R (L, M^{\ast})$.

{\tmem{Solution:}} For brevity, let $X = \text{$\tmop{Hom}_R (L, \tmop{Hom}_R
(M, N))$}$ and $Y = \text{$\tmop{Hom}_R (L \otimes_R M, N)$}$. Using the hint,
we let $\phi : X \rightarrow Y$, where $\phi (f) = \hat{f}$ and $\hat{f} (a
\otimes b) = f_a (b) = [f (a)] (b)$, where $a \in L, b \in M$. Let $f, g \in
X$, then $\phi (f + g) = \widehat{f + g} = \hat{f} + \hat{g} = \phi (f) + \phi
(g)$, once we show that $\widehat{f + g} = \hat{f} + \hat{g}$. So consider $a
\otimes b \in L \otimes_R M$, then $( \widehat{f + g}) (a \otimes b) = [(f +
g)_a] (b) = [(f + g) (a)] (b) = [f (a) + g (a)] (b) = (f_a + g_a) (b) = f_a
(b) + g_a (b) = \hat{f} (a \otimes b) + \hat{g} (a \otimes b)$ and thus $\phi$
is a group homomorphism. Now let $r \in R$ and $f \in X$ so that $\phi (r f) =
\widehat{(r f)} = r \hat{f} = r \phi (f)$, when we show that $\widehat{(r f)}
= r \hat{f}$. So consider $a \otimes b \in L \otimes_R M$, then $\widehat{(r
f)} (a \otimes b) = [(r f) (a)] (b) = [f (r a)] (b) = f_{r a} (b) = \hat{f} (r
a \otimes b) = r \hat{f} (a \otimes b)$ and thus $\phi$ is an $R$-module
homomorphism. Using definition from exercise 13 and the faact that $R$ is an
$R$-module, $(L \otimes_R M)^{\ast} = \tmop{Hom}_R (L \otimes_R M, R) \cong
\tmop{Hom}_R (L, \tmop{Hom}_R (M, R)) = \text{$\tmop{Hom}_R (L, M^{\ast})$}$
follows directly from above.

\end{document}
