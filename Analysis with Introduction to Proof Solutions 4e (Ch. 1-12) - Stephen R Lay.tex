\documentclass[12pt]{article}

\pagestyle{empty}
\usepackage{wrapfig}
\usepackage{shapepar}
\usepackage{fullpage}
\usepackage{amssymb}
\usepackage{amsthm}
%\usepackage{calligra}
\usepackage{amsmath}
%\usepackage[symbol*]{footmisc}
%\usepackage[enableskew]{youngtab}
\usepackage[all]{xy}
\usepackage{graphicx}

\begin{document}
Analysis with an introduction to Proof (4th ed.) by Steven R. Lay
\begin{enumerate}

\item[1.1] Mark each statement True or False. Justify each answer.
\begin{enumerate}
\item[a)] In order to be classified as a statement, a sentence must be true.\\
False; "If a sentence can be classified as true or false, it is called a {\bf statement}"(pg.1)
\item[b)] Some statements are both true and false.\\
False; "For a sentence to be a statement...whether it is true or false, but it must clearly be the case that it is one or the other." (pg. 1)
\item[c)] When statement $p$ is true, then it's negation $\sim$$p$ is false. \\
True; "When $p$ is true, then $\sim$$p$ is false..." (pg. 3)
\item[d)] A statement and its negation may both be false.\\
False; when $p$ is false, then $\sim$$p$ is true.
\item[e)] In mathematical logic, the word "or" has an inclusive meaning.\\
True; by the truth table on pg. 4, when $p$ is true and $q$ is true then $p \vee q$ is true.
\end{enumerate}

\item[1.2] Mark each statement True or False. Justify each answer.
\begin{enumerate}
\item[a)] In an implication $p \Rightarrow q$, statement $p$ is referred to as the proposition.\\
False; "The if-statement $p$ in the implication is called the {\bf antecedent}..."(pg. 4)
\item[b)] The only case where $p \Rightarrow q$ is false is when $p$ is true and $q$ is false.\\
True; This follows from the truth table located on page 5.
\item[c)] "If $p$, then $q$" is equivalent to "$p$ whenever $q$".\\
False; should read "$q$ whenever $p$" (pg. 5)
\item[d)] The negation of a conjunction is the disjunction of the negations of the individual parts.\\
True; The solution to Practice Problem 1.6(a) proves that this is true.
\item[e)] The negation of $p \Rightarrow q$ is $q \Rightarrow p$.\\
False; The solution to Practice Problem 1.6(c) shows that this statement is false.
\end{enumerate}

\item[1.4] Write the negation of each statement.
\begin{enumerate}
\item[a)] The relation {\large R} is transitive.\\
\emph{Negation:} The relation {\large R} is not transitive.
\item[b)] The set of rational numbers is bounded.\\
\emph{Negation:} The set of rational numbers is unbounded. 
\item[c)] The function $f$ is injective and surjective.\\
\emph{Negation:} The function $f$ is not injective or not surjective. 
\item[d)] $x < 5$ or $x > 7$.\\
\emph{Negation:} $x \geq 5$ and $x \leq 7$. 
\item[e)] If $x$ is in $A$, then $f(x)$ is not in $B$.\\
\emph{Negation:} $x$ is in $A$ and $f(x)$ is in $B$. 
\item[f)] If $f$ is continuous, then $f(S)$ is closed and bounded.
\emph{Negation:} $f$ is continuous and $f(S)$ is not closed or unbounded. 
\item[g)] If $K$ is closed and bounded, then $K$ is compact.\\
\emph{Negation:} $K$ is closed and bounded and $K$ is not compact. 
\end{enumerate}

\item[1.8] Construct a truth table for each statement.
\begin{enumerate}
\item[a)]
\begin{tabular}{c|c|c|c}
\hline
$p$ & $q$ & $\sim$$p$ & $\sim$$p$$\vee$$q$ \\
\hline
T & T & F & T \\ 
T & F & F & F \\ 
F & T & T & T \\
F & F & T & T \\
\hline
\end{tabular}

\item[b)]
\begin{tabular}{c|c|c}
\hline
$p$ & $\sim$$p$ & $p$ $\wedge$ $\sim$$p$ \\
\hline
T & F & F \\
F & T & F \\
\hline
\end{tabular}

\item[c)]
\begin{tabular}{c|c|c|c|c|c|c}
\hline
$p$ & $q$ & $\sim$$p$ & $\sim$$q$ & $p \Rightarrow q$ & $\sim$$q$ $\wedge (p \Rightarrow q)$ & 
$[\sim$$q$ $\wedge (p \Rightarrow q)] \Rightarrow $$\sim$$p$\\
\hline
T & T & F & F & T & F & T \\
T & F & T & F & F & F & T \\
F & T & F & T & T & F & T \\
F & F & T & T & T & T & T \\
\hline
\end{tabular}
\end{enumerate}

\item[1.12] Let $p$ be the statement "Buford got a C on the exam," and let $q$ be the statement "Buford passed the class." Express each of the following statements in symbols.
\begin{enumerate}
\item[a)] Buford did not get a C on the exam, but he passed the class. \\
$\sim$$p \wedge q$
\item[b)] Buford got neither a C on the exam nor did he pass the class.\\
$\sim$$p$ $\wedge \sim$$q$
\item[c)] If buford passed the class, he did not get a C on the exam.\\
$q \Rightarrow \sim$$p$
\item[d)] It was necessary for Buford to get a C on the exam in order for him to pass the class.\\
$q \Rightarrow p$
\item[e)] Buford passed the class only if he got a C on the exam.\\
$q \Rightarrow p$
\end{enumerate}

\item[2.4] Write the negation of each statement.
\begin{enumerate}
\item[a)] Some basketball players at Central High are short.\\
\emph{Negation:} All basketball players at Central High are not short.
\item[b)] All of the lights are on.\\
\emph{Negation:} Some of the lights are not on.
\item[c)] No bounded interval contains infinite many integers.\\
\emph{Negation:} There exists a bounded interval that contains infinite many integers.
\item[d)] $\exists x \in S \ni x \geq 5$. \\
\emph{Negation:} $\forall x \in S, x < 5$
\item[e)] $\forall x \ni 0 < x < 1, f(x) < 2$ or $f(x) > 5$.\\
\emph{Negation:} $\exists x, 0 < x < 1$ such that $f(x) \geq 2 \wedge f(x) \leq 5$
\item[f)] If $x > 5$, then $\exists y > 0 \ni x^2 > 25 + y$.\\
\emph{Negation:} $x > 5$ and $\forall y > 0, x^2 \leq 25 + y$
\end{enumerate}

\item[2.6] Determine the truth value of each statement, assuming that $x, y$ and $z$ are real numbers.
\begin{enumerate}
\item[a)] $\forall x$ and $\forall y, \exists z \ni x + y = z$.\\
True; we can choose some $z$ to equal $x + y$.
\item[b)] $\forall x \exists y \ni \forall z, x + y = z$.\\
False; impossible for every $x$ and it's corresponding $y$ to be summed up to every $z$.
\item[c)] $\exists x \ni \forall y, \exists z \ni xz = y$.\\
True; we can choose $x = 1$ for any $y$ and $z = y$.
\item[d)] $\forall x$ and $\forall y, \exists z \ni yz = x$.\\
True; for any $x$ and $y$, we can choose $z = \frac{x}{y}$ as long as $y \not= 0$.
\item[e)] $\forall x \exists y \ni \forall z, z > y$ implies that $z > x + y$.\\
False; we choose $y = x$ then $\forall z, z > y$, let's say $z = \frac{3x}{2}$, clearly
$\frac{3x}{2} \not> x + x$.
\item[f)] $\forall x$ and $\forall y, \exists z \ni z > y$ implies that $z > x + y$.\\
True, given any $x$ and $y$ value we can always choose a $z \ni z > |x| + |y|$.
\end{enumerate}

\item[2.10] Determine the truth value of each statement, assuming $x$ is a real number.
\begin{enumerate}
\item[a)] $\exists x \in [3, 5] \ni x \geq 4$.\\
True; we can choose any real number in the range [4, 5] to satisfy the existence condition, let's
choose $x = 4.01$, we can see $4.01 \geq 4$.
\item[b)] $\forall x \in [3, 5], x \geq 4$.\\
False; choose $x$ equal to any real number in the interval [3, 4), which shows that
$x \not\geq 4$.
\item[c)] $\exists x \ni x^2 \neq 3$.\\
True; choose $x = 4$, then $x^2 = 16 \neq 3$.
\item[d)] $\forall x, x^2 \neq 3$.\\
False; can't say "for all" $x$ because when $x = \sqrt{3}$ then $x^2 = 3$.
\item[e)] $\exists x \ni x^2 = -5$.\\
False; by definition, any real number whether it's negative or positive raised to an even power will not yield a negative number, therefore there doesn't exist an $x$ we can choose to satisfy $x^2 = -5$.
\item[f)] $\forall x, x^2 = -5$.\\
False; in part (e) we said that \underline{no real number exists} which can 
satisfy $x^2 = -5$, therefore it's quite clear that applying a universal quantifier to 
the problem will not help any.
\item[g)] $\exists x \ni x - x = 0$.\\
True; simply put, choose $x = 1$ then $x - x = 1 - 1 = 0$.
\item[h)] $\forall x, x - x = 0$.\\
True; starting with the additive identity axiom, $\forall x, x + 0 = x$ and then
using the addition property of equality, we can see $\forall x, x - x = 0$.
\end{enumerate}

\item[2.14] A function is \emph{strictly decreasing} iff for every $x$ and for every $y$, if $x < y$, then $f(x) < f(y)$.
\begin{enumerate}
\item[a)] $\forall x$ and $\forall y, x < y \Rightarrow f(x) > f(y)$
\item[b)] $\exists x$ and $\exists y \ni x < y$ and $f(x) \leq f(y)$
\end{enumerate}

\item[2.16] A function $f: A \rightarrow B$ is \emph{surjective} iff for every $y$ in $B$ there exists an $x$ in $A$ such that $f(x) = y$.
\begin{enumerate}
\item[a)] $\forall y \in B, \exists x \in A$ such that $f(x) = y$
\item[b)] $\exists y \in B \ni \forall x \in A, f(x) \neq y$
\end{enumerate}

\item[4.2] Mark each statement True or False. Justify each answer.
\begin{enumerate}
\item[a)] A proof by contradiction may use the tautology $(\sim$$p \Rightarrow c) \Leftrightarrow p$.\\
True; pg. 28 states "The two basic forms of a proof by contradiction are based on tautologies
(f) and (g) in Example 3.12. Tautology (f) has the form \\
$(\sim$$p \Rightarrow c) \Leftrightarrow p$". 
\item[b)] A proof by contradiction may use the tautology $[(p \vee \sim$$q) \Rightarrow c] 
\Leftrightarrow (p \Rightarrow q)$.\\
False; following from the same page as part (a), "Tautology (g) has the form \\
$(p \Rightarrow q) \Leftrightarrow [(p\,\, \wedge \sim$$q) \Rightarrow c]$".
\item[c)] Definitions often play an important role in proofs.\\
True; pg. 32 claims that "Often a proof will be little more than unraveling definitions
and applying them to specific cases.
\end{enumerate}

\item[4.4] Prove: There exists a rational number $x$ such that $x^2 + 3x/2 = 1$. Is this rational number unique?
\begin{enumerate}
\item[]
Let $x = \frac{1}{2}$. Then \[
x^2 + \frac{3x}{2} = \left(\frac{1}{2}\right)^2 + \left(\frac{3}{2}\right)\left(\frac{1}{2}\right) 
= \frac{1}{4} + \frac{3}{4} = 1, \]
as required. The rational number is in fact unique. $\blacklozenge$
\end{enumerate}

\item[4.10] Prove: If $x$ is a real number, then $|x - 2| \leq 3$ implies that $-1 \leq x \leq 5$.
\begin{enumerate}
\item[]
Proof by contradiction: Tautology (g) states $(p \Rightarrow q) \Leftrightarrow [(p\,\, \wedge \sim$$q) \Rightarrow c]$ \\
So, let $p: |x - 2| \leq 3$ and $q: -1 \leq x \leq 5$ and in order to prove $p \Rightarrow q$, we need to show
that $p$ and $\sim$$q$ will lead us to a contradiction. \\
Let's algebraically manipulate $p$, by definition of absolute-value inequalities \[ |x - 2| \leq 3 \] is the same as
\[ -3 \leq x - 2 \leq 3 \] which can be manipulated to \[ -1 \leq x \leq 5 \] Now $\sim$$q$, translates to \[
x < -1 \mbox{ or } x > 5 \] Thus, $p\,\, \wedge \sim$$q$ leaves us with a contradiction for $x$ is a real number
and proves that $p \Rightarrow q$ or that $|x - 2| \leq 3$ implies $-1 \leq x \leq 5$. $\diamondsuit$
\end{enumerate}

\item[4.12] Consider the following theorem: "If $xy = 0$, then $x = 0$ or $y = 0$." Indicate what,
if anything, is wrong with each of the following "proofs."
\begin{enumerate}
\item[a)] Suppose $xy = 0$ and $x \neq 0$. Then dividing both sides of the first equation by $x$ we have $y = 0$. Thus if $xy = 0$, then $x = 0$ or $y = 0$. \\
There is nothing wrong with this proof. It demonstrates perfectly the tautology 3.12(p), which is also talked about in example 4.7.
\item[b)] There are two cases to consider. First suppose that $x = 0$. Then $x \cdot y = x \cdot 0 = 0$. Similarly, suppose that $y = 0$. Then $x \cdot y = x \cdot 0 = 0$. In either case, $x \cdot y = 0$. Thus if $xy = 0$, then $x = 0$ or $y = 0$. \\
The second sentence of this proof says "First suppose that $x = 0$." Then goes on to show that $xy = x \cdot 0 = 0$ in the third sentence when it should actually be $xy = 0 \cdot y = 0$. Plus, the proof, logically, isn't proving that "if $xy = 0$ then $x = 0$ or $y = 0$". It's actually using tautology 3.12(q) to prove that "if $x = 0$ or $y = 0$, then $xy = 0$".
\end{enumerate}

\item[4.18] Consider the following theorem: There do not exist three consecutive odd integers $a, b,$ and $c$ such that $a^2 + b^2 = c^2$.
\begin{enumerate}
\item[a)] For every three consecutive odd integers $a$, $b$, and $c$, \underline{$a^2 + b^2 \neq c^2$}.
\item[b)] If $a$, $b$, and $c$ are consecutive odd integers, then \underline{$a^2 + b^2 \neq c^2$}.
\item[c)] Let $a$, $b$, and $c$ be consecutive odd integers. Then $a = 2k + 1$, $b = $\underline{
$2k + 3$}, and $c = 2k + 5$ for some integer $k$. Suppose $a^2 + b^2 = c^2$. Then \\
$(2k + 1)^2 + ($\underline{$2k + 3$}$)^2 = (2k + 5)^2$. It follows that $8k^2 + 16k + 10 =
4k^2 + 20k + 25$ and $4k^2 - 4k\, - $\underline{$\,15$} $= 0$. Thus $k$ = 5/2 or $k$ = 
\underline{$-$(3/2)}. This contradicts $k$ being an \underline{integer}. Therefore, there do not
exist three consecutive odd integers $a$, $b$, and $c$ such that $a^2 + b^2 = c^2$. $\diamondsuit$
\item[d)] Which of the tautologies in Example 3.12 best describes the structure of the proof?
The tautology that best describes the proof from part(c) is 3.12(g). In part(b), we defined $q$ as $a^2 + b^2 \neq c^2$, and use $\sim$$q: a^2 + b^2 = c^2$ in part(c) and the original $p$ from part(b) to show a contradiction which proves our implication from part(b) to be true.
\end{enumerate}

\item[4.19] Prove or give a counterexample: The sum of any five consecutive integers is divisible by five.
\begin{enumerate}
\item[] We can prove the sum of any five consecutive integers is divisible by 5 directly. \\
Let $k$, $k + 1$, $k + 2$, $k + 3$, $k + 4$ be our five consecutive integers
for any integer $k$. Then $k + (k + 1) + (k + 2) + (k + 3) + (k + 4) = 5k + 10 = 5(k + 2)$ which will always
be divisible by 5. $\diamondsuit$
\end{enumerate}

\item[6.7] Let $A = \{ a \}$ and $B = \{ 1, 2, 3 \}$. List all possible relations between $A$ and $B$.
\begin{enumerate}
\item[]
All possible relations between $\{a\}$ and $\{1, 2, 3\}$. \\
$\emptyset$ \\
\{(a, 1)\}  \\ 
\{(a, 1), (a, 2)\} \\ 
\{(a, 1), (a, 2), (a, 3)\} \\ 
\{(a, 2)\} \\
\{(a, 2), (a, 3)\} \\
\{(a, 3)\} \\
\{(a, 1), (a, 3)\} 
\end{enumerate}

\item[6.11] Determine which of the three properties (reflexive, symmetric, and transitive) apply to each relation.
\begin{enumerate}
\item[a)] Let {\large R} be the relation on $\mathbb{N}$ given by $x${\large R}$y$ iff $x$ divides $y$.\\
Reflexive and transitive. Reflexivity by $\forall x \in \mathbb{N}, x$/$x = 1$. To show transitivity, we have
to say that if $x$ divides $y$ by a multiple of $n$ and $y$ divides $z$ by a multiple of $m$, then 
$x$ will divide $z$ by a multiple of $nm$. To show that this isn't symmetric, we need one counterexample, so let's take 5 divides 10 but 10 does not divide 5.
\item[b)] Let $X$ be a set and let {\large R} be the relation "$\subseteq$" defined on subsets of $X$. \\
Reflexive and transitive. According to the book, the symbol $\subseteq$ does not imply equality.
Let $A, B$, and $C$ be subsets of $X$. For reflexiveness, $A \subseteq A$ holds true, since this equivalent to $(\forall x)[(x \in A) \Rightarrow (x \in A)]$, which happens to be a tautology. Not symmetric because if $A \subseteq B$, then A might be a proper subset, in which case $B \subseteq A$ does not hold. If $A \subseteq B$ and $B \subseteq C$, then $A \subseteq C$ will hold true by the tautology 3.12(l).
\item[c)] Let $S$ be the set of people in the school. Define {\large R} on $S$ by $x${\large R}$y$ iff "$x$ likes $y$." \\
None. Not reflexive because there might exist a person in the set S that likes someone else but not
him/herself. Not symmetric because there might exist a person who likes someone else but isn't "liked" in return.
Not transitive due to the fact that if there was a person A who liked person B and person B liked person C, then
there is no guarantee that person A would like person C.
\item[d)] Let {\large R} be the relation $\{ (1, 1), (1, 2), (2, 2), (1, 3), (3, 3) \}$ on the set $\{ 1, 2, 3 \}$. \\ 
Reflexive. Since the relation {\large R} has the ordered pairs (1, 1), (2, 2), (3, 3), and the set is $\{1, 2, 3\}$, it is indeed reflexive. The relation does not contain (2, 1) nor (3, 1) therefore it cannot be
symmetric. Here is one example of transitivity that fails in this relation, (1, 2) and (2, 3) thus (1, 3), because the ordered pair (2, 3) does not exist in this relation, therefore making {\large R} not transitive.
\item[e)] Let {\large R} be the relation on $\mathbb{R}$ givey by $x${\large R}$y$ iff $x - y$ is rational. \\
Reflexive, Symmetric, and Transitive. Reflexiveness by $\forall x \in \mathbb{R}, x - x = 0$, and zero is a rational number. Symmetric because if $x - y$ is a rational number then $y - x = -(x - y)$, which means it
is also a rational number. Transitive through the fact that if $x - y$ is rational and $y - z$ are rational, then $x - y + y - z = x - z$ is rational.
\item[f)] Let {\large R} be the relation on $\mathbb{R}$ givey by $x${\large R}$y$ iff $x - y$ is irrational. \\
Symmetric and Transitive. In part(e) we said that 0 is rational, therefore this can't be reflexive since 0 is not irrational. Symmetry and transitivity both follow the same argumentation from part(e), that is if $x - y$ is irrational, then $y - x = -(x - y)$ is irrational and if $x - y$ is irrational and $y - z$ are irrational, then $x - y + y - z = x - z$ is irrational as well.
\item[g)] Let {\large R} be the relation on $\mathbb{R}$ givey by $x${\large R}$y$ iff $(x - y)^2 < 0$. \\
Symmetric and Transitive. This relation happens to an empty relation on a non-empty set, therefore it is not reflexive. Symmetric because $x${\large R}$y$ is always false. Transitive for the same reason as symmetric.
\item[h)] Let {\large R} be the relation on $\mathbb{R}$ givey by $x${\large R}$y$ iff $| x - y | \leq 2$. \\
Reflexive and Symmetric. Reflexivity by $\forall x \in \mathbb{R}, |x - x| = 0 \leq 2$. Symmetric by definition of absolute value, $|x - y| = |y - x| \leq 2$. Not transitive because while $|10 - 12| \leq 2$ and $|12 - 14| \leq 2$, $|10 - 14| \not\leq 2$.
\end{enumerate}

\item[6.20] Define a relation {\large R} on the set of all integers $\mathbb{Z}$ by $x${\large R}$y$ iff $x + y = 2k$ for some integer $k$. Is {\large R} an equivalence relation on $\mathbb{Z}$? Why or why not?
\begin{enumerate}
\item[] To verify that this relation is in fact an equivalence relation, we need to show that it is reflexive,
symmetric, and transitive. \\
Reflexive by the fact that $x - x = 0 = 3k$, where $k$ is some integer and in this case, always $0$.
Symmetry by $x - y = 3k$ for some integer $k$ and $y - x = -(x - y) = -3k$, which means it's also a multiple of 3.
Transitive by $x - y = 3k$ for some integer $k$ and $y - z = 3n$ for some integer $n$, then 
$x - y + y - z = 3k + 3n \Leftrightarrow x - z = 3(k + n)$, which means it's also a multiple of 3. 
The relation displays all three properties necessary to call it an equivalence relation. \\
To find our equivalence class, $E_5$, we set $x = 5$ and let $k$ be some integer, then find the $y \in \mathbb{Z}$ that
satisfies $y${\large R}$x$. Let's find a few elements; $5 - y = 3(0)$ if $k = 0$, then $y$ must be 5. Let $k = 1$, then
$5 - y = 3(1)$ and thus $y$ has to be 2. Let $k = 2$, then $5 - y = 3(2)$ which means $y$ has to be -1.
Going in the other direction, let $k = -1$ then $5 - y = 3(-1)$ and $y$ has
to be 8. \\
Now let's describe our equivalence class from the elements we found. $E_5 = \{-1, 2, 5, 8\}$. There appears to be 
a pattern here in which case the next elements to add will be 3 tacked on to the numbers on the end and since 
$\mathbb{Z}$ goes to infinity in both directions, we have $E_5 = \{\ldots, -4, -1, 2, 5, 8, 11,\ldots \}$. \\
There should be 3 distinct equivalence classes because as you saw from above $E_5 = E_2 = E_{-1}$ and so forth.
So let's show our 3 distinct equivalence classes: \\
$\ldots \,\,= E_{-3} = E_0 = E_3 = E_6 = \ldots$ \\
$\ldots \,\,= E_{-2} = E_1 = E_4 = E_7 = \ldots$ \\
$\ldots \,\,= E_{-1} = E_2 = E_5 = E_8 = \ldots$ 
\end{enumerate}

\item[7.2] Mark each statement True or False. Justify each answer.
\begin{enumerate}
\item[a)] If $f: A \rightarrow B$ and $C$ is a nonempty subset of $A$, then $f(C)$ is a nonempty subset of $B$. \\
True; pg. 65 states "If $C \subseteq A$, we let $f(C)$ represent the subset
$\{f(x): x \in C\}$ of $B$."
\item[b)] If $f: A \rightarrow B$ is surjective and $y \in B$, then $f^{-1}(y) \in A$. \\
False; pg. 65 states "...we let $f^{-1}(D)$ represent the subset $\{x \in A: f(x) \in D\}$
of $A$." 
\item[c)] If $f: A \rightarrow B$ and $D$ is a nonempty subset of $B$, then $f^{-1}(D)$ is a nonempty subset of $A$. \\
True; pg. 65 claims that "If $D \subseteq B$, we let $f^{-1}(D)$ represent the subset 
$\{x \in A: f(x) \in D\}$ of A." Therfore if $\exists f(x) \in D$ then there must be a corresponding $x \in A$
and $f^{-1}(D)$ is not empty.
\item[d)] The composition of two surjective functions is always surjective. \\
True; pg. 68, Theorem 7.19(a) proves that the composition of two surjective functions is also
surjective.
\item[e)] If $f: A \rightarrow B$ is bijective, then $f^{-1}: B \rightarrow A$ is bijective. \\
True; pg. 69 claims and then explains that "If $f: A \rightarrow B$ is bijective", then 
it follows that $f^{-1}: B \rightarrow A$ is also bijective.".
\item[f)] The identity function maps $\mathbb{R}$ onto $\{ 1 \}$. \\
False; pg. 69 defines the identity function as "A function...on a set A that maps each element
in A onto itself...".
\end{enumerate}

\item[7.5] Suppose that $A$ has exactly three elements and $B$ has exactly two. How many different functions are there from $A$ to $B$? How many of these are injective? How many are surjective?
\begin{enumerate}
\item[b)] If A has 3 elements and B has 2 elements, there will be 8 functions from A to B.
To show this, let $A = \{a, b, c\}$ and $B = \{1, 2\}$. Then \\
$f_1 = \{(a, 1), (b, 1), (c, 1)\}$ \\
$f_2 = \{(a, 1), (b, 1), (c, 2)\}$ \\
$f_3 = \{(a, 1), (b, 2), (c, 1)\}$ \\
$f_4 = \{(a, 1), (b, 2), (c, 2)\}$ \\
$f_5 = \{(a, 2), (b, 1), (c, 1)\}$ \\
$f_6 = \{(a, 2), (b, 1), (c, 2)\}$ \\
$f_7 = \{(a, 2), (b, 2), (c, 1)\}$ \\
$f_8 = \{(a, 2), (b, 2), (c, 2)\}$ \\
There are 6 surjective functions, with $f_1$ and $f_8$ being the only functions that are not. \\
There are no injective functions.
\end{enumerate}

\item[7.7] Classify each function as injective, surjective, bijective, or none of these.
\begin{enumerate}
\item[d)] $f: [1, \infty ) \rightarrow [0, \infty )$ defined by $f(x) = x^3 - x$ \\
bijective; like part(c) except the domain and range are limited in order to make the 
function both surjective and injective, which means it's bijective.
\item[e)] $f: \mathbb{N} \rightarrow \mathbb{Z}$ defined by $f(n) = n^2 - n$ \\
injective; not surjective because 5 has no pre-image and $\forall n, n' \in \mathbb{N}, f(n) = f(n')$ implies $n = n'$.
\item[f)] $f: [3, \infty ) \rightarrow [5, \infty )$ defined by $f(x) = (x - 3)^2 + 5$ \\
bijective; $f(3) = 5$ and every number in the interval $[5, \infty]$ can be obtained uniquely
from every number in the interval $[3, \infty]$, therefore this function is bijective.
\item[g)] $f: \mathbb{N} \rightarrow \mathbb{Q}$ defined by $f(n) = 1/n$ \\
injective; not surjective because 2/3 has no pre-image and $\forall n, n' \in \mathbb{N}, f(n) = f(n')$ implies $n = n'$.
\end{enumerate}

\item[7.19] Prove Theorem 17.9(b). That is, suppose that $f: A \rightarrow B$ and $g: B \rightarrow C$ are both injective. Prove that $g \circ f: A \rightarrow C$ is injective.
\begin{enumerate}
\item[] Let $f: A \rightarrow B$ and $g: B \rightarrow C$ both be injective functions. Assume $n_1, n_2
\in A$ and $g \circ f$ is injective, then we have $g \circ f(n_1) = g \circ f(n_2)$.
Formula notation states that this is $g(f(n_1)) = g(f(n_2))$. Since $g$ is injective then 
$g(f(n_1)) = g(f(n_2))$ implies $f(n_1) = f(n_2)$ and since $f$ is injective, $f(n_1) = f(n_2)$ implies
that $n_1 = n_2$. Therefore, since $g \circ f(n_1) = g \circ f(n_2)$ implies $n_1 = n_2$, then
$g \circ f$ is in fact, injective. $\diamondsuit$
\end{enumerate}

\item[7.21] Suppose that $f: A \rightarrow B$ and let $C$ be a subset of $A$.
\begin{enumerate}
\item[b)] Prove or give a counterexample: $f(A)\setminus f(C) \subseteq f(A\setminus C)$.
Counterexample: Let $A = \mathbb{Z}$ and $C = \mathbb{N}$, so $C \subseteq A$. So, now we
have \[
f(\mathbb{Z}\backslash \mathbb{N}) \subseteq f(\mathbb{Z})\backslash f(\mathbb{N})
\] which looks like \[
f(\{x \in \mathbb{Z}: x \leq 0\}) \subseteq f(\mathbb{Z})\backslash f(\mathbb{N}).
\] Now take $f$ to be the square function, we have \[
(\{0\} \cup \{x: x = k^2, \forall k \in \mathbb{N}\})
\subseteq 
(\{0\} \cup \{x: x = k^2, \forall k \in \mathbb{N}\})
\backslash 
\{x: x = k^2, \forall k \in \mathbb{N}\}
\] which we know is \[
(\{0\} \cup \{x: x = k^2, \forall k \in \mathbb{N}\})
\subseteq \{0\} \] and clearly not true. Therefore we can conclude that 
$f(A\backslash C) \not\subseteq f(A)\backslash f(C)$.
\end{enumerate}

\item[7.27] Find an example of functions $f: A \rightarrow B$ and $g: B \rightarrow C$ such that $g \circ f$ is bijective, but neither $f$ nor $g$ is bijective.
\begin{enumerate}
\item[]
Take the example of when A and C only have one element and B has two elements. With the only element of A mapped onto only one element of B and both elements of B mapped onto the only element of C. We can conclude that $g$ is surjective and $g \circ f$ is surjective but $f$ is not.
\end{enumerate}

\item[8.1] Mark each statement True or False. Justify each answer.
\begin{enumerate}
\item[a)] Two sets $S$ and $T$ are equinumerous if there exists a bijection $f: S \rightarrow T$. \\
True; This is definition 8.1 from pg. 78.
\item[b)] If a set $S$ is finite, then $S$ is equinumerous with $I_n$ for some $n \in \mathbb{N}$. \\
True; pg. 78 says "$S$ is finite if\,f $S = \emptyset$ or $S$ is equinumerous with $I_n$ for some
$n \in \mathbb{N}$.
\item[c)] If a cardinal number is not finite, it is said to be infinite.\\
False; Definition 8.4 says that "If a cardinal number is not finite, it is called transfinite."
\item[d)] A set $S$ is denumerable if there exists a bijection $f: \mathbb{R} \rightarrow S$.\\
False; Definition 8.6 says "A set $S$ is said to be denumerable if there exists a bijection
$f: \mathbb{N} \rightarrow S$."
\item[e)] Every subset of a countable set is countable.\\
True; the proof from theorem 8.9 proves it's true.
\item[f)] Every subset of a denumerable set is denumerable.\\
False; a subset of a denumerable set could be finite, not always denumerable.
\end{enumerate}

\item[8.3] Show that the following pairs of sets $S$ and $T$ are equinumerous by finding a specific bijection between the sets in each pair.
\begin{enumerate}
\item[a)] $S = [0, 1]$ and $T = [1, 3]$ \\
We could use a linear equation for this bijection. If we use the point-slope form to find the slope, we get $m = 2$, then again use the point-slope form to find our bijection in slope-intercept form. After all is done, we are left with the bijection $f: [0, 1] \rightarrow [1, 3]$ by $f(x) = 2x - 1$.
\item[b)] $S = [0, 1]$ and $T = [0, 1)$ \\
The answer in the back of the book states "Let $f(x) = 1/(n + 1)$ if there exists $n \in
\mathbb{N}$ such that $x = 1/n$, and $f(x) = x$ otherwise."
\item[c)] $S = [0, 1)$ and $T = (0, 1)$\\
This bijection is similar to part(b), so let's make $f(0) = 1/2$ and $f(x) = 1/(n + 2)$
if there exists $n \in \mathbb{N}$ such that $x = 1/n$ and $f(x) = x$ otherwise.
\item[d)] $S = (0, 1)$ and $T = (0, \infty )$ \\
Let's see if we can find a bijection from $(0, \infty )$ to $(0, 1)$ first. In fact, it's not that
hard to see that the bijection would be $f(x) = x/(1 + x)$ will work. So now find the inverse, which, by definition, will be our bijection from $(0, 1)$ to $(0, \infty )$. A little algebraic manipulation: \[
y = x/(1 + x) \Leftrightarrow y(1 + x) - x = 0 \]
\[ \Leftrightarrow y - x(-y + 1) = 0 
\Leftrightarrow y = x(1 - y) \Leftrightarrow x = y/(1 - y) \]
and thus we have our bijection from $(0, 1)$ to $(0, \infty )$ by $g(x) = x/(1 - x)$.
\item[e)] $S = (0, 1)$ and $T = \mathbb{R}$ \\
We could use the composition of two functions to show that (0, 1) and $\mathbb{R}$ are equinumerous. Let's take the bijection $f: (0, 1) \rightarrow (-\pi /2, \pi /2)$ by $f(x) = \pi (x - \frac{1}{2})$ and the bijection $g: (-\pi /2, \pi /2) \rightarrow \mathbb{R}$ by $g(x) = \tan (x)$. The composition of two bijective functions is bijective, thus there exists a bijection between $(0, 1)$ and $\mathbb{R}$, making them equinumerous.
\end{enumerate}

\item[8.9] A real number is said to be {\bf algebraic} if it is a root of a polynomial equation
\[
a_nx^n + \cdots + a_1x + a_0 = 0
\]
with integer coefficients. Note that the algebraic numbers include the rationals and all roots of rationals (such as $\sqrt{2}, \sqrt[3]{5}$, etc.). If a number is not algebraic, it is called {\bf transcendental}.
\begin{enumerate}
\item[a)] Show that the set of polynomials with integer coefficients is countable.\\
Let $P$ be the set of all polynomials with integer coefficients and $P_n \subseteq P$, 
where $P_n$ is the set of all polynomials with integral coefficients of being at most degree $n$, which makes $P_n$ a finite set. Since, $P_n$ is a set of all polynomials with integral coefficients, the polynomial $a_nx^n + \cdots + a_1x + a_0$ will be in the set. Since the
aforementioned polynomial has $n + 1$ integral coefficients, we can let $a_k$, where $0 \leq k \leq n$, be drawn from the subset of integers. We use a subset of $\mathbb{Z}$ here for ease of demonstration. Also, because here we can also choose a large enough $n$ such that all the integers of $\mathbb{Z}$ will be found in the subset and we already know that $\mathbb{Z}$ is countable, thus any subset, proper or equal, will also be countable. Namely, $\{-n, \ldots, -1, 0, 1, \ldots ,n\}$, which has $2n + 1$ elements. \\
Thus $|P_n| = (2n + 1)^{n + 1}$, making it countable. Using what is written in Example 8.11(d), we now know that $P$ must be countable since the union of countable sets is countable. To show this, we can choose a large enough $n$, such that all polynomials of $P$ are in $P_n$, that is to say, $P$ is comprised of $P_1 \cup P_2 \cup \ldots \cup P_n$.
\item[b)] Show that the set of algebraic numbers is countable.\\
This follows from part(a) and the fact that a polynomial of degree $n$ has at most
$n$ roots. Since $P$ is countable and each polynomial in $P$ has a finite number of roots,
then the set of algebraic numbers is countable.
\item[c)] Are there more algebraic numbers or transcendental numbers?\\
There are more algebraic numbers such as the complex number, $i$, as well as there being more
transcendental numbers like, $\pi$ or $e$.
\end{enumerate}

\item[10.4] Prove that $1^3 + 2^3 + \cdots + n^3 = \frac{1}{4}n^2(n + 1)^2$ for all $n \in \mathbb{N}$.
\begin{enumerate}
\item[] Let $P(n)$ be the statement \[
1^3 + 2^3 + \cdots + n^3 = \frac{1}{4} n^2(n + 1)^2.
\]
Since, we want to prove for all natural numbers, then $P(1)$ will establish our
basis for induction. Therefore, $1^3 = \frac{1}{4} 1^2(1 + 1)^2$, which is true, as they both
equal 1. To verify the induction step, we suppose that $P(k)$ is true, where $k \in \mathbb{N}$.
That is we assume, \[
1^3 + 2^3 + \cdots + k^3 = \frac{1}{4} k^2(k + 1)^2.
\]
Since we wish to conclude that $P(k + 1)$ is true, we add $k + 1$ to both sides to obtain
\begin{eqnarray*}
1^3 + 2^3 + \cdots + k^3 + (k + 1)^3 &=& \frac{1}{4} k^2(k + 1)^2 + (k + 1)^3 \\
&=& \frac{1}{4} k^2(k + 1)^2 + (k + 1)(k + 1)^2 \\
&=& [\frac{1}{4} k^2 + (k + 1)](k + 1)^2 \\
&=& [\frac{k^2 + 4k + 4}{4}](k + 1)^2 \\
&=& [\frac{(k + 2)^2}{4}](k + 1)^2 \\
&=& [\frac{1}{4}](k + 1)^2(k + 2)^2.
\end{eqnarray*}
Thus $P(k + 1)$ is true whenever $P(k)$ is true, and by the principle of 
mathematical induction, we conclude that $P(n)$ is true for all $n$. $\blacklozenge$
\end{enumerate}

\item[10.6] Prove that 
\[
\frac{1}{1 \cdot 2} + \frac{1}{2 \cdot 3} + \frac{1}{3 \cdot 4} + \cdots + \frac{1}{n(n+1)} = 
\frac{n}{n+1}, \mbox{ for all $n \in \mathbb{N}$.}
\]
\begin{enumerate}
\item[] Let $P(n)$ be the statement \[
\frac{1}{1 \cdot 2} + \frac{1}{2 \cdot 3} + \cdots + \frac{1}{n(n + 1)} = \frac{n}{n + 1}.
\]
Since, we want to prove for all natural numbers, then $P(1)$ will establish our
basis for induction. Therefore, \[ \frac{1}{1(1 + 1)} = \frac{1}{1 + 1} \] which is true, as they both
equal 1/2. To verify the induction step, we suppose that $P(k)$ is true, where $k \in \mathbb{N}$.
That is we assume, \[
\frac{1}{1 \cdot 2} + \frac{1}{2 \cdot 3} + \cdots + \frac{1}{k(k + 1)} = \frac{k}{k + 1}.
\]
Since we wish to conclude that $P(k + 1)$ is true, we add $k + 1$ to both sides to obtain
\begin{eqnarray*}
\frac{1}{1 \cdot 2} + \frac{1}{2 \cdot 3} + \cdots + \frac{1}{k(k + 1)} + \frac{1}{(k + 1)(k + 2)} 
&=& \frac{k}{k + 1} + \frac{1}{(k + 1)(k + 2)} \\
&=& \frac{k(k + 2) + 1}{(k + 1)(k + 2)} \\
&=& \frac{k^2 + 2k + 1}{(k + 1)(k + 2)} \\
&=& \frac{(k + 1)^2}{(k + 1)(k + 2)} \\
&=& \frac{k + 1}{k + 2} 
\end{eqnarray*}
Thus $P(k + 1)$ is true whenever $P(k)$ is true, and by the principle of 
mathematical induction, we conclude that $P(n)$ is true for all $n$. $\blacklozenge$
\end{enumerate}

\item[10.9] Prove that $1 + 2 + 2^2 + \cdots + 2^{n-1} = 2^n - 1$, for all $n \in \mathbb{N}$.
\begin{enumerate}
\item[] Let $P(n)$ be the statement \[
1 + 2 + 2^2 + \cdots + 2^{n - 1} = 2^n - 1
\]
Since, we want to prove for all natural numbers, then $P(1)$ will establish our
basis for induction. Therefore, $2^{1 - 1} = 2^1 - 1$ which is true, as they both
equal 1. To verify the induction step, we suppose that $P(k)$ is true, where $k \in \mathbb{N}$.
That is we assume, \[
1 + 2 + 2^2 + \cdots + 2^{k - 1} = 2^k - 1
\]
Since we wish to conclude that $P(k + 1)$ is true, we add $k + 1$ to both sides to obtain
\begin{eqnarray*}
1 + 2 + 2^2 + \cdots + 2^{k - 1} + 2^k &=&  (2^k - 1) + 2^k \\
&=& 2 \cdot 2^k - 1 \\
&=& 2^{k + 1} - 1.
\end{eqnarray*}
Thus $P(k + 1)$ is true whenever $P(k)$ is true, and by the principle of 
mathematical induction, we conclude that $P(n)$ is true for all $n$. $\blacklozenge$
\end{enumerate}

\item[10.14] Prove that $9^n - 4^n$ is a multiple of 5 for all $n \in \mathbb{N}$.
\begin{enumerate}
\item[] We have want to prove by induction that $9^n - 4^n$ is a multiple of $5$ for all
$n \in \mathbb{N}$. Clearly, this is true when $n = 1$, since $9^1 - 4^1 = 5$. Now let 
$k \in \mathbb{N}$ and suppose that $9^k - 4^k$ is a multiple of $5$. That is, 
$9^k - 4^k = 5m$ for some $m \in \mathbb{N}$. It follows that
\begin{eqnarray*}
9^{k + 1} - 4^{k + 1} &=&  9^{k + 1} - 9 \cdot 4^k + 9 \cdot 4^k - 4^{k + 1} \\
&=& 9(9^k - 4^k) + 5 \cdot 4^k \\
&=& 9(5k) + 5 \cdot 4^k \\
&=& 5(9k + 4^k).
\end{eqnarray*}
Since $m$ and $k$ are natural numbers, so is $9k + 4^k$. Thus $9^{k + 1} + 4^{k + 1}$ is
also a multiple of 3, and by induction we conclude that $9^n - 4^n$ is a multiple of
$3$ for all $n \in \mathbb{N}$. $\blacklozenge$
\end{enumerate}

\item[10.15] Indicate what is wrong with each of the following induction "proofs."
\begin{enumerate}
\item[a)] {\bf Theorem:} For each $n \in \mathbb{N}$, let $P(n)$ be the statement "Any collection of $n$ marbles consists of marbles of the same color." Then $P(n)$ is true for all $n \in \mathbb{N}$. \\
{\bf Proof:} Clearly, $P(1)$ is a true statement. Now suppose that $P(k)$ is a true statement for some $k \in \mathbb{N}$. Let $S$ be a collection of $k + 1$ marbles. If one marble, call it $x$, is removed, then the induction hypothesis applied to the remaining $k$ marbles implies that these $k$ marbles all have the same color. Call this color $C$. Now if $x$ is returned to the set $S$ and a different marble is removed, then again the remaining $k$ marbles must all be of the same color $C$. But one of these marbles is $x$, so in fact all $k + 1$ marbles have the same color $C$. Thus $P(k + 1)$ is true, and by induction we conclude that $P(n)$ is true for all $n \in \mathbb{N}$. $\blacklozenge$ \\
\emph{Answer:} The implication $P(k) \Rightarrow P(k + 1)$ does not always hold to be true. According to the proof given, suppose that the collection, $S$, of marbles had 2 marbles in it, one white and the other black. We remove one marble, $x$, in this case happens to be black, now
the color C becomes white. Now we put $x$ back into the bag and pull out the other marble, 
which is the white one. The proof says by inductive hypothesis, the marble in the bag is 
color C, white, and this proves that $P(k + 1)$ is true. Yet, we know for a fact that the
marble in the bag was black, thereby showing the implication $P(1) \Rightarrow P(2)$ is not true.
\end{enumerate}

\item[10.22] Use induction to prove Bernoulli's inequality: If $1 + x > 0$, then $(1 + x)^n \geq 1 + nx$ for all $n \in \mathbb{N}$.
\begin{enumerate}
\item[] Let $P(n)$ be the statement \[
\forall x \ni x + 1 > 0, (1 + x)^n \geq 1 + nx
\]
Since, we want to prove for all natural numbers, then $P(1)$ will establish our
basis for induction. Therefore, $(1 + x)^1 \geq 1 + x$ which is true, since they are
equal to each other. To verify the induction step, we suppose that $P(k)$ is 
true, where $k \in \mathbb{N}$. That is we assume, \[
\forall x \ni x + 1 > 0, (1 + x)^k \geq 1 + kx
\]
Since we wish to conclude that $P(k + 1)$ is true, we add $k + 1$ to both sides to obtain
\begin{eqnarray*}
(1 + x)^{k + 1} &=& (1 + x)(1 + x)^k \\
&\geq& (1 + x)(1 + kx) \\
&=& 1 + x + kx + kx^2 \\
&=& 1 + (k + 1)x + kx^2 \\
&\geq& 1 + (k + 1)x.
\end{eqnarray*}
The last two steps follow from $kx^2 \geq 0$. We know this because $k \in \mathbb{N}$ and 
$\forall x, x > -1$ implies that $x^2 > 0$. Thus $kx^2$ must be greater than $0$.
Thus $P(k + 1)$ is true whenever $P(k)$ is true, 
and by the principle of mathematical induction, we conclude that $P(n)$ is true for 
all $n$. $\blacklozenge$
\end{enumerate}

\item[11.3]
\begin{enumerate}
\item[b)] 
$(-x) \cdot y = -(xy)$ and $(-x) \cdot (-y) = xy$
\begin{eqnarray*}
xy + (-x) \cdot y &=& y(-x + x) \quad \mbox{  by DL} \\
&=& y \cdot 0 \hskip 1.55cm \mbox{  by A5} \\
&=& 0 \hskip 2.05cm \mbox{  by Theorem 11.1(b)}
\end{eqnarray*}
Thus $(-x) \cdot y = (-xy)$ by the uniqueness of $-(xy)$ in A5. 
\[ \]
The second part wants use to prove that $(-x)(-y) = xy$.
\begin{eqnarray*}
(-x)(-y) &=& -[(x)(-y)] \hskip 0.7cm \mbox{by the above proof} \\
&=& -[-(xy)] \hskip 1.02cm \mbox{by the above proof} \\
&=& xy \hskip 2.12cm \mbox{  by Exercise 11.3(a)}
\end{eqnarray*}
$\blacklozenge$

\item[c)] If $x \neq 0$, then $(1/x) \neq 0$ and $1/(1/x) = x$. \\
If $x \neq 0$, then we know by M5 that there is a unique real number $1/x$ such that
$x \cdot (1/x) = 1$. For the sake of contradiction, let $1/x = 0$, thus $x \cdot 0 = 1$ and by
Theorem 11.1(b), we are left with $1 = 0$, which is a contradiction. Thus $1/x$ must not be
equal to 0.
\[ \]
To prove the second part, if $x \neq 0$, then from M5, there exists a unique real number, $1/x$, such that
$x \cdot 1/x = 1$. We will use the other notation for $1/x$, which is $x^{-1}$. 
By the above proof, $x^{-1} \neq 0$, so by M5 there must
exist a unique real number such that, $(x^{-1})^{-1} \cdot x^{-1} = 1$. 
By, the uniqueness of the $x^{-1}$ in M5, $(x^{-1})^{-1} \cdot x^{-1} = 1$ can 
only be true if and only if $(x^{-1})^{-1} = x$. $\blacklozenge$

\item[d)] If $x \cdot z = y \cdot z$ and $z \neq 0$, then $x = y$. \\
If $x \cdot z = y \cdot z$ and $z \neq 0$, then 
\begin{eqnarray*}
(x \cdot z) \cdot \frac{1}{z} &=& (y \cdot z) \cdot \frac{1}{z} \qquad \mbox{by M5 and M1} \\
x \cdot (z \cdot \frac{1}{z}) &=& y \cdot (z \cdot \frac{1}{z}) \qquad \mbox{by M3} \\
x \cdot 1 &=& y \cdot 1 \hskip 1.78cm \mbox{by M5} \\
x &=& y \hskip 2.30cm \mbox{by M4}
\end{eqnarray*}$\blacklozenge$ 

\item[g)] If $x > 1$, then $x^2 > x$. \\
If $x > 1$ then $x > 0$ by Exercise 11.3(f) and O2. Since $0 < x$ and $1 < x$
, we get $1 \cdot x < x \cdot x$, by
O4 and then $x < x \cdot x$ by M4. From the real numbers, $x \cdot x$ is denoted by $x^2$, thus proving $x < x^2$.
$\blacklozenge$

\item[h)] If $0 < x < 1$, then $x^2 < 1$. \\
We have $0 < x < 1$, so $1 - x > 0$ by O3 and $x > 0$. Since $1 > 0$ from Exercise 11.3(f) and
$x > 0$, it follows from O3 and O2, that $1 + x > 0$. Then by O4, $(1 + x)(1 - x) = 1 - x^2 > 0$. 
Hence, by O3 again, $x^2 < 1$. $\blacklozenge$

\item[i)] If $x > 0$, then $1/x > 0$. If $x < 0$, then $1/x < 0$. \\
If $x > 0$, then $1/x > 0$. For the sake of contradiction, let $1/x \leq 0$, thus by Theorem 11.1(e) we have $-(1/x) \geq 0$. By O4 and Theorem 11.1(b), $-(1/x) \cdot x \geq 0$, which by M3 and M5, happens to be $-1 \geq 0$, which is a contradiction. Hence $1/x > 0$. \[ \]
The second part says, if $x < 0$, then $1/x < 0$. 
\begin{eqnarray*}
0 < -x &=& (-1) \cdot x \hskip 2.1cm \mbox{by Theorem 11.1(e) then (c)} \\
&=& -(1 \cdot x) \hskip 2.1cm \mbox{by Exercise 11.3(b)} \\
&=& -[1/x \cdot (x \cdot x)] \hskip .9cm \mbox{by M5 and M3} \\
&=& -(1/x) \hskip 2.2cm \mbox{by O4 (twice)}
\end{eqnarray*}
Since $0 < -(1/x)$, by Theorem 11.1(e), this must be that $0 > 1/x$. $\blacklozenge$
\end{enumerate}

\item[11.6] 
\begin{enumerate}
\item[b)] Prove: If $|x - y| < c$, then $|x| < |y| + c$. \\
To show that $|x| < |y| + c$ is true if $|x - y| < c$ is true, let's first show that
$|x| - |y| \leq |x - y|$. 
\begin{eqnarray*}
|x| - |y| &=& |x - y + y| - |y| \qquad \mbox{by A4 and A5} \\
&\leq& |x - y| + |y| - |y| \hskip 0.59cm \mbox{by Theorem 11.9(d)} \\
&=& |x - y| + 0 \hskip 2.00cm \mbox{by A5} \\
&=& |x - y| \hskip 2.53cm \mbox{by A4} 
\end{eqnarray*}
Thus if $|x| - |y| \leq |x - y|$ and $|x - y| < c$, then by O2, $|x| - |y| < c$. Then by O3, 
we have $|x| - |y| + |y| < c + |y|$, then by A5, A4 and A2, we have proven that $|x| < |y| + c$.
$\blacklozenge$

\item[c)] Prove: If $|x - y| < \epsilon$ for all $\epsilon > 0$, then $x = y$. \\
By theorem 11.9(a), $|x - y| \geq 0$. Since $0 \leq |x - y| < \varepsilon, \forall \varepsilon > 0$ then this is only possible if $|x - y| = 0$. Hence, $0 = |0| = |x - y|$ provided that 
$x - y = 0$, thus by the uniqueness defined in A5, this is true iff $y = x$.
\end{enumerate}

\item[12.1] Mark each statement True or False. Justify each answer.
\begin{enumerate}
\item[a)] If a nonempty subset of $\mathbb{R}$ has an upper bound, then it has a least upper bound. \\
True; by the Completeness Axiom.
\item[b)] If a nonempty subset of $\mathbb{R}$ has an infimum, then it is bounded. \\ 
False; a bounded subset must be bounded from above and below.
\item[c)] Every nonempty bounded subset of $\mathbb{R}$ has a maximum and a minimum. \\
False; take Example 12.3(c) to show that a set might not have a minimum. Similarly, we can
also say that it's possible to not have a maximum or both.
\item[d)] If $m$ is an upper bound for $S$ and $m' < m$, then $m'$ is not an upper bound for $S$.\\
False; since $m \neq \sup S$ then $m'$ could be an upper bound as well. 
\item[e)] If $m = $ inf $S$ and $m' < m$, then $m'$ is a lower bound for $S$. \\
True; by definition of infimum, $m$ is the greatest lower bound and if $m' < m$, then $m'$ must
be a lower bound as well.
\item[f)] For each real number $x$ and each $\epsilon > 0$, there exists $n \in \mathbb{N}$ such that $n\epsilon > x$. \\
True; as proven in Theorem 12.10(b).
\end{enumerate}

\item[12.4] Give the minimum and infimum of each set.
\begin{enumerate}
\item[a)] $\{ 1, 3 \}$ \\
$\inf \{1, 3\} = 1$ and $\min \{1, 3\} = 1$
\item[b)] $\{ \pi , 3 \}$ \\
$\inf \{\pi, 3\} = 3$ and $\min \{\pi, 3\} = 3$
\item[c)] $[0, 4]$ \\
$\inf [0, 4] = 0$ and $\min [0, 4] = 0$
\item[d)] $(0, 4)$ \\
$\inf (0, 4) = 0$ and there is no minimum
\item[e)] $\{ 1/n : n \in \mathbb{N} \}$ \\
$\inf \{1/n : n\in \mathbb{N}\} = 0$ and there is no minimum
\item[f)] $\{ 1 - (1/n) : n \in \mathbb{N} \}$ \\
$\inf \{1 - (1/n) : n\in \mathbb{N}\} = 0$ and $\min \{1 - (1/n) : n\in \mathbb{N}\} = 0$
\item[g)] $\{ n/(n + 1) : n \in \mathbb{N} \}$ \\
$\inf \{n/(n + 1) : n\in \mathbb{N}\} = 1/2$ and $\min \{n/(n + 1) : n\in \mathbb{N}\} = 1/2$
\item[h)] $\{ (-1)^n[1 + (1/n)] : n \in \mathbb{N} \}$ \\
$\inf \{(-1)^n \left(1 + \frac{1}{n}\right) : n \in \mathbb{N}\} = -2$ and 
$\min \{(-1)^n \left(1 + \frac{1}{n}\right) : n \in \mathbb{N}\} = -2$
\end{enumerate}

\item[12.6] 
\begin{enumerate}
\item[a)] Let $S$ be a nonempty bounded subset of $\mathbb{R}$. Prove that sup $S$ is unique. \\
Let $a = \sup S$ and $b = \sup S$. By the definition of the supremum, since $a$ is the supremum, it is the lowest of the upper bounds and since $b$ is an upper bound, then $a \leq b$. Again, by the definition of the supremum, $b$ is the supremum, thus it is the lowest of the upper bounds and since $a$ is an upper bound, then $b \leq a$. This is only possible when $a = b$, hence $\sup S$ must be \emph{unique}.
\item[b)] Suppose that $m$ and $n$ are both maxima of a set $S$. Prove that $m = n$. \\
By the definition of the maximum, since $n$ is the maximum, it is the element in $S$ such that
$\forall s \in S, s \leq n$. Again, by the definition of the maximum, $m$ is the maximum, 
thus $\forall s \in S, s \leq m$. Thus by definition of the maximum, $n \leq m$ and $m \leq n$ implies that this is only possible when $n = m$.
\end{enumerate}

\end{enumerate}
\end{document}
