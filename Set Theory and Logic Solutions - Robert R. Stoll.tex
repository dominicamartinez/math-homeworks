\documentclass{letter}
\usepackage{geometry,amsmath,amssymb,bbm}
\geometry{letterpaper}
\usepackage{fullpage}

%%%%%%%%%% Start TeXmacs macros
\newcommand{\nin}{\not\in}
\newcommand{\tmem}[1]{{\em #1\/}}
\newcommand{\tmname}[1]{\textsc{#1}}
\newcommand{\tmop}[1]{\ensuremath{\operatorname{#1}}}
\newcommand{\tmtextit}[1]{{\itshape{#1}}}
\newcommand{\um}{-}
%%%%%%%%%% End TeXmacs macros

\begin{document}

Set Theory and Logic by Robert R. Stoll

1.2.1 Explain why $2 \in \{1, 2, 3\}$.

\tmtextit{Solution:} The set $\{1, 2, 3\}$ contains 2 in it.

1.2.2 Is $\{1, 2\} \in \{\{1, 2, 3\}, \{1, 3\}, 1, 2\}$? Justify your answer.

\tmtextit{Solution:} No, the only elements of the set are $\{1, 2, 3\}, \{1,
3\}, 1, \tmop{and} 2$. Thus $\{1, 2\}$ is not contained in the set.

1.2.3 Try to devise a set which is a member of itself.

\tmtextit{Solution:} $A =\{A\}$

1.2.4 Given an example of sets $A, B, \tmop{and} C$ such that $A \in B, B \in
C, \tmop{and} A \nin C$.

\tmtextit{Solution:} $A =\{1\}, B =\{A\}, C =\{B\}$

1.2.5 Describe in prose each of the following sets.

(a) $\{x \in \mathbbm{Z}| x \tmop{is} \tmop{divisible} \tmop{by} 2 \tmop{and}
x \tmop{is} \tmop{divisible} \tmop{by} 3\}$

\tmtextit{Solution:} All the integers that are divisible by 6.

(b) $\{x | x \in A \tmop{and} x \in B\}$

\tmtextit{Solution:} All the elements that are in $A$ and in $B$.

(c) $\{x| x \in A \tmop{or} x \in B\}$

\tmtextit{Solution:} All the elements included in $A \tmop{or} B$.

(d) $\{x \in \mathbbm{Z}^+ | x \in \{x \in \mathbbm{Z}| \tmop{for} \tmop{some}
\tmop{integer} y, x = 2 y\} \tmop{and} x \in \{x \in \mathbbm{Z}| \tmop{for}
\tmop{some} \tmop{integer} y, x = 3 y\}\}$

\tmtextit{Solution:} All positive integers that are divisible by 6.

(e) $\{x^2 | x$ is a prime\}

\tmtextit{Solution:} The set of all the primes squared.

(f) $\{a / b \in \mathbbm{Q}| a + b = 1 \tmop{and} a, b \in \mathbbm{Q}\}$

\tmtextit{Solution:} The set of all rationals such that the sum of the
numerator and denominator is 1.

(g) $\{(x, y) \in \mathbbm{R}^2 | x^2 + y^2 = 1\}$

\tmtextit{Solution:} The set of all points that lie on the circle of radius 1.

(h) $\{(x, y) \in \mathbbm{R}^2 | y = 2 x \tmop{and} y = 3 x\}$

\tmtextit{Solution:} Since $2 x = 3 x$ only when $x = 0$, then the set only
contains the ordered pair $(0, 0)$.

1.2.6 Prove that if $a, b, c, \tmop{and} d$ are any objects, not necessarily
distinct from one another, then

$\{\{a\}, \{a, b\}=\{\{c\}, \{c, d\}\}$ iff both $a = c$ and $b = d$.

\tmtextit{Solution:} If $a = c \tmop{and} b = d$, then $\{a\}=\{c\} \tmop{and}
\{a, b\}=\{c, d\}$. More importantly,

$\{\{a\}, \{a, b\}=\{\{c\}, \{c, d\}\}$. Now if $\{\{a\}, \{a, b\}\}=\{\{c\},
\{c, d\}\}$, then $\{a\}=\{c\}$ since these are the only unit set elements in
their respective sets and this implies that $a = c$. Furthermore $\{a,
b\}=\{c, d\}$ and with $a = c$, we can only conclude that $b = d$.

1.3.1 Prove each of the following, using any properties of numbers that may
be needed.

(a) $\{x \in \mathbbm{Z}| \tmop{for} \tmop{an} \tmop{integer} y, x = 6 y\}=\{x
\in \mathbbm{Z}| \tmop{for} \tmop{integers} u \tmop{and} v, x = 2 u \tmop{and}
x = 3 v\}$

\tmtextit{Solution:} Since $x = 6 y$ can be written as $x = 2 \cdot 3 \cdot
y$, we see that $x = 2 u \tmop{and} x = 3 v$, when $u = 3 y \tmop{and} v = 2
y$. Thus $\{x \in \mathbbm{Z}| \tmop{for} \tmop{an} \tmop{integer} y, x = 6
y\} \subseteq \{x \in \mathbbm{Z}| \tmop{for} \tmop{integers} u \tmop{and} v,
x = 2 u \tmop{and} x = 3 v\}$. Now, $x = 2 u = 3 v$ and thus $\tmop{lcm} (2,
3) = 6$ must divide $x$ by a factor of some integer $y$. So $\{x \in
\mathbbm{Z}| \tmop{for} \tmop{an} \tmop{integer} y, x = 6 y\} \supseteq \{x
\in \mathbbm{Z}| \tmop{for} \tmop{integers} u \tmop{and} v, x = 2 u \tmop{and}
x = 3 v\}$.

(b) $\{x \in \mathbbm{R}|$for a real number $y, x = y^2 \}=\{x \in
\mathbbm{R}| x \geq 0\}$

\tmtextit{Solution:} For any $y \in \mathbbm{R}, y^2 \geq 0$. Since $x = y^2$,
then $\{x \in \mathbbm{R}|$for a real number $y, x = y^2 \} \subseteq \{x \in
\mathbbm{R}| x \geq 0\}$. Now $x \geq 0 \Rightarrow \sqrt{x} \geq 0$. Thus $y
= \sqrt{x} \in \mathbbm{R}$ and we have $\{x \in \mathbbm{R}|$for a real
number $y, x = y^2 \} \supseteq \{x \in \mathbbm{R}| x \geq 0\}$.

(c) $\{x \in \mathbbm{Z}| \tmop{for} \tmop{an} \tmop{integer} y, x = 6 y\}
\subseteq \{x \in \mathbbm{Z}| \tmop{for} \tmop{an} \tmop{integer} y, x = 2
y\}$

\tmtextit{Solution:} Since $x = 6 y = 2 \cdot 3 \cdot y$, then $x$ is
divisible by 2.

1.3.2 Prove each of the following for sets $A, B, \tmop{and} C .$

(a) If $A \subseteq B \tmop{and} B \subseteq C, \tmop{then} A \subseteq C$.

\tmtextit{Solution:} Since any element in $A$ is found inside $B$ and any
element of $B$ is found inside $C$, then $C$ must contain those elements of
$A$ since it contains all of $B$.

(b) If $A \subseteq B \tmop{and} B \subset C, \tmop{then} A \subset C$.

\tmtextit{Solution:} Any element in $A$ is found in $B$ and $C$ contains $B$
and some elements not in $B$, hence $C$ contains some elements not in $A$
since all of $A$ is contained by $B$.

(c) If $A \subset B \tmop{and} B \subseteq C, \tmop{then} A \subset C$.

\tmtextit{Solution:} All elements from $A$ are in $B$, while $B$ contains some
not in $A$. Also $C$ contains all of $B$, thus $C$ must contain some elements
not in $A$, whether they are in $B$ or not.

(d) If $A \subset B \tmop{and} B \subset C, \tmop{then} A \subset C$.

\tmtextit{Solution:} So $A$ is contained in $B$ with $B$ having some elements
not in $A$ and $C$ contains $B$ with some elements not in $B$, thus $C$ must
also contain elements which cannot be in $A$, seeing as $B$ properly contains
$A$.

1.3.3 Give an example of sets $A, B, C, D$ and $E$ which satisfy the following
conditions simultaneously:

$A \subset B, B \in C, C \subset D, \tmop{and} D \subset E$

\tmtextit{Solution:} $A =\{1\}, B =\{1, 2\}, C =\{B\}, D =\{B, 3\}, E =\{B, 3,
4\}$

1.3.4 Which of the following are true for all sets $A, B, \tmop{and} C$?

(a) If $A \nin B \tmop{and} B \nin C, \tmop{then} A \nin C$.

\tmtextit{Solution:} False; $A =\{1\}, B =\{2\}, C =\{A\}$

(b) If $A \neq B \tmop{and} B \neq C, \tmop{then} A \neq C$.

\tmtextit{Solution:} False; $A =\{1\}, B =\{1, 2\}, C =\{1\}$

(c) If $A \in B \tmop{and} B \nsubseteq C, \tmop{then} A \nin C$.

\tmtextit{Solution:} False; $A =\{1\}, B =\{A, \{2\}\}, C =\{B, A\}$

(d) If $A \subset B \tmop{and} B \subseteq C, \tmop{then} A \nsubseteq C$.

\tmtextit{Solution:} True; at most $B = C$, in which case $A \subset C$.

(e) If $A \subseteq B \tmop{and} B \in C, \tmop{then} A \nin C$.

\tmtextit{Solution:} False; $A =\{1\}, B =\{1, 2\}, C =\{A, B\}$

1.3.5 Show that for every set $A, A \subseteq \varnothing \tmop{iff} A =
\varnothing$.

\tmtextit{Solution:} Assume $A \subseteq \varnothing$. We know that the empty
set is a subset of every set so $\varnothing \subseteq A$ and hence $A =
\varnothing$. Now assume that $A = \varnothing$. But this means that $A
\subseteq \varnothing \tmop{and} \varnothing \subseteq A$. So $A \subseteq
\varnothing$ is trivial.

1.3.6 Let $A_1, A_2, \ldots, A_n$ be $n$ sets. Show that $A_1 \subseteq A_2
\subseteq \cdots \subseteq A_n \subseteq A_1 \tmop{iff} A_1 = A_2 = \cdots =
A_n$

\tmtextit{Solution:} If $A_1 \subseteq A_2 \subseteq \cdots \subseteq A_n
\subseteq A_1, \tmop{then} A_1 = A_n$ since $A_1 \subseteq A_n \tmop{and} A_n
\subseteq A_1$. Now for any $A_i, 1 < i < n$, $A_i \subseteq A_n$, yet $A_1
\subseteq A_i$ and thus $A_n \subseteq A_i$ since $A_1 = A_n$. Hence $A_i =
A_n = A_1$. Now, if $A_1 = A_2 = \cdots = A_n$, then it's clear that $A_1
\subseteq A_2 \subseteq \cdots \subseteq A_n$, yet $A_1 = A_n \Rightarrow A_1
\supseteq A_n$. Hence $A_1 \subseteq A_2 \subseteq \cdots \subseteq A_n
\subseteq A_1$.

1.3.7 Give several examples of a set $X$ such that each element of $X$ is a
subset of $X$.

\tmtextit{Solution:} $X =\{X\}, X =\{\}= \varnothing$

1.3.8 List the members of $\mathcal{P}(A) \tmop{if} A =\{\{1, 2\}, \{3\},
1\}$.

\tmtextit{Solution:} $\mathcal{P}(A) =\{\varnothing, \{1\}, \{\{3\}\}, \{\{1,
2\}\}, \{1, \{3\}\}, \{1, \{1, 2\}\}, \{\{3\}, \{1, 2\}\}, A\}$

1.3.9 For each positive integer $n$, give an example of a set $A_n$ of $n$
elements such that for each pair of elements of $A_n$, one member is an
element of the other.

\tmtextit{Solution:} $A_n =\{a_1, a_2, \ldots, a_n \}=\{a_i | a_i =\{a_j | 1
\leq j < i\} \tmop{and} 1 \leq i \leq n\}$

1.4.1 Prove that for all sets $A$ and $B, \varnothing \subseteq A \cap B
\subseteq A \cup B$.

\tmtextit{Solution:} First, we know that the $\varnothing$ is a subset of
every set, so it must be a subset of $A \cap B$. Now, assume that $x \in A
\cap B$, that means that $x \in A \tmop{and} x \in B$. Since $x$ is both, then
it's clearly in the set containing all elements from $A$ or $B$. Thus $x \in A
\cup B$.

1.4.2 Let $\mathbbm{Z}$ be the universal set and let $A =\{x \in \mathbbm{Z}|
\tmop{for} \tmop{some} \tmop{positive} \tmop{integer} y, x = 2 y\}$,

$B =\{x \in \mathbbm{Z}| \tmop{for} \tmop{some} \tmop{positive} \tmop{integer}
y, x = 2 y - 1\}, C =\{x \in \mathbbm{Z}| x < 10\}$. Describe $\overline{A},
\overline{A \cup B}, \overline{C}, A - \overline{C},$ and $C - (A \cup B)$,
either in prose or by a defining property.

\tmtextit{Solution:} $\overline{A}$ is all positive odd integers, $\overline{A
\cup B}$ is all nonpositive integers, $\overline{C}$ is all positive integers
greater than or equal to 10, $A - \overline{C} =\{2, 4, 6, 8\}$, $C - (A \cup
B) = C \cap \overline{A \cup B} = \overline{A \cup B}$.

1.4.3 Consider the following subsets of $\mathbbm{Z}^+$, the set of positive
integers: $A =\{x \in \mathbbm{Z}^+ | \tmop{for} \tmop{some} \tmop{integer} y,
x = 2 y\}, B =\{x \in \mathbbm{Z}^+ | \tmop{for} \tmop{some} \tmop{integer} y,
x = 2 y + 1\}, C =\{x \in \mathbbm{Z}^+ | \tmop{for} \tmop{some}
\tmop{integer} y, x = 3 y\}$.

(a) Describe $A \cap C, B \cup C, \tmop{and} B - C$.

\tmtextit{Solution:} $A \cap C =\{x \in \mathbbm{Z}^+ | \tmop{for} \tmop{some}
\tmop{integer} y, x = 6 y\}$, $B \cup C =\{x \in \mathbbm{Z}^+ | x$ is a
positive odd integer or $x$ is divisible by $6\}$, $B - C =\{x \in
\mathbbm{Z}^+ | x$ is a positive odd integer and $x$ is not divisible by
$3\}$.

(b) Verify that $A \cap (B \cup C) = (A \cap B) \cup (A \cap C) .$

\tmtextit{Solution:} The intersection between the set that contains all
positive even integers, $A$, and the set containing all positive odd integers
and those positive integers divisible by 6, $B \cup C$, will be all those
positive integers divisible by 6. Now the union between all the positive
integers divisible by 6, $A \cap C$, and the empty set, $A \cap B$, is just
the set all the positive integers divisible by $6$.

1.4.4 If $A$ is any set, what are each of the following sets? $A \cap
\varnothing, A \cup \varnothing, A - \varnothing, A - A, \varnothing - A$.

\tmtextit{Solution:} $A \cap \varnothing = \varnothing, A \cup \varnothing =
A, A - \varnothing = A, A - A = \varnothing, \varnothing - A = \varnothing$

1.4.5 Determine $\varnothing \cap \{\varnothing\}, \{\varnothing\} \cap
\{\varnothing\}, \{\varnothing, \{\varnothing\}\}- \varnothing, \{\varnothing,
\{\varnothing\}\}-\{\varnothing\}, \{\varnothing,
\{\varnothing\}\}-\{\{\varnothing\}\}$.

\tmtextit{Solution:} $\varnothing \cap \{\varnothing\}= \varnothing,
\{\varnothing\} \cap \{\varnothing\}=\{\varnothing\}, \{\varnothing,
\{\varnothing\}\}- \varnothing =\{\varnothing, \{\varnothing\}\},
\{\varnothing, \{\varnothing\}\}-\{\varnothing\}=\{\{\varnothing\}\},
\{\varnothing, \{\varnothing\}\}-\{\{\varnothing\}\}=\{\varnothing\}$

1.4.6 Suppose $A$ and $B$ are subsets of $U$. Show that in each of (a), (b),
and (c) below, if any one of the relations stated holds, then each of the
others holds.

(a) $A \subseteq B, \overline{B} \subseteq \overline{A}, A \cup B = B, A \cap
B = A$

\tmtextit{Solution:} Well let's start with $A \cap B = A$. This means that all
the elements that $A$ and $B$ have in common are $A$, so all the elements of
$A$ are contained in $B$, hence $A \subseteq B$. Then clearly $A \cup B = B$,
since $A$ does not ``add'' anything to $B$. Moreover, all the elements not in
$A$, including those in $B$ contain all those elements not in $B$, so
$\overline{B} \subseteq \overline{A}$.

(b) $A \cap B = \varnothing, A \subseteq \overline{B}, B \subseteq
\overline{A}$

\tmtextit{Solution:} $A \cap B = \varnothing$ means that $A$ and $B$ are
disjoint. Thus all elements not in $B$ would contain $A$ and all those
elements not in $A$ would contain $B$, hence $A \subseteq \overline{B}
\tmop{and} B \subseteq \overline{A}$.

(c) $A \cup B = U, \overline{A} \subseteq B, \overline{B} \subseteq A$

\tmtextit{Solution:} With $A \cup B = U$, then $A$ and $B$ combined have all
elements of the universe of discourse. The complement of $B$ must then be a
subset of $A$ since $A$ might have elements in the universe not in $B$,
clearly then $A$ would have those in common with both $A$ and $B$, while the
complement of $B$ does not have those. By same argument, $\overline{B}
\subseteq A$.

1.4.7 Prove that for all sets $A, B, \tmop{and} C ; (A \cap B) \cup C = A \cap
(B \cup C) \tmop{iff} C \subseteq A$.

\tmtextit{Solution:} Suppose that $(A \cap B) \cup C = A \cap (B \cup C)$.
This means that $(A \cap B) \cup C \subseteq A \cap (B \cup C) \\
$and $(A \cap B) \cup C \supseteq A \cap (B \cup C)$. Consider $(A \cap B)
\cup C \subseteq A \cap (B \cup C)$ and the element $x \in (A \cap B) \cup C$,
which translates to $x \in (A \cap B)$ or $x \in C$. If $x \in A \cap B$, then
$x \in A$ and $x \in B$ which means that $x \in B \cup C$ and $x \in A$. If $x
\in C$, then $x \in B \cup C$ but for $x$ to be an element of $A \cap (B \cup
C)$, then it must be that $x$ is also in $A$ and this implies that $C
\subseteq A$.

1.4.8 Prove that for all sets $A, B, \tmop{and} C ; (A - B) - C = (A - C) -
(B - C)$

\tmtextit{Solution:} Firstly, let $x \in (A - B) - C$. This means that $x \in
(A \cap \overline{B}) \cap \overline{C} $ and hence we have $x \in A$ and $x
\in \overline{B}$ and $x \in \overline{C}$. Since $x \in A \tmop{and} x \nin
C$, then $x \in A - C$. Yet, since $x \nin B \tmop{and} x \in \overline{C}$,
then $x \nin B \cap \overline{C}$ hence $x \in \overline{(B \cap \overline{C}}
$). So $x \in A - C$ and $x \in \overline{(B \cap \overline{C}} $)
$\Rightarrow x \in (A \cap \overline{C}) \cap \text{$\overline{(B \cap
\overline{C}} $)}$. Thus we have $x \in (A - B) - (B - C)$. Now consider $x
\in (A - C) - (B - C)$, so $x \in (A \cap \overline{C}) \cap \overline{(B \cap
\overline{C}})$ which means that $x \in (A \cap \overline{C}$) so $x \in A$
and $x \in \overline{C}$. Plus $x \in \overline{B \cap \overline{C}}$ then $x
\nin B \cap \overline{C} \Rightarrow x \nin B$ or $x \nin \overline{C}$ but we
know that $x \in \overline{C}$, therefore $x \nin B \Rightarrow x \in
\overline{B}$. In conclusion, we have $x \in A$ and $x \in \overline{B}$ and
$x \in \overline{C}$. Hence $x \in (A \cap \overline{B}) \cap \overline{C}$,
so $x \in (A - B) - C$.

1.4.9 (c) Show that for every set $A, A + A = \varnothing \tmop{and} A +
\varnothing = A$.

\tmtextit{Solution:} So $A + A = (A - A) \cup (A - A) = (A \cap \overline{A})
\cup (A \cap \overline{A}) = \varnothing \cup \varnothing = \varnothing$ \ and
$\\
A + \varnothing = (A - \varnothing) \cup (A - \varnothing) = (A \cap U) \cup
(A \cap U) = A \cup A = A$.

1.5.1 Prove that parts $3', 4', \tmop{and} 5'$ of Theorem 5.1 are identities.

\tmtextit{Solution:} $3' : A \cap (B \cup C) = (A \cap B) \cup (A \cap C)$

Say $x \in A \cap (B \cup C)$, then $x \in A \tmop{and} x \in B \tmop{or} x
\in C$. If $x \in B$ and $x \in A$, then $x \in A \cap B$. Otherwise, $x \in
C$ and $x \in A$, which means $x \in A \cap C$. So, either $x \in A \cap C
\tmop{or} x \in A \cap B$. Hence $x \in (A \cap B) \cup (A \cap C)$.
Conversely, say $x \in (A \cap B) \cup (A \cap C)$, then $x \in A \cap B$ or
$x \in A \cap C$. In either case, we have $x \in A$ but $x \in B$ or $x \in
C$. Hence $x \in A \cap (B \cup C)$.

$4' : A \cap U = A$

Say $x \in A \cap U$, then $x \in A \tmop{and} x \in U$. Since $A \subseteq
U$, then $x$ is definitely in $A$. So $A \cap U \subseteq A$. Now, say $x \in
A$. Again since $A \subseteq U$, then $x \in U$ and thus $x \in A \cap U$
showing that $A \subseteq A \cap U$. We have shown our equality.

1.5.2 Prove the unprimed parts of Theorem 5.2 using the membership relation.
Try to prove the same results using only Theorem 5.1. In at least one such
proof write out the dual of each step to demonstrate that a proof of the dual
results.

\tmtextit{Solution:} 11: $A \cup U = U$

Say $x \in A \cup U$, then $x \in A \tmop{or} x \in U$. If $x \in U$, then
we're done. If $x \in A$, then $x \in U$ since $A \subseteq U$ and again we're
done. Conversely, say $x \in U$, then $x \in U \cup A$, no matter what $A$ is.

1.5.3 Using only the identities in Theorems 5.1 and 5.2, show that each of the
following equations is an identity.

(a) $(A \cap B \cap X) \cup (A \cap B \cap C \cap X \cap Y) \cup (A \cap X
\cap \bar{A}) = A \cap B \cap X$

\tmtextit{Solution:} $\begin{array}{ll}
  (A \cap B \cap X) \cup (A \cap B \cap C \cap X \cap Y) \cup (A \cap X \cap
  \bar{A}) & = \text{$(A \cap B \cap X) \cup (U \cup X \cap Y) \cup
  \varnothing$}\\
  & = (A \cap B \cap X)
\end{array}$

(b) $(A \cap B \cap C) \cup ( \overline{A} \cap B \cap C) \cup \overline{B}
\cup \overline{C} = U$

\tmtextit{Solution:} $\begin{array}{ll}
  (A \cap B \cap C) \cup ( \bar{A} \cap B \cap C) \cup \bar{B} \cup \bar{C} &
  = (A \cup \bar{A}) \cap (B \cap C) \cup \bar{B} \cup \bar{C}\\
  & = (B \cup \bar{B}) \cap ( \bar{B} \cup C) \cup \bar{C}\\
  & = \bar{B} \cup (C \cup \bar{C})\\
  & = U
\end{array}$

1.5.4 Rework Exercise 4.9(b), using solely the algebra of sets developed in
this section.

\tmtextit{Solution:} Associative:

$(A - B) - C = (A - C) - (B - C) = (A \cap \bar{C}) \cap ( \overline{B \cap
\bar{C}}) = A \cap \overline{C} \cap ( \overline{B} \cup C) = A \cap ( \bar{C}
\cap \bar{B}) \cup (C \cap \bar{C}) = A \cap ( \bar{C} \cap \bar{B}) \cup
\varnothing = A \cap \bar{B} \cap \bar{C} = (A - B) - C$

1.5.7 Referring to Example 5.2, prove the following.

(a) For all sets $A$ and $B, A = B \tmop{iff} A + B = \varnothing$.

\tmtextit{Solution:} If $A = B$, then $A + B = (A - B) \cup (B - A) = (A - A)
\cup (B - B) = \varnothing \cup \varnothing = \varnothing$. Conversely, if $A
+ B = \varnothing$, then $(A - B) \cup (B - A) = \varnothing$. Of course, this
is only possible when $A - B = \varnothing$ and $B - A = \varnothing$. Yet $A
- B = \varnothing \Rightarrow A \subseteq B$ and $B - A = \varnothing
\Rightarrow B \subseteq A$. Thus we have $A = B$.

(c) For all sets $A$ and $B, A = B = \varnothing \tmop{iff} A \cup B =
\varnothing$.

\tmtextit{Solution:} If $A = B = \varnothing$, then $A \cup B = \varnothing
\cup \varnothing = \varnothing$. Now, if $A \cup B = \varnothing$, then we
know that $A = \varnothing = B$ otherwise this is not possible.

1.6.1 Show that if $(x, y, z) = (u, v, w)$, then $x = u, y = v, \tmop{and} z =
w$.

\tmtextit{Solution:} Since an ordered triple can be defined in terms of
ordered pairs, we have if $(x, y, z) = ((x, y), z)$ $= ((u, v), w) = (u, v,
w)$, then by Theorem 6.1, we know that $(x, y) = (u, v)$ and $z = w$. Thus by
applying Theorem 6.1 again, we get that $x = u \tmop{and} y = v$. QED

1.6.2 Write the members of $\{1, 2\} \times \{2, 3, 4\}$. What are the domain
and range of this relation? What is its graph?

\tmtextit{Solution:} $\{1, 2\} \times \{2, 3, 4\}=\{(1, 2), (1, 3), (1, 4),
(2, 2), (2, 3), (2, 4)\}$

$\tmop{Domain} =\{1, 2\} \tmop{Range} =\{2, 3, 4\}$

1.6.3 State the domain and the range of each of the following relations, and
then draw it's graph.

(a) $\{(x, y) \in \mathbbm{R} \times \mathbbm{R}| x^2 + 4 y^2 = 1\}$

\tmtextit{Solution:} $\tmop{Domain} =\{x| x = \sqrt{1 - 4 y^2}, \tmop{for}
\tmop{some} y \in \mathbbm{R}\}=\{x| 0 \leq x \leq 1\} \\
\tmop{Range} =\{y| y = \frac{\sqrt{1 - x^2}}{2}, \tmop{for} \tmop{some} x \in
\mathbbm{R}\}=\{y| 0 \leq y \leq \frac{1}{2} \}$

(b) $\{(x, y) \in \mathbbm{R} \times \mathbbm{R}| x^2 = y^2 \}$

\tmtextit{Solution:} $\tmop{Domain} =\mathbbm{R}= \tmop{Range}$

(d) $\{(x, y) \in \mathbbm{R} \times \mathbbm{R}| x^2 + y^2 < 1 \tmop{and} x >
0\}$

\tmtextit{Solution:} $\tmop{Domain} =\{x | x = \sqrt{1 - y^2} \tmop{and} x >
0\}=\{x | 0 < x < 1\}; \tmop{Range} =\{y | \um 1 < y < 1\}$

(e) {\tmname{$\{(x, y) \in \mathbbm{R} \times \mathbbm{R}| y \geq 0 \tmop{and}
y \leq x \tmop{and} x + y \leq 1\}$}}

\tmtextit{Solution:} $\tmop{Domain} =\{x | x \leq 1 - y \tmop{and} 0 \leq y
\leq x\}=\{x | 0 \leq x \leq 1\}; \tmop{Range} =\{y | 0 \leq y \leq 1 / 2\}$

1.6.4 Write the relation in Exercise 6.3(c) as the union of four relations and
that in Exercise 6.3(e) as the intersection of three relations.

\tmtextit{Solution:} (e) Let $\rho =\{(x, y) \in \mathbbm{R} \times
\mathbbm{R}| y \geq 0\}, \sigma =\{(x, y) \in \mathbbm{R} \times \mathbbm{R}|
y \leq x\}, \beta =\{(x, y) \in \mathbbm{R} \times \mathbbm{R}| x + y \leq
1\}$, then $\rho \cap \sigma \cap \beta$.

1.6.5 The formation of the cartesian product of two sets is a binary operation
for sets. Show by examples that this operation is neither commutative nor
associative.

\tmtextit{Solution:} To show that the cartesian product is not associative:

$\{a\} \times (\{b\} \times \{c\}) =\{a\} \times \{(b, c)\}=\{(a, (b, c))\}
\neq \{(a, b), c\}=\{(a, b)\} \times \{c\}= (\{a\} \times \{b\}) \times \{c\}$

To show that the cartesian product is not commutative

$\{a\} \times \{b\}=\{(a, b)\} \neq \{(b, a)\}=\{b\} \times \{a\}$

1.6.6 Let $\beta$ be the relation ``is a brother of,'' and let $\sigma$ be the
relation ``is a sister of.'' Describe $\beta \cup \sigma, \beta \cap \sigma,
\tmop{and} \beta - \sigma$.

\tmtextit{Solution:} $\beta \cup \sigma$ would be the relation ``is a sibling
of'', $\beta \cap \sigma$ must be the empty set since it is impossible for a
person to be both a brother and a sister of someone, and $\beta - \sigma$ must
be simply $\beta$ since $\beta \tmop{and} \sigma$ are disjoint.

1.6.7 Let $\beta \tmop{and} \sigma$ have the same meaning as in Exercise 6.6.
Let $A$ be the set of students now in the reader's school. What is $\beta
[A]$? What is $(\beta \cup \sigma) [A]$?

\tmtextit{Solution:} $\beta [A]$ is the set of all people who have a brother
in my school. $(\beta \cup \sigma) [A]$ is the set of all people who have a
brother or sister in my school.

1.6.8 Prove that if $A, B, C, \tmop{and} D$ are sets, then $(A \cap B) \times
(C \cap D) = (A \times C) \cap (B \times D) .$ Deduce that the cartesian
multiplication of sets distributes over the operation of intersection, that
is, that $(A \cap B) \times C = (A \times C) \cap (B \times D)$ and $A \times
(B \cap C) = (A \times B) \cap (A \times C)$ for all $A, B, \tmop{and} C.$

\tmtextit{Solution:} Say $x \in (A \cap B) \times (C \cap D)$. Then $x = (y,
z)$ for some $y \in A \cap B$ and some $z \in C \cap D$. So $y \in A$ and $y
\in B$ and $z \in C$ and $z \in D$. Now $y \in A \tmop{and} z \in C$, so $x
\in A \times C$ and with $y \in B$ and $z \in D$, then $x \in B \times D$.
Hence $x \in (A \times C) \cap (B \times D)$. Now, suppose that $x \in (A
\times C) \cap (B \times D)$, then $x \in A \times C$ and $x \in B \times D$.
But this means for some $y \in A$ and some $z \in C$ such that $x = (y, z)$,
then $y \in B$ and $z \in D$. Hence $y \in A$ and $y \in B$, so $y \in A \cap
B$ and $z \in C$ and $z \in D$, thus $z \in C \cap D$. Conclusively we get $x
\in (A \cap B) \times (C \cap D)$.

1.6.9 Exhibit four sets $A, B, C, \tmop{and} D$ for which $(A \cup B) \times
(C \cup D) \neq (A \times C) \cup (B \times D) .$

\tmtextit{Solution:} Let $A =\{1\}, B =\{2\}, C =\{3\}, \tmop{and} D =\{4\}$,
then $(A \cup B) \times (C \cup D) =\{1, 2\} \times \{3, 4\}=\{(1, 3), (1, 4),
(2, 3), (2, 4)\} \neq \{(1, 3), (2, 4)\}=\{(1, 3)\} \cup \{(2, 4)\}= (A \times
C) \cup (B \times D) .$

1.6.10 In spite of the result in the preceding exercise, cartesian
multiplication distributes over the operation of union. Prove this.

\tmtextit{Solution:} Say $x \in A \times (B \cup C)$, then $x = (y, z)$ for
some $y \in A$ and $z \in B$ or $z \in C$. If $z \in B$, then $x \in A \times
B$. Otherwise, $z \in C \Rightarrow x \in A \times C$. Thus $x \in A \times B
\tmop{or} x \in A \times C$, so $x \in (A \times B) \cup (A \times C)$. Now,
suppose that $x \in (A \times B) \cup (A \times C)$, which means $x \in A
\times B$ or $x \in A \times C$. So for some $z \in B \tmop{or} z \in C$ and
some $y \in A$, $x = (y, x)$. But $z \in B \tmop{or} z \in C$, so $z \in B
\cup C$. Thus $y \in A$ and $z \in B \cup C$, hence $x \in A \times (B \cup
C)$.

1.7.4 Let $\rho \tmop{and} \sigma$ be equivalence relations. Prove that $\rho
\cap \sigma$ is an equivalence relation.

\tmtextit{Solution:} To show that $\rho \cap \sigma$ is an equivalence
relation, we show that it is reflexive, symmetric, and transitive. For
reflexivity, we remark that any element $(x, x) \in \rho \tmop{and} (x, x) \in
\sigma$ means that $(x, x) \in \rho \cap \sigma$. To show that it is
symmetric, if $(x, y) \in \rho \cap \sigma$, then we know that $(x, y) \in
\rho \tmop{and} (x, y) \in \sigma$, but since $\rho \tmop{and} \sigma$ are
equivalence relations then $(y, x) \in \rho \tmop{and} (y, x) \in \sigma$,
hence it must be that $(y, x) \in \rho \cap \sigma$. For transitivity, if $(x,
y) \tmop{and} (y, z)$ both in $\rho \cap \sigma$, then $(x, y), (y, z) \in
\rho$ and $(x, y), (y, z) \in \sigma$, but since both $\sigma \tmop{and} \rho$
are both equivalence relations then $(x, z) \in \rho$ and $(x, z) \in \sigma$,
hence $(x, z) \in \rho \cap \sigma$.

1.7.5 Let $\rho$ be an equivalence relation on $X \tmop{and} \tmop{let} Y$ be
a set. Show that $\rho \cap (Y \times Y)$ is an equivalence relation on $X
\cap Y$.

\tmtextit{Solution:} Let $\sigma$ denote the relation $\rho \cap (Y \times Y)$
and we say that $x \sigma y$ iff $(x, y) \in \sigma$. Now, $x \sigma x$ since
for all $x \in X$, $x \rho x$ and

1.7.6 Give an example of these relations.

(a) A relation which is reflexive and symmetric but not transitive.

\tmtextit{Solution:} The relation $\rho$ where $x \rho y$ iff $| x - y | \leq
1$.

(b) A relation which is reflexive and transitive but not symmetric.

\tmtextit{Solution:} The relation $\leq$ is reflexive and transitive but not
symmetric.

(c) A relation which is symmetric and transitive but not reflexive in some
set.

\tmtextit{Solution:} The relation $\rho$ such that $x \rho y$ is not true.

1.7.10 Let $\rho$ be a relation which is reflexive and transitive in the set
$A$. For $a, b \in A$, define $a \sim b \tmop{iff} a \rho b \tmop{and} b \rho
a$.

(a) Show that $\sim$ is an equivalence relation on $A$.

\tmtextit{Solution:} To show $\sim$ is reflexive, we note that for any $a \in
A$, that $a \rho a$ since $\rho$ is reflexive, thus $a \sim a$. If $a \sim b$,
then $a \rho b \tmop{and} b \rho a$, but $b \sim a$ iff $b \rho a \tmop{and} a
\rho b$, so $\sim$ is symmetric. Now suppose that $a \sim b \tmop{and} b \sim
c$, this means that $a \rho b \tmop{and} b \rho a \tmop{and} b \rho c
\tmop{and} c \rho b$, yet $\rho$ is transitive, so $a \rho c$ and $c \rho a$,
hence $a \sim c$ and $\sim$ is transitive.

(b) For $[a], [b] \in A / \sim,$define $[a] \rho' [b] \tmop{iff} a \rho b$.
Show that this definition is independent of $a$ and $b$ in the sense that if
$a' \in [a], b' \in [b], \tmop{and} a \rho b, \tmop{then} a' \rho b' .$

\tmtextit{Solution:} Firstly, if $a' \in [a] \tmop{and} b' \in [b]$, then we
know that means that $[a'] = [a] \tmop{and} [b'] = [b]$. Thus if $a \rho b$
and $a' \in [a] \tmop{and} b' \in [b]$, then $[a] \rho' [b]$ and likewise it
must also be that $[a'] \rho' [b']$. Hence by definition of $\rho'$, $a' \rho
b'$.

(c) Show that $\rho'$ is reflexive and transitive. Further, show that if $[a]
\rho' [b] \tmop{and} [b] \rho' [a], \tmop{then} [a] = [b]$.

\tmtextit{Solution:} $[a] \rho' [a] \tmop{iff} a \rho a$, but $\rho$ is
reflexive so $a \rho a$ holds true for all $a$ and thus $\rho'$ is reflexive.
If $[a] \rho' [b] \tmop{and} [b] \rho' [c]$, then $a \rho b \tmop{and} b \rho
c$, but $\rho$ is transitive so $a \rho c$ and hence $[a] \rho' [c]$. Now if
$[a] \rho' [b] \tmop{and} [b] \rho' [a]$, then $a \rho b \tmop{and} b \rho a$.
But $\rho$ is not symmetric, so $a \rho b \tmop{and} b \rho a$ are only
possible when $a = b$, due to the reflexiveness of $\rho$. So it must be that
$[a] = [b]$.

1.7.11 In the set $\mathbbm{Z}^+ \times \mathbbm{Z}^+$ define $(a, b) \sim (c,
d) \tmop{iff} a + d = b + c$. Show that $\sim$ is an equivalence relation on
this set. Indicate the graph $\mathbbm{Z}^+ \times \mathbbm{Z}^+$, and
describe the $\sim$-equivalence classes.

\tmtextit{Solution:} To show $\sim$ is reflexive, $(a, b) \sim (a, b)$ since
$a + b = b + a$ which we know is true by commutativity of integers. Now if
$(a, b) \sim (c, d)$, then $a + d = b + c$ and $(c, d) \sim (a, b)$ since $c +
b = d + a$, which we know is true again by commutativity, hence we have shown
that $\sim$ is symmetric. To see that $\sim$ is transitive, if $(a, b) \sim
(c, d) \tmop{and} (c, d) \sim (e, f)$, so $a + d = b + c \tmop{and} c + f = d
+ e$, then $a + d = b + d + e - f \Leftrightarrow a + f = b + e$ thus $(a, b)
\sim (e, f)$.

1.8.1 Give an example of a function on $\mathbbm{R} \tmop{onto} \mathbbm{Z}$.

\tmtextit{Solution:} The floor function $f : \mathbbm{R} \rightarrow
\mathbbm{Z}$, such that $f (x)$ is the biggest integer that is $\leq$ to $x$.

1.8.2 Show that if $A \subseteq X, \tmop{then} i_X | A = i_A$.

\tmtextit{Solution:} We know that $i_X (x) = x$ for all $x \in X$. So let $A
\subseteq X$, then for all $a \in A$, $(i_X | A) (a) = i_X (a) = a$. Thus $i_X
|A =\{(a, a) | a \in A\}$ but this is just the definition of $i_A$, so $i_X |A
= i_A$.

1.8.3 If $X \tmop{and} Y$ are sets of $n$ and $m$ elements, respectively,
$Y^X$ has how many elements? How many members of $\mathcal{P}(X \times Y)$ are
functions?

\tmtextit{Solution:} We have $m$ possible choices for each of the $n$ elements
of $X$, thus we have $m^n$ elements in $Y^X$. $Y^X$ are all the functions
possible from $X \rightarrow Y$, thus it must be that the only members of
$\mathcal{P}(X \times Y)$ are those that are in $Y^X$ since $Y^X \subseteq
\mathcal{P}(X \times Y)$, so the answer is $m^n$.

1.8.4 Using only mappings of the form $f : \text{$\mathbbm{Z}^+ \rightarrow
\mathbbm{Z}^+$}$, give an example of a function which

(a) is one-to-one but not onto;

\tmtextit{Solution:} $f (x) = 2 x$

(b) is onto but not one-to-one.

\tmtextit{Solution:} $f (x) = \left\{ \begin{array}{l}
  x \tmop{when} x = 1\\
  x - 1 \tmop{when} x \neq 1
\end{array} \right.$

1.8.5 Let $A =\{1, 2, \cdots, n\}$. Prove that if a map $f : A \rightarrow A$
is onto, then it is one-to-one, and that if a map $g : A \rightarrow A$ is
one-to-one, then it is onto.

\tmtextit{Solution:} Suppose that $f : A \rightarrow A$ is onto, then since
for all $a \in f [A] = A$, there being $n$, there must exist an element $x \in
A$ such that $f (x) = a$. But since there are $n$ elements in the domain that
means that $x$ gets mapped uniquely to an element in $f [A]$ since otherwise
it would violate the definition of a function, that is to say either we didn't
use all the elements in the domain or some $x$ maps to more than one element
in the range. Hence $f$ is one-to-one. Now, if $g : A \rightarrow A$ is
one-to-one, then the $n$ elements of the domain get mapped to exactly $n$
elements in the range. But this is the size of the range, hence $g$ is onto.

1.8.6 Let $f : \mathbbm{R}^+ \rightarrow \mathbbm{R}$, where $f (x) = \int_1^x
\frac{\tmop{dt}}{t}$. Show as best you can that $f$ is a one-to-one and onto
function.

\tmtextit{Solution:} Since $\int^x_1 \frac{\tmop{dt}}{t} = \ln x$, then $f (x)
= \ln x$. Suppose that $f (x_1) = f (x_2)$, which means that $\ln x_1 = \ln
x_2$, applying the exponential function to both sides of the equation, yields
$e^{\ln x_!} = e^{\ln x_2} \Leftrightarrow x_1 = x_2$ and thus $f$ is
one-to-one. \

1.8.8 Referring to Example 8.8, prove that if $f$ is a function and $A
\tmop{and} B$ are sets, then $f [A \cup B] = f [A] \cup f [B]$.

\tmtextit{Solution:} Let $y \in f [A \cup B]$, then that means that there
exists an $x \in A \cup B$ such that $f (x) = y$. Since $x \in A \cup B$, then
$x \in A$ or $x \in B$. If $x \in A$, then $y \in f [A]$ and clearly $y \in f
[A] \cup f [B]$. Otherwise, $x \in B$, in which case $y \in f [B]$ and clearly
$y \in f [A] \cup f [B]$. Hence, in either case, $y \in f [A] \cup f [B]$. Now
suppose that $y \in f [A] \cup f [B]$, then $y \in f [A] \tmop{or} y \in f
[B]$. If $y \in f [A]$, then there is an $x \in A \tmop{such} \tmop{that} f
(x) = y$. Since $x \in A$, then clearly $x \in A \cup B$ and thus $y \in f [A
\cup B]$. Otherwise $y \in f [B]$, which means that there is an $x \in B$ such
that $f (x) = y$. Since $x \in B$ then clearly $x \in A \cup B$ and thus $y
\in f [A \cup B]$. In both cases, $y \in f [A \cup B]$.

1.8.9 Referring to the preceding exercise, prove further that $f [A \cap B]
\subseteq f [A] \cap f [B]$, and show that proper inclusion can occur.

\tmtextit{Solution:} Let $y \in f [A \cap B]$. This means that there exists an
$x \in A \cap B \tmop{such} \tmop{that} f (x) = y$. But since $x \in A \cap
B$, then $x \in A$ and $x \in B$. Thus $y \in f [A] \tmop{and} y \in f [B]$.
Hence $y \in f [A] \cap f [B]$. To show that the proper inclusion can occur,
let $A =\{1, 2\}, B =\{1, 3\}, \tmop{and} f =\{(1, a), (2, b), (3, b$\}. Then
$f [A \cap B] =\{a\} \subset \{a, b\}= f [A] \cap f [B]$.

1.8.10 Prove that a function $f$ is one-to-one iff for all sets $A \tmop{and}
B$, $f [A \cap B] = f [A] \cap f [B]$.

\tmtextit{Solution:} Suppose that $f$ is one-to-one and let $A$ and $B$ be any
set. It suffices to show that $f [A] \cap f [B] \subseteq f [A \cap B]$ since
we already know $f [A \cap B] \subseteq f [A] \cap f [B]$ by exercise 1.8.9.
So say $y \in f [A] \cap f [B]$, then we know $y \in f [A] \tmop{and} y \in f
[B]$, which means that there exists an $x_1 \in A \tmop{such} \tmop{that} f
(x_1) = y$ and $x_2 \in B$ such that $f (x_2) = y$. Hence $f (x_1) = f (x_2) =
y$, yet $f$ is one-to-one, so $x_1 = x_2$. Thus $x_1 = x_2 \in A$ and $x_1 =
x_2 \in B$, so $x_1 = x_2 \in A \cap B$. We conclude then that $y \in f [A
\cap B]$. To show the converse, let us consider the contrapositive, so $f$ is
not injective, then we know that there exists a $x_1 \tmop{and} x_2$ such that
$f (x_1) = f (x_2)$. Now, let $A =\{x_1 \}$ and $B =\{x_2 \}$ to show that $f
[A \cap B] \neq f [A] \cap f [B]$.

1.8.11 Prove that a function $f : X \rightarrow Y$ is onto $Y \tmop{iff} f [X
- A] \supseteq Y - f [A]$ for all sets $A$.

\tmtextit{Solution:} Suppose that $f$ is onto. For any set $A$ either $A \cap
X = \varnothing$ or $A \cap X \neq \varnothing$. If $A \cap X = \varnothing$,
then $f [X - A] = f [X] = Y = Y - f [A]$ since $f$ is onto which makes $f [X -
A] \supseteq Y - f [A]$ clear. Now assume that $A \cap X \neq \varnothing$ and
let $y \in (Y - f [A])$. This means that if $f (x) = y$, then $x \in X - A$.
Thus $y \in f [X - A]$ and we have our inclusion. Conversely, suppose that $Y
- f [A] \subseteq f [X - A]$. \ So $f : X \rightarrow Y$ and it's clear that
if $A \cap X = \varnothing$, then $Y - \varnothing = Y \subseteq f [X] = f [X
- A]$. Since $f [X] \subseteq Y$ by definition, then $f [X] = Y$ and $f$ is
onto. Otherwise, $A \cap X \neq \varnothing$ and we see that $(Y - f [A]) \cup
f [A] = Y \subseteq f [X] = f [(X - A) \cup A] = f [X - A] \cup f [A]$. Again
$f [X] \subseteq Y$, so $Y = f [X]$ and $f$ is onto.

1.8.12 Prove that a function $f : X \rightarrow Y$ is one-to-one and onto iff
$f [X - A] = Y - f [A]$ for all sets $A$.

\tmtextit{Solution:} Suppose that $f$ is one-to-one and onto. Then by 1.8.11,
$Y - f [A] \subseteq f [X - A]$, so let $y \in f [X - A]$. Now we know
$\exists x \in (X - A)$ such that $f (x) = y$. Then $x \in X$ and $x \nin A$,
so $y \in f [A]$ but $y \in Y$. So $y \in Y - f [A]$ and $f [X - A] \subseteq
Y - f [A]$. Hence $f [X - A] = Y - f [A]$. Conversely, suppose that $f [X - A]
= Y - f [A]$. Then by 1.8.11, $f$ is onto. Now we use 1.8.10 and will know
that $f$ is one-to-one if we show that $f [X \cap \overline{A}] = f [X] \cap f
[ \overline{A]}$. Well in the case where $X \cap A = \varnothing$, then we
have $f [X \cap \overline{A}] = f [X] = f [X] \cap f [X] = f [X] \cap f [
\overline{A}]$, where we get the last equality since the function is only
defined on $X$, and $f$ is one-to-one. Now for when $X \cap A \neq
\varnothing$,

1.9.1 Let $f : \mathbbm{R} \rightarrow \mathbbm{R}$ where $f (x) = (1 + (1 -
x)^{1 / 2})^{1 / 5}$. Express $f$ as the composite of four functions, none of
which is the identity function.

\tmtextit{Solution:} Let $g, h, j, k$ all be functions from $\mathbbm{R}
\rightarrow \mathbbm{R}$. Then $g (x) = 1 - x$, $h (x) = x^{1 / 2}$, $j (x) =
1 + x$, $k (x) = x^{1 / 5}$ so that we have $f = k \circ j \circ h \circ g$.

1.9.2 If $f : X \rightarrow Y$ and $A \subseteq X$, show that $f | A = f \circ
i_A$.

\tmtextit{Solution:} We know that $i_A : A \rightarrow A \subseteq X$ where
$f (a) = a$ for all $a \in A$. Then $f \circ i_A =\{(a, f (a)) | a \in A\}$,
but this is exactly $f \cap (A \times Y)$.

1.9.3 Complete the proof of the assertions made in Example 9.2.

\tmtextit{Solution:} We must show that $g : X / \rho \rightarrow f [X]
\tmop{with} g ([x]) = f (x)$ is one-to-one and onto. So, if $g ([x_1]) = g
([x_2])$, then $f (x_1) = f (x_2)$. But $f (x_1) = f (x_2)$ iff $x_1 \rho
x_2$. Hence, $x_2 \in [x_1]$ and $x_1 \in [x_2]$, thus $[x_1] = [x_2]$. Now,
for any $f (x)$ there exists an $[x]$ such that $g ([x]) = f (x)$. \

1.9.5 Prove that $f : A \rightarrow B$ is a one-to-one correspondence between
$A$ and $B$ iff there exists a map $g : B \rightarrow A$ such that $g \circ f
= i_A$ and $f \circ g = i_B$.

\tmtextit{Solution:} Suppose that $f : A \rightarrow B$ is a one-to-one
correspondnce between $A$ and $B$. Then we know that $f^{- 1} : B \rightarrow
A$ is also a one-to-one correspondence between $B$ and $A$. Thus $f^{- 1}
\circ f = i_A$ and $f \circ f^{- 1} = i_B$. Conversely, suppose that there
exists a map $g : B \rightarrow A$ such that $g \circ f = i_A$ and $f \circ g
= i_B$. Now we note that $g = f^{- 1}$ (or $f = g^{- 1}$, if one were so
inclined) and that $g$ is a one-to-one correspondence between $B$ and $A$ by
definition. Thus $f : A \rightarrow B$ must also be a one-to-one
correspondence between $A$ and B.

1.9.6 If $f : A \rightarrow B$ and $g : B \rightarrow C$ are both one-to-one
and onto, show that $g \circ f : A \rightarrow C$ is one-to-one and onto and
that $(g \circ f)^{\um 1} = f^{\um 1} \circ g^{\um 1}$.

\tmtextit{Solution:} Suppose that $g (f (x_1)) = g (f (x_2))$, then since $g$
is one-to-one we have $f (x_1) = f (x_2)$. Yet $f$ is also one-to-one which we
conclude that $x_1 = x_2$ and thus $g \circ f$ is one-to-one. For any element
$c \in C$, we know that there exists an element $b \in B$ such that $g (b) =
c$ since $g$ is onto and that there exists an element $a \in A$ such that $f
(a) = b$, since $f$ is onto. Hence we know that for any element $c \in C$
there exists an element $a \in A$ such that $g (f (a)) = c$, so $g \circ f$ is
onto. For the last part, we refer to the solution of exercise 1.9.11.

1.9.7 For a function $f : A \rightarrow A$, $f^n$ is the standard abbreviation
for $f \circ f \circ \cdots \circ f$ with $n$ occurrences of $f$. Suppose that
$f^n = i_A$. Show that $f$ is one-to-one and onto.

\tmtextit{Solution:} If $f^n = i_A$, then it must be that $f^{n - 1} = f^{-
1}$, since $f^n = f \circ f^{n - 1} = i_A$. Certainly $f^{- 1}$ cannot exist
without $f$ being one-to-one. Furthermore, by the proof to Exercise 1.8.5 that
with $f$ being a map from $A$ to $A$ and also one-to-one, then this implies
that $f$ is also onto.

1.9.8 Justify the following restatement of Theorem 7.1. Let $X$ be a set. Then
there exists a one-to-one correspondence between the equivalence relations on
$X$ and the partitions of $X$.

\tmtextit{Solution:} Consider $f$ the function which takes an equivalence
relation $\rho$ to the collection of equivalence classes of $\rho$, also known
as a partition.

1.9.9 Prove that if the inverse of the function $f$ in $\mathbbm{R}$ exists,
then the graph of $f^{- 1}$ may be obtained from that of $f$ by a reflection
in the line $y = x$.

\tmtextit{Solution:} The graph to $f$ is all the points $(x, y)$ on the
Cartesian plane such that $(x, y) \in f$. Reflection about the line $y = x$
means that every point $(x, y)$ in the Cartesian plane will end up as $(y,
x)$. Thus the graph of the reflection of $f$ will be the points taken from the
set $\{(y, x) | (x, y) \in f\}$, but this is also the definition of the $f^{-
1}$.

1.9.10 Show that each of the following functions has an inverse. Determine the
domain of each inverse and its value at each member of its domain. Further,
sketch the graph of each inverse.

(a) $f : \mathbbm{R} \rightarrow \mathbbm{R}$ where $f (x) = 2 x - 1$

\tmtextit{Solution:} $f^{- 1} : \mathbbm{R} \rightarrow \mathbbm{R}$ where
$f^{- 1} (x) = \frac{1}{2} (x + 1)$. The domain is all real numbers.

(b) $f : \mathbbm{R} \rightarrow \mathbbm{R}$ where $f (x) = x^3$

\tmtextit{Solution:} $f^{- 1} (x) = \sqrt[3]{x}$ where the domain is
$\mathbbm{R}$.

(c) $f =\{(x, (1 - x^2)^{1 / 2}) | 0 \leq x \leq 1\}$

\tmtextit{Solution:} $f = f^{- 1}$ and the domain remains the same.

(d) $f =\{(x, \frac{x}{x - 1}) | - 2 \leq x < 1\}$

\tmtextit{Solution:} $f^{- 1} =\{(x, \frac{x}{x - 1}) | - \infty < x \leq 2 /
3\}$

1.9.11 Establish the identity $(g \circ f)^{- 1} = f^{- 1} \circ g^{- 1}$ for
one-to-one functions $f$ and $g$.

\tmtextit{Solution:} We know that the composition of functions is associative.
Furthermore, for any invertible function $h$, $h \circ h^{- 1} = h^{- 1} \circ
h = i$ and $i \circ h = h \circ i = h$. Now, $(g \circ f) \circ (f^{- 1} \circ
g^{- 1}) = g \circ (f \circ f^{- 1}) \circ g^{- 1} = g \circ i \circ g^{- 1} =
g \circ g^{- 1} = i$ and $(f^{- 1} \circ g^{- 1}) \circ (g \circ f) = f^{- 1}
\circ (g^{- 1} \circ g) \circ f = f^{- 1} \circ i \circ f = f^{- 1} \circ f =
i$. Thus, we have shown that $f^{- 1} \circ g^{- 1}$ is the inverse of $g
\circ f$, namely that $(g \circ f)^{- 1} = f^{- 1} \circ g^{- 1}$.

1.9.12 Prove that if the inverse of $f$ exists, then $f^{- 1} [A \cap B] =
f^{- 1} [A] \cap f^{- 1} [B]$.

\tmtextit{Solution:} Since we know that $f^{- 1}$ is one-to-one by definition,
then the equality holds by the the proof from the solution to Exercise 1.8.10.

1.9.13 The definition of the composite of two functions is applicable to any
pair of relations. With this in mind, show that if $f$ is any function and $g
=\{(y, x) | (x, y) \in f\}$ then $g \circ f$ is an equivalence relation.

\tmtextit{Solution:} We first note that $g \circ f =\{(x, x) | (x, y) \in f
\tmop{and} (y, x) \in g\}$. This is also the identity relation and thus an
equivalence relation. It's clearly reflexive. Symmetric since for any element
$(x, y) \in g \circ f$, then $(y, x) \in g \circ f$ which is true since $y =
x$. Transitivity follows by a similar argument.

1.9.14 Let $A, B, A', \tmop{and} B'$ be sets such that $A$ and $A'$ are in
one-to-one correspondence and $B$ and $B'$ are in one-to-one correspondence.
Show that

(a) there exists a one-to-one correspondence between $A \times B$ and $A'
\times B'$

\tmtextit{Solution:} Since $A$ is in a one-to-one correspondence then we know
that there exists an function $f$ such that for any unique element $a \in A$,
then $f (a) \in A'$. Moreover $f (a)$ is unique and $f [A] = A'$. The same
holds for $B$ and $B'$ except we can call this function $g$. Now, to show that
$A \times B$ is in one-to-one correspondence with $A' \times B'$, then we can
define a function $h$ such that $h ((a, b)) = (f (a), g (b))$ for all $(a, b)
\in A \times B$.

(c) if, further, $A \cap B = \varnothing$and $A' \cap B' = \varnothing$, then
there exists a one-to-one correspondence between $A \cup B$ and $A' \cup B'$

\tmtextit{Solution:} Using $f$ and $g$ from part (a), we define the one-to-one
corresponding function $h : A \cup B \rightarrow A' \cup B' $ by $h (x) =
\left\{ \begin{array}{l}
  f (x) \tmop{when} x \in A\\
  g (x) \tmop{when} x \in B
\end{array} \right.$.

1.9.15 For sets $A, B, \tmop{and} C$ show that

(a) $A \times B$ is in one-to-one correspondence with $B \times A$

\tmtextit{Solution:} Let us consider the mapping $f : A \times B \rightarrow B
\times A$ such that $(a, b) \rightarrow (b, a)$. If $f ((a, b)) = f ((a',
b'))$, then we have $(b, a) = (b', a')$ and $f$ is one-to-one. Furthermore,
consider $(b, a) \in B \times A$, then there exists $(a, b) \in A \times B$
such that $f ((a, b)) = (b, a)$, hence $f$ is onto.

(b) $(A \times B) \times C$ is in one-to-one correspondence with $A \times (B
\times C)$

\tmtextit{Solution:} Consider the mapping $f : (A \times B) \times C
\rightarrow A \times (B \times C)$ such that $((a, b), c) \rightarrow (a, (b,
c))$.

(c) $A \times (B \cup C)$ is in one-to-one correspondence with $(A \times B)
\cup (A \times C)$.

\tmtextit{Solution:} Since $A \times (B \cup C) = (A \times B) \cup (A \times
C)$, we just need to consider the identity mapping here.

1.10.1 Let $\rho$ be a relation, that is, a set each of whose members is an
ordered pair. Show that $\rho$ is a relation in $\bigcup \bigcup \rho$ .

\tmtextit{Solution:} $\bigcup \bigcup \rho = \text{$\bigcup \{ \bigcup
\{\rho\}\}= \text{$\bigcup \{\rho\}= \rho$}$}$ and $\rho$ is a relation.

1.11.1 Show that if $\rho$ is a partial ordering relation, then so is
$\breve{\rho}$.

\tmtextit{Solution:} $\breve{\rho}$ is reflexive since $x
\text{$\breve{\rho}$} x$ iff $x \rho x$, which it must since $\rho$ is a
partial ordering relation. $\breve{\rho}$ is transitive because if $z
\text{$\breve{\rho}$} y$ and $y \text{$\breve{\rho}$} x$ then $z
\text{$\breve{\rho}$} x$ since $\rho$ is transitive. $\breve{\rho}$ being
antisymmetric is straightforward as $x \text{$\breve{\rho}$} y$ and $y
\text{$\breve{\rho}$} x$ mean that $x = y$.

1.11.2 For the set of real-valued continuous functions with the nonnegative
reals as domain, define $f = O (g)$ to mean that there exist positive
constants $M$ and $N$ such that $f (x) \leq \tmop{Mg} (x)$ for all $x > N$.
Show that this is a preordering, and define the associated equivalence
relation.

\tmtextit{Solution:} We show that $f = O (f)$. Clearly, letting $M = 1$, then
$f (x) = M f (x)$ for all $x$ and so the inequality is satisfied. Let $f = O
(g)$ and $g = O (h)$. This means that $f (x) \leq M g (x)$ for all $x > N$ and
$g (y) \leq S h (y)$ for all $y > N'$, thus $f (z) \leq M g (z) \leq M S h
(z)$ for $z > N N'$. So $f = O (h)$. We have shown reflexitivity and
transitivity so this is in fact a preordering.

2.3.1 Show that similarity is an equivalence relation on any collection of
sets.

{\tmem{Solution:}} Reflexive: $A \sim A$ since the identity function is a
bijection.

Symmetric: Suppose $A \sim B$. Let $f$ be the bijection from $A$ to $B$. Then
we know that $f^{- 1}$ is a bijection from $B$ to $A$ and thus $B \sim A$.

Transitive: Suppose $A \sim B$ and $B \sim C$. Let $f$ be a bijection from $A$
to $B$ and $g$ be a bijection from $B$ to $C$. Then $g \circ f$ is a bijection
from $A$ to $C$, hence $A \sim C$.

3.3.5 From properties (1), (3), (4) of addition, prove that

(i) addition has the cancellation property

{\tmem{Solution:}} If $a + x = b + x$. By 4, we know that $x + z = 0$, so $(a
+ x) + z = (b + x) + z$. Then by associativity, we have $a + (x + z) = b + (x
+ z)$ and we conclude that by 3, $a = b$.

(ii) for each $x$ the solution of $z + x = 0$ is unique

{\tmem{Solution:}} Assume that $z + x = 0$ and $z' + x = 0$. Then $z + x = z'
+ x$ and by the previous exercise $z = z'$.

(iii) for each $x$ and $y$, the equation $z + x = y$ has a unique solution.

{\tmem{Solution:}} Assume that $z + x = y$ and $z' + x = y$. Then $z + x = z'
+ x$ and by the previous exercise $z = z'$.

3.3.8 Prove each of the following properties of the system of integers.

(iii) $<$ is transitive

{\tmem{Solution:}} If $x < y$ and $y < z$, then $y - x \in \mathbbm{Z}^+$ and
$z - y \in \mathbbm{Z}^+$. By theorem 3.3.1 (11), $(y - x) + (z - y) \in
\mathbbm{Z}^+$ and by associativity and commutativity, we have $(y - y) + (z -
x) = z - x \in \mathbbm{Z}^+$. Hence by theorem 3.3.1 (3), $x < z$.

4.1.1 Translate the following composite sentences into symbolic notation,
using letters to stand for the prime components (which here we understand to
mean sentences which contain no connectives).

(a) Either it is raining or someone left the shower on.

{\tmem{Solution:}} $R \vee S$

(b) If it is foggy tonight, then either John must stay home or he must take a
taxi.

{\tmem{Solution:}} $F \rightarrow S \vee T$

(c) John will sit, and he or George will wait.

{\tmem{Solution:}} $S \wedge (J \vee G)$

(d) John will sit and wait, or George will wait.

{\tmem{Solution:}} $(S \wedge W) \vee G$

(e) I will go either by bus or by taxi.

{\tmem{Solution:}} $G \rightarrow (B \vee T)$

(f) Neither the North nor the South won the Civil War.

{\tmem{Solution:}} $W \rightarrow \sim (N \wedge S)$

(g) If, and only if, irrigation ditches are dug will the crops survive; should
the crops not survive, then the farmers will go bankrupt and leave.

{\tmem{Solution:}} $(D \leftrightarrow C) \wedge (\sim C \rightarrow (B \wedge
L))$

(h) If I am either tired or hungry, then I cannot study.

{\tmem{Solution:}} $(T \vee H) \rightarrow \sim S$

(i) If John gets up and goes to school, he will be happy; and if he does not
get up, he will not be happy.

{\tmem{Solution:}} $(U \wedge S \rightarrow H) \wedge (\sim U \rightarrow \sim
H)$

4.1.2 Let $C$ be ``Today is clear,'' $R$ be ``It is raining today,'' $S$ be
``It is snowing today,'' and $Y$ be ``Yesterday was cloudy.'' Translate into
acceptable English the following.

(b) $Y \leftrightarrow C$

{\tmem{Solution:}} Yesterday was cloudy if and only if today is clear.

(c) $Y \wedge (C \vee R)$

{\tmem{Solution:}} Yesterday was cloudy and either today is clear or it's
raining today.

(d) $(Y \rightarrow R) \vee C$

{\tmem{Solution:}} Today is clear or if yesterday was cloudy, then it is
raining today.

(e) $C \leftrightarrow (R \wedge \neg S) \vee Y$

{\tmem{Solution:}} If and only if, it's clear today either yesterday was
cloudy or it is the case that it is raining today and not snowing.

(f) $(C \leftrightarrow R) \wedge (\neg S \vee Y)$

{\tmem{Solution:}} Today is clear if and only if it is raining today, but
either it's not snowing today or yesterday was cloudy.

4.2.1 Suppose that the statements $P, Q, R$ and $S$ are assigned the truth
values T, F, F, and T, respectively. Find the truth value or each of the
following statements.

(a) $(P \vee Q) \vee R$

{\tmem{Solution:}} T

(b) $P \vee (Q \vee R)$

{\tmem{Solution:}} T

(c) $R \rightarrow (S \wedge P)$

{\tmem{Solution:}} T

(d) $P \rightarrow (R \rightarrow S)$

{\tmem{Solution:}} T

(e) $P \rightarrow (R \vee S)$

{\tmem{Solution:}} T

(f) $P \vee R \leftrightarrow R \wedge \neg S$

{\tmem{Solution:}} F

(g) $S \leftrightarrow P \rightarrow (\neg P \vee S)$

{\tmem{Solution:}} T

(h) $Q \wedge \neg S \rightarrow (P \leftrightarrow S)$

{\tmem{Solution:}} T

(i) $R \wedge S \rightarrow (P \rightarrow \neg Q \vee S)$

{\tmem{Solution:}} T

(j) $(P \vee \neg Q) \vee R \rightarrow (S \wedge \neg S)$

{\tmem{Solution:}} F

4.2.3 Suppose the value of $P \rightarrow Q$ is T; what can be said about the
value of $\neg P \wedge Q \leftrightarrow P \vee Q$ ?

{\tmem{Solution:}} When $P$ is T and $Q$ is T, then $\neg P \wedge Q
\leftrightarrow P \vee Q$ is F. When $P$ is F and $Q$ is T, then $\neg P
\wedge Q \leftrightarrow P \vee Q$ is T. When $P$ is F and $Q$ is F, then
$\neg P \wedge Q \leftrightarrow P \vee Q$ is T. Ultimately, if you do not
know $P$ and $Q$, then you can say nothing about the value of $\neg P \wedge Q
\leftrightarrow P \vee Q$.

4.2.4 (a) Suppose the value of $P \leftrightarrow Q$ is T; what can be said
about the values of $P \leftrightarrow \neg Q$ and $\neg P \leftrightarrow Q$
?

{\tmem{Solution:}} Both $P \leftrightarrow \neg Q$ and $\neg P \leftrightarrow
Q$ are F.

(b) Suppose the value of $P \leftrightarrow Q$ is F; what can be said about
the values of $P \leftrightarrow \neg Q$ and $\neg P \leftrightarrow Q$ ?

{\tmem{Solution:}} $P \leftrightarrow \neg Q$ and $\neg P \leftrightarrow Q$
will be T.

4.2.7 Referring to the statement in the preceding exercise, suppose it is
agreed that: If the government obtains an injunction, then troops will be sent
into the mills. If the troops are sent into the mills, then the strike will
not be settled. The strike will be settled. Management is stubborn. Determine
wehther the statement in question is true or not.

{\tmem{Solution:}} We have $G \rightarrow R, R \rightarrow \neg S, M,
\tmop{and} S$ as all being true statements. Thus we know $\neg (G \rightarrow
R) = G \wedge \neg R$ is false. Now $\begin{array}{lllllllll}
  L & \vee & M & \rightarrow & (S & \leftrightarrow & [G & \wedge & \neg R])\\
  & T &  & \rightarrow & (T & \leftrightarrow &  & F) & \\
  & T &  & \rightarrow &  & F &  &  & \\
  &  &  & F &  &  &  &  & 
\end{array}$ and the statement is false.

5.2.3 Let $\mathbbm{Z}_n$ be the set of residue classes $[a]$ of $\mathbbm{Z}$
modulo $n$. Show that the relation $\{(([a], [b]), [a + b]) | [a], [b] \in
\mathbbm{Z}_n \}$ is a binary operation in $\mathbbm{Z}_n$. Show that
$\mathbbm{Z}_n$ together with this operation and $[0]$ is a commutative group.

{\tmem{Solution:}} We could clearly see that given $[a], [b] \in
\mathbbm{Z}_n$ that $(a + b) \tmop{mod} n \in [c]$ and $[c] \in
\mathbbm{Z}_n$. To show commutativity, we note that $a + b = b + a, \forall a,
b \in \mathbbm{Z}_n$. Thus $f ([a], [b]) = [a + b] = [b + a] = f ([b], [a])$.

5.2.5 Show that $\mathbbm{R}$ together with the operation $\star$ such that
$x \star y = (x^3 + y^3)^{1 / 3}$ and $0$ is a group.

{\tmem{Solution:}} Let $x, y, z \in \mathbbm{R}$, then $x \star (y \star z) =
x \star \sqrt[3]{y^3 + z^3} = \sqrt[3]{x^3 + ( \sqrt[3]{y^3 + z^3})^3 =}
\sqrt[3]{x^3 + y^3 + z^3}^{}$ and $(x \star y) \star z = \sqrt[3]{x^3 + y^3}
\star z = \sqrt[3]{( \sqrt[3]{x^3 + y^3})^3 + z^3 =} \sqrt[3]{x^3 + y^3 +
z^3}^{}$, hence $x \star (y \star z) = (x \star y) \star z$ and we have shown
associativity. We note that $0 \in \mathbbm{R}$ is the identity since $x \star
0 = \sqrt[3]{x^3 + 0^3} = \sqrt[3]{x^3} = x, \forall x \in \mathbbm{R}$. We
know that if $x \in \mathbbm{R}$, then $- x \in \mathbbm{R}$, hence we have $x
\star - x = \sqrt[3]{x^3 + (- x)^3} = \sqrt[3]{x^3 - x^3} = \sqrt[3]{0} = 0$
and $- x = x^{- 1}, \forall x \in \mathbbm{R}$.

5.2.6 Write out the elements of $G (X)$ for $X =\{1, 2\}$ and for $X =\{1, 2,
3\}$. Show that the group associated with the latter set of mappings is not
commutative.

{\tmem{Solution:}} If $X =\{1, 2\}$, then $G (X) =\{\{(1, 1), (2, 2)\}, \{(1,
2), (2, 1)\}\}$. Now, if $X =\{1, 2, 3\}$, then $G (X) =\{\{(1, 1), (2, 2),
(3, 3)\}, \{(1, 2), (2, 3), (3, 1)\}, \{(1, 3), (2, 1), (3, 2)\}, \{(1, 2), 
(2, 1), (3, 3)\}, \{(1, 3), (2, 2), (3, 1)\}, \\
\{(1, 1), (2, 3), (3, 2)\}\}$.
Let $g =\{(1, 2), (2, 3), (3, 1)\}$ and $g' =\{(1, 3), (2, 2), (3, 1)\}$. Then
$g \circ g' =\{(1, 1), (2, 3), (3, 2)\} \neq \{(1, 2), (2, 1), (3, 3)\}= g'
\circ g$.

8.3.3 Show that $\{(1 + 2 m) / (1 + 2 n) | m, n \in \mathbbm{Z}\}$ is a group
under ordinary multiplication.

{\tmem{Solution:}} Let $A =\{(1 + 2 m) / (1 + 2 n) | n, m \in \mathbbm{Z}\}$
under multiplication. Check closure, so let $(1 + 2 m) / (1 + 2 n)$ and $(1 +
2 k) / (1 + 2 r)$ be elements in $A$.
\[ ( \frac{1 + 2 m}{1 + 2 n}) ( \frac{1 + 2 k}{1 + 2 r}) = \frac{1 + 2 m + 2 k
   + 4 k m}{1 + 2 n + 2 r + 4 n r} = \frac{1 + 2 (m + k + 2 k m)}{1 + 2 (n + r
   + 2 n r)} \in A \]
Clearly $1 = \frac{1}{1}$ is the identity. The inverse for any element, $(1 +
2 m) / (1 + 2 n)$, is $(1 + 2 n) / (1 + 2 m)$. Now to show associativity:

8.4.7 Show that if $a, b$ and $a b$ are groups elements each of order 2, then
$a b = b a$.

{\tmem{Solution:}} $(a a) = (a b) (a b)$. Then multiply on the left by $a$
and use association to get $(a a) a = (a a) b (a b)$. Now multiply on the
right by $b$ and use association to yield $(a a) a b = (a a) b a (b b)$. The
result after using what was given $a b = b a$.

\end{document}
