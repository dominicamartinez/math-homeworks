\documentclass[12pt]{article}

\pagestyle{empty}
\usepackage{wrapfig}
\usepackage{shapepar}
\usepackage{fullpage}
\usepackage{amssymb}
\usepackage{amsthm}
\usepackage{calrsfs}
\usepackage{amsmath}
%\usepackage[symbol*]{footmisc}
%\usepackage[enableskew]{youngtab}
\usepackage[all]{xy}
\usepackage{graphicx}

\begin{document}
Analysis with an introduction to Proof (4th ed.) by Steven R. Lay
\begin{enumerate}

\item[13.2] Let $S \subseteq \mathbb{R}$. Mark each statement True or False. Justify each answer.
\begin{enumerate}
\item[a)] bd $S = $ bd $(\mathbb{R}\backslash S)$. \\
True; pg. 131 states, "\ldots since bd $S = $ bd$(\mathbb{R}\backslash S)$\ldots".
\item[b)] bd $S \subseteq \mathbb{R}\backslash S$. \\
False; this is true only if the set $S$ is open.
\item[c)] $S \subseteq S' \subseteq $ cl $S$. \\
False; Example 13.16(c) shows us that it's possible for $S \not\subseteq S'$.
\item[d)] $S$ is closed iff cl $S \subseteq S$. \\
True; by theorem 13.17(c).
\item[e)] $S$ is closed iff $S' \subseteq S$. \\
True; by theorem 13.17(a).
\item[f)] If $x \in S$ and $x$ is not an isolated point of $S$, then $x \in S'$. \\
True; a point is either isolated or an accumulation point.
\item[g)] The set $\mathbb{R}$ of real numbers is neither open nor closed. \\
False; Example 13.8 says that $\mathbb{R}$ is both open and closed.
\item[h)] The intersection of any collection of open sets is open. \\
False; by Theorem 13.10(b), "The intersection of any finite collection of open sets is open.",
but not necessary true for an infinite collection of open sets. See Exercise 13.5(d).
\item[i)] The intersection of any collection of closed sets is closed. \\
True; this is Corollary 13.11(a).
\end{enumerate}

\item[13.5] Classify each of the following sets as open, closed, neither, or both. 
\begin{enumerate}
\item[a)] $\{ 1/n: n \in \mathbb{N} \}$ \\
Neither. bd $\{ 1/n: n \in \mathbb{N} \} = \{ 0, 1 \}$ and $0 \in \mathbb{R}\backslash 
\{ 1/n: n \in \mathbb{N} \}$ \\
and $1 \in \{ 1/n: n \in \mathbb{N} \}$.
\item[b)] $\mathbb{N}$ \\
Closed. bd $\mathbb{N} = \mathbb{N}$ and since $\mathbb{N} \subseteq \mathbb{N}$.
\item[c)] $\mathbb{Q}$ \\
Neither. By theorem 12.12 and 12.14, $\mathbb{Q}$ is not open because we can't find an open
interval that is a subset of $\mathbb{Q}$, since there exists a rational and irrational number between
the interval. For the same reason, $\mathbb{R}\backslash \mathbb{Q}$ is not open, thus $\mathbb{Q}$
cannot be closed.
\item[d)] $\bigcap_{n = 1}^{\infty} \left(0, \frac{1}{n}\right)$ \\
Both. $\emptyset = \bigcap_{n = 1}^{\infty} \left(0, \frac{1}{n}\right)$ and the empty set is
both open and closed.
\item[e)] $\{ x : |x - 5| \leq \frac{1}{2} \}$ \\
Closed. $\{ x : |x - 5| \leq \frac{1}{2} \} = [4\frac{1}{2}, 5\frac{1}{2}]$, thus bd 
$[4\frac{1}{2}, 5\frac{1}{2}] = \{4\frac{1}{2}, 5\frac{1}{2}\}$, both of which are elements of 
$\{ x : |x - 5| \leq \frac{1}{2} \}$.
\item[f)] $\{x: x^2 > 0\}$ \\
Open. bd $\{x: x^2 > 0\} = \{0\}$ and $0 \in \mathbb{R}\backslash \{x: x^2 > 0\}$.
\end{enumerate}

\item[13.6] Find the closure of each set in Exercise 13.5.
\begin{enumerate}
\item[a)] cl $\{ 1/n: n \in \mathbb{N} \} = \{ 1/n: n \in \mathbb{N} \} \cup \,\{0\}$
\item[b)] cl $\mathbb{N} = \mathbb{N} \cup \,\emptyset = \mathbb{N}$.
\item[c)] cl $\mathbb{Q} = \mathbb{Q} \cup \mathbb{R}\backslash \mathbb{Q} = \mathbb{R}$.
\item[d)] cl $\emptyset = \emptyset$.
\item[e)] cl $\{ x : |x - 5| \leq \frac{1}{2} \} = \{ x : |x - 5| \leq \frac{1}{2} \}$ by Theorem 13.17(c).
\item[f)] cl $\{x: x^2 > 0\} = \{x: x^2 > 0\} \cup \{0\}$.
\end{enumerate}

\item[13.7] Let $S$ and $T$ be subsets of $\mathbb{R}$. Find a counterexample for each of the following.
\begin{enumerate}
\item[e)] bd (bd $S$) = bd $S$ \\
Let $S$ be the set of all rational numbers between 0 and 1. Thus bd $S = [0, 1]$, yet
bd (bd $S$) $= \{0, 1\}$, which goes to show that these two are not equal.
\item[g)] bd ($S \cap T$) = (bd $S$) $\cap$ (bd $T$) \\
Let $S = (0, 1)$ and $T = [1, 2)$. So, bd $S = \{0, 1\}$ and bd $T = \{1, 2\}$. Thus
$(\mbox{bd } S) \cap (\mbox{bd } T) = \{1\} \neq \emptyset = \mbox{bd }\emptyset = \mbox{bd }(S \cap T)$.
\end{enumerate}

\item[13.9] Prove the following.
\begin{enumerate}
\item[a)] An accumulation point of a set $S$ is either an interior point of $S$ or a boundary point of $S$. \\
For the sake of contradiction, suppose that $s \in S$ is an accumulation point of $S$ but is not an interior point nor a boundary point of $S$. This means that $s$ must be an interior point of $\mathbb{R}\backslash S$ and that there is a neighborhood of $s$, $N(s, \varepsilon )$, such that $N(s, \varepsilon ) \subseteq \mathbb{R}\backslash S$. Thus, $N(s, \varepsilon ) \cap S = \emptyset$ and $N^*(s, \varepsilon ) \cap S = \emptyset$ but this violates the definition of an accumulation point.
$\blacklozenge$
\end{enumerate}

\item[13.16] 
\begin{enumerate}
\item[a)] Prove: bd $S$ = (cl $S) \cap [$cl $(\mathbb{R}\backslash S)]$. \\
From definition, $x \in \mbox{bd } S$ iff every neighborhood of $x$, $N(x, \varepsilon)$, 
$N(x, \varepsilon) \cap S \neq \emptyset$ and $N(x, \varepsilon) \cap \,\mathbb{R}\backslash S 
\neq \emptyset$. This is the same as saying that every neighborhood of $x$, 
$N(x, \varepsilon) \cap S \neq \emptyset$ and for every neighborhood of $x$, 
$N(x, \varepsilon) \cap \,\mathbb{R}\backslash S \neq \emptyset$. Since this is the definition of
closure, we have $x \in \mbox{cl }S$ and $x \in \mbox{cl }\mathbb{R}\backslash S$. Hence, 
$x \in \mbox{cl }S \cap \mbox{cl }\mathbb{R}\backslash S$. $\blacklozenge$

\item[b)] Prove: bd $S$ is a closed set. \\
By the above proof, bd $S = (\mbox{cl } S) \cap [\mbox{cl }(\mathbb{R}\backslash S)]$. 
By Theorem 13.17(b), $\mbox{cl } S$ and $\mbox{cl }(\mathbb{R}\backslash S)$ are both closed.
Since bd $S$ is an intersection of two closed sets, by corollary 13.11(a), it too must be closed.
$\blacklozenge$
\end{enumerate}

\item[13.19] Let $A$ be a nonempty open subset of $\mathbb{R}$ and let $\mathbb{Q}$ be the set of rationals. Prove that $A \cap \mathbb{Q} \neq \emptyset$.
\begin{enumerate}
\item[] By Theorem 12.12 (the density of $\mathbb{Q}$ in $\mathbb{R}$), then for any $x, y \in \mathbb{R}$, $x < y$, there exists a rational number, $w$, such that $x < w < y$. Since $A$ is an open and nonempty set, this means that there is an interior point, $a \in A$. By definition, there is a $N(a, \varepsilon) \subseteq A$, which means that $a - \varepsilon < a < a + \varepsilon$. Thus, by Theorem 12.12 and $\varepsilon > 0$, we can find a rational number, $n \in A$, such that $a - \varepsilon < n < a + \varepsilon$. We have shown that there exists a rational number in $A$, so that $A \cap \mathbb{Q}$ cannot possibly be the empty set. $\blacklozenge$
\end{enumerate}

\item[13.20] Let $S$ and $T$ be subsets of $\mathbb{R}$. Prove the following.
\begin{enumerate}
\item[a)] cl (cl $S$) = cl $S$ \\
This just follows from definition. By theorem 13.17(c), cl (cl $S$) = cl $S$ iff 
cl $S$ is a closed set. Thus by theorem 13.17(b), we know that cl $S$ is in fact a closed set.
$\blacklozenge$

\item[c)] cl $(S \cap T) \subseteq$ (cl $S$) $\cap$ (cl $T$) \\
If $x \in (S \cap T)'$, then $x \in S'$ and $x \in T'$, which is the same as $x \in S' \cap T'$.
Thus, $(S \cap T)' \subseteq S' \cap T'$. Therefore by definition, 
\begin{eqnarray*}
\mbox{cl } (S \cap T) &=& (S \cap T) \cup (S \cap T)' \\
&=& [S \cup (S \cap T)'] \cap [T \cup (S \cap T)'] \hskip .6cm \mbox{by distribution}\\
&\subseteq & [S \cup (S' \cap T')] \cap [T \cup (S' \cap T')] \\
&\subseteq & (S \cup S') \cap (T \cup T') \\
&=& \mbox{cl } S \cap\, \mbox{cl } T
\end{eqnarray*}
\hskip 14.0cm $\blacklozenge$

\item[d)] Find an example to show that equality need not hold in part (c). \\
Let $S = (0, 1)$ and $T = (1, 2)$, then cl $S = [0, 1]$ and cl $T = [1, 2]$. 
Also, $S \cap T = \emptyset$ and cl $\emptyset = \emptyset$. Thus,
cl $(S \cap T) = \mbox{cl } \emptyset = \emptyset \neq \{1\} = [0, 1] \cap [1, 2] = (\mbox{cl }S) \cap (\mbox{cl }T)$.
\end{enumerate}

\item[13.21] Let $S$ and $T$ be subsets of $\mathbb{R}$. Prove the following.
\begin{enumerate}
\item[a)] int $S$ is an open set. \\
int $S$ is the set of all interior points in $S$. This means that for every $s \in \mbox{int}\,S$, there exists a neighborhood, $N$ of $s$, by definition that is a subset of $S$. Since int $S$ is comprised of all these neighborhoods and we know from Example 13.8 that all neighborhoods are open sets, then int $S$ is the union of all the open sets in $S$. By Theorem 13.10(a), we know that the union of any collection of open sets is an open set. Thus int $S$ is an open set. $\blacklozenge$

\item[b)] int (int $S$) = int $S$ \\
Since int (int $S$) is the set of all interior points in int $S$, and all the points 
in $S$ are interior points, it must follow that the inclusion int (int $S$) $\subseteq$ int $S$ always holds. Thus we have int (int $S$) $=$ int $S$. $\blacklozenge$

\item[c)] int $(S \cap T) = ($int $S) \cap ($int $T)$ \\
To prove int $(S \cap T) = ($int $S) \cap ($int $T)$, need to show that \\
int $(S \cap T) \subseteq ($int $S) \cap ($int $T)$ and (int $S)\, \cap $ (int $T) \subseteq $ int$(S \cap T)$.
Firstly, let's show that int $(S \cap T) \subseteq ($int $S) \cap ($int $T)$. By definition, if
$x \in \mbox{int }(S \cap T)$ then there is a neighborhood, $N$ of $x$, such that $N \subseteq S \cap T$.
This means, that $N \subseteq S$ and $N \subseteq T$ which implies that $x \in \mbox{int }S$ and 
$x \in \mbox{int }T$. Thus $x \in (\mbox{int } S) \cap (\mbox{int } T)$, which shows that
int $(S \cap T) \subseteq ($int $S) \cap ($int $T)$. \\
Secondly, we need to show that (int $S)\, \cap $ (int $T) \subseteq $ int$(S \cap T)$.
If $x \in (\mbox{int }S) \cap (\mbox{int }T)$, then $x \in \mbox{int }S$ and $x \in \mbox{int }T$.
By definition there is a neighborhood, $N$ of $x$, such that $N \subseteq S$ and $N \subseteq T$.
Thus $N \subseteq S \cap T$, which implies that $x \in \mbox{int }(S \cap T)$. \\
Thus proven that int $(S \cap T) \subseteq ($int $S) \cap ($int $T)$ and 
(int $S)\, \cap $ (int $T) \subseteq $ int$(S \cap T)$, we have 
int $(S \cap T) = ($int $S) \cap ($int $T)$. $\blacklozenge$

\item[d)] (int $S) \cup ($int $T) \subseteq$ int $(S \cup T)$ \\
To show this, if $x \in (\mbox{int }S) \cup (\mbox{int } T)$, then either 
$x \in \mbox{int } S$ or $x \in \mbox{int }T$. If $x \in \mbox{int } S$ then by definition
there is a neighborhood, $N$ of $x$, such that $N \subseteq S$. Since $S \subseteq (S \cup T)$, then $N \subseteq (S \cup T)$ and thus $x \in \mbox{int } (S \cup T)$. Similarly, if $x \in \mbox{int } T$ then by definition there is a neighborhood, $N$ of $x$, such that $N \subseteq T$. 
Since $T \subseteq (S \cup T)$, then $N \subseteq S \cup T$ and thus $x \in \mbox{int }(S \cup T)$.
$\blacklozenge$
\item[e)] Find an example to show that equality need not hold in part (d). \\
Let $S = (1, 2)$ and $T = [2, 3)$. We have $S \cup T = (1, 3)$ and int $(S \cup T) = (1, 3)$, 
yet int $S = (1, 2)$ and int $T = (2, 3)$. We can now see that \\
(int $S$) $\cup$ (int $T$) $= (1, 2)\cup (2, 3) \subseteq (1, 3)$ but $(1, 2)\cup (2, 3) \neq (1, 3)$.
\end{enumerate}

\item[14.2] Mark each statement True or False. Justify each answer.
\begin{enumerate}
\item[a)] Some unbounded sets are compact.\\
False; this goes against the Heine-Borel theorem.
\item[b)] If $S$ is a compact subset of $\mathbb{R}$, then there is at least one point in $\mathbb{R}$ that is an accumulation point of $S$. \\
False; the Bolzano-Weierstrass theorem states, "If a \emph{bounded} subset $S$ of
$\mathbb{R}$ contains infinitely man points, then there exists a least one point in
$\mathbb{R}$ that is an accumulation point of $S$." Moreover, this problem is false because
of what is said in part(c), as well. 
\item[c)] If $S$ is compact and $x$ is an accumulation point of $S$, then $x \in S$. \\
True; since $S$ is compact, then it is closed. We know that if $S$ is closed, by 
theorem 13.17(a), then $S$ contains all of its accumulation points. 
\item[d)] If $S$ is unbounded, then $S$ has at least one accumulation point. \\
False; the set $\mathbb{N}$ is unbounded and has no accumulation points.
\item[e)] Let $\mathcal{F} = \{ A_i : i \in \mathbb{N} \}$ and suppose that the intersection of any finite subfamily of $\mathcal{F}$ is nonempty. If $\bigcap \mathcal{F} = \emptyset$, then for some $k \in \mathbb{N}, A_k$ is not compact. \\
True; the set from Exercise 13.5(d) is one example.
\end{enumerate}

\item[14.3] Show that each subset of $\mathbb{R}$ is not compact by describing an open cover for it that has no finite subcover.
\begin{enumerate}
\item[a)] [1, 3) \\
$A_n = ${\Large $\left( 0, 2 + \frac{(n - 1)}{n} \right)$} for all $n \in \mathbb{N}$.
\end{enumerate}

\item[14.5] 
\begin{enumerate}
\item[a)] If $S_1$ and $S_2$ area compact subsets of $\mathbb{R}$, prove that $S_1 \cup S_2$ is compact. \\
Let $\mathcal{F}$ be any open cover of both $S_1$ and $S_2$. Since $S_1$ and $S_2$
are both compact, we can let $\mathcal{G}$ be a finite subcover of $S_1$ and 
$\mathcal{H}$ be a finite subcover of $S_2$, such that $\mathcal{G} \subseteq \mathcal{F}$ 
and $\mathcal{H} \subseteq \mathcal{F}$. 
It's clear that $\mathcal{G} \cup \mathcal{H} \subseteq \mathcal{F}$, hence $S_1 \cup S_2$
has a finite subcover in $\mathcal{F}$ and thus is compact.
\item[b)] Find an infinite collection $\{ S_n : n \in \mathbb{N} \}$ of compact sets in $\mathbb{R}$ such taht $\bigcup^\infty_{n=1}S_n$ is not compact. \\
$\{S_n = [0, n] : n \in \mathbb{N}\}$.
\end{enumerate}

\item[14.6] Show that compactness is necessary in Corollary 14.8. That is, find a family of intervals $\{ A_n : n \in \mathbb{N} \}$ with $A_{n+1} \subseteq A_n$ for all $n, \bigcap^\infty_{n=1} A_n = \emptyset$, such that
\begin{enumerate}
\item[a)] The sets $A_n$ are all closed. \\
$\{ A_n = [n, \infty ) : n \in \mathbb{N} \}$.
\item[b)] The sets $A_n$ are all bounded. \\
The set from Exercise 13.5(d); $\{ A_n = (0, 1/n) : n \in \mathbb{N} \}$.
\end{enumerate}

\item[14.7]
\begin{enumerate}
\item[b)] Find an example of a collection of disjoint closed subsets of $\mathbb{R}$ that is not countable. \\
The collection of sets, where each set is a singleton set of an irrational number. 
A singleton set is closed and the set of all irrational numbers is uncountable, thus
since every irrational number is will be represented by a singleton set in this collection, 
this collection must also be uncountable.
\end{enumerate}

\item[14.12] Let $S$ be a subset of $\mathbb{R}$. Prove that $S$ is compact iff every infinite subset of $S$ has an accumulation point in $S$.
\begin{enumerate}
\item[] If $S$ is compact, then every infinite subset of $S$ has an accumulation point in $S$, follows
directly from the Bolzano-Weierstrass theorem. The rest of the proof is verbatim from the one found at \\
http://www.econ.upenn.edu/Courses/2003/summer2/econ897/part1/chpt5.pdf: 
\\ Suppose that every infinite subset of $S$ has an accumulation point in
$S$ but $S$ is not compact. Then $S$ is not bounded or not closed. Suppose first that $S$ is
not bounded. This means that for every $n \in N$ there exists $x_n \in S$ such that $|x_n| > n$.
Construct a subset $T \subseteq S$ as \\
$T = \{x_n \in S : |x_n| > n, |x_n| > |x_m| \wedge |x_n - x_m| > 1, \forall m,n \in \mathbb{N} \wedge m < n\}$
This subset is nonempty and is infinite, for if not we would be contradicting the
unboundedness assumption of $S$. Indeed take $T$ to be a finite set $T = \{x_1, x_2,\ldots,x_N\}$.
This means that there is no $x \in S$ such that $|x| > N + 1$ and $|x| > |x_N|$ and
$|x - x_m| > 1$ for all $m < N$. In other words, for all $x \in S$ we have that $|x| \leq N + 1$
or $|x| < |x-N|$ or $|x - x_m| \leq 1$ meaning the S is bounded. 
Finally, by construction $T$ has no accumulation point. 
Thus we have constructed an infinite subset of $S$ which
does not have an accumulation point in $S$, contradicting the initial assumption. Then $S$ must
be bounded. \\
Suppose now $S$ is not closed. Then there exists $x \in \mathbb{R}\backslash S$ such that $x \in S'$.
Since $x \in S'$, for every $\varepsilon > 0$ there is $x_\varepsilon \in S$ such that 
$|x_\varepsilon - x| < \varepsilon$. Construct a
subset $T \subseteq S$ as \\
$T = \{x_n \in S : |x_n - x| < |x_m - x| \wedge |x_n - x| < 1/n$ for every $m, n \in \mathbb{N}$ and $m< n\}$
This subset $T$ is nonempty and infinite, for if not we would be contradicting that $x$
is an accumulation point of $S$. Moreover $x$ is the unique limit point of $T$ in $\mathbb{R}$.
To show this assume it were not, and let $y \in R$ and $y \neq x$ be another accumulation point of $T$.
Let $\delta = \frac{|x - y|}{3}$. By the Archimedean property there exists $N \in \mathbb{N}$ such that 
$1/N < \delta$. This
implies that for all $n \geq N$ and $x_n \in T, |x_n - y| > \delta$. This leaves at most a finite set
$T = {x_1, x_2..., x_N}$ such that $|x_i - y| < \delta$. Without loss of generality we can assume
that all $x_i \in T$ are different from $y$\ldots otherwise we can remove those $x_i = y$ from $T$,
since we are interested in the properties of deleted neighborhoods of $y$. If $T$ is empty
then we are done, since there exists $\varepsilon = \delta > 0$ for which 
$N^* (y, \varepsilon ) \cap T = \emptyset $, meaning that $y$ is not an accumulation point. 
Now let $T \neq \emptyset$ and call $\varepsilon = \min (|x_i - y|)$ for $x_i \in T$.
For such $\varepsilon > 0$ we also have $N^* (y, \varepsilon) \cap T = \emptyset$, 
leading to the desired result. Thus we
have constructed an infinite subset of $S$ whose only accumulation point is in 
$\mathbb{R}\backslash S$ and
not in $S$, contradicting the initial assumption. Therefore $S$ must be both closed and
bounded, hence compact.
\end{enumerate}

\item[16.2] Mark each statement True or False. Justify each answer.
\begin{enumerate}
\item[a)] If $s_n \rightarrow 0$, then for every $\epsilon > 0$ there exists $N \in \mathbb{R}$ such that $n > N$ implies $s_n < \epsilon$. \\
False; the statement should say, "\ldots $|s_n| < \varepsilon$ \ldots ".
\item[b)] If for every $\epsilon > 0$ there exists $N \in \mathbb{R}$ such that $n > N$ implies $s_n < \epsilon$, then $s_n \rightarrow 0$. \\
False; for same reason as above.
\item[c)] Given sequences $(s_n)$ and $(a_n)$, if for some $s \in \mathbb{R}, k > 0$ and $m \in \mathbb{N}$ we have $|s_n - s| \leq k|a_n|$ for all $n > m$, then lim $s_n = s$. \\
False; the statement is similar to theorem 16.8 except it
leaves out that the limit of $a_n$ must be $0$. 
\item[d)] If $s_n \rightarrow s$ and $s_n \rightarrow t$, then $s = t$. \\
True; by theorem 16.14.
\end{enumerate}

\item[16.4] Find $k > 0$ and $m \in \mathbb{N}$ so that $7n^3 + 13n \leq kn^3$ for all integers $n \geq m$.
\begin{enumerate}
\item[] If $13 \leq n^2$ when $n \geq 4$ then we can choose $m = 4$. Thus \\
$7n^3 + 13n \leq 7n^3 + n^3 = 8n^3$, from which we can choose $k = 8$.
\end{enumerate}

\item[16.6] Using only Definition 16.2, prove the following.
\begin{enumerate}
\item[a)] For any real number $k$, $\lim_{n \to \infty}(k/n) = 0$. \\
In Example 16.3, they chose $N = 1/\varepsilon$, in this problem we can choose $N = |k|/\varepsilon$. Hence, given $\varepsilon > 0$, let $N = |k|/\varepsilon$. Then for any $n > N$ we have
$|(k/n) - 0| = |k/n| = |k|/n < |k|/N = \varepsilon$. Thus, lim $(k/n) = 0$. $\blacklozenge$
\item[c)]
\[
\lim \frac{3n + 1}{n + 2} = 3
\]
Given $\varepsilon > 0$, let $N = 5/\varepsilon$. Then for any $n > N$ we have 
\[
\left| \frac{3n + 1}{n + 2} - 0 \right| = \left| \frac{3n + 1 - 3n - 6}{n + 2}\right|
= \left| \frac{-5}{n + 2} \right| = \frac{5}{n + 2} <  \frac{5}{n} < \frac{5}{N} = \varepsilon
\]
Thus lim $\frac{3n + 1}{n + 2} = 3$. $\blacklozenge$
\end{enumerate}

\item[16.7] Use any of the results in this section, prove the following.
\begin{enumerate}
\item[a)] 
\[
\lim \frac{1}{1 + 3n} = 0
\]

Clearly, $3n < 3n + 1$ for all $n \in \mathbb{N}$, meaning that $1/(3n + 1) < 1/3n$. Thus,
\[ 
\left| \frac{1}{1 + 3n} - 0 \right| = \frac{1}{1 + 3n} < \frac{1}{3n} = \left( \frac{1}{3} \right) 
\left( \frac{1}{n} \right)
\]
Since the lim $1/n = 0$, theorem 16.8 implies that that lim $1/(1 + 3n) = 0$.
\item[c)] 
\[
\lim \frac{6n^2 + 5}{2n^2 - 3n} = 3
\]

If $n \geq 6$, then $2n^2 - 3n > n^2$ and $5 + 9n < 10n$. Thus,
\[
\left| \frac{6n^2 + 5}{2n^2 - 3n} - 3 \right| = \left| \frac{5 + 9n}{2n^2 - 3n} \right| < 
\frac{10n}{n^2} = 10 \cdot \left( \frac{1}{n} \right) 
\]
Since the lim $1/n = 0$, theorem 16.8 implies that that lim $\frac{6n^2 + 5}{2n^2 - 3n} = 3$.

\item[e)] 
\[
\lim \frac{n^2}{n!} = 0
\]

If $n \geq 6$, then $(n - 1)! > n^2$. Thus, 
\[
\left| \frac{n^2}{n!} - 0 \right| = \left| \frac{n^2}{n \cdot (n - 1)!} \right| =
\left| \frac{n}{(n - 1)!} \right| < \frac{n}{n^2} = 1 \cdot \frac{1}{n}
\]
Since the lim $1/n = 0$, theorem 16.8 implies that that lim $\frac{n^2}{n!} = 0$.
\end{enumerate}

\item[16.8] Show that each of the following sequences is divergent.
\begin{enumerate}
\item[a)] $a_n = 2n$ \\
To prove that the sequence $a_n = 2n$ is divergent, let us suppose that $a_n$ converges
to some real number $s$. Letting $\varepsilon = 1$ in the definition of convergence, we find that
there exists a number $N$ such that $n > N$ implies that $|2n - s| < 1$. We obtain 
$-1 < 2n - s < 1$, which is the same as $2n - 1 < s < 2n + 1$. It's just as well that $2n > N$
which implies $|2(2n) - s| < 1$, which we obtain the inequality $-1 < 4n - s < 1$, which
is the same as $4n - 1 < s < 4n + 1$. Since $s$ cannot satisfy both inequalities, we have 
reached a contradiction. Thus the sequence $(a_n)$ is divergent. $\blacklozenge$

\item[c)] $c_n = \cos (n\pi/3)$ \\
To prove that the sequence $c_n =$ cos$\left( \frac{n\pi}{3} \right)$ 
is divergent, let us suppose that $c_n$ converges
to some real number $s$. Letting $\varepsilon = 1$ in the definition of convergence, we find that
there exists a number $N$ such that $n > N$ implies that \\
$|$ cos$[(n\pi )/3] - s| < 1$. If $n > N$ and $n$ is of a number such that $6$ divided by $n$ 
will yield a remainder of $3$, then we have the inequality $| -1 - s | < 1$, which is the same
as $-2 < s < 0$. On the other hand, if $n > N$ and $n$ is of a number such that $6$ divided
by $n$ will yield a remainder of $0$, then we have the inequality $|\, 1 - s | < 1$, which is
the same as $0 < s < 2$. Since $s$ cannot satisfy both inequalities, we have reached
a contradiction. Thus the sequence $(c_n)$ is divergent. $\blacklozenge$
\end{enumerate}

\item[16.9] For each of the following, prove or give a counterexample.
\begin{enumerate}
\item[a)] If $(s_n)$ converges to $s$, then $(|s_n|)$ converges to $|s|$.\\
As proven in class: Let $\varepsilon > 0$. By definition of convergence, 
$\exists N \in \mathbb{R}$ such that $\forall n > N$, we have $|s_n - s| < \varepsilon$.
By exercise 11.6(a), $||s_n| - |s|| \leq |s_n - s|$. Therefore $\forall \varepsilon > 0$,
$\exists N \in \mathbb{R}$ such that $\forall n > N$, we have $||s_n| - |s|| \leq |s_n - s| < 
\varepsilon$. Thus, we see that lim $|s_n| = |s|$. $\blacklozenge$

\item[b)] If $(|s_n|)$ is convergent, then $(s_n)$ is convergent.\\
Counterexample: Let $s_n = (-1)^n$. We can see that $(|s_n|)$ converges to $1$, 
while $(s_n)$ does not converge as it alternates between $1$ and $-1$.

\item[c)] $\lim s_n = 0$ iff $\lim |s_n| = 0$. \\
If lim $s_n = 0$, then lim $|s_n| = 0$ follows trivially from part (a), since $|\,0| = 0$.
Conversly, if lim $|s_n| = 0$ then by definition, for each $\varepsilon > 0$, 
$\exists N \in \mathbb{R}$ such that $\forall n \in \mathbb{N}$, $n > N$ implies
$||s_n| - 0| < \varepsilon$. By idempotence, $||s_n|| = |s_n|$ and thus, 
the implication $||s_n| - 0| = ||s_n|| = |s_n| < \varepsilon$ is also true, 
which also happens to show that the limit of $(s_n)$ is $0$. $\blacklozenge$
\end{enumerate}

\item[17.2] Mark each statement True or False. Justify each answer.
\begin{enumerate}
\item[a)] If lim $s_n = s$ and lim $t_n = t$, then lim $(s_nt_n) = st$. \\
True; this is theorem 17.1(c).
\item[b)] If lim $s_n = +\infty$, then $(s_n)$ is said to converge to $+\infty$.\\
False; $(s_n)$ is said to \emph{diverge} to $+\infty$.
\item[c)] Given sequences $(s_n)$ and $(t_n)$ with $s_n \leq t_n$ for all $n \in \mathbb{N}$, if lim $s_n = +\infty$, then lim $t_n = +\infty$. \\
True; by theorem 17.4.
\item[d)] Suppose $(s_n)$ is a sequence such that the sequence of ratios $(s_{n+1}/s_n)$ converges to $L$. If $L < 1$, then lim $s_n = 0$. \\
False; $(s_n)$ must be a sequence of positive terms.
\end{enumerate}

\item[17.4] Prove Theorem 17.1(b).
\begin{enumerate}
\item[b)] In one case, we can say that $(t_n)$ is an increasing sequence, that is for all
$t_n \leq t_{n+1}$, and that it converges to $t$. Since by theorem 16.13, $(t_n)$ is bounded
and we can also say that for this type of sequence the supremum would be $t$. 
Thus, it follows that since $t_n \geq 0$
for all $n \in \mathbb{N}$, that $0 \leq t_1 \leq \cdots \leq t_n \leq t$. \\
In another case, we can say that $(t_n)$ is an decreasing sequence, that is for all
$t_n \geq t_{n+1}$, and that it converges to $t$. Since by theorem 16.13, $(t_n)$ is bounded
and we can also say that the infimum would be $t$. Thus, it follows that since $t_n \geq 0$
for all $n \in \mathbb{N}$, that $t_1 \geq \cdots \geq t_n \geq t \geq 0$. \\
In our third case, let $(t_n)$ converge to some $t$ such that \\
$\min\{t_1, t_2, \ldots, t_n\} \leq t \leq \max\{t_1, t_2, \ldots, t_n\}$. Hence,
it follows that since $t_n \geq 0$ for all $n \in \mathbb{N}$, that 
$0 \leq \min\{t_1, t_2, \ldots, t_n\} \leq t$.
\end{enumerate}

\item[17.5] 
\begin{enumerate}
\item[a)] 
\[
s_n = \frac{3 - 2n}{1 + n} = \frac{(3/n) - 2}{(1/n) + 1}
\]
Now lim $(1/n) = 0$ by example 16.3 and lim $(3/n) = 0$ by theorem 17.1(b). Thus
\[
\mbox{lim } \left[ \frac{(3/n) - 2}{(1/n) + 1} \right]
\,\, = \frac{0 - 2}{0 + 1} = \,-2
\]
by theorem 17.1(a). Hence, $(s_n)$ converges to $-2$.

\item[c)] 
\[ 
s_n = \frac{(-1)^nn}{2n - 1} = (-1)^n \cdot \frac{n}{2n - 1}
\]
By theorem 17.1(c), we can find the limit of $(-1)^n$ and $n/(2n - 1)$, which is the same as 
$1/(2 - 1/n)$. Since we know that lim $1/n = 0$, thus the lim $[1/(2 - 1/n)] = 1/2$ and
$(-1)^n$ diverges because it alternates between $1$ and $-1$. No matter the
limit for $n/(2n - 1)$, we can see that $(s_n)$ is going to diverge with no limit due to
$(-1)^n$.

\item[e)] 
\[
s_n = \frac{n^2 - 2}{n + 1}
\]
Given any $M \in \mathbb{R}$, let $N = \max\{1, 4M\}$. Then $n > N$ implies that
$n > 1$ and $n > 4M$. Since $n > 1$, $n + 1 \leq 2n$ and $n^2 - 2 \geq n^2/2$.
Thus for $n \geq N$ we have
\[
\frac{n^2 - 2}{n + 1} \geq \frac{n^2/2}{2n} = \frac{n^2}{4n} = \frac{n}{4} > M
\]
Hence lim $(n^2 - 2)/(n + 1) = +\infty$.

\item[g)] 
\[
s_n = \frac{1 - n}{2^n}
\]
Just by observation we can see that this is probably going to go to $0$. 
To show this is true, for $n \geq 5$, $n^2 < 2^n$ and $1 - n > -n$, thus
\[
\left| \frac{1 - n}{2^n} - 0 \right| = \frac{1 - n}{2^n} < \frac{-n}{n^2} = (-1) \cdot \frac{1}{n}
\]
by theorem 16.8, lim $(s_n) = 0$.

\end{enumerate}

\item[17.6] For each of the following, prove or give a counterexample.
\begin{enumerate}
\item[c)] If $(s_n)$ and $(s_n + t_n)$ are convergent sequences, then $(t_n)$ converges. \\
\emph{Proof by contrapositive:} If $(t_n)$ is a divergent sequence, then either $(s_n)$ is 
divergent or $(s_n + t_n)$ is divergent. If $(s_n)$ is a divergent sequence, then we are done.
On the other hand, $(s_n)$ could be a convergent sequence. For the sake of contradiction, let
$(s_n + t_n)$ be a convergent sequence. Thus it follows from theorem 17.1(a), that $(s_n)$ and
$(t_n)$ both must be convergent, which contradicts that $(t_n)$ is divergent. Hence
$(s_n + t_n)$ must be divergent. $\blacklozenge$

\item[d)] If $(s_n)$ and $(s_nt_n)$ are convergent sequences, then $(t_n)$ converges. \\
\emph{Counterexample:} Let $(s_n)$ be the sequence given by $s_n = 1/n$ 
and $(s_nt_n)$ be the sequence given by $s_nt_n = 1$. Thus, $(s_n)$ and $(s_nt_n)$
are convergent. Hence, $(t_n)$ is the sequence given by $t_n = n$, 
which we know diverges to $+\infty$.
\end{enumerate}

\item[17.8] Give an example of a divergent sequence $(t_n)$ of positive numbers such that lim $(t_{n+1}/t_n) = 1$.
\begin{enumerate}
\item[b)] The sequence given by $t_n = n$ is divergent as well as the lim $(t_{n+1}/t_n) = 1$.
\[
\mbox{lim } \left( \frac{n + 1}{n} \right) = \mbox{lim } \left( \frac{1 + (1/n)}{1} \right) =
\frac{1 + 0}{1} = 1
\]
\end{enumerate}

\item[17.9] Prove Theorem 17.12.
\begin{enumerate}
\item[a)] If $(s_n)$ diverges to $+\infty$, by definition we have, 
for every $M \in \mathbb{R}$ there exists a number $N$ such that
$n > N$ implies that $s_n > M$. Since $s_n \leq t_n$ for all $n \in \mathbb{N}$, 
the implication $t_n > M$ will also be true. Thus, $t_n$ also diverges to
$+\infty$.
\item[b)] If $(t_n)$ diverges to $-\infty$, by definition we have, 
for every $M \in \mathbb{R}$ there exists a number $N$ such that
$n > N$ implies that $t_n < M$. Since $s_n \leq t_n$ for all $n \in \mathbb{N}$, 
the implication $s_n < M$ will also be true. Thus, $s_n$ also diverges to
$-\infty$.
\end{enumerate}

\item[17.15] Prove the following.
\begin{enumerate}
\item[c)] $\lim (\sqrt{n^2 + n} - n) = 1/2$
\[
\sqrt{n^2 + n} - n = \frac{(\sqrt{n^2 + n} - n)(\sqrt{n^2 + n} + n)}{\sqrt{n^2 + n} + n}
= \frac{n}{\sqrt{n^2 + n} + n}
\]
Thus,
\[
\mbox{lim }\left( \frac{n}{\sqrt{n^2 + n} + n} \right) = 
\mbox{lim }\left( \frac{1}{\sqrt{1 + (1/n)} + 1} \right) =
\frac{\mbox{lim }1}{\mbox{lim } [\sqrt{1 + (1/n)}] + \mbox{lim } 1}
\]
and from that we get $1/(1 + 1) = 1/2$. \hskip 7.2cm $\blacklozenge$ 
\end{enumerate}

\item[18.2] Mark each statement True or False. Justify each answer.
\begin{enumerate}
\item[a)] If a convergent sequence is bounded, then it is monotone. \\
False; the convergent sequence, {\large $\left( \frac{(-1)^n}{n} \right)$},
is bounded and it is not monotone.
\item[b)] If $(s_n)$ is an unbounded increasing sequence, then lim $s_n = +\infty$. \\
True; this is theorem 18.8(a).
\item[c)] The Cauchy convergence criterion holds in $\mathbb{Q}$, the ordered field of rational numbers. \\
False; the Cauchy convergence criterion depends on the completeness of $\mathbb{R}$
and we already established in section 12 that $\mathbb{Q}$ is not complete.
\end{enumerate}

\item[18.4] Find an example of a sequence of real numbers satisfying each set of properties.
\begin{enumerate}
\item[a)] Cauchy, but not monotone: {\large $\left( \frac{(-1)^n}{n} \right)$}
\item[b)] Monotone, but not Cauchy: $(n)$
\item[c)] Bounded, but not Cauchy: $s(n) = 1 + (-1)^n$
\end{enumerate}

\item[18.7] Let $s_1 = \sqrt{6}, s_2 = \sqrt{6 + \sqrt{6}}, s_3 = \sqrt{6 + \sqrt{6 + \sqrt{6}}}$,
and in general define $s_{n+1} = \sqrt{6 + s_n}$. Prove that $(s_n)$ converges and find its limit.
\begin{enumerate}
\item[] Let $(s_n)$ be the sequence defined by $s_1 = \sqrt{6}$ and 
$s_{n + 1} = \sqrt{6 + s_n}$ for $n \geq 1$. We shall show that $(s_n)$ is a
bounded increasing sequence. Computing the next three terms of the sequence we find
\begin{eqnarray*}
s_2 &=& \sqrt{6 + \sqrt{6}} \hskip 3.17cm \approx 2.91 \\
s_3 &=& \sqrt{6 + \sqrt{6 + \sqrt{6}}} \hskip 2.05cm \approx 2.98 \\
s_4 &=& \sqrt{6 + \sqrt{6 + \sqrt{6 + \sqrt{6}}}} \hskip 0.9cm \approx 3.0
\end{eqnarray*}
where the decimals have been rounded off. It appears that the sequence is bounded
above by 4. To see if this conjecture is true, let us try to prove it using 
induction. Certainly, $s_1 = \sqrt{6} < 4$. Now suppose that $s_k < 4$ for 
some $k \in \mathbb{N}$. Then
\[
s_{k + 1} = \sqrt{6 + s_k} < \sqrt{6 + 4} = \sqrt{10} < 4.
\]
Thus we may conclude by induction that $s_n < 4$ for all $n \in \mathbb{N}$. \\
To verify that $(s_n)$ is an increasing sequence, we also argue by induction.
Since $s_1 = \sqrt{6}$ and $s_2 = \sqrt{6 + \sqrt{6}}$, we have $s_1 < s_2$, which
establishes the basis for induction. Now suppose that $s_k < s_{k + 1}$ for some
$k \in \mathbb{N}$. Then we have
\[
s_{k + 1} = \sqrt{6 + s_k} < \sqrt{6 + s_{k + 1}} = s_{k + 2}
\]
Thus the induction step holds and we conclude that $s_n < s_{n + 1}$ for all 
$n \in \mathbb{N}$. \\
Thus $(s_n)$ is an increasing sequence and it is bounded by the interval [2, 4].
We conclude from the monotone convergence therorem (18.3) that $(s_n)$ is 
convergent. The only question that remains is to find the value $s$ to which
it converges. Since lim $s_{n + 1} = $lim $s_n$ (Exercise 16.11), we see that $s$
must satisfy the equation 
\[
s = \sqrt{6 + s}
\]
Solving algebraically for $s$, we obtain $s = 3$ or $s = -2$. Since $s_n \geq \sqrt{6}$
for all $n$, $-2$ cannot be the limit. We conclude that lim $s_n = s = 3$.
\end{enumerate}


\item[18.8] Let $s_1 = k$ and define $s_{n+1} = \sqrt{4s_n - 1}$ for $n \in \mathbb{N}$. Determine for what values of $k$ the sequence $(s_n)$ will be monotone increasing and for what values of $k$ it will be monotone decreasing.
\begin{enumerate}
\item[] In order for $\sqrt{4x - 1}$ to be a real number, then $x \geq 1/4$. We can find
the roots of $x = \sqrt{4x - 1}$ to be $2 \pm \sqrt{3}$. Thus when $k = 2 \pm \sqrt{3}$, it
follows that the sequence will be both monotone increasing and monotone decreasing, since
$s_1 = s_2 = \cdots = s_n$. Also, since we can say the domain of $x^2 - 4x + 1$ is
$[1/4, +\infty)$, after taking the roots into account, we need to 
say whether the intervals $[1/4, 2 - \sqrt{3}), (2 - \sqrt{3}, 2 + \sqrt{3})$, and 
$(2 + \sqrt{3}, +\infty)$ are monotone increasing or decreasing. \\
Let's start with $[1/4, 2 - \sqrt{3})$. Let $k = 1/4$, then $s_1 = 1/4$ and 
$s_2 = 0$. Now suppose that $s_k > s_{k + 1}$ for some $k \in \mathbb{N}$, then
\[
s_{k + 1} = \sqrt{4s_k - 1} > \sqrt{4s_{k+1} - 1} = s_{k + 2}
\]
This shows that if $k = 1/4$ then it is a monotone decreasing sequence.
More generally, the inequality $k > \sqrt{4k - 1}$ holds true $\forall k \in [1/4, 2 - \sqrt{3})$,
thus if $k \in [1/4, 2 - \sqrt{3})$, then the sequence will also be monotone decreasing. The 
aforementioned inequality also happens to hold true $\forall k \in (2 + \sqrt{3}, +\infty)$, 
thus if $k \in (2 + \sqrt{3}, +\infty)$, then the sequence will also be monotone decreasing.\\
Lastly, the inequality, $k < \sqrt{4k - 1}$ holds true $\forall k \in (2 - \sqrt{3}, 2 + \sqrt{3})$.
This shows that $s_1 < s_2$ and assuming that $s_k < s_{k + 1}$ for some $k \in \mathbb{N}$, then
we can show that $s_k = \sqrt{4s_k - 1} < \sqrt{4s_{k+1} - 1} = s_{k + 2}$. By induction,
we have shown that for all $k$ in this interval will make the sequence monotone increasing.
\end{enumerate}

\item[18.13] Prove Lemma 18.11.
\begin{enumerate}
\item[] By theorem 18.12, we know that every Cauchy sequence is convergent and by theorem 16.13, 
we know that every convergent sequence is bounded. Thus every Cauchy sequence must be bounded.
\end{enumerate}

\item[20.2] Let $f: D \rightarrow \mathbb{R}$ and let $c$ be an accumulation point of $D$. Mark each statement True or False. Justify each answer.
\begin{enumerate}
\item[a)] For any polynomial $P$ and any $c \in \mathbb{R}, \lim_{x \to c}P(x) = P(c)$. \\
True; this is word for word what was said in example 20.14.
\item[b)] For any polynomials $P$ and $Q$ and any $c \in \mathbb{R}$,
\[
\lim_{x \to c}\frac{P(x)}{Q(x)} = \frac{P(c)}{Q(c)}.
\]
True; follows from what was said in part (a) and the quotient of
definition 20.12.
\item[c)] In evaluating $\lim_{x \to a-}f(x)$ we only consider points $x$ that are greater than $a$. \\
False; this is a left-handed limit, which is defined as 
taking into consideration all points less than but no equal to $a$.
\item[d)] If $f$ is defined in a deleted neighborhood of $c$, then $\lim_{x \to c}f(x) = L$ iff 
\\ $\lim_{x \to c+}f(x) = \lim_{x \to c-}f(x) = L$. \\
True; this is the last two sentences of the section.
\end{enumerate}

\item[20.3] Determine the following limits.
\begin{enumerate}
\item[a)]
\[
\lim_{x \to 1} \frac{x^3 + 5}{x^2 + 2} = \frac{1 + 5}{1 + 2} = \frac{6}{3} = 2
\]
\item[c)]
\[
\lim_{x \to 1} \frac{\sqrt{x} - 1}{x - 1} = 
\lim_{x \to 1} \frac{(\sqrt{x} - 1)(\sqrt{x} + 1)}{(x - 1)(\sqrt{x} + 1)} =
\lim_{x \to 1} \frac{1}{\sqrt{x} + 1} = \frac{1}{2}
\]
\item[e)]
\[
\lim_{x \to 0} \frac{x^2 + 3x}{x^2 + 1} = \frac{0 + 0}{0 + 1} = 0
\]
\item[g)] As, $x < 0, |x| = -x$, therefore
\[
\lim_{x \to 0-} \frac{4x}{|x|} = \lim_{x \to 0-} \frac{4x}{-x} = \lim_{x \to 0-} -4 = -4
\]
\end{enumerate}

\item[20.5] Find a $\delta > 0$ so that $|x - 2| < \delta$ implies that $|x^2 - 7x + 10| < 1/3$.
\begin{enumerate}
\item[] First note that $|x^2 - 7x + 10| = |x - 5||x - 2|$. Thus we need to have an upper
bound on the size of $|x - 5|$. Now if $|x - 2| < 1/3$, then 
$|x - 5| = |x - 2 - 3| \leq |x - 2| + |-3| < 4$, so that 
$|x^2 - 7x + 10| = |x - 5||x - 2| < 4|x - 2|$. Thus we take $\delta = 1/12$.
\end{enumerate}


\item[20.6] Use definition 20.1 to prove each limit.
\begin{enumerate}
\item[b)] $\lim_{x \rightarrow -2}(x^2 + 2x + 7) = 7$
\[
|x^2 + 2x + 7 - 7 | = |x^2 + 2x| = |x||x + 2|
\]
If we bound $|x|$ when $x$ is approaching $2$ from the left side, then we make
$|x| < 2$. Thus we can choose $|x + 2| < \varepsilon /2$. Hence we can choose 
$\delta = \min\{2, \varepsilon /2 \}$.
\[
|x^2 + 2x + 7 - 7 | = |x^2 + 2x| = |x||x + 2| < 2|x + 2| < 2\delta \leq \varepsilon
\]
\end{enumerate}

\item[20.11] Prove Theorem 20.10.
\begin{enumerate}
\item[] To prove that (a) $\Rightarrow$ (b), suppose that (b) is false. Let $(s_n)$ be a sequence
in $D$ with $s_n \rightarrow c$ and $s_n \neq c$ for all $n$. Since (b) is false, $(f(s_n))$ 
converges to some value, say $L$. We must now show that any given sequence $(t_n)$ in $D$ with
$t_n \rightarrow c$ and $t_n \neq c$ for all $n$, we have $\lim f(t_n) = L$. We only know from the
negation of (b) that $(f(t_n))$ is convergent. To see the $\lim f(t_n) = L$, consider the sequence
$(u_n) = (s_1, t_1, s_2, t_2, \ldots )$, clearly this sequence converges to $c$. Now let's say
that $f(u_n)$ converged to $M$, $M \neq L$. By theorem 19.4, since $f(u_n)$ converges to 
$M$, then $f(s_n)$ would also converge to $M$ as well. We know that $f(s_n)$ converges to $L$, thus
since it is a subsequence of $f(u_n)$, then $f(u_n)$ converges to $L$ as well. Again, by 
theorem 19.4, since $f(t_n)$ is a subsequence of $f(u_n)$ then it must also converge to $L$,
which following the negation and theorem 20.8, implies that $f$ has a limit at $c$. To prove
(b) $\Rightarrow$ (a), for the sake of contradiction suppose that (a) is false. Then (a) reads
that $f$ has a limit at $c$, which contradicts theorem 20.8 where every sequence $(s_n)$ converging
to $c$ with $s_n \neq c$ for all $n$, then the sequence $f(s_n)$ converges. Thus $f$ must not
have a limit at $c$. $\blacklozenge$
\end{enumerate}

\item[20.16] Let $f: D \rightarrow \mathbb{R}$ and let $c$ be and accumulation point of $D$. Suppose that $\lim_{x \rightarrow c}f(x) > 0$. Prove that there exists a deleted neighborhood $U$ of $c$ such that $f(x) > 0$ for all $x \in U \cap D$.
\begin{enumerate}
\item[] Since $f: D \rightarrow \mathbb{R}$ and $c$ is an accumulation point in $D$ and 
$\lim_{x \to c}f(x) = L > 0$, then that means for each neighborhood $V$ of $L$ there exists
a deleted neighborhood $U$ of $c$ such that $f(U \cap D) \subseteq V$, by theorem 20.2.
Using the $\varepsilon - \delta$ definition, then $V$ is a open interval bounded by
$L - \varepsilon$ and $L + \varepsilon$. Thus since $L > 0$, for any $\varepsilon$,
$0 < \varepsilon < L$ implies that $L - \varepsilon > 0$ and since this happens
to a lower bound of $V$ then we have $0 < L - \varepsilon < f(x)$, for all $x \in U \cap D$. 
Since by definition 20.1, the $\delta$ depends on the $\varepsilon$, 
thus it follows there there exist a $U^*(c, \delta)$ that satisfies our requirements.
$\blacklozenge$
\end{enumerate}

\item[21.2] Let $f: D \rightarrow \mathbb{R}$ and let $c \in D$. Mark each statement True or False. Justify each answer.
\begin{enumerate}
\item[a)] If $f$ is continuous at $c$ and $c$ is an accumulation point of $D$, then $\lim_{x \to c}f(x) = f(c)$. \\
True; as stated in theorem 21.2(d).
\item[b)] Every polynomial is continuous at each point in $\mathbb{R}$. \\
True; as stated in example 21.3.
\item[c)] If $(x_n)$ is a Cauchy sequence in $D$, then $(f(x_n))$ is convergent. \\
False; consider the function $f(1/x) = x$ and the Cauchy sequence (1/n).
\item[d)] If $f: \mathbb{R} \rightarrow \mathbb{R}$ is continuous at each irrational number, then $f$ is continuous on $\mathbb{R}$. \\
False; the modified Dirichlet function is a counterexample.
\item[e)] If $f: \mathbb{R} \rightarrow \mathbb{R}$ and $g: \mathbb{R} \rightarrow \mathbb{R}$ are both continuous on $\mathbb{R}$, then $f \circ g$ and $g \circ f$ are both continuous on $\mathbb{R}$. \\
True; by theorem 21.12 by letting $D = \mathbb{R}$ and $E = \mathbb{R}$.
\end{enumerate}

\item[21.4] Define $f: \mathbb{R} \rightarrow \mathbb{R}$ by $f(x) = x^2 - 3x + 5$. Use Definition 21.1 to prove that $f$ is continuous at 2.
\begin{enumerate}
\item[] 
\[ 
|f(x) - f(c)| = |x^2 - 3x + 5 - (4 - 6 + 5)| = |x^2 - 3x + 2| = |x - 2||x - 1|
\]
We arbitrarily want $|x - 2| < 1$ that way we can bound $|x - 1|$. Hence, 
$|x - 1| = |x - 2 + 1| = \leq |x - 2| + |1| \leq 1 + 1 = 2$. Thus for any given
$\varepsilon > 0$, let $\delta = \min \{1, \varepsilon /2 \}$. Then for all $x$
satisfying $|x - 2| < \delta$ we have $|x - 2| < 1$, so that $|x - 1| < 2$. It
foloows that for these $x$ we have 
\[
|f(x) - f(c)| = |x - 2||x - 1| \leq 2|x - 2| \leq \varepsilon
\]
\end{enumerate}

\item[21.6] Prove or give a counterexample for each statement.
\begin{enumerate}
\item[a)] If $f$ is continuous on $D$ and $k \in \mathbb{R}$, then $kf$ is continuous on $D$. \\
To show that $kf$ is continuous on $D$, we show that 
$\lim kf(x_n) = k \lim f(x_n) = kf(c)$, by the fact that $f$ is continuous on $D$
and theorem 17.1(b).
\item[b)] If $f$ and $f + g$ are continuous on $D$, then $g$ is continuous on $D$. \\
As proven in class: If $f$ is continuous, then $-f$ is also continuous. Since
we can represent $g$ as $(-f) + (f + g)$ and we know that $(f + g)$ is also continuous,
hence $g$ must be continuous.
\item[c)] If $f$ and $fg$ are continuous on $D$, then $g$ is continuous on $D$. \\
If $f$ is continuous and $f(c) \neq 0$ then $1/f$ is also continuous. Since
we can represent $g = (fg)/f$ and we know that $fg$ is continuous then we know that
$g$ is continuous by theorem 21.10(b).
\item[d)] If $f^2$ is continuous on $D$, then $f$ is continuous on $D$. \\
As shown in class: Counterexample: Let $f(x)$ be $1$ when $x \geq 0$ and 
$-1$ when $x < 0$.
\item[e)] If $f$ is continuous on $D$, then $f(D)$ is a bounded set. \\
Counterexample: Let $f(x) = 1/x$ and $D = (0, 1]$.
\item[f)] If $f$ and $g$ are not continuous on $D$, then $f+g$ is not continuous on $D$. \\
Counterexample: Let $f(x)$ be $1$ when $x \geq 0$ and 
$-1$ when $x < 0$, also, let $g(x)$ be $-1$ when $x \geq 0$ and 
$1$ when $x < 0$.
\item[g)] If $f$ and $g$ are not continuous on $D$, then $fg$ is not continuous on $D$. \\
Counterexample: Let $f(x)$ and $g(x)$ be $1$ when $x \geq 0$ and 
$-1$ when $x < 0$.
\item[h)] If $f: D \rightarrow E$ and $g: E \rightarrow F$ are not continuous on $D$ and $E$, respectively, then $g \circ f : D \rightarrow F$ is not continuous on $D$. \\
Counterexample: Consider $ f(x) = 
\begin{cases}
-1, & x < 0 \\
1, & x \geq 0
\end{cases}
$ \quad and \\ 
$g(x) = 
\begin{cases}
1, & x = 1 \vee x = -1 \\
-1, & \mbox{for anything else} 
\end{cases}
$
\end{enumerate}

\item[21.14] Let $f: D \rightarrow \mathbb{R}$ be continuous at $c \in D$. Prove that there exists an $M > 0$ and a neighborhood $U$ of $c$ such that $|f(x)| \leq M$ for all $x \in U \cap D$.
\begin{enumerate}
\item[] Since $f$ is continuous at $c \in D$, by theorem 21.2, for every neighborhood
$V$ of $f(c)$ there exists a neighborhood $U$ of $c$ such that 
$f(U \cap D) \subseteq V$. Using the $\varepsilon - \delta$ definition, 
$V$ is the open interval bounded by $f(c) - \varepsilon$ and $f(c) + \varepsilon$. 
Hence we can choose $M = \max \{ |f(c) - \varepsilon |, |f(c) + \varepsilon |\} > 0$,
such that $\forall x \in U \cap D$, $|f(x)| \leq M$. $\blacklozenge$
\end{enumerate}

\item[21.16] Let $f: \mathbb{R} \rightarrow \mathbb{R}$. Prove that $f$ is continuous on $\mathbb{R}$ iff $f^{-1}(H)$ is a closed set whenever $H$ is a closed set.
\begin{enumerate}
\item[] Let us note that $\mathbb{R} = \mathbb{R}\backslash H \cup H$ and that
$\mathbb{R} = f^{-1}(\mathbb{R}) = f^{-1}(\mathbb{R}\backslash H) \cup f^{-1}(H)$. 
We also know that $H$ is closed whenever $\mathbb{R}\backslash H$ is open and 
vice versa. So if $H$ is closed, then $f^{-1}(H)$ must also be closed if and 
only if $\mathbb{R}\backslash f^{-1}(H)$ is open which also happens to be equal 
to $f^{-1}(\mathbb{R}\backslash H)$, which we know follows from corollary 21.15.
Thus, $f$ is continuous on $\mathbb{R}$ iff $f^{-1}(H)$ is a closed set
whenever $H$ is a closed set. $\blacklozenge$
\end{enumerate}

\item[22.2] Mark each statement True or False. Justify each answer.
\begin{enumerate}
\item[a)] Let $f : [a, b] \rightarrow \mathbb{R}$ be continuous and suppose $f(a) < 0 < f(b)$. Then there exists a point $c$ in $(a, b)$ such that $f(c) = 0$. \\
True; by lemma 22.5.
\item[b)] Let $f : [a, b] \rightarrow \mathbb{R}$ be continuous and suppose $f(a) \leq k \leq f(b)$. Then there exists a point $c \in [a, b]$ such that $f(c) = k$. \\
False; $f(a) < k < f(b)$ must be true.
\item[c)] If $f: D \rightarrow \mathbb{R}$ is continuous and bounded on $D$, then $f$ assumes maximum and minimum values on $D$. \\
False; $D$ must be a compact subset of $\mathbb{R}$.
\end{enumerate}

\item[22.3] Let $f: D \rightarrow \mathbb{R}$ be continuous. For each of the following, prove or give a counterexample.
\begin{enumerate}
\item[a)] If $D$ is open, then $f(D)$ is open. \\
\emph{Counterexample:} Consider $D = \mathbb{R}$ and $f(x) = \sin x$.
\item[b)] If $D$ is closed, then $f(D)$ is closed. \\
\emph{Counterexample:} Consider $D = \mathbb{N}$ and $f(x) = 1/x$.
\item[c)] If $D$ is not open, then $f(D)$ is not open. \\
\emph{Counterexample:} Consider $D = \mathbb{N}$ and $f(x) = 1/x$.
\item[d)] If $D$ is not closed, then $f(D)$ is not closed. \\
\emph{Counterexample:} Consider $D = (0, 1)$ and $f(x) = 5$.
\item[e)] If $D$ is not compact, then $f(D)$ is not compact. \\
\emph{Counterexample:} Consider $D = (0, 1)$ and $f(x) = 5$.
\item[f)] If $D$ is unbounded, then $f(D)$ is unbounded. \\
\emph{Counterexample:} Consider $D = [1, \infty)$ and $f(x) = 1/x$.
\item[g)] If $D$ is finite, then $f(D)$ is finite. \\
This is clearly true as the definition of a function states that
there is exactly 1 mapping for all elements in $D$ 
to some element in the range. Since $D$ is finite, then the 
image, $f(D)$, can have at most the same amount of elements in $D$.
\item[h)] If $D$ is infinite, then $f(D)$ is infinite. \\
\emph{Counterexample:} Consider $D = \mathbb{R}$ and $f(x) = 0$.
\item[j)] If $D$ is an interval that is not open, then $f(D)$ is an interval that is not open. \\
\emph{Counterexample:} Consider $D = \mathbb{N}$ and $f(x) = 1/x$.
\end{enumerate}

\item[22.6] Show that any polynomial of odd degree has at least one real root.
\begin{enumerate}
\item[] Let $P(x) = a_nx^n + \cdots + a_0$ be our odd degree polynomial.
Now either $P(x) \rightarrow \infty$ or $P(x) \rightarrow -\infty$ as
$x \rightarrow \infty$, depending on whether $a_n$ is positive or negative. 
When $P(x) \rightarrow \infty$ as $x \rightarrow \infty$ then
$P(x) \rightarrow -\infty$ as $x \rightarrow -\infty$. It's just as well, that
when $P(x) \rightarrow -\infty$ as $x \rightarrow \infty$, then $P(x) \to \infty$
as $x \rightarrow -\infty$. This property implies that $P(x)$ takes both
positive and negative values, so we can apply the Intermediate Value Theorem
so that we know there exists at least one $c$ such that $P(c) = 0$, meaning
there is at least one root. 
\end{enumerate}

\item[22.10] Suppose that $f: [a, b] \rightarrow \mathbb{R}$ is two-to-one. That is, for each $y \in \mathbb{R}, f^{-1}(\{y\})$ either is empty of contains exactly two points.
\begin{enumerate}
\item[a)] Find an example of such a function. \\
If we can turn a blind eye to $x = 0$ then $f(x) = x^2$ is a 
two-to-one function for $f: \mathbb{R} \rightarrow \mathbb{R}$.
\end{enumerate}

\item[22.13] Let $f$ be a function defined on an interval $I$. We say that $f$ is {\bf strictly increasing} if $x_1 < x_2$ in $I$ implies that $f(x_1) < f(x_2)$. Similarly, $f$ is {\bf strictly decreasing} if $x_1 < x_2$ in $I$ implies that $f(x_1) > f(x_2)$. Prove the following.
\footnote{
The answer to part(a) and (b) was taken from\\
http://students.csci.unt.edu/\%7Ehgc0005/homeworks/h8ra.pdf
}
\begin{enumerate}
\item[a)] If $f$ is continuous and injective on $I$, then $f$ is strictly increasing or strictly decreasing. \\
If $f$ is not strictly increasing nor decreasing, then there exists
$x \in I$ such that $f(x) < f(z)$ for some $z < x$ in $I$. Now choose any 
$y < z \in I$ such that $f(y) < f(x)$. If there exist no such $y$, then let
$x = c$ such that $f(y) < f(c) < f(z)$ and $c \in (z, x)$, this is guaranteed
by the Intermediate Value Theorem applied to $f(z)$ and $f(x)$. Now, applying
again the Intermediate Value Theorem to $f(y)$ and $f(z)$ there exists 
$d \in (y, z)$ such that $f(d) = f(c)$, clearly $d \neq c$, thus this 
contradicts our assumption that $f$ is injective.
\item[b)] If $f$ is strictly increasing and if $f(I)$ is an interval, then $f$ is continuous. Furthermore, $f^{-1}$ is a strictly increasing continuous function on $f(I)$. \\
Let $V$ be the neighborhood $(y_0 - \varepsilon, y_0 + \varepsilon)$ 
such that $V \subseteq f(I)$, then there exists $x_1, x_2 \in I$ such that
$f(x_1) = x - \varepsilon$ and $f(x_2) = x + \varepsilon$, 
clearly $x_1 < x_2$, let $U$ be the neighborhood $(x_1, x_2)$. Now for every
$y \in V$ there exists an $x \in U$ such that $f(x) = y$ (otherwise if 
$y_0 - \varepsilon < y < y_0 + \varepsilon$ and $x < x_1$ or $x > x_2$ would
contradict the fact that the function is strictly increasing), then $f(U) 
\subseteq V$, then by theorem 21.2(c), $f$ is continuous. Now since $f$
is a strictly increasing function on $I$, then for every $x_1 < x_2 \in I$, 
$f(x_1) < f(x_2)$, this implies that for every $f(x_1) < f(x_2) \in f(I)$,
$f^{-1}(f(x_1)) < f^{-1}(f(x_2))$, in other words for every
$f(x_1) < f(x_2) \in f(I)$, $x_1 < x_2$.
\end{enumerate}

\item[23.2] Let $f: D \rightarrow \mathbb{R}$. Mark each statement True or False. Justify each answer.
\begin{enumerate}
\item[a)] In the definition of uniform continuity, the positive $\delta$ depends only on the function $f$ and the given $\epsilon > 0$.\\
False; $\delta$ only depends on $\varepsilon$ in the definition of
uniform continuity.
\item[b)] If $f$ is continuous and $(x_n)$ is a Cauchy sequence in $D$, then $(f(x_n))$ is a Cauchy sequence. \\
False; $f$ must be uniformly continuous by theorem 23.8.
\item[c)] If $f: (a, b) \rightarrow \mathbb{R}$ can be extended to a function that is continuous on $[a, b]$, then $f$ is uniformly continuous on $(a, b)$. \\
True; by theorem 23.9.
\end{enumerate}

\item[23.3] Determine which of the following continuous functions are uniformly continuous on the given set. Justify your answers.
\begin{enumerate}
\item[a)] $f(x) = e^x/x$ on [2, 5] \\
It is uniformly continuous since it is continuous on a compact set.
\item[b)] $f(x) = e^x/x$ on (0, 2) \\
Not uniformly continuous because it cannot be extended to a function
that is continuous because $e^x/x$ is undefined at $0$ on the compact set $[0, 2]$.
\item[c)] $f(x) = x^2 + 3x - 5$ on [0, 4] \\
It is uniformly continuous since it is continuous on a compact set.
\item[d)] $f(x) = x^2 + 3x - 5$ on (1, 3) \\
It is uniformly continuous since it is continuous on the compact set, [1, 3].
\item[e)] $f(x) = 1/x^2$ on (0, 1) \\
It is uniformly continuous since it is continuous on a compact set.
\item[f)] $f(x) = 1/x^2$ on $(0, \infty)$ \\
It is not uniformly continuous since it cannot be extended to a function
that is continuous on $[0, \infty]$ because $\lim_{x \to 0} f(x) = \infty$.
\item[g)] $f(x) = x \sin (1/x)$ on $(0, 1)$. \\
It is uniformly continuous since it can be extended to a function that
is continuous on [0, 1] by example 21.5.
\end{enumerate}

\item[23.4] Prove that each function is uniformly continuous on the given set by directly verifying the $\epsilon-\delta$ property in Definition 23.1.
\begin{enumerate}
\item[b)] $f(x) = \frac{1}{x}$ on $[2, \infty )$
\[
|f(x) - f(y)| = |\frac{1}{x} - \frac{1}{y}| = \frac{|x - y|}{|xy|}
\]
Hence $x$ and $y$ are both positive so $|xy| = xy$. Also, since both 
$x \geq 2$ and $y \geq 2$ then $\frac{|x - y|}{|xy|} \leq |x - y|$.
Given $\varepsilon$, take $\delta = \varepsilon$.
If $|x - y| < \delta$, then \[
|\frac{1}{x} - \frac{1}{y}| = \frac{|x - y|}{|xy|} \leq |x - y| < \delta =
\varepsilon
\]. Thus $f(x)$ is uniformly continuous on $[2, \infty)$.
\end{enumerate}

\item[23.5] Prove that $f(x) = \sqrt{x}$ is uniformly continuous on $[0, \infty )$.
\begin{enumerate}
\item[] Let $\varepsilon > 0$. Choose $\delta = \varepsilon^2$. Since
$x, y \in [0, \infty)$ then we have two cases. \\
Case 1: $x, y \in [0, \varepsilon^2)$. Then $\sqrt{x}\sqrt{y} \in [0, \varepsilon)$,
thus \[ |f(x) - f(y)| = |\sqrt{x} - \sqrt{y}| < |\varepsilon - 0| = \varepsilon \]
Case 2: At least one of $x, y$ is greater or equal to $\varepsilon^2$. In this
case, $\sqrt{x} + \sqrt{y} \geq \sqrt{\varepsilon^2} = \varepsilon$, then
\[
|\sqrt{x} - \sqrt{y}| = \frac{|x - y|}{|\sqrt{x} + \sqrt{y}|} \leq 
\frac{|x - y|}{\varepsilon} < \delta/\varepsilon = \varepsilon^2/\varepsilon 
= \varepsilon 
\].
\end{enumerate}

\item[23.10] Find two real-valued functions $f$ and $g$ that are uniformly continuous on a set $D$, but such that their product $fg$ is not uniformly continuous on $D$.
\begin{enumerate}
\item[] Consider when $D = \mathbb{R}$ and $f(x) = g(x) = x$. By Example 23.4, we
know that $fg = x^2$, which is not uniformly continuous on $\mathbb{R}$.
\end{enumerate}

\item[23.13] Suppose that $f$ is uniformly continuous on $[a, b]$ and uniformly continuous on $[b, c]$. Prove that $f$ is uniformly continuous on $[a, c]$.
\begin{enumerate}
\item[] Given any $\varepsilon > 0$, find $\delta_1$ so that if $x, y \in [a, b]$ 
with $|x - y| < \delta_1$, then $|f(x) - f(y)| < \varepsilon /2$. Similarly, find
$\delta_2$ so that if $x, y \in [b, c]$, and $|x - y| < \delta_2$, then 
$|f(x) - f(y)| < \varepsilon /2$. Let $\delta = \min \{\delta_1, \delta_2 \}$.
Now, suppose that $x, y \in [a, b]$ with $|x - y| < \delta$. If $x$ and $y$ are both
elements of $[a, b]$, we can conclude that $|f(x) - f(y)| < \varepsilon /2 < \varepsilon$.
Similarly, if $x$ and $y$ are both elements of $[b, c]$, then we conclude that
$|f(x) - f(y)| < \varepsilon$.
Suppose instead that $x \in [a, b]$ and $y \in [b, c]$. We then can conclude
that $|x - b| < \delta$ and $|b - y| < \delta$. Therefore, 
$|f(x) - f(b)| < \varepsilon /2$ and $|f(b) - f(y)| < \varepsilon /2$. So 
$|f(x) - f(y)| = |f(x) - f(b) + f(b) - f(y)| \leq |f(x) - f(b)| - |f(b) - f(y)|
< \varepsilon /2 + \varepsilon /2 = \varepsilon$.
\end{enumerate}


\end{enumerate}
\end{document}
