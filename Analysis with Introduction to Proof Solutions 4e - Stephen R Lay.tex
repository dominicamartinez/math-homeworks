\documentclass[12pt]{article}

\pagestyle{empty}
\usepackage{wrapfig}
\usepackage{shapepar}
\usepackage{fullpage}
\usepackage{amssymb}
\usepackage{amsthm}
\usepackage{calrsfs}
%\usepackage{calligra}
\usepackage{amsmath}
%\usepackage[symbol*]{footmisc}
%\usepackage[enableskew]{youngtab}
\usepackage[all]{xy}
\usepackage{graphicx}

\begin{document}
Analysis with an introduction to Proof (4th ed.) by Steven R. Lay
\begin{enumerate}

\item[1.1] Mark each statement True or False. Justify each answer.
\begin{enumerate}
\item[a)] In order to be classified as a statement, a sentence must be true.\\
False; "If a sentence can be classified as true or false, it is called a {\bf statement}"(pg.1)
\item[b)] Some statements are both true and false.\\
False; "For a sentence to be a statement...whether it is true or false, but it must clearly be the case that it is one or the other." (pg. 1)
\item[c)] When statement $p$ is true, then it's negation $\sim$$p$ is false. \\
True; "When $p$ is true, then $\sim$$p$ is false..." (pg. 3)
\item[d)] A statement and its negation may both be false.\\
False; when $p$ is false, then $\sim$$p$ is true.
\item[e)] In mathematical logic, the word "or" has an inclusive meaning.\\
True; by the truth table on pg. 4, when $p$ is true and $q$ is true then $p \vee q$ is true.
\end{enumerate}

\item[1.2] Mark each statement True or False. Justify each answer.
\begin{enumerate}
\item[a)] In an implication $p \Rightarrow q$, statement $p$ is referred to as the proposition.\\
False; "The if-statement $p$ in the implication is called the {\bf antecedent}..."(pg. 4)
\item[b)] The only case where $p \Rightarrow q$ is false is when $p$ is true and $q$ is false.\\
True; This follows from the truth table located on page 5.
\item[c)] "If $p$, then $q$" is equivalent to "$p$ whenever $q$".\\
False; should read "$q$ whenever $p$" (pg. 5)
\item[d)] The negation of a conjunction is the disjunction of the negations of the individual parts.\\
True; The solution to Practice Problem 1.6(a) proves that this is true.
\item[e)] The negation of $p \Rightarrow q$ is $q \Rightarrow p$.\\
False; The solution to Practice Problem 1.6(c) shows that this statement is false.
\end{enumerate}

\item[1.4] Write the negation of each statement.
\begin{enumerate}
\item[a)] The relation {\large R} is transitive.\\
\emph{Negation:} The relation {\large R} is not transitive.
\item[b)] The set of rational numbers is bounded.\\
\emph{Negation:} The set of rational numbers is unbounded. 
\item[c)] The function $f$ is injective and surjective.\\
\emph{Negation:} The function $f$ is not injective or not surjective. 
\item[d)] $x < 5$ or $x > 7$.\\
\emph{Negation:} $x \geq 5$ and $x \leq 7$. 
\item[e)] If $x$ is in $A$, then $f(x)$ is not in $B$.\\
\emph{Negation:} $x$ is in $A$ and $f(x)$ is in $B$. 
\item[f)] If $f$ is continuous, then $f(S)$ is closed and bounded.
\emph{Negation:} $f$ is continuous and $f(S)$ is not closed or unbounded. 
\item[g)] If $K$ is closed and bounded, then $K$ is compact.\\
\emph{Negation:} $K$ is closed and bounded and $K$ is not compact. 
\end{enumerate}

\item[1.8] Construct a truth table for each statement.
\begin{enumerate}
\item[a)]
\begin{tabular}{c|c|c|c}
\hline
$p$ & $q$ & $\sim$$p$ & $\sim$$p$$\vee$$q$ \\
\hline
T & T & F & T \\ 
T & F & F & F \\ 
F & T & T & T \\
F & F & T & T \\
\hline
\end{tabular}

\item[b)]
\begin{tabular}{c|c|c}
\hline
$p$ & $\sim$$p$ & $p$ $\wedge$ $\sim$$p$ \\
\hline
T & F & F \\
F & T & F \\
\hline
\end{tabular}

\item[c)]
\begin{tabular}{c|c|c|c|c|c|c}
\hline
$p$ & $q$ & $\sim$$p$ & $\sim$$q$ & $p \Rightarrow q$ & $\sim$$q$ $\wedge (p \Rightarrow q)$ & 
$[\sim$$q$ $\wedge (p \Rightarrow q)] \Rightarrow $$\sim$$p$\\
\hline
T & T & F & F & T & F & T \\
T & F & T & F & F & F & T \\
F & T & F & T & T & F & T \\
F & F & T & T & T & T & T \\
\hline
\end{tabular}
\end{enumerate}

\item[1.12] Let $p$ be the statement "Buford got a C on the exam," and let $q$ be the statement "Buford passed the class." Express each of the following statements in symbols.
\begin{enumerate}
\item[a)] Buford did not get a C on the exam, but he passed the class. \\
$\sim$$p \wedge q$
\item[b)] Buford got neither a C on the exam nor did he pass the class.\\
$\sim$$p$ $\wedge \sim$$q$
\item[c)] If buford passed the class, he did not get a C on the exam.\\
$q \Rightarrow \sim$$p$
\item[d)] It was necessary for Buford to get a C on the exam in order for him to pass the class.\\
$q \Rightarrow p$
\item[e)] Buford passed the class only if he got a C on the exam.\\
$q \Rightarrow p$
\end{enumerate}

\item[2.4] Write the negation of each statement.
\begin{enumerate}
\item[a)] Some basketball players at Central High are short.\\
\emph{Negation:} All basketball players at Central High are not short.
\item[b)] All of the lights are on.\\
\emph{Negation:} Some of the lights are not on.
\item[c)] No bounded interval contains infinite many integers.\\
\emph{Negation:} There exists a bounded interval that contains infinite many integers.
\item[d)] $\exists x \in S \ni x \geq 5$. \\
\emph{Negation:} $\forall x \in S, x < 5$
\item[e)] $\forall x \ni 0 < x < 1, f(x) < 2$ or $f(x) > 5$.\\
\emph{Negation:} $\exists x, 0 < x < 1$ such that $f(x) \geq 2 \wedge f(x) \leq 5$
\item[f)] If $x > 5$, then $\exists y > 0 \ni x^2 > 25 + y$.\\
\emph{Negation:} $x > 5$ and $\forall y > 0, x^2 \leq 25 + y$
\end{enumerate}

\item[2.6] Determine the truth value of each statement, assuming that $x, y$ and $z$ are real numbers.
\begin{enumerate}
\item[a)] $\forall x$ and $\forall y, \exists z \ni x + y = z$.\\
True; we can choose some $z$ to equal $x + y$.
\item[b)] $\forall x \exists y \ni \forall z, x + y = z$.\\
False; impossible for every $x$ and it's corresponding $y$ to be summed up to every $z$.
\item[c)] $\exists x \ni \forall y, \exists z \ni xz = y$.\\
True; we can choose $x = 1$ for any $y$ and $z = y$.
\item[d)] $\forall x$ and $\forall y, \exists z \ni yz = x$.\\
True; for any $x$ and $y$, we can choose $z = \frac{x}{y}$ as long as $y \not= 0$.
\item[e)] $\forall x \exists y \ni \forall z, z > y$ implies that $z > x + y$.\\
False; we choose $y = x$ then $\forall z, z > y$, let's say $z = \frac{3x}{2}$, clearly
$\frac{3x}{2} \not> x + x$.
\item[f)] $\forall x$ and $\forall y, \exists z \ni z > y$ implies that $z > x + y$.\\
True, given any $x$ and $y$ value we can always choose a $z \ni z > |x| + |y|$.
\end{enumerate}

\item[2.10] Determine the truth value of each statement, assuming $x$ is a real number.
\begin{enumerate}
\item[a)] $\exists x \in [3, 5] \ni x \geq 4$.\\
True; we can choose any real number in the range [4, 5] to satisfy the existence condition, let's
choose $x = 4.01$, we can see $4.01 \geq 4$.
\item[b)] $\forall x \in [3, 5], x \geq 4$.\\
False; choose $x$ equal to any real number in the interval [3, 4), which shows that
$x \not\geq 4$.
\item[c)] $\exists x \ni x^2 \neq 3$.\\
True; choose $x = 4$, then $x^2 = 16 \neq 3$.
\item[d)] $\forall x, x^2 \neq 3$.\\
False; can't say "for all" $x$ because when $x = \sqrt{3}$ then $x^2 = 3$.
\item[e)] $\exists x \ni x^2 = -5$.\\
False; by definition, any real number whether it's negative or positive raised to an even power will not yield a negative number, therefore there doesn't exist an $x$ we can choose to satisfy $x^2 = -5$.
\item[f)] $\forall x, x^2 = -5$.\\
False; in part (e) we said that \underline{no real number exists} which can 
satisfy $x^2 = -5$, therefore it's quite clear that applying a universal quantifier to 
the problem will not help any.
\item[g)] $\exists x \ni x - x = 0$.\\
True; simply put, choose $x = 1$ then $x - x = 1 - 1 = 0$.
\item[h)] $\forall x, x - x = 0$.\\
True; starting with the additive identity axiom, $\forall x, x + 0 = x$ and then
using the addition property of equality, we can see $\forall x, x - x = 0$.
\end{enumerate}

\item[2.14] A function is \emph{strictly decreasing} iff for every $x$ and for every $y$, if $x < y$, then $f(x) < f(y)$.
\begin{enumerate}
\item[a)] $\forall x$ and $\forall y, x < y \Rightarrow f(x) > f(y)$
\item[b)] $\exists x$ and $\exists y \ni x < y$ and $f(x) \leq f(y)$
\end{enumerate}

\item[2.16] A function $f: A \rightarrow B$ is \emph{surjective} iff for every $y$ in $B$ there exists an $x$ in $A$ such that $f(x) = y$.
\begin{enumerate}
\item[a)] $\forall y \in B, \exists x \in A$ such that $f(x) = y$
\item[b)] $\exists y \in B \ni \forall x \in A, f(x) \neq y$
\end{enumerate}

\item[4.2] Mark each statement True or False. Justify each answer.
\begin{enumerate}
\item[a)] A proof by contradiction may use the tautology $(\sim$$p \Rightarrow c) \Leftrightarrow p$.\\
True; pg. 28 states "The two basic forms of a proof by contradiction are based on tautologies
(f) and (g) in Example 3.12. Tautology (f) has the form \\
$(\sim$$p \Rightarrow c) \Leftrightarrow p$". 
\item[b)] A proof by contradiction may use the tautology $[(p \vee \sim$$q) \Rightarrow c] 
\Leftrightarrow (p \Rightarrow q)$.\\
False; following from the same page as part (a), "Tautology (g) has the form \\
$(p \Rightarrow q) \Leftrightarrow [(p\,\, \wedge \sim$$q) \Rightarrow c]$".
\item[c)] Definitions often play an important role in proofs.\\
True; pg. 32 claims that "Often a proof will be little more than unraveling definitions
and applying them to specific cases.
\end{enumerate}

\item[4.4] Prove: There exists a rational number $x$ such that $x^2 + 3x/2 = 1$. Is this rational number unique?
\begin{enumerate}
\item[]
Let $x = \frac{1}{2}$. Then \[
x^2 + \frac{3x}{2} = \left(\frac{1}{2}\right)^2 + \left(\frac{3}{2}\right)\left(\frac{1}{2}\right) 
= \frac{1}{4} + \frac{3}{4} = 1, \]
as required. The rational number is in fact unique. $\blacklozenge$
\end{enumerate}

\item[4.10] Prove: If $x$ is a real number, then $|x - 2| \leq 3$ implies that $-1 \leq x \leq 5$.
\begin{enumerate}
\item[]
Proof by contradiction: Tautology (g) states $(p \Rightarrow q) \Leftrightarrow [(p\,\, \wedge \sim$$q) \Rightarrow c]$ \\
So, let $p: |x - 2| \leq 3$ and $q: -1 \leq x \leq 5$ and in order to prove $p \Rightarrow q$, we need to show
that $p$ and $\sim$$q$ will lead us to a contradiction. \\
Let's algebraically manipulate $p$, by definition of absolute-value inequalities \[ |x - 2| \leq 3 \] is the same as
\[ -3 \leq x - 2 \leq 3 \] which can be manipulated to \[ -1 \leq x \leq 5 \] Now $\sim$$q$, translates to \[
x < -1 \mbox{ or } x > 5 \] Thus, $p\,\, \wedge \sim$$q$ leaves us with a contradiction for $x$ is a real number
and proves that $p \Rightarrow q$ or that $|x - 2| \leq 3$ implies $-1 \leq x \leq 5$. $\diamondsuit$
\end{enumerate}

\item[4.12] Consider the following theorem: "If $xy = 0$, then $x = 0$ or $y = 0$." Indicate what,
if anything, is wrong with each of the following "proofs."
\begin{enumerate}
\item[a)] Suppose $xy = 0$ and $x \neq 0$. Then dividing both sides of the first equation by $x$ we have $y = 0$. Thus if $xy = 0$, then $x = 0$ or $y = 0$. \\
There is nothing wrong with this proof. It demonstrates perfectly the tautology 3.12(p), which is also talked about in example 4.7.
\item[b)] There are two cases to consider. First suppose that $x = 0$. Then $x \cdot y = x \cdot 0 = 0$. Similarly, suppose that $y = 0$. Then $x \cdot y = x \cdot 0 = 0$. In either case, $x \cdot y = 0$. Thus if $xy = 0$, then $x = 0$ or $y = 0$. \\
The second sentence of this proof says "First suppose that $x = 0$." Then goes on to show that $xy = x \cdot 0 = 0$ in the third sentence when it should actually be $xy = 0 \cdot y = 0$. Plus, the proof, logically, isn't proving that "if $xy = 0$ then $x = 0$ or $y = 0$". It's actually using tautology 3.12(q) to prove that "if $x = 0$ or $y = 0$, then $xy = 0$".
\end{enumerate}

\item[4.18] Consider the following theorem: There do not exist three consecutive odd integers $a, b,$ and $c$ such that $a^2 + b^2 = c^2$.
\begin{enumerate}
\item[a)] For every three consecutive odd integers $a$, $b$, and $c$, \underline{$a^2 + b^2 \neq c^2$}.
\item[b)] If $a$, $b$, and $c$ are consecutive odd integers, then \underline{$a^2 + b^2 \neq c^2$}.
\item[c)] Let $a$, $b$, and $c$ be consecutive odd integers. Then $a = 2k + 1$, $b = $\underline{
$2k + 3$}, and $c = 2k + 5$ for some integer $k$. Suppose $a^2 + b^2 = c^2$. Then \\
$(2k + 1)^2 + ($\underline{$2k + 3$}$)^2 = (2k + 5)^2$. It follows that $8k^2 + 16k + 10 =
4k^2 + 20k + 25$ and $4k^2 - 4k\, - $\underline{$\,15$} $= 0$. Thus $k$ = 5/2 or $k$ = 
\underline{$-$(3/2)}. This contradicts $k$ being an \underline{integer}. Therefore, there do not
exist three consecutive odd integers $a$, $b$, and $c$ such that $a^2 + b^2 = c^2$. $\diamondsuit$
\item[d)] Which of the tautologies in Example 3.12 best describes the structure of the proof?
The tautology that best describes the proof from part(c) is 3.12(g). In part(b), we defined $q$ as $a^2 + b^2 \neq c^2$, and use $\sim$$q: a^2 + b^2 = c^2$ in part(c) and the original $p$ from part(b) to show a contradiction which proves our implication from part(b) to be true.
\end{enumerate}

\item[4.19] Prove or give a counterexample: The sum of any five consecutive integers is divisible by five.
\begin{enumerate}
\item[] We can prove the sum of any five consecutive integers is divisible by 5 directly. \\
Let $k$, $k + 1$, $k + 2$, $k + 3$, $k + 4$ be our five consecutive integers
for any integer $k$. Then $k + (k + 1) + (k + 2) + (k + 3) + (k + 4) = 5k + 10 = 5(k + 2)$ which will always
be divisible by 5. $\diamondsuit$
\end{enumerate}

\item[6.7] Let $A = \{ a \}$ and $B = \{ 1, 2, 3 \}$. List all possible relations between $A$ and $B$.
\begin{enumerate}
\item[]
All possible relations between $\{a\}$ and $\{1, 2, 3\}$. \\
$\emptyset$ \\
\{(a, 1)\}  \\ 
\{(a, 1), (a, 2)\} \\ 
\{(a, 1), (a, 2), (a, 3)\} \\ 
\{(a, 2)\} \\
\{(a, 2), (a, 3)\} \\
\{(a, 3)\} \\
\{(a, 1), (a, 3)\} 
\end{enumerate}

\item[6.11] Determine which of the three properties (reflexive, symmetric, and transitive) apply to each relation.
\begin{enumerate}
\item[a)] Let {\large R} be the relation on $\mathbb{N}$ given by $x${\large R}$y$ iff $x$ divides $y$.\\
Reflexive and transitive. Reflexivity by $\forall x \in \mathbb{N}, x$/$x = 1$. To show transitivity, we have
to say that if $x$ divides $y$ by a multiple of $n$ and $y$ divides $z$ by a multiple of $m$, then 
$x$ will divide $z$ by a multiple of $nm$. To show that this isn't symmetric, we need one counterexample, so let's take 5 divides 10 but 10 does not divide 5.
\item[b)] Let $X$ be a set and let {\large R} be the relation "$\subseteq$" defined on subsets of $X$. \\
Reflexive and transitive. According to the book, the symbol $\subseteq$ does not imply equality.
Let $A, B$, and $C$ be subsets of $X$. For reflexiveness, $A \subseteq A$ holds true, since this equivalent to $(\forall x)[(x \in A) \Rightarrow (x \in A)]$, which happens to be a tautology. Not symmetric because if $A \subseteq B$, then A might be a proper subset, in which case $B \subseteq A$ does not hold. If $A \subseteq B$ and $B \subseteq C$, then $A \subseteq C$ will hold true by the tautology 3.12(l).
\item[c)] Let $S$ be the set of people in the school. Define {\large R} on $S$ by $x${\large R}$y$ iff "$x$ likes $y$." \\
None. Not reflexive because there might exist a person in the set S that likes someone else but not
him/herself. Not symmetric because there might exist a person who likes someone else but isn't "liked" in return.
Not transitive due to the fact that if there was a person A who liked person B and person B liked person C, then
there is no guarantee that person A would like person C.
\item[d)] Let {\large R} be the relation $\{ (1, 1), (1, 2), (2, 2), (1, 3), (3, 3) \}$ on the set $\{ 1, 2, 3 \}$. \\ 
Reflexive. Since the relation {\large R} has the ordered pairs (1, 1), (2, 2), (3, 3), and the set is $\{1, 2, 3\}$, it is indeed reflexive. The relation does not contain (2, 1) nor (3, 1) therefore it cannot be
symmetric. Here is one example of transitivity that fails in this relation, (1, 2) and (2, 3) thus (1, 3), because the ordered pair (2, 3) does not exist in this relation, therefore making {\large R} not transitive.
\item[e)] Let {\large R} be the relation on $\mathbb{R}$ givey by $x${\large R}$y$ iff $x - y$ is rational. \\
Reflexive, Symmetric, and Transitive. Reflexiveness by $\forall x \in \mathbb{R}, x - x = 0$, and zero is a rational number. Symmetric because if $x - y$ is a rational number then $y - x = -(x - y)$, which means it
is also a rational number. Transitive through the fact that if $x - y$ is rational and $y - z$ are rational, then $x - y + y - z = x - z$ is rational.
\item[f)] Let {\large R} be the relation on $\mathbb{R}$ givey by $x${\large R}$y$ iff $x - y$ is irrational. \\
Symmetric and Transitive. In part(e) we said that 0 is rational, therefore this can't be reflexive since 0 is not irrational. Symmetry and transitivity both follow the same argumentation from part(e), that is if $x - y$ is irrational, then $y - x = -(x - y)$ is irrational and if $x - y$ is irrational and $y - z$ are irrational, then $x - y + y - z = x - z$ is irrational as well.
\item[g)] Let {\large R} be the relation on $\mathbb{R}$ givey by $x${\large R}$y$ iff $(x - y)^2 < 0$. \\
Symmetric and Transitive. This relation happens to an empty relation on a non-empty set, therefore it is not reflexive. Symmetric because $x${\large R}$y$ is always false. Transitive for the same reason as symmetric.
\item[h)] Let {\large R} be the relation on $\mathbb{R}$ givey by $x${\large R}$y$ iff $| x - y | \leq 2$. \\
Reflexive and Symmetric. Reflexivity by $\forall x \in \mathbb{R}, |x - x| = 0 \leq 2$. Symmetric by definition of absolute value, $|x - y| = |y - x| \leq 2$. Not transitive because while $|10 - 12| \leq 2$ and $|12 - 14| \leq 2$, $|10 - 14| \not\leq 2$.
\end{enumerate}

\item[6.20] Define a relation {\large R} on the set of all integers $\mathbb{Z}$ by $x${\large R}$y$ iff $x + y = 2k$ for some integer $k$. Is {\large R} an equivalence relation on $\mathbb{Z}$? Why or why not?
\begin{enumerate}
\item[] To verify that this relation is in fact an equivalence relation, we need to show that it is reflexive,
symmetric, and transitive. \\
Reflexive by the fact that $x - x = 0 = 3k$, where $k$ is some integer and in this case, always $0$.
Symmetry by $x - y = 3k$ for some integer $k$ and $y - x = -(x - y) = -3k$, which means it's also a multiple of 3.
Transitive by $x - y = 3k$ for some integer $k$ and $y - z = 3n$ for some integer $n$, then 
$x - y + y - z = 3k + 3n \Leftrightarrow x - z = 3(k + n)$, which means it's also a multiple of 3. 
The relation displays all three properties necessary to call it an equivalence relation. \\
To find our equivalence class, $E_5$, we set $x = 5$ and let $k$ be some integer, then find the $y \in \mathbb{Z}$ that
satisfies $y${\large R}$x$. Let's find a few elements; $5 - y = 3(0)$ if $k = 0$, then $y$ must be 5. Let $k = 1$, then
$5 - y = 3(1)$ and thus $y$ has to be 2. Let $k = 2$, then $5 - y = 3(2)$ which means $y$ has to be -1.
Going in the other direction, let $k = -1$ then $5 - y = 3(-1)$ and $y$ has
to be 8. \\
Now let's describe our equivalence class from the elements we found. $E_5 = \{-1, 2, 5, 8\}$. There appears to be 
a pattern here in which case the next elements to add will be 3 tacked on to the numbers on the end and since 
$\mathbb{Z}$ goes to infinity in both directions, we have $E_5 = \{\ldots, -4, -1, 2, 5, 8, 11,\ldots \}$. \\
There should be 3 distinct equivalence classes because as you saw from above $E_5 = E_2 = E_{-1}$ and so forth.
So let's show our 3 distinct equivalence classes: \\
$\ldots \,\,= E_{-3} = E_0 = E_3 = E_6 = \ldots$ \\
$\ldots \,\,= E_{-2} = E_1 = E_4 = E_7 = \ldots$ \\
$\ldots \,\,= E_{-1} = E_2 = E_5 = E_8 = \ldots$ 
\end{enumerate}

\item[7.2] Mark each statement True or False. Justify each answer.
\begin{enumerate}
\item[a)] If $f: A \rightarrow B$ and $C$ is a nonempty subset of $A$, then $f(C)$ is a nonempty subset of $B$. \\
True; pg. 65 states "If $C \subseteq A$, we let $f(C)$ represent the subset
$\{f(x): x \in C\}$ of $B$."
\item[b)] If $f: A \rightarrow B$ is surjective and $y \in B$, then $f^{-1}(y) \in A$. \\
False; pg. 65 states "...we let $f^{-1}(D)$ represent the subset $\{x \in A: f(x) \in D\}$
of $A$." 
\item[c)] If $f: A \rightarrow B$ and $D$ is a nonempty subset of $B$, then $f^{-1}(D)$ is a nonempty subset of $A$. \\
True; pg. 65 claims that "If $D \subseteq B$, we let $f^{-1}(D)$ represent the subset 
$\{x \in A: f(x) \in D\}$ of A." Therfore if $\exists f(x) \in D$ then there must be a corresponding $x \in A$
and $f^{-1}(D)$ is not empty.
\item[d)] The composition of two surjective functions is always surjective. \\
True; pg. 68, Theorem 7.19(a) proves that the composition of two surjective functions is also
surjective.
\item[e)] If $f: A \rightarrow B$ is bijective, then $f^{-1}: B \rightarrow A$ is bijective. \\
True; pg. 69 claims and then explains that "If $f: A \rightarrow B$ is bijective", then 
it follows that $f^{-1}: B \rightarrow A$ is also bijective.".
\item[f)] The identity function maps $\mathbb{R}$ onto $\{ 1 \}$. \\
False; pg. 69 defines the identity function as "A function...on a set A that maps each element
in A onto itself...".
\end{enumerate}

\item[7.5] Suppose that $A$ has exactly three elements and $B$ has exactly two. How many different functions are there from $A$ to $B$? How many of these are injective? How many are surjective?
\begin{enumerate}
\item[b)] If A has 3 elements and B has 2 elements, there will be 8 functions from A to B.
To show this, let $A = \{a, b, c\}$ and $B = \{1, 2\}$. Then \\
$f_1 = \{(a, 1), (b, 1), (c, 1)\}$ \\
$f_2 = \{(a, 1), (b, 1), (c, 2)\}$ \\
$f_3 = \{(a, 1), (b, 2), (c, 1)\}$ \\
$f_4 = \{(a, 1), (b, 2), (c, 2)\}$ \\
$f_5 = \{(a, 2), (b, 1), (c, 1)\}$ \\
$f_6 = \{(a, 2), (b, 1), (c, 2)\}$ \\
$f_7 = \{(a, 2), (b, 2), (c, 1)\}$ \\
$f_8 = \{(a, 2), (b, 2), (c, 2)\}$ \\
There are 6 surjective functions, with $f_1$ and $f_8$ being the only functions that are not. \\
There are no injective functions.
\end{enumerate}

\item[7.7] Classify each function as injective, surjective, bijective, or none of these.
\begin{enumerate}
\item[d)] $f: [1, \infty ) \rightarrow [0, \infty )$ defined by $f(x) = x^3 - x$ \\
bijective; like part(c) except the domain and range are limited in order to make the 
function both surjective and injective, which means it's bijective.
\item[e)] $f: \mathbb{N} \rightarrow \mathbb{Z}$ defined by $f(n) = n^2 - n$ \\
injective; not surjective because 5 has no pre-image and $\forall n, n' \in \mathbb{N}, f(n) = f(n')$ implies $n = n'$.
\item[f)] $f: [3, \infty ) \rightarrow [5, \infty )$ defined by $f(x) = (x - 3)^2 + 5$ \\
bijective; $f(3) = 5$ and every number in the interval $[5, \infty]$ can be obtained uniquely
from every number in the interval $[3, \infty]$, therefore this function is bijective.
\item[g)] $f: \mathbb{N} \rightarrow \mathbb{Q}$ defined by $f(n) = 1/n$ \\
injective; not surjective because 2/3 has no pre-image and $\forall n, n' \in \mathbb{N}, f(n) = f(n')$ implies $n = n'$.
\end{enumerate}

\item[7.19] Prove Theorem 17.9(b). That is, suppose that $f: A \rightarrow B$ and $g: B \rightarrow C$ are both injective. Prove that $g \circ f: A \rightarrow C$ is injective.
\begin{enumerate}
\item[] Let $f: A \rightarrow B$ and $g: B \rightarrow C$ both be injective functions. Assume $n_1, n_2
\in A$ and $g \circ f$ is injective, then we have $g \circ f(n_1) = g \circ f(n_2)$.
Formula notation states that this is $g(f(n_1)) = g(f(n_2))$. Since $g$ is injective then 
$g(f(n_1)) = g(f(n_2))$ implies $f(n_1) = f(n_2)$ and since $f$ is injective, $f(n_1) = f(n_2)$ implies
that $n_1 = n_2$. Therefore, since $g \circ f(n_1) = g \circ f(n_2)$ implies $n_1 = n_2$, then
$g \circ f$ is in fact, injective. $\diamondsuit$
\end{enumerate}

\item[7.21] Suppose that $f: A \rightarrow B$ and let $C$ be a subset of $A$.
\begin{enumerate}
\item[b)] Prove or give a counterexample: $f(A)\setminus f(C) \subseteq f(A\setminus C)$.
Counterexample: Let $A = \mathbb{Z}$ and $C = \mathbb{N}$, so $C \subseteq A$. So, now we
have \[
f(\mathbb{Z}\backslash \mathbb{N}) \subseteq f(\mathbb{Z})\backslash f(\mathbb{N})
\] which looks like \[
f(\{x \in \mathbb{Z}: x \leq 0\}) \subseteq f(\mathbb{Z})\backslash f(\mathbb{N}).
\] Now take $f$ to be the square function, we have \[
(\{0\} \cup \{x: x = k^2, \forall k \in \mathbb{N}\})
\subseteq 
(\{0\} \cup \{x: x = k^2, \forall k \in \mathbb{N}\})
\backslash 
\{x: x = k^2, \forall k \in \mathbb{N}\}
\] which we know is \[
(\{0\} \cup \{x: x = k^2, \forall k \in \mathbb{N}\})
\subseteq \{0\} \] and clearly not true. Therefore we can conclude that 
$f(A\backslash C) \not\subseteq f(A)\backslash f(C)$.
\end{enumerate}

\item[7.27] Find an example of functions $f: A \rightarrow B$ and $g: B \rightarrow C$ such that $g \circ f$ is bijective, but neither $f$ nor $g$ is bijective.
\begin{enumerate}
\item[]
Take the example of when A and C only have one element and B has two elements. With the only element of A mapped onto only one element of B and both elements of B mapped onto the only element of C. We can conclude that $g$ is surjective and $g \circ f$ is surjective but $f$ is not.
\end{enumerate}

\item[8.1] Mark each statement True or False. Justify each answer.
\begin{enumerate}
\item[a)] Two sets $S$ and $T$ are equinumerous if there exists a bijection $f: S \rightarrow T$. \\
True; This is definition 8.1 from pg. 78.
\item[b)] If a set $S$ is finite, then $S$ is equinumerous with $I_n$ for some $n \in \mathbb{N}$. \\
True; pg. 78 says "$S$ is finite if\,f $S = \emptyset$ or $S$ is equinumerous with $I_n$ for some
$n \in \mathbb{N}$.
\item[c)] If a cardinal number is not finite, it is said to be infinite.\\
False; Definition 8.4 says that "If a cardinal number is not finite, it is called transfinite."
\item[d)] A set $S$ is denumerable if there exists a bijection $f: \mathbb{R} \rightarrow S$.\\
False; Definition 8.6 says "A set $S$ is said to be denumerable if there exists a bijection
$f: \mathbb{N} \rightarrow S$."
\item[e)] Every subset of a countable set is countable.\\
True; the proof from theorem 8.9 proves it's true.
\item[f)] Every subset of a denumerable set is denumerable.\\
False; a subset of a denumerable set could be finite, not always denumerable.
\end{enumerate}

\item[8.3] Show that the following pairs of sets $S$ and $T$ are equinumerous by finding a specific bijection between the sets in each pair.
\begin{enumerate}
\item[a)] $S = [0, 1]$ and $T = [1, 3]$ \\
We could use a linear equation for this bijection. If we use the point-slope form to find the slope, we get $m = 2$, then again use the point-slope form to find our bijection in slope-intercept form. After all is done, we are left with the bijection $f: [0, 1] \rightarrow [1, 3]$ by $f(x) = 2x - 1$.
\item[b)] $S = [0, 1]$ and $T = [0, 1)$ \\
The answer in the back of the book states "Let $f(x) = 1/(n + 1)$ if there exists $n \in
\mathbb{N}$ such that $x = 1/n$, and $f(x) = x$ otherwise."
\item[c)] $S = [0, 1)$ and $T = (0, 1)$\\
This bijection is similar to part(b), so let's make $f(0) = 1/2$ and $f(x) = 1/(n + 2)$
if there exists $n \in \mathbb{N}$ such that $x = 1/n$ and $f(x) = x$ otherwise.
\item[d)] $S = (0, 1)$ and $T = (0, \infty )$ \\
Let's see if we can find a bijection from $(0, \infty )$ to $(0, 1)$ first. In fact, it's not that
hard to see that the bijection would be $f(x) = x/(1 + x)$ will work. So now find the inverse, which, by definition, will be our bijection from $(0, 1)$ to $(0, \infty )$. A little algebraic manipulation: \[
y = x/(1 + x) \Leftrightarrow y(1 + x) - x = 0 \]
\[ \Leftrightarrow y - x(-y + 1) = 0 
\Leftrightarrow y = x(1 - y) \Leftrightarrow x = y/(1 - y) \]
and thus we have our bijection from $(0, 1)$ to $(0, \infty )$ by $g(x) = x/(1 - x)$.
\item[e)] $S = (0, 1)$ and $T = \mathbb{R}$ \\
We could use the composition of two functions to show that (0, 1) and $\mathbb{R}$ are equinumerous. Let's take the bijection $f: (0, 1) \rightarrow (-\pi /2, \pi /2)$ by $f(x) = \pi (x - \frac{1}{2})$ and the bijection $g: (-\pi /2, \pi /2) \rightarrow \mathbb{R}$ by $g(x) = \tan (x)$. The composition of two bijective functions is bijective, thus there exists a bijection between $(0, 1)$ and $\mathbb{R}$, making them equinumerous.
\end{enumerate}

\item[8.9] A real number is said to be {\bf algebraic} if it is a root of a polynomial equation
\[
a_nx^n + \cdots + a_1x + a_0 = 0
\]
with integer coefficients. Note that the algebraic numbers include the rationals and all roots of rationals (such as $\sqrt{2}, \sqrt[3]{5}$, etc.). If a number is not algebraic, it is called {\bf transcendental}.
\begin{enumerate}
\item[a)] Show that the set of polynomials with integer coefficients is countable.\\
Let $P$ be the set of all polynomials with integer coefficients and $P_n \subseteq P$, 
where $P_n$ is the set of all polynomials with integral coefficients of being at most degree $n$, which makes $P_n$ a finite set. Since, $P_n$ is a set of all polynomials with integral coefficients, the polynomial $a_nx^n + \cdots + a_1x + a_0$ will be in the set. Since the
aforementioned polynomial has $n + 1$ integral coefficients, we can let $a_k$, where $0 \leq k \leq n$, be drawn from the subset of integers. We use a subset of $\mathbb{Z}$ here for ease of demonstration. Also, because here we can also choose a large enough $n$ such that all the integers of $\mathbb{Z}$ will be found in the subset and we already know that $\mathbb{Z}$ is countable, thus any subset, proper or equal, will also be countable. Namely, $\{-n, \ldots, -1, 0, 1, \ldots ,n\}$, which has $2n + 1$ elements. \\
Thus $|P_n| = (2n + 1)^{n + 1}$, making it countable. Using what is written in Example 8.11(d), we now know that $P$ must be countable since the union of countable sets is countable. To show this, we can choose a large enough $n$, such that all polynomials of $P$ are in $P_n$, that is to say, $P$ is comprised of $P_1 \cup P_2 \cup \ldots \cup P_n$.
\item[b)] Show that the set of algebraic numbers is countable.\\
This follows from part(a) and the fact that a polynomial of degree $n$ has at most
$n$ roots. Since $P$ is countable and each polynomial in $P$ has a finite number of roots,
then the set of algebraic numbers is countable.
\item[c)] Are there more algebraic numbers or transcendental numbers?\\
There are more algebraic numbers such as the complex number, $i$, as well as there being more
transcendental numbers like, $\pi$ or $e$.
\end{enumerate}

\item[10.4] Prove that $1^3 + 2^3 + \cdots + n^3 = \frac{1}{4}n^2(n + 1)^2$ for all $n \in \mathbb{N}$.
\begin{enumerate}
\item[] Let $P(n)$ be the statement \[
1^3 + 2^3 + \cdots + n^3 = \frac{1}{4} n^2(n + 1)^2.
\]
Since, we want to prove for all natural numbers, then $P(1)$ will establish our
basis for induction. Therefore, $1^3 = \frac{1}{4} 1^2(1 + 1)^2$, which is true, as they both
equal 1. To verify the induction step, we suppose that $P(k)$ is true, where $k \in \mathbb{N}$.
That is we assume, \[
1^3 + 2^3 + \cdots + k^3 = \frac{1}{4} k^2(k + 1)^2.
\]
Since we wish to conclude that $P(k + 1)$ is true, we add $k + 1$ to both sides to obtain
\begin{eqnarray*}
1^3 + 2^3 + \cdots + k^3 + (k + 1)^3 &=& \frac{1}{4} k^2(k + 1)^2 + (k + 1)^3 \\
&=& \frac{1}{4} k^2(k + 1)^2 + (k + 1)(k + 1)^2 \\
&=& [\frac{1}{4} k^2 + (k + 1)](k + 1)^2 \\
&=& [\frac{k^2 + 4k + 4}{4}](k + 1)^2 \\
&=& [\frac{(k + 2)^2}{4}](k + 1)^2 \\
&=& [\frac{1}{4}](k + 1)^2(k + 2)^2.
\end{eqnarray*}
Thus $P(k + 1)$ is true whenever $P(k)$ is true, and by the principle of 
mathematical induction, we conclude that $P(n)$ is true for all $n$. $\blacklozenge$
\end{enumerate}

\item[10.6] Prove that 
\[
\frac{1}{1 \cdot 2} + \frac{1}{2 \cdot 3} + \frac{1}{3 \cdot 4} + \cdots + \frac{1}{n(n+1)} = 
\frac{n}{n+1}, \mbox{ for all $n \in \mathbb{N}$.}
\]
\begin{enumerate}
\item[] Let $P(n)$ be the statement \[
\frac{1}{1 \cdot 2} + \frac{1}{2 \cdot 3} + \cdots + \frac{1}{n(n + 1)} = \frac{n}{n + 1}.
\]
Since, we want to prove for all natural numbers, then $P(1)$ will establish our
basis for induction. Therefore, \[ \frac{1}{1(1 + 1)} = \frac{1}{1 + 1} \] which is true, as they both
equal 1/2. To verify the induction step, we suppose that $P(k)$ is true, where $k \in \mathbb{N}$.
That is we assume, \[
\frac{1}{1 \cdot 2} + \frac{1}{2 \cdot 3} + \cdots + \frac{1}{k(k + 1)} = \frac{k}{k + 1}.
\]
Since we wish to conclude that $P(k + 1)$ is true, we add $k + 1$ to both sides to obtain
\begin{eqnarray*}
\frac{1}{1 \cdot 2} + \frac{1}{2 \cdot 3} + \cdots + \frac{1}{k(k + 1)} + \frac{1}{(k + 1)(k + 2)} 
&=& \frac{k}{k + 1} + \frac{1}{(k + 1)(k + 2)} \\
&=& \frac{k(k + 2) + 1}{(k + 1)(k + 2)} \\
&=& \frac{k^2 + 2k + 1}{(k + 1)(k + 2)} \\
&=& \frac{(k + 1)^2}{(k + 1)(k + 2)} \\
&=& \frac{k + 1}{k + 2} 
\end{eqnarray*}
Thus $P(k + 1)$ is true whenever $P(k)$ is true, and by the principle of 
mathematical induction, we conclude that $P(n)$ is true for all $n$. $\blacklozenge$
\end{enumerate}

\item[10.9] Prove that $1 + 2 + 2^2 + \cdots + 2^{n-1} = 2^n - 1$, for all $n \in \mathbb{N}$.
\begin{enumerate}
\item[] Let $P(n)$ be the statement \[
1 + 2 + 2^2 + \cdots + 2^{n - 1} = 2^n - 1
\]
Since, we want to prove for all natural numbers, then $P(1)$ will establish our
basis for induction. Therefore, $2^{1 - 1} = 2^1 - 1$ which is true, as they both
equal 1. To verify the induction step, we suppose that $P(k)$ is true, where $k \in \mathbb{N}$.
That is we assume, \[
1 + 2 + 2^2 + \cdots + 2^{k - 1} = 2^k - 1
\]
Since we wish to conclude that $P(k + 1)$ is true, we add $k + 1$ to both sides to obtain
\begin{eqnarray*}
1 + 2 + 2^2 + \cdots + 2^{k - 1} + 2^k &=&  (2^k - 1) + 2^k \\
&=& 2 \cdot 2^k - 1 \\
&=& 2^{k + 1} - 1.
\end{eqnarray*}
Thus $P(k + 1)$ is true whenever $P(k)$ is true, and by the principle of 
mathematical induction, we conclude that $P(n)$ is true for all $n$. $\blacklozenge$
\end{enumerate}

\item[10.14] Prove that $9^n - 4^n$ is a multiple of 5 for all $n \in \mathbb{N}$.
\begin{enumerate}
\item[] We have want to prove by induction that $9^n - 4^n$ is a multiple of $5$ for all
$n \in \mathbb{N}$. Clearly, this is true when $n = 1$, since $9^1 - 4^1 = 5$. Now let 
$k \in \mathbb{N}$ and suppose that $9^k - 4^k$ is a multiple of $5$. That is, 
$9^k - 4^k = 5m$ for some $m \in \mathbb{N}$. It follows that
\begin{eqnarray*}
9^{k + 1} - 4^{k + 1} &=&  9^{k + 1} - 9 \cdot 4^k + 9 \cdot 4^k - 4^{k + 1} \\
&=& 9(9^k - 4^k) + 5 \cdot 4^k \\
&=& 9(5k) + 5 \cdot 4^k \\
&=& 5(9k + 4^k).
\end{eqnarray*}
Since $m$ and $k$ are natural numbers, so is $9k + 4^k$. Thus $9^{k + 1} + 4^{k + 1}$ is
also a multiple of 3, and by induction we conclude that $9^n - 4^n$ is a multiple of
$3$ for all $n \in \mathbb{N}$. $\blacklozenge$
\end{enumerate}

\item[10.15] Indicate what is wrong with each of the following induction "proofs."
\begin{enumerate}
\item[a)] {\bf Theorem:} For each $n \in \mathbb{N}$, let $P(n)$ be the statement "Any collection of $n$ marbles consists of marbles of the same color." Then $P(n)$ is true for all $n \in \mathbb{N}$. \\
{\bf Proof:} Clearly, $P(1)$ is a true statement. Now suppose that $P(k)$ is a true statement for some $k \in \mathbb{N}$. Let $S$ be a collection of $k + 1$ marbles. If one marble, call it $x$, is removed, then the induction hypothesis applied to the remaining $k$ marbles implies that these $k$ marbles all have the same color. Call this color $C$. Now if $x$ is returned to the set $S$ and a different marble is removed, then again the remaining $k$ marbles must all be of the same color $C$. But one of these marbles is $x$, so in fact all $k + 1$ marbles have the same color $C$. Thus $P(k + 1)$ is true, and by induction we conclude that $P(n)$ is true for all $n \in \mathbb{N}$. $\blacklozenge$ \\
\emph{Answer:} The implication $P(k) \Rightarrow P(k + 1)$ does not always hold to be true. According to the proof given, suppose that the collection, $S$, of marbles had 2 marbles in it, one white and the other black. We remove one marble, $x$, in this case happens to be black, now
the color C becomes white. Now we put $x$ back into the bag and pull out the other marble, 
which is the white one. The proof says by inductive hypothesis, the marble in the bag is 
color C, white, and this proves that $P(k + 1)$ is true. Yet, we know for a fact that the
marble in the bag was black, thereby showing the implication $P(1) \Rightarrow P(2)$ is not true.
\end{enumerate}

\item[10.22] Use induction to prove Bernoulli's inequality: If $1 + x > 0$, then $(1 + x)^n \geq 1 + nx$ for all $n \in \mathbb{N}$.
\begin{enumerate}
\item[] Let $P(n)$ be the statement \[
\forall x \ni x + 1 > 0, (1 + x)^n \geq 1 + nx
\]
Since, we want to prove for all natural numbers, then $P(1)$ will establish our
basis for induction. Therefore, $(1 + x)^1 \geq 1 + x$ which is true, since they are
equal to each other. To verify the induction step, we suppose that $P(k)$ is 
true, where $k \in \mathbb{N}$. That is we assume, \[
\forall x \ni x + 1 > 0, (1 + x)^k \geq 1 + kx
\]
Since we wish to conclude that $P(k + 1)$ is true, we add $k + 1$ to both sides to obtain
\begin{eqnarray*}
(1 + x)^{k + 1} &=& (1 + x)(1 + x)^k \\
&\geq& (1 + x)(1 + kx) \\
&=& 1 + x + kx + kx^2 \\
&=& 1 + (k + 1)x + kx^2 \\
&\geq& 1 + (k + 1)x.
\end{eqnarray*}
The last two steps follow from $kx^2 \geq 0$. We know this because $k \in \mathbb{N}$ and 
$\forall x, x > -1$ implies that $x^2 > 0$. Thus $kx^2$ must be greater than $0$.
Thus $P(k + 1)$ is true whenever $P(k)$ is true, 
and by the principle of mathematical induction, we conclude that $P(n)$ is true for 
all $n$. $\blacklozenge$
\end{enumerate}

\item[11.3]
\begin{enumerate}
\item[b)] 
$(-x) \cdot y = -(xy)$ and $(-x) \cdot (-y) = xy$
\begin{eqnarray*}
xy + (-x) \cdot y &=& y(-x + x) \quad \mbox{  by DL} \\
&=& y \cdot 0 \hskip 1.55cm \mbox{  by A5} \\
&=& 0 \hskip 2.05cm \mbox{  by Theorem 11.1(b)}
\end{eqnarray*}
Thus $(-x) \cdot y = (-xy)$ by the uniqueness of $-(xy)$ in A5. 
\[ \]
The second part wants use to prove that $(-x)(-y) = xy$.
\begin{eqnarray*}
(-x)(-y) &=& -[(x)(-y)] \hskip 0.7cm \mbox{by the above proof} \\
&=& -[-(xy)] \hskip 1.02cm \mbox{by the above proof} \\
&=& xy \hskip 2.12cm \mbox{  by Exercise 11.3(a)}
\end{eqnarray*}
$\blacklozenge$

\item[c)] If $x \neq 0$, then $(1/x) \neq 0$ and $1/(1/x) = x$. \\
If $x \neq 0$, then we know by M5 that there is a unique real number $1/x$ such that
$x \cdot (1/x) = 1$. For the sake of contradiction, let $1/x = 0$, thus $x \cdot 0 = 1$ and by
Theorem 11.1(b), we are left with $1 = 0$, which is a contradiction. Thus $1/x$ must not be
equal to 0.
\[ \]
To prove the second part, if $x \neq 0$, then from M5, there exists a unique real number, $1/x$, such that
$x \cdot 1/x = 1$. We will use the other notation for $1/x$, which is $x^{-1}$. 
By the above proof, $x^{-1} \neq 0$, so by M5 there must
exist a unique real number such that, $(x^{-1})^{-1} \cdot x^{-1} = 1$. 
By, the uniqueness of the $x^{-1}$ in M5, $(x^{-1})^{-1} \cdot x^{-1} = 1$ can 
only be true if and only if $(x^{-1})^{-1} = x$. $\blacklozenge$

\item[d)] If $x \cdot z = y \cdot z$ and $z \neq 0$, then $x = y$. \\
If $x \cdot z = y \cdot z$ and $z \neq 0$, then 
\begin{eqnarray*}
(x \cdot z) \cdot \frac{1}{z} &=& (y \cdot z) \cdot \frac{1}{z} \qquad \mbox{by M5 and M1} \\
x \cdot (z \cdot \frac{1}{z}) &=& y \cdot (z \cdot \frac{1}{z}) \qquad \mbox{by M3} \\
x \cdot 1 &=& y \cdot 1 \hskip 1.78cm \mbox{by M5} \\
x &=& y \hskip 2.30cm \mbox{by M4}
\end{eqnarray*}$\blacklozenge$ 

\item[g)] If $x > 1$, then $x^2 > x$. \\
If $x > 1$ then $x > 0$ by Exercise 11.3(f) and O2. Since $0 < x$ and $1 < x$
, we get $1 \cdot x < x \cdot x$, by
O4 and then $x < x \cdot x$ by M4. From the real numbers, $x \cdot x$ is denoted by $x^2$, thus proving $x < x^2$.
$\blacklozenge$

\item[h)] If $0 < x < 1$, then $x^2 < 1$. \\
We have $0 < x < 1$, so $1 - x > 0$ by O3 and $x > 0$. Since $1 > 0$ from Exercise 11.3(f) and
$x > 0$, it follows from O3 and O2, that $1 + x > 0$. Then by O4, $(1 + x)(1 - x) = 1 - x^2 > 0$. 
Hence, by O3 again, $x^2 < 1$. $\blacklozenge$

\item[i)] If $x > 0$, then $1/x > 0$. If $x < 0$, then $1/x < 0$. \\
If $x > 0$, then $1/x > 0$. For the sake of contradiction, let $1/x \leq 0$, thus by Theorem 11.1(e) we have $-(1/x) \geq 0$. By O4 and Theorem 11.1(b), $-(1/x) \cdot x \geq 0$, which by M3 and M5, happens to be $-1 \geq 0$, which is a contradiction. Hence $1/x > 0$. \[ \]
The second part says, if $x < 0$, then $1/x < 0$. 
\begin{eqnarray*}
0 < -x &=& (-1) \cdot x \hskip 2.1cm \mbox{by Theorem 11.1(e) then (c)} \\
&=& -(1 \cdot x) \hskip 2.1cm \mbox{by Exercise 11.3(b)} \\
&=& -[1/x \cdot (x \cdot x)] \hskip .9cm \mbox{by M5 and M3} \\
&=& -(1/x) \hskip 2.2cm \mbox{by O4 (twice)}
\end{eqnarray*}
Since $0 < -(1/x)$, by Theorem 11.1(e), this must be that $0 > 1/x$. $\blacklozenge$
\end{enumerate}

\item[11.6] 
\begin{enumerate}
\item[b)] Prove: If $|x - y| < c$, then $|x| < |y| + c$. \\
To show that $|x| < |y| + c$ is true if $|x - y| < c$ is true, let's first show that
$|x| - |y| \leq |x - y|$. 
\begin{eqnarray*}
|x| - |y| &=& |x - y + y| - |y| \qquad \mbox{by A4 and A5} \\
&\leq& |x - y| + |y| - |y| \hskip 0.59cm \mbox{by Theorem 11.9(d)} \\
&=& |x - y| + 0 \hskip 2.00cm \mbox{by A5} \\
&=& |x - y| \hskip 2.53cm \mbox{by A4} 
\end{eqnarray*}
Thus if $|x| - |y| \leq |x - y|$ and $|x - y| < c$, then by O2, $|x| - |y| < c$. Then by O3, 
we have $|x| - |y| + |y| < c + |y|$, then by A5, A4 and A2, we have proven that $|x| < |y| + c$.
$\blacklozenge$

\item[c)] Prove: If $|x - y| < \epsilon$ for all $\epsilon > 0$, then $x = y$. \\
By theorem 11.9(a), $|x - y| \geq 0$. Since $0 \leq |x - y| < \varepsilon, \forall \varepsilon > 0$ then this is only possible if $|x - y| = 0$. Hence, $0 = |0| = |x - y|$ provided that 
$x - y = 0$, thus by the uniqueness defined in A5, this is true iff $y = x$.
\end{enumerate}

\item[12.1] Mark each statement True or False. Justify each answer.
\begin{enumerate}
\item[a)] If a nonempty subset of $\mathbb{R}$ has an upper bound, then it has a least upper bound. \\
True; by the Completeness Axiom.
\item[b)] If a nonempty subset of $\mathbb{R}$ has an infimum, then it is bounded. \\ 
False; a bounded subset must be bounded from above and below.
\item[c)] Every nonempty bounded subset of $\mathbb{R}$ has a maximum and a minimum. \\
False; take Example 12.3(c) to show that a set might not have a minimum. Similarly, we can
also say that it's possible to not have a maximum or both.
\item[d)] If $m$ is an upper bound for $S$ and $m' < m$, then $m'$ is not an upper bound for $S$.\\
False; since $m \neq \sup S$ then $m'$ could be an upper bound as well. 
\item[e)] If $m = $ inf $S$ and $m' < m$, then $m'$ is a lower bound for $S$. \\
True; by definition of infimum, $m$ is the greatest lower bound and if $m' < m$, then $m'$ must
be a lower bound as well.
\item[f)] For each real number $x$ and each $\epsilon > 0$, there exists $n \in \mathbb{N}$ such that $n\epsilon > x$. \\
True; as proven in Theorem 12.10(b).
\end{enumerate}

\item[12.4] Give the minimum and infimum of each set.
\begin{enumerate}
\item[a)] $\{ 1, 3 \}$ \\
$\inf \{1, 3\} = 1$ and $\min \{1, 3\} = 1$
\item[b)] $\{ \pi , 3 \}$ \\
$\inf \{\pi, 3\} = 3$ and $\min \{\pi, 3\} = 3$
\item[c)] $[0, 4]$ \\
$\inf [0, 4] = 0$ and $\min [0, 4] = 0$
\item[d)] $(0, 4)$ \\
$\inf (0, 4) = 0$ and there is no minimum
\item[e)] $\{ 1/n : n \in \mathbb{N} \}$ \\
$\inf \{1/n : n\in \mathbb{N}\} = 0$ and there is no minimum
\item[f)] $\{ 1 - (1/n) : n \in \mathbb{N} \}$ \\
$\inf \{1 - (1/n) : n\in \mathbb{N}\} = 0$ and $\min \{1 - (1/n) : n\in \mathbb{N}\} = 0$
\item[g)] $\{ n/(n + 1) : n \in \mathbb{N} \}$ \\
$\inf \{n/(n + 1) : n\in \mathbb{N}\} = 1/2$ and $\min \{n/(n + 1) : n\in \mathbb{N}\} = 1/2$
\item[h)] $\{ (-1)^n[1 + (1/n)] : n \in \mathbb{N} \}$ \\
$\inf \{(-1)^n \left(1 + \frac{1}{n}\right) : n \in \mathbb{N}\} = -2$ and 
$\min \{(-1)^n \left(1 + \frac{1}{n}\right) : n \in \mathbb{N}\} = -2$
\end{enumerate}

\item[12.6] 
\begin{enumerate}
\item[a)] Let $S$ be a nonempty bounded subset of $\mathbb{R}$. Prove that sup $S$ is unique. \\
Let $a = \sup S$ and $b = \sup S$. By the definition of the supremum, since $a$ is the supremum, it is the lowest of the upper bounds and since $b$ is an upper bound, then $a \leq b$. Again, by the definition of the supremum, $b$ is the supremum, thus it is the lowest of the upper bounds and since $a$ is an upper bound, then $b \leq a$. This is only possible when $a = b$, hence $\sup S$ must be \emph{unique}.
\item[b)] Suppose that $m$ and $n$ are both maxima of a set $S$. Prove that $m = n$. \\
By the definition of the maximum, since $n$ is the maximum, it is the element in $S$ such that
$\forall s \in S, s \leq n$. Again, by the definition of the maximum, $m$ is the maximum, 
thus $\forall s \in S, s \leq m$. Thus by definition of the maximum, $n \leq m$ and $m \leq n$ implies that this is only possible when $n = m$.
\end{enumerate}

\begin{enumerate}

\item[13.2] Let $S \subseteq \mathbb{R}$. Mark each statement True or False. Justify each answer.
\begin{enumerate}
\item[a)] bd $S = $ bd $(\mathbb{R}\backslash S)$. \\
True; pg. 131 states, "\ldots since bd $S = $ bd$(\mathbb{R}\backslash S)$\ldots".
\item[b)] bd $S \subseteq \mathbb{R}\backslash S$. \\
False; this is true only if the set $S$ is open.
\item[c)] $S \subseteq S' \subseteq $ cl $S$. \\
False; Example 13.16(c) shows us that it's possible for $S \not\subseteq S'$.
\item[d)] $S$ is closed iff cl $S \subseteq S$. \\
True; by theorem 13.17(c).
\item[e)] $S$ is closed iff $S' \subseteq S$. \\
True; by theorem 13.17(a).
\item[f)] If $x \in S$ and $x$ is not an isolated point of $S$, then $x \in S'$. \\
True; a point is either isolated or an accumulation point.
\item[g)] The set $\mathbb{R}$ of real numbers is neither open nor closed. \\
False; Example 13.8 says that $\mathbb{R}$ is both open and closed.
\item[h)] The intersection of any collection of open sets is open. \\
False; by Theorem 13.10(b), "The intersection of any finite collection of open sets is open.",
but not necessary true for an infinite collection of open sets. See Exercise 13.5(d).
\item[i)] The intersection of any collection of closed sets is closed. \\
True; this is Corollary 13.11(a).
\end{enumerate}

\item[13.5] Classify each of the following sets as open, closed, neither, or both. 
\begin{enumerate}
\item[a)] $\{ 1/n: n \in \mathbb{N} \}$ \\
Neither. bd $\{ 1/n: n \in \mathbb{N} \} = \{ 0, 1 \}$ and $0 \in \mathbb{R}\backslash 
\{ 1/n: n \in \mathbb{N} \}$ \\
and $1 \in \{ 1/n: n \in \mathbb{N} \}$.
\item[b)] $\mathbb{N}$ \\
Closed. bd $\mathbb{N} = \mathbb{N}$ and since $\mathbb{N} \subseteq \mathbb{N}$.
\item[c)] $\mathbb{Q}$ \\
Neither. By theorem 12.12 and 12.14, $\mathbb{Q}$ is not open because we can't find an open
interval that is a subset of $\mathbb{Q}$, since there exists a rational and irrational number between
the interval. For the same reason, $\mathbb{R}\backslash \mathbb{Q}$ is not open, thus $\mathbb{Q}$
cannot be closed.
\item[d)] $\bigcap_{n = 1}^{\infty} \left(0, \frac{1}{n}\right)$ \\
Both. $\emptyset = \bigcap_{n = 1}^{\infty} \left(0, \frac{1}{n}\right)$ and the empty set is
both open and closed.
\item[e)] $\{ x : |x - 5| \leq \frac{1}{2} \}$ \\
Closed. $\{ x : |x - 5| \leq \frac{1}{2} \} = [4\frac{1}{2}, 5\frac{1}{2}]$, thus bd 
$[4\frac{1}{2}, 5\frac{1}{2}] = \{4\frac{1}{2}, 5\frac{1}{2}\}$, both of which are elements of 
$\{ x : |x - 5| \leq \frac{1}{2} \}$.
\item[f)] $\{x: x^2 > 0\}$ \\
Open. bd $\{x: x^2 > 0\} = \{0\}$ and $0 \in \mathbb{R}\backslash \{x: x^2 > 0\}$.
\end{enumerate}

\item[13.6] Find the closure of each set in Exercise 13.5.
\begin{enumerate}
\item[a)] cl $\{ 1/n: n \in \mathbb{N} \} = \{ 1/n: n \in \mathbb{N} \} \cup \,\{0\}$
\item[b)] cl $\mathbb{N} = \mathbb{N} \cup \,\emptyset = \mathbb{N}$.
\item[c)] cl $\mathbb{Q} = \mathbb{Q} \cup \mathbb{R}\backslash \mathbb{Q} = \mathbb{R}$.
\item[d)] cl $\emptyset = \emptyset$.
\item[e)] cl $\{ x : |x - 5| \leq \frac{1}{2} \} = \{ x : |x - 5| \leq \frac{1}{2} \}$ by Theorem 13.17(c).
\item[f)] cl $\{x: x^2 > 0\} = \{x: x^2 > 0\} \cup \{0\}$.
\end{enumerate}

\item[13.7] Let $S$ and $T$ be subsets of $\mathbb{R}$. Find a counterexample for each of the following.
\begin{enumerate}
\item[e)] bd (bd $S$) = bd $S$ \\
Let $S$ be the set of all rational numbers between 0 and 1. Thus bd $S = [0, 1]$, yet
bd (bd $S$) $= \{0, 1\}$, which goes to show that these two are not equal.
\item[g)] bd ($S \cap T$) = (bd $S$) $\cap$ (bd $T$) \\
Let $S = (0, 1)$ and $T = [1, 2)$. So, bd $S = \{0, 1\}$ and bd $T = \{1, 2\}$. Thus
$(\mbox{bd } S) \cap (\mbox{bd } T) = \{1\} \neq \emptyset = \mbox{bd }\emptyset = \mbox{bd }(S \cap T)$.
\end{enumerate}

\item[13.9] Prove the following.
\begin{enumerate}
\item[a)] An accumulation point of a set $S$ is either an interior point of $S$ or a boundary point of $S$. \\
For the sake of contradiction, suppose that $s \in S$ is an accumulation point of $S$ but is not an interior point nor a boundary point of $S$. This means that $s$ must be an interior point of $\mathbb{R}\backslash S$ and that there is a neighborhood of $s$, $N(s, \varepsilon )$, such that $N(s, \varepsilon ) \subseteq \mathbb{R}\backslash S$. Thus, $N(s, \varepsilon ) \cap S = \emptyset$ and $N^*(s, \varepsilon ) \cap S = \emptyset$ but this violates the definition of an accumulation point.
$\blacklozenge$
\end{enumerate}

\item[13.16] 
\begin{enumerate}
\item[a)] Prove: bd $S$ = (cl $S) \cap [$cl $(\mathbb{R}\backslash S)]$. \\
From definition, $x \in \mbox{bd } S$ iff every neighborhood of $x$, $N(x, \varepsilon)$, 
$N(x, \varepsilon) \cap S \neq \emptyset$ and $N(x, \varepsilon) \cap \,\mathbb{R}\backslash S 
\neq \emptyset$. This is the same as saying that every neighborhood of $x$, 
$N(x, \varepsilon) \cap S \neq \emptyset$ and for every neighborhood of $x$, 
$N(x, \varepsilon) \cap \,\mathbb{R}\backslash S \neq \emptyset$. Since this is the definition of
closure, we have $x \in \mbox{cl }S$ and $x \in \mbox{cl }\mathbb{R}\backslash S$. Hence, 
$x \in \mbox{cl }S \cap \mbox{cl }\mathbb{R}\backslash S$. $\blacklozenge$

\item[b)] Prove: bd $S$ is a closed set. \\
By the above proof, bd $S = (\mbox{cl } S) \cap [\mbox{cl }(\mathbb{R}\backslash S)]$. 
By Theorem 13.17(b), $\mbox{cl } S$ and $\mbox{cl }(\mathbb{R}\backslash S)$ are both closed.
Since bd $S$ is an intersection of two closed sets, by corollary 13.11(a), it too must be closed.
$\blacklozenge$
\end{enumerate}

\item[13.19] Let $A$ be a nonempty open subset of $\mathbb{R}$ and let $\mathbb{Q}$ be the set of rationals. Prove that $A \cap \mathbb{Q} \neq \emptyset$.
\begin{enumerate}
\item[] By Theorem 12.12 (the density of $\mathbb{Q}$ in $\mathbb{R}$), then for any $x, y \in \mathbb{R}$, $x < y$, there exists a rational number, $w$, such that $x < w < y$. Since $A$ is an open and nonempty set, this means that there is an interior point, $a \in A$. By definition, there is a $N(a, \varepsilon) \subseteq A$, which means that $a - \varepsilon < a < a + \varepsilon$. Thus, by Theorem 12.12 and $\varepsilon > 0$, we can find a rational number, $n \in A$, such that $a - \varepsilon < n < a + \varepsilon$. We have shown that there exists a rational number in $A$, so that $A \cap \mathbb{Q}$ cannot possibly be the empty set. $\blacklozenge$
\end{enumerate}

\item[13.20] Let $S$ and $T$ be subsets of $\mathbb{R}$. Prove the following.
\begin{enumerate}
\item[a)] cl (cl $S$) = cl $S$ \\
This just follows from definition. By theorem 13.17(c), cl (cl $S$) = cl $S$ iff 
cl $S$ is a closed set. Thus by theorem 13.17(b), we know that cl $S$ is in fact a closed set.
$\blacklozenge$

\item[c)] cl $(S \cap T) \subseteq$ (cl $S$) $\cap$ (cl $T$) \\
If $x \in (S \cap T)'$, then $x \in S'$ and $x \in T'$, which is the same as $x \in S' \cap T'$.
Thus, $(S \cap T)' \subseteq S' \cap T'$. Therefore by definition, 
\begin{eqnarray*}
\mbox{cl } (S \cap T) &=& (S \cap T) \cup (S \cap T)' \\
&=& [S \cup (S \cap T)'] \cap [T \cup (S \cap T)'] \hskip .6cm \mbox{by distribution}\\
&\subseteq & [S \cup (S' \cap T')] \cap [T \cup (S' \cap T')] \\
&\subseteq & (S \cup S') \cap (T \cup T') \\
&=& \mbox{cl } S \cap\, \mbox{cl } T
\end{eqnarray*}
\hskip 14.0cm $\blacklozenge$

\item[d)] Find an example to show that equality need not hold in part (c). \\
Let $S = (0, 1)$ and $T = (1, 2)$, then cl $S = [0, 1]$ and cl $T = [1, 2]$. 
Also, $S \cap T = \emptyset$ and cl $\emptyset = \emptyset$. Thus,
cl $(S \cap T) = \mbox{cl } \emptyset = \emptyset \neq \{1\} = [0, 1] \cap [1, 2] = (\mbox{cl }S) \cap (\mbox{cl }T)$.
\end{enumerate}

\item[13.21] Let $S$ and $T$ be subsets of $\mathbb{R}$. Prove the following.
\begin{enumerate}
\item[a)] int $S$ is an open set. \\
int $S$ is the set of all interior points in $S$. This means that for every $s \in \mbox{int}\,S$, there exists a neighborhood, $N$ of $s$, by definition that is a subset of $S$. Since int $S$ is comprised of all these neighborhoods and we know from Example 13.8 that all neighborhoods are open sets, then int $S$ is the union of all the open sets in $S$. By Theorem 13.10(a), we know that the union of any collection of open sets is an open set. Thus int $S$ is an open set. $\blacklozenge$

\item[b)] int (int $S$) = int $S$ \\
Since int (int $S$) is the set of all interior points in int $S$, and all the points 
in $S$ are interior points, it must follow that the inclusion int (int $S$) $\subseteq$ int $S$ always holds. Thus we have int (int $S$) $=$ int $S$. $\blacklozenge$

\item[c)] int $(S \cap T) = ($int $S) \cap ($int $T)$ \\
To prove int $(S \cap T) = ($int $S) \cap ($int $T)$, need to show that \\
int $(S \cap T) \subseteq ($int $S) \cap ($int $T)$ and (int $S)\, \cap $ (int $T) \subseteq $ int$(S \cap T)$.
Firstly, let's show that int $(S \cap T) \subseteq ($int $S) \cap ($int $T)$. By definition, if
$x \in \mbox{int }(S \cap T)$ then there is a neighborhood, $N$ of $x$, such that $N \subseteq S \cap T$.
This means, that $N \subseteq S$ and $N \subseteq T$ which implies that $x \in \mbox{int }S$ and 
$x \in \mbox{int }T$. Thus $x \in (\mbox{int } S) \cap (\mbox{int } T)$, which shows that
int $(S \cap T) \subseteq ($int $S) \cap ($int $T)$. \\
Secondly, we need to show that (int $S)\, \cap $ (int $T) \subseteq $ int$(S \cap T)$.
If $x \in (\mbox{int }S) \cap (\mbox{int }T)$, then $x \in \mbox{int }S$ and $x \in \mbox{int }T$.
By definition there is a neighborhood, $N$ of $x$, such that $N \subseteq S$ and $N \subseteq T$.
Thus $N \subseteq S \cap T$, which implies that $x \in \mbox{int }(S \cap T)$. \\
Thus proven that int $(S \cap T) \subseteq ($int $S) \cap ($int $T)$ and 
(int $S)\, \cap $ (int $T) \subseteq $ int$(S \cap T)$, we have 
int $(S \cap T) = ($int $S) \cap ($int $T)$. $\blacklozenge$

\item[d)] (int $S) \cup ($int $T) \subseteq$ int $(S \cup T)$ \\
To show this, if $x \in (\mbox{int }S) \cup (\mbox{int } T)$, then either 
$x \in \mbox{int } S$ or $x \in \mbox{int }T$. If $x \in \mbox{int } S$ then by definition
there is a neighborhood, $N$ of $x$, such that $N \subseteq S$. Since $S \subseteq (S \cup T)$, then $N \subseteq (S \cup T)$ and thus $x \in \mbox{int } (S \cup T)$. Similarly, if $x \in \mbox{int } T$ then by definition there is a neighborhood, $N$ of $x$, such that $N \subseteq T$. 
Since $T \subseteq (S \cup T)$, then $N \subseteq S \cup T$ and thus $x \in \mbox{int }(S \cup T)$.
$\blacklozenge$
\item[e)] Find an example to show that equality need not hold in part (d). \\
Let $S = (1, 2)$ and $T = [2, 3)$. We have $S \cup T = (1, 3)$ and int $(S \cup T) = (1, 3)$, 
yet int $S = (1, 2)$ and int $T = (2, 3)$. We can now see that \\
(int $S$) $\cup$ (int $T$) $= (1, 2)\cup (2, 3) \subseteq (1, 3)$ but $(1, 2)\cup (2, 3) \neq (1, 3)$.
\end{enumerate}

\item[14.2] Mark each statement True or False. Justify each answer.
\begin{enumerate}
\item[a)] Some unbounded sets are compact.\\
False; this goes against the Heine-Borel theorem.
\item[b)] If $S$ is a compact subset of $\mathbb{R}$, then there is at least one point in $\mathbb{R}$ that is an accumulation point of $S$. \\
False; the Bolzano-Weierstrass theorem states, "If a \emph{bounded} subset $S$ of
$\mathbb{R}$ contains infinitely man points, then there exists a least one point in
$\mathbb{R}$ that is an accumulation point of $S$." Moreover, this problem is false because
of what is said in part(c), as well. 
\item[c)] If $S$ is compact and $x$ is an accumulation point of $S$, then $x \in S$. \\
True; since $S$ is compact, then it is closed. We know that if $S$ is closed, by 
theorem 13.17(a), then $S$ contains all of its accumulation points. 
\item[d)] If $S$ is unbounded, then $S$ has at least one accumulation point. \\
False; the set $\mathbb{N}$ is unbounded and has no accumulation points.
\item[e)] Let $\mathcal{F} = \{ A_i : i \in \mathbb{N} \}$ and suppose that the intersection of any finite subfamily of $\mathcal{F}$ is nonempty. If $\bigcap \mathcal{F} = \emptyset$, then for some $k \in \mathbb{N}, A_k$ is not compact. \\
True; the set from Exercise 13.5(d) is one example.
\end{enumerate}

\item[14.3] Show that each subset of $\mathbb{R}$ is not compact by describing an open cover for it that has no finite subcover.
\begin{enumerate}
\item[a)] [1, 3) \\
$A_n = ${\Large $\left( 0, 2 + \frac{(n - 1)}{n} \right)$} for all $n \in \mathbb{N}$.
\end{enumerate}

\item[14.5] 
\begin{enumerate}
\item[a)] If $S_1$ and $S_2$ area compact subsets of $\mathbb{R}$, prove that $S_1 \cup S_2$ is compact. \\
Let $\mathcal{F}$ be any open cover of both $S_1$ and $S_2$. Since $S_1$ and $S_2$
are both compact, we can let $\mathcal{G}$ be a finite subcover of $S_1$ and 
$\mathcal{H}$ be a finite subcover of $S_2$, such that $\mathcal{G} \subseteq \mathcal{F}$ 
and $\mathcal{H} \subseteq \mathcal{F}$. 
It's clear that $\mathcal{G} \cup \mathcal{H} \subseteq \mathcal{F}$, hence $S_1 \cup S_2$
has a finite subcover in $\mathcal{F}$ and thus is compact.
\item[b)] Find an infinite collection $\{ S_n : n \in \mathbb{N} \}$ of compact sets in $\mathbb{R}$ such taht $\bigcup^\infty_{n=1}S_n$ is not compact. \\
$\{S_n = [0, n] : n \in \mathbb{N}\}$.
\end{enumerate}

\item[14.6] Show that compactness is necessary in Corollary 14.8. That is, find a family of intervals $\{ A_n : n \in \mathbb{N} \}$ with $A_{n+1} \subseteq A_n$ for all $n, \bigcap^\infty_{n=1} A_n = \emptyset$, such that
\begin{enumerate}
\item[a)] The sets $A_n$ are all closed. \\
$\{ A_n = [n, \infty ) : n \in \mathbb{N} \}$.
\item[b)] The sets $A_n$ are all bounded. \\
The set from Exercise 13.5(d); $\{ A_n = (0, 1/n) : n \in \mathbb{N} \}$.
\end{enumerate}

\item[14.7]
\begin{enumerate}
\item[b)] Find an example of a collection of disjoint closed subsets of $\mathbb{R}$ that is not countable. \\
The collection of sets, where each set is a singleton set of an irrational number. 
A singleton set is closed and the set of all irrational numbers is uncountable, thus
since every irrational number is will be represented by a singleton set in this collection, 
this collection must also be uncountable.
\end{enumerate}

\item[14.12] Let $S$ be a subset of $\mathbb{R}$. Prove that $S$ is compact iff every infinite subset of $S$ has an accumulation point in $S$.
\begin{enumerate}
\item[] If $S$ is compact, then every infinite subset of $S$ has an accumulation point in $S$, follows
directly from the Bolzano-Weierstrass theorem. The rest of the proof is verbatim from the one found at \\
http://www.econ.upenn.edu/Courses/2003/summer2/econ897/part1/chpt5.pdf: 
\\ Suppose that every infinite subset of $S$ has an accumulation point in
$S$ but $S$ is not compact. Then $S$ is not bounded or not closed. Suppose first that $S$ is
not bounded. This means that for every $n \in N$ there exists $x_n \in S$ such that $|x_n| > n$.
Construct a subset $T \subseteq S$ as \\
$T = \{x_n \in S : |x_n| > n, |x_n| > |x_m| \wedge |x_n - x_m| > 1, \forall m,n \in \mathbb{N} \wedge m < n\}$
This subset is nonempty and is infinite, for if not we would be contradicting the
unboundedness assumption of $S$. Indeed take $T$ to be a finite set $T = \{x_1, x_2,\ldots,x_N\}$.
This means that there is no $x \in S$ such that $|x| > N + 1$ and $|x| > |x_N|$ and
$|x - x_m| > 1$ for all $m < N$. In other words, for all $x \in S$ we have that $|x| \leq N + 1$
or $|x| < |x-N|$ or $|x - x_m| \leq 1$ meaning the S is bounded. 
Finally, by construction $T$ has no accumulation point. 
Thus we have constructed an infinite subset of $S$ which
does not have an accumulation point in $S$, contradicting the initial assumption. Then $S$ must
be bounded. \\
Suppose now $S$ is not closed. Then there exists $x \in \mathbb{R}\backslash S$ such that $x \in S'$.
Since $x \in S'$, for every $\varepsilon > 0$ there is $x_\varepsilon \in S$ such that 
$|x_\varepsilon - x| < \varepsilon$. Construct a
subset $T \subseteq S$ as \\
$T = \{x_n \in S : |x_n - x| < |x_m - x| \wedge |x_n - x| < 1/n$ for every $m, n \in \mathbb{N}$ and $m< n\}$
This subset $T$ is nonempty and infinite, for if not we would be contradicting that $x$
is an accumulation point of $S$. Moreover $x$ is the unique limit point of $T$ in $\mathbb{R}$.
To show this assume it were not, and let $y \in R$ and $y \neq x$ be another accumulation point of $T$.
Let $\delta = \frac{|x - y|}{3}$. By the Archimedean property there exists $N \in \mathbb{N}$ such that 
$1/N < \delta$. This
implies that for all $n \geq N$ and $x_n \in T, |x_n - y| > \delta$. This leaves at most a finite set
$T = {x_1, x_2..., x_N}$ such that $|x_i - y| < \delta$. Without loss of generality we can assume
that all $x_i \in T$ are different from $y$\ldots otherwise we can remove those $x_i = y$ from $T$,
since we are interested in the properties of deleted neighborhoods of $y$. If $T$ is empty
then we are done, since there exists $\varepsilon = \delta > 0$ for which 
$N^* (y, \varepsilon ) \cap T = \emptyset $, meaning that $y$ is not an accumulation point. 
Now let $T \neq \emptyset$ and call $\varepsilon = \min (|x_i - y|)$ for $x_i \in T$.
For such $\varepsilon > 0$ we also have $N^* (y, \varepsilon) \cap T = \emptyset$, 
leading to the desired result. Thus we
have constructed an infinite subset of $S$ whose only accumulation point is in 
$\mathbb{R}\backslash S$ and
not in $S$, contradicting the initial assumption. Therefore $S$ must be both closed and
bounded, hence compact.
\end{enumerate}

\item[16.2] Mark each statement True or False. Justify each answer.
\begin{enumerate}
\item[a)] If $s_n \rightarrow 0$, then for every $\epsilon > 0$ there exists $N \in \mathbb{R}$ such that $n > N$ implies $s_n < \epsilon$. \\
False; the statement should say, "\ldots $|s_n| < \varepsilon$ \ldots ".
\item[b)] If for every $\epsilon > 0$ there exists $N \in \mathbb{R}$ such that $n > N$ implies $s_n < \epsilon$, then $s_n \rightarrow 0$. \\
False; for same reason as above.
\item[c)] Given sequences $(s_n)$ and $(a_n)$, if for some $s \in \mathbb{R}, k > 0$ and $m \in \mathbb{N}$ we have $|s_n - s| \leq k|a_n|$ for all $n > m$, then lim $s_n = s$. \\
False; the statement is similar to theorem 16.8 except it
leaves out that the limit of $a_n$ must be $0$. 
\item[d)] If $s_n \rightarrow s$ and $s_n \rightarrow t$, then $s = t$. \\
True; by theorem 16.14.
\end{enumerate}

\item[16.4] Find $k > 0$ and $m \in \mathbb{N}$ so that $7n^3 + 13n \leq kn^3$ for all integers $n \geq m$.
\begin{enumerate}
\item[] If $13 \leq n^2$ when $n \geq 4$ then we can choose $m = 4$. Thus \\
$7n^3 + 13n \leq 7n^3 + n^3 = 8n^3$, from which we can choose $k = 8$.
\end{enumerate}

\item[16.6] Using only Definition 16.2, prove the following.
\begin{enumerate}
\item[a)] For any real number $k$, $\lim_{n \to \infty}(k/n) = 0$. \\
In Example 16.3, they chose $N = 1/\varepsilon$, in this problem we can choose $N = |k|/\varepsilon$. Hence, given $\varepsilon > 0$, let $N = |k|/\varepsilon$. Then for any $n > N$ we have
$|(k/n) - 0| = |k/n| = |k|/n < |k|/N = \varepsilon$. Thus, lim $(k/n) = 0$. $\blacklozenge$
\item[c)]
\[
\lim \frac{3n + 1}{n + 2} = 3
\]
Given $\varepsilon > 0$, let $N = 5/\varepsilon$. Then for any $n > N$ we have 
\[
\left| \frac{3n + 1}{n + 2} - 0 \right| = \left| \frac{3n + 1 - 3n - 6}{n + 2}\right|
= \left| \frac{-5}{n + 2} \right| = \frac{5}{n + 2} <  \frac{5}{n} < \frac{5}{N} = \varepsilon
\]
Thus lim $\frac{3n + 1}{n + 2} = 3$. $\blacklozenge$
\end{enumerate}

\item[16.7] Use any of the results in this section, prove the following.
\begin{enumerate}
\item[a)] 
\[
\lim \frac{1}{1 + 3n} = 0
\]

Clearly, $3n < 3n + 1$ for all $n \in \mathbb{N}$, meaning that $1/(3n + 1) < 1/3n$. Thus,
\[ 
\left| \frac{1}{1 + 3n} - 0 \right| = \frac{1}{1 + 3n} < \frac{1}{3n} = \left( \frac{1}{3} \right) 
\left( \frac{1}{n} \right)
\]
Since the lim $1/n = 0$, theorem 16.8 implies that that lim $1/(1 + 3n) = 0$.
\item[c)] 
\[
\lim \frac{6n^2 + 5}{2n^2 - 3n} = 3
\]

If $n \geq 6$, then $2n^2 - 3n > n^2$ and $5 + 9n < 10n$. Thus,
\[
\left| \frac{6n^2 + 5}{2n^2 - 3n} - 3 \right| = \left| \frac{5 + 9n}{2n^2 - 3n} \right| < 
\frac{10n}{n^2} = 10 \cdot \left( \frac{1}{n} \right) 
\]
Since the lim $1/n = 0$, theorem 16.8 implies that that lim $\frac{6n^2 + 5}{2n^2 - 3n} = 3$.

\item[e)] 
\[
\lim \frac{n^2}{n!} = 0
\]

If $n \geq 6$, then $(n - 1)! > n^2$. Thus, 
\[
\left| \frac{n^2}{n!} - 0 \right| = \left| \frac{n^2}{n \cdot (n - 1)!} \right| =
\left| \frac{n}{(n - 1)!} \right| < \frac{n}{n^2} = 1 \cdot \frac{1}{n}
\]
Since the lim $1/n = 0$, theorem 16.8 implies that that lim $\frac{n^2}{n!} = 0$.
\end{enumerate}

\item[16.8] Show that each of the following sequences is divergent.
\begin{enumerate}
\item[a)] $a_n = 2n$ \\
To prove that the sequence $a_n = 2n$ is divergent, let us suppose that $a_n$ converges
to some real number $s$. Letting $\varepsilon = 1$ in the definition of convergence, we find that
there exists a number $N$ such that $n > N$ implies that $|2n - s| < 1$. We obtain 
$-1 < 2n - s < 1$, which is the same as $2n - 1 < s < 2n + 1$. It's just as well that $2n > N$
which implies $|2(2n) - s| < 1$, which we obtain the inequality $-1 < 4n - s < 1$, which
is the same as $4n - 1 < s < 4n + 1$. Since $s$ cannot satisfy both inequalities, we have 
reached a contradiction. Thus the sequence $(a_n)$ is divergent. $\blacklozenge$

\item[c)] $c_n = \cos (n\pi/3)$ \\
To prove that the sequence $c_n =$ cos$\left( \frac{n\pi}{3} \right)$ 
is divergent, let us suppose that $c_n$ converges
to some real number $s$. Letting $\varepsilon = 1$ in the definition of convergence, we find that
there exists a number $N$ such that $n > N$ implies that \\
$|$ cos$[(n\pi )/3] - s| < 1$. If $n > N$ and $n$ is of a number such that $6$ divided by $n$ 
will yield a remainder of $3$, then we have the inequality $| -1 - s | < 1$, which is the same
as $-2 < s < 0$. On the other hand, if $n > N$ and $n$ is of a number such that $6$ divided
by $n$ will yield a remainder of $0$, then we have the inequality $|\, 1 - s | < 1$, which is
the same as $0 < s < 2$. Since $s$ cannot satisfy both inequalities, we have reached
a contradiction. Thus the sequence $(c_n)$ is divergent. $\blacklozenge$
\end{enumerate}

\item[16.9] For each of the following, prove or give a counterexample.
\begin{enumerate}
\item[a)] If $(s_n)$ converges to $s$, then $(|s_n|)$ converges to $|s|$.\\
As proven in class: Let $\varepsilon > 0$. By definition of convergence, 
$\exists N \in \mathbb{R}$ such that $\forall n > N$, we have $|s_n - s| < \varepsilon$.
By exercise 11.6(a), $||s_n| - |s|| \leq |s_n - s|$. Therefore $\forall \varepsilon > 0$,
$\exists N \in \mathbb{R}$ such that $\forall n > N$, we have $||s_n| - |s|| \leq |s_n - s| < 
\varepsilon$. Thus, we see that lim $|s_n| = |s|$. $\blacklozenge$

\item[b)] If $(|s_n|)$ is convergent, then $(s_n)$ is convergent.\\
Counterexample: Let $s_n = (-1)^n$. We can see that $(|s_n|)$ converges to $1$, 
while $(s_n)$ does not converge as it alternates between $1$ and $-1$.

\item[c)] $\lim s_n = 0$ iff $\lim |s_n| = 0$. \\
If lim $s_n = 0$, then lim $|s_n| = 0$ follows trivially from part (a), since $|\,0| = 0$.
Conversly, if lim $|s_n| = 0$ then by definition, for each $\varepsilon > 0$, 
$\exists N \in \mathbb{R}$ such that $\forall n \in \mathbb{N}$, $n > N$ implies
$||s_n| - 0| < \varepsilon$. By idempotence, $||s_n|| = |s_n|$ and thus, 
the implication $||s_n| - 0| = ||s_n|| = |s_n| < \varepsilon$ is also true, 
which also happens to show that the limit of $(s_n)$ is $0$. $\blacklozenge$
\end{enumerate}

\item[17.2] Mark each statement True or False. Justify each answer.
\begin{enumerate}
\item[a)] If lim $s_n = s$ and lim $t_n = t$, then lim $(s_nt_n) = st$. \\
True; this is theorem 17.1(c).
\item[b)] If lim $s_n = +\infty$, then $(s_n)$ is said to converge to $+\infty$.\\
False; $(s_n)$ is said to \emph{diverge} to $+\infty$.
\item[c)] Given sequences $(s_n)$ and $(t_n)$ with $s_n \leq t_n$ for all $n \in \mathbb{N}$, if lim $s_n = +\infty$, then lim $t_n = +\infty$. \\
True; by theorem 17.4.
\item[d)] Suppose $(s_n)$ is a sequence such that the sequence of ratios $(s_{n+1}/s_n)$ converges to $L$. If $L < 1$, then lim $s_n = 0$. \\
False; $(s_n)$ must be a sequence of positive terms.
\end{enumerate}

\item[17.4] Prove Theorem 17.1(b).
\begin{enumerate}
\item[b)] In one case, we can say that $(t_n)$ is an increasing sequence, that is for all
$t_n \leq t_{n+1}$, and that it converges to $t$. Since by theorem 16.13, $(t_n)$ is bounded
and we can also say that for this type of sequence the supremum would be $t$. 
Thus, it follows that since $t_n \geq 0$
for all $n \in \mathbb{N}$, that $0 \leq t_1 \leq \cdots \leq t_n \leq t$. \\
In another case, we can say that $(t_n)$ is an decreasing sequence, that is for all
$t_n \geq t_{n+1}$, and that it converges to $t$. Since by theorem 16.13, $(t_n)$ is bounded
and we can also say that the infimum would be $t$. Thus, it follows that since $t_n \geq 0$
for all $n \in \mathbb{N}$, that $t_1 \geq \cdots \geq t_n \geq t \geq 0$. \\
In our third case, let $(t_n)$ converge to some $t$ such that \\
$\min\{t_1, t_2, \ldots, t_n\} \leq t \leq \max\{t_1, t_2, \ldots, t_n\}$. Hence,
it follows that since $t_n \geq 0$ for all $n \in \mathbb{N}$, that 
$0 \leq \min\{t_1, t_2, \ldots, t_n\} \leq t$.
\end{enumerate}

\item[17.5] 
\begin{enumerate}
\item[a)] 
\[
s_n = \frac{3 - 2n}{1 + n} = \frac{(3/n) - 2}{(1/n) + 1}
\]
Now lim $(1/n) = 0$ by example 16.3 and lim $(3/n) = 0$ by theorem 17.1(b). Thus
\[
\mbox{lim } \left[ \frac{(3/n) - 2}{(1/n) + 1} \right]
\,\, = \frac{0 - 2}{0 + 1} = \,-2
\]
by theorem 17.1(a). Hence, $(s_n)$ converges to $-2$.

\item[c)] 
\[ 
s_n = \frac{(-1)^nn}{2n - 1} = (-1)^n \cdot \frac{n}{2n - 1}
\]
By theorem 17.1(c), we can find the limit of $(-1)^n$ and $n/(2n - 1)$, which is the same as 
$1/(2 - 1/n)$. Since we know that lim $1/n = 0$, thus the lim $[1/(2 - 1/n)] = 1/2$ and
$(-1)^n$ diverges because it alternates between $1$ and $-1$. No matter the
limit for $n/(2n - 1)$, we can see that $(s_n)$ is going to diverge with no limit due to
$(-1)^n$.

\item[e)] 
\[
s_n = \frac{n^2 - 2}{n + 1}
\]
Given any $M \in \mathbb{R}$, let $N = \max\{1, 4M\}$. Then $n > N$ implies that
$n > 1$ and $n > 4M$. Since $n > 1$, $n + 1 \leq 2n$ and $n^2 - 2 \geq n^2/2$.
Thus for $n \geq N$ we have
\[
\frac{n^2 - 2}{n + 1} \geq \frac{n^2/2}{2n} = \frac{n^2}{4n} = \frac{n}{4} > M
\]
Hence lim $(n^2 - 2)/(n + 1) = +\infty$.

\item[g)] 
\[
s_n = \frac{1 - n}{2^n}
\]
Just by observation we can see that this is probably going to go to $0$. 
To show this is true, for $n \geq 5$, $n^2 < 2^n$ and $1 - n > -n$, thus
\[
\left| \frac{1 - n}{2^n} - 0 \right| = \frac{1 - n}{2^n} < \frac{-n}{n^2} = (-1) \cdot \frac{1}{n}
\]
by theorem 16.8, lim $(s_n) = 0$.

\end{enumerate}

\item[17.6] For each of the following, prove or give a counterexample.
\begin{enumerate}
\item[c)] If $(s_n)$ and $(s_n + t_n)$ are convergent sequences, then $(t_n)$ converges. \\
\emph{Proof by contrapositive:} If $(t_n)$ is a divergent sequence, then either $(s_n)$ is 
divergent or $(s_n + t_n)$ is divergent. If $(s_n)$ is a divergent sequence, then we are done.
On the other hand, $(s_n)$ could be a convergent sequence. For the sake of contradiction, let
$(s_n + t_n)$ be a convergent sequence. Thus it follows from theorem 17.1(a), that $(s_n)$ and
$(t_n)$ both must be convergent, which contradicts that $(t_n)$ is divergent. Hence
$(s_n + t_n)$ must be divergent. $\blacklozenge$

\item[d)] If $(s_n)$ and $(s_nt_n)$ are convergent sequences, then $(t_n)$ converges. \\
\emph{Counterexample:} Let $(s_n)$ be the sequence given by $s_n = 1/n$ 
and $(s_nt_n)$ be the sequence given by $s_nt_n = 1$. Thus, $(s_n)$ and $(s_nt_n)$
are convergent. Hence, $(t_n)$ is the sequence given by $t_n = n$, 
which we know diverges to $+\infty$.
\end{enumerate}

\item[17.8] Give an example of a divergent sequence $(t_n)$ of positive numbers such that lim $(t_{n+1}/t_n) = 1$.
\begin{enumerate}
\item[b)] The sequence given by $t_n = n$ is divergent as well as the lim $(t_{n+1}/t_n) = 1$.
\[
\mbox{lim } \left( \frac{n + 1}{n} \right) = \mbox{lim } \left( \frac{1 + (1/n)}{1} \right) =
\frac{1 + 0}{1} = 1
\]
\end{enumerate}

\item[17.9] Prove Theorem 17.12.
\begin{enumerate}
\item[a)] If $(s_n)$ diverges to $+\infty$, by definition we have, 
for every $M \in \mathbb{R}$ there exists a number $N$ such that
$n > N$ implies that $s_n > M$. Since $s_n \leq t_n$ for all $n \in \mathbb{N}$, 
the implication $t_n > M$ will also be true. Thus, $t_n$ also diverges to
$+\infty$.
\item[b)] If $(t_n)$ diverges to $-\infty$, by definition we have, 
for every $M \in \mathbb{R}$ there exists a number $N$ such that
$n > N$ implies that $t_n < M$. Since $s_n \leq t_n$ for all $n \in \mathbb{N}$, 
the implication $s_n < M$ will also be true. Thus, $s_n$ also diverges to
$-\infty$.
\end{enumerate}

\item[17.15] Prove the following.
\begin{enumerate}
\item[c)] $\lim (\sqrt{n^2 + n} - n) = 1/2$
\[
\sqrt{n^2 + n} - n = \frac{(\sqrt{n^2 + n} - n)(\sqrt{n^2 + n} + n)}{\sqrt{n^2 + n} + n}
= \frac{n}{\sqrt{n^2 + n} + n}
\]
Thus,
\[
\mbox{lim }\left( \frac{n}{\sqrt{n^2 + n} + n} \right) = 
\mbox{lim }\left( \frac{1}{\sqrt{1 + (1/n)} + 1} \right) =
\frac{\mbox{lim }1}{\mbox{lim } [\sqrt{1 + (1/n)}] + \mbox{lim } 1}
\]
and from that we get $1/(1 + 1) = 1/2$. \hskip 7.2cm $\blacklozenge$ 
\end{enumerate}

\item[18.2] Mark each statement True or False. Justify each answer.
\begin{enumerate}
\item[a)] If a convergent sequence is bounded, then it is monotone. \\
False; the convergent sequence, {\large $\left( \frac{(-1)^n}{n} \right)$},
is bounded and it is not monotone.
\item[b)] If $(s_n)$ is an unbounded increasing sequence, then lim $s_n = +\infty$. \\
True; this is theorem 18.8(a).
\item[c)] The Cauchy convergence criterion holds in $\mathbb{Q}$, the ordered field of rational numbers. \\
False; the Cauchy convergence criterion depends on the completeness of $\mathbb{R}$
and we already established in section 12 that $\mathbb{Q}$ is not complete.
\end{enumerate}

\item[18.4] Find an example of a sequence of real numbers satisfying each set of properties.
\begin{enumerate}
\item[a)] Cauchy, but not monotone: {\large $\left( \frac{(-1)^n}{n} \right)$}
\item[b)] Monotone, but not Cauchy: $(n)$
\item[c)] Bounded, but not Cauchy: $s(n) = 1 + (-1)^n$
\end{enumerate}

\item[18.7] Let $s_1 = \sqrt{6}, s_2 = \sqrt{6 + \sqrt{6}}, s_3 = \sqrt{6 + \sqrt{6 + \sqrt{6}}}$,
and in general define $s_{n+1} = \sqrt{6 + s_n}$. Prove that $(s_n)$ converges and find its limit.
\begin{enumerate}
\item[] Let $(s_n)$ be the sequence defined by $s_1 = \sqrt{6}$ and 
$s_{n + 1} = \sqrt{6 + s_n}$ for $n \geq 1$. We shall show that $(s_n)$ is a
bounded increasing sequence. Computing the next three terms of the sequence we find
\begin{eqnarray*}
s_2 &=& \sqrt{6 + \sqrt{6}} \hskip 3.17cm \approx 2.91 \\
s_3 &=& \sqrt{6 + \sqrt{6 + \sqrt{6}}} \hskip 2.05cm \approx 2.98 \\
s_4 &=& \sqrt{6 + \sqrt{6 + \sqrt{6 + \sqrt{6}}}} \hskip 0.9cm \approx 3.0
\end{eqnarray*}
where the decimals have been rounded off. It appears that the sequence is bounded
above by 4. To see if this conjecture is true, let us try to prove it using 
induction. Certainly, $s_1 = \sqrt{6} < 4$. Now suppose that $s_k < 4$ for 
some $k \in \mathbb{N}$. Then
\[
s_{k + 1} = \sqrt{6 + s_k} < \sqrt{6 + 4} = \sqrt{10} < 4.
\]
Thus we may conclude by induction that $s_n < 4$ for all $n \in \mathbb{N}$. \\
To verify that $(s_n)$ is an increasing sequence, we also argue by induction.
Since $s_1 = \sqrt{6}$ and $s_2 = \sqrt{6 + \sqrt{6}}$, we have $s_1 < s_2$, which
establishes the basis for induction. Now suppose that $s_k < s_{k + 1}$ for some
$k \in \mathbb{N}$. Then we have
\[
s_{k + 1} = \sqrt{6 + s_k} < \sqrt{6 + s_{k + 1}} = s_{k + 2}
\]
Thus the induction step holds and we conclude that $s_n < s_{n + 1}$ for all 
$n \in \mathbb{N}$. \\
Thus $(s_n)$ is an increasing sequence and it is bounded by the interval [2, 4].
We conclude from the monotone convergence therorem (18.3) that $(s_n)$ is 
convergent. The only question that remains is to find the value $s$ to which
it converges. Since lim $s_{n + 1} = $lim $s_n$ (Exercise 16.11), we see that $s$
must satisfy the equation 
\[
s = \sqrt{6 + s}
\]
Solving algebraically for $s$, we obtain $s = 3$ or $s = -2$. Since $s_n \geq \sqrt{6}$
for all $n$, $-2$ cannot be the limit. We conclude that lim $s_n = s = 3$.
\end{enumerate}


\item[18.8] Let $s_1 = k$ and define $s_{n+1} = \sqrt{4s_n - 1}$ for $n \in \mathbb{N}$. Determine for what values of $k$ the sequence $(s_n)$ will be monotone increasing and for what values of $k$ it will be monotone decreasing.
\begin{enumerate}
\item[] In order for $\sqrt{4x - 1}$ to be a real number, then $x \geq 1/4$. We can find
the roots of $x = \sqrt{4x - 1}$ to be $2 \pm \sqrt{3}$. Thus when $k = 2 \pm \sqrt{3}$, it
follows that the sequence will be both monotone increasing and monotone decreasing, since
$s_1 = s_2 = \cdots = s_n$. Also, since we can say the domain of $x^2 - 4x + 1$ is
$[1/4, +\infty)$, after taking the roots into account, we need to 
say whether the intervals $[1/4, 2 - \sqrt{3}), (2 - \sqrt{3}, 2 + \sqrt{3})$, and 
$(2 + \sqrt{3}, +\infty)$ are monotone increasing or decreasing. \\
Let's start with $[1/4, 2 - \sqrt{3})$. Let $k = 1/4$, then $s_1 = 1/4$ and 
$s_2 = 0$. Now suppose that $s_k > s_{k + 1}$ for some $k \in \mathbb{N}$, then
\[
s_{k + 1} = \sqrt{4s_k - 1} > \sqrt{4s_{k+1} - 1} = s_{k + 2}
\]
This shows that if $k = 1/4$ then it is a monotone decreasing sequence.
More generally, the inequality $k > \sqrt{4k - 1}$ holds true $\forall k \in [1/4, 2 - \sqrt{3})$,
thus if $k \in [1/4, 2 - \sqrt{3})$, then the sequence will also be monotone decreasing. The 
aforementioned inequality also happens to hold true $\forall k \in (2 + \sqrt{3}, +\infty)$, 
thus if $k \in (2 + \sqrt{3}, +\infty)$, then the sequence will also be monotone decreasing.\\
Lastly, the inequality, $k < \sqrt{4k - 1}$ holds true $\forall k \in (2 - \sqrt{3}, 2 + \sqrt{3})$.
This shows that $s_1 < s_2$ and assuming that $s_k < s_{k + 1}$ for some $k \in \mathbb{N}$, then
we can show that $s_k = \sqrt{4s_k - 1} < \sqrt{4s_{k+1} - 1} = s_{k + 2}$. By induction,
we have shown that for all $k$ in this interval will make the sequence monotone increasing.
\end{enumerate}

\item[18.13] Prove Lemma 18.11.
\begin{enumerate}
\item[] By theorem 18.12, we know that every Cauchy sequence is convergent and by theorem 16.13, 
we know that every convergent sequence is bounded. Thus every Cauchy sequence must be bounded.
\end{enumerate}

\item[20.2] Let $f: D \rightarrow \mathbb{R}$ and let $c$ be an accumulation point of $D$. Mark each statement True or False. Justify each answer.
\begin{enumerate}
\item[a)] For any polynomial $P$ and any $c \in \mathbb{R}, \lim_{x \to c}P(x) = P(c)$. \\
True; this is word for word what was said in example 20.14.
\item[b)] For any polynomials $P$ and $Q$ and any $c \in \mathbb{R}$,
\[
\lim_{x \to c}\frac{P(x)}{Q(x)} = \frac{P(c)}{Q(c)}.
\]
True; follows from what was said in part (a) and the quotient of
definition 20.12.
\item[c)] In evaluating $\lim_{x \to a-}f(x)$ we only consider points $x$ that are greater than $a$. \\
False; this is a left-handed limit, which is defined as 
taking into consideration all points less than but no equal to $a$.
\item[d)] If $f$ is defined in a deleted neighborhood of $c$, then $\lim_{x \to c}f(x) = L$ iff 
\\ $\lim_{x \to c+}f(x) = \lim_{x \to c-}f(x) = L$. \\
True; this is the last two sentences of the section.
\end{enumerate}

\item[20.3] Determine the following limits.
\begin{enumerate}
\item[a)]
\[
\lim_{x \to 1} \frac{x^3 + 5}{x^2 + 2} = \frac{1 + 5}{1 + 2} = \frac{6}{3} = 2
\]
\item[c)]
\[
\lim_{x \to 1} \frac{\sqrt{x} - 1}{x - 1} = 
\lim_{x \to 1} \frac{(\sqrt{x} - 1)(\sqrt{x} + 1)}{(x - 1)(\sqrt{x} + 1)} =
\lim_{x \to 1} \frac{1}{\sqrt{x} + 1} = \frac{1}{2}
\]
\item[e)]
\[
\lim_{x \to 0} \frac{x^2 + 3x}{x^2 + 1} = \frac{0 + 0}{0 + 1} = 0
\]
\item[g)] As, $x < 0, |x| = -x$, therefore
\[
\lim_{x \to 0-} \frac{4x}{|x|} = \lim_{x \to 0-} \frac{4x}{-x} = \lim_{x \to 0-} -4 = -4
\]
\end{enumerate}

\item[20.5] Find a $\delta > 0$ so that $|x - 2| < \delta$ implies that $|x^2 - 7x + 10| < 1/3$.
\begin{enumerate}
\item[] First note that $|x^2 - 7x + 10| = |x - 5||x - 2|$. Thus we need to have an upper
bound on the size of $|x - 5|$. Now if $|x - 2| < 1/3$, then 
$|x - 5| = |x - 2 - 3| \leq |x - 2| + |-3| < 4$, so that 
$|x^2 - 7x + 10| = |x - 5||x - 2| < 4|x - 2|$. Thus we take $\delta = 1/12$.
\end{enumerate}


\item[20.6] Use definition 20.1 to prove each limit.
\begin{enumerate}
\item[b)] $\lim_{x \rightarrow -2}(x^2 + 2x + 7) = 7$
\[
|x^2 + 2x + 7 - 7 | = |x^2 + 2x| = |x||x + 2|
\]
If we bound $|x|$ when $x$ is approaching $2$ from the left side, then we make
$|x| < 2$. Thus we can choose $|x + 2| < \varepsilon /2$. Hence we can choose 
$\delta = \min\{2, \varepsilon /2 \}$.
\[
|x^2 + 2x + 7 - 7 | = |x^2 + 2x| = |x||x + 2| < 2|x + 2| < 2\delta \leq \varepsilon
\]
\end{enumerate}

\item[20.11] Prove Theorem 20.10.
\begin{enumerate}
\item[] To prove that (a) $\Rightarrow$ (b), suppose that (b) is false. Let $(s_n)$ be a sequence
in $D$ with $s_n \rightarrow c$ and $s_n \neq c$ for all $n$. Since (b) is false, $(f(s_n))$ 
converges to some value, say $L$. We must now show that any given sequence $(t_n)$ in $D$ with
$t_n \rightarrow c$ and $t_n \neq c$ for all $n$, we have $\lim f(t_n) = L$. We only know from the
negation of (b) that $(f(t_n))$ is convergent. To see the $\lim f(t_n) = L$, consider the sequence
$(u_n) = (s_1, t_1, s_2, t_2, \ldots )$, clearly this sequence converges to $c$. Now let's say
that $f(u_n)$ converged to $M$, $M \neq L$. By theorem 19.4, since $f(u_n)$ converges to 
$M$, then $f(s_n)$ would also converge to $M$ as well. We know that $f(s_n)$ converges to $L$, thus
since it is a subsequence of $f(u_n)$, then $f(u_n)$ converges to $L$ as well. Again, by 
theorem 19.4, since $f(t_n)$ is a subsequence of $f(u_n)$ then it must also converge to $L$,
which following the negation and theorem 20.8, implies that $f$ has a limit at $c$. To prove
(b) $\Rightarrow$ (a), for the sake of contradiction suppose that (a) is false. Then (a) reads
that $f$ has a limit at $c$, which contradicts theorem 20.8 where every sequence $(s_n)$ converging
to $c$ with $s_n \neq c$ for all $n$, then the sequence $f(s_n)$ converges. Thus $f$ must not
have a limit at $c$. $\blacklozenge$
\end{enumerate}

\item[20.16] Let $f: D \rightarrow \mathbb{R}$ and let $c$ be and accumulation point of $D$. Suppose that $\lim_{x \rightarrow c}f(x) > 0$. Prove that there exists a deleted neighborhood $U$ of $c$ such that $f(x) > 0$ for all $x \in U \cap D$.
\begin{enumerate}
\item[] Since $f: D \rightarrow \mathbb{R}$ and $c$ is an accumulation point in $D$ and 
$\lim_{x \to c}f(x) = L > 0$, then that means for each neighborhood $V$ of $L$ there exists
a deleted neighborhood $U$ of $c$ such that $f(U \cap D) \subseteq V$, by theorem 20.2.
Using the $\varepsilon - \delta$ definition, then $V$ is a open interval bounded by
$L - \varepsilon$ and $L + \varepsilon$. Thus since $L > 0$, for any $\varepsilon$,
$0 < \varepsilon < L$ implies that $L - \varepsilon > 0$ and since this happens
to a lower bound of $V$ then we have $0 < L - \varepsilon < f(x)$, for all $x \in U \cap D$. 
Since by definition 20.1, the $\delta$ depends on the $\varepsilon$, 
thus it follows there there exist a $U^*(c, \delta)$ that satisfies our requirements.
$\blacklozenge$
\end{enumerate}

\item[21.2] Let $f: D \rightarrow \mathbb{R}$ and let $c \in D$. Mark each statement True or False. Justify each answer.
\begin{enumerate}
\item[a)] If $f$ is continuous at $c$ and $c$ is an accumulation point of $D$, then $\lim_{x \to c}f(x) = f(c)$. \\
True; as stated in theorem 21.2(d).
\item[b)] Every polynomial is continuous at each point in $\mathbb{R}$. \\
True; as stated in example 21.3.
\item[c)] If $(x_n)$ is a Cauchy sequence in $D$, then $(f(x_n))$ is convergent. \\
False; consider the function $f(1/x) = x$ and the Cauchy sequence (1/n).
\item[d)] If $f: \mathbb{R} \rightarrow \mathbb{R}$ is continuous at each irrational number, then $f$ is continuous on $\mathbb{R}$. \\
False; the modified Dirichlet function is a counterexample.
\item[e)] If $f: \mathbb{R} \rightarrow \mathbb{R}$ and $g: \mathbb{R} \rightarrow \mathbb{R}$ are both continuous on $\mathbb{R}$, then $f \circ g$ and $g \circ f$ are both continuous on $\mathbb{R}$. \\
True; by theorem 21.12 by letting $D = \mathbb{R}$ and $E = \mathbb{R}$.
\end{enumerate}

\item[21.4] Define $f: \mathbb{R} \rightarrow \mathbb{R}$ by $f(x) = x^2 - 3x + 5$. Use Definition 21.1 to prove that $f$ is continuous at 2.
\begin{enumerate}
\item[] 
\[ 
|f(x) - f(c)| = |x^2 - 3x + 5 - (4 - 6 + 5)| = |x^2 - 3x + 2| = |x - 2||x - 1|
\]
We arbitrarily want $|x - 2| < 1$ that way we can bound $|x - 1|$. Hence, 
$|x - 1| = |x - 2 + 1| = \leq |x - 2| + |1| \leq 1 + 1 = 2$. Thus for any given
$\varepsilon > 0$, let $\delta = \min \{1, \varepsilon /2 \}$. Then for all $x$
satisfying $|x - 2| < \delta$ we have $|x - 2| < 1$, so that $|x - 1| < 2$. It
foloows that for these $x$ we have 
\[
|f(x) - f(c)| = |x - 2||x - 1| \leq 2|x - 2| \leq \varepsilon
\]
\end{enumerate}

\item[21.6] Prove or give a counterexample for each statement.
\begin{enumerate}
\item[a)] If $f$ is continuous on $D$ and $k \in \mathbb{R}$, then $kf$ is continuous on $D$. \\
To show that $kf$ is continuous on $D$, we show that 
$\lim kf(x_n) = k \lim f(x_n) = kf(c)$, by the fact that $f$ is continuous on $D$
and theorem 17.1(b).
\item[b)] If $f$ and $f + g$ are continuous on $D$, then $g$ is continuous on $D$. \\
As proven in class: If $f$ is continuous, then $-f$ is also continuous. Since
we can represent $g$ as $(-f) + (f + g)$ and we know that $(f + g)$ is also continuous,
hence $g$ must be continuous.
\item[c)] If $f$ and $fg$ are continuous on $D$, then $g$ is continuous on $D$. \\
If $f$ is continuous and $f(c) \neq 0$ then $1/f$ is also continuous. Since
we can represent $g = (fg)/f$ and we know that $fg$ is continuous then we know that
$g$ is continuous by theorem 21.10(b).
\item[d)] If $f^2$ is continuous on $D$, then $f$ is continuous on $D$. \\
As shown in class: Counterexample: Let $f(x)$ be $1$ when $x \geq 0$ and 
$-1$ when $x < 0$.
\item[e)] If $f$ is continuous on $D$, then $f(D)$ is a bounded set. \\
Counterexample: Let $f(x) = 1/x$ and $D = (0, 1]$.
\item[f)] If $f$ and $g$ are not continuous on $D$, then $f+g$ is not continuous on $D$. \\
Counterexample: Let $f(x)$ be $1$ when $x \geq 0$ and 
$-1$ when $x < 0$, also, let $g(x)$ be $-1$ when $x \geq 0$ and 
$1$ when $x < 0$.
\item[g)] If $f$ and $g$ are not continuous on $D$, then $fg$ is not continuous on $D$. \\
Counterexample: Let $f(x)$ and $g(x)$ be $1$ when $x \geq 0$ and 
$-1$ when $x < 0$.
\item[h)] If $f: D \rightarrow E$ and $g: E \rightarrow F$ are not continuous on $D$ and $E$, respectively, then $g \circ f : D \rightarrow F$ is not continuous on $D$. \\
Counterexample: Consider $ f(x) = 
\begin{cases}
-1, & x < 0 \\
1, & x \geq 0
\end{cases}
$ \quad and \\ 
$g(x) = 
\begin{cases}
1, & x = 1 \vee x = -1 \\
-1, & \mbox{for anything else} 
\end{cases}
$
\end{enumerate}

\item[21.14] Let $f: D \rightarrow \mathbb{R}$ be continuous at $c \in D$. Prove that there exists an $M > 0$ and a neighborhood $U$ of $c$ such that $|f(x)| \leq M$ for all $x \in U \cap D$.
\begin{enumerate}
\item[] Since $f$ is continuous at $c \in D$, by theorem 21.2, for every neighborhood
$V$ of $f(c)$ there exists a neighborhood $U$ of $c$ such that 
$f(U \cap D) \subseteq V$. Using the $\varepsilon - \delta$ definition, 
$V$ is the open interval bounded by $f(c) - \varepsilon$ and $f(c) + \varepsilon$. 
Hence we can choose $M = \max \{ |f(c) - \varepsilon |, |f(c) + \varepsilon |\} > 0$,
such that $\forall x \in U \cap D$, $|f(x)| \leq M$. $\blacklozenge$
\end{enumerate}

\item[21.16] Let $f: \mathbb{R} \rightarrow \mathbb{R}$. Prove that $f$ is continuous on $\mathbb{R}$ iff $f^{-1}(H)$ is a closed set whenever $H$ is a closed set.
\begin{enumerate}
\item[] Let us note that $\mathbb{R} = \mathbb{R}\backslash H \cup H$ and that
$\mathbb{R} = f^{-1}(\mathbb{R}) = f^{-1}(\mathbb{R}\backslash H) \cup f^{-1}(H)$. 
We also know that $H$ is closed whenever $\mathbb{R}\backslash H$ is open and 
vice versa. So if $H$ is closed, then $f^{-1}(H)$ must also be closed if and 
only if $\mathbb{R}\backslash f^{-1}(H)$ is open which also happens to be equal 
to $f^{-1}(\mathbb{R}\backslash H)$, which we know follows from corollary 21.15.
Thus, $f$ is continuous on $\mathbb{R}$ iff $f^{-1}(H)$ is a closed set
whenever $H$ is a closed set. $\blacklozenge$
\end{enumerate}

\item[22.2] Mark each statement True or False. Justify each answer.
\begin{enumerate}
\item[a)] Let $f : [a, b] \rightarrow \mathbb{R}$ be continuous and suppose $f(a) < 0 < f(b)$. Then there exists a point $c$ in $(a, b)$ such that $f(c) = 0$. \\
True; by lemma 22.5.
\item[b)] Let $f : [a, b] \rightarrow \mathbb{R}$ be continuous and suppose $f(a) \leq k \leq f(b)$. Then there exists a point $c \in [a, b]$ such that $f(c) = k$. \\
False; $f(a) < k < f(b)$ must be true.
\item[c)] If $f: D \rightarrow \mathbb{R}$ is continuous and bounded on $D$, then $f$ assumes maximum and minimum values on $D$. \\
False; $D$ must be a compact subset of $\mathbb{R}$.
\end{enumerate}

\item[22.3] Let $f: D \rightarrow \mathbb{R}$ be continuous. For each of the following, prove or give a counterexample.
\begin{enumerate}
\item[a)] If $D$ is open, then $f(D)$ is open. \\
\emph{Counterexample:} Consider $D = \mathbb{R}$ and $f(x) = \sin x$.
\item[b)] If $D$ is closed, then $f(D)$ is closed. \\
\emph{Counterexample:} Consider $D = \mathbb{N}$ and $f(x) = 1/x$.
\item[c)] If $D$ is not open, then $f(D)$ is not open. \\
\emph{Counterexample:} Consider $D = \mathbb{N}$ and $f(x) = 1/x$.
\item[d)] If $D$ is not closed, then $f(D)$ is not closed. \\
\emph{Counterexample:} Consider $D = (0, 1)$ and $f(x) = 5$.
\item[e)] If $D$ is not compact, then $f(D)$ is not compact. \\
\emph{Counterexample:} Consider $D = (0, 1)$ and $f(x) = 5$.
\item[f)] If $D$ is unbounded, then $f(D)$ is unbounded. \\
\emph{Counterexample:} Consider $D = [1, \infty)$ and $f(x) = 1/x$.
\item[g)] If $D$ is finite, then $f(D)$ is finite. \\
This is clearly true as the definition of a function states that
there is exactly 1 mapping for all elements in $D$ 
to some element in the range. Since $D$ is finite, then the 
image, $f(D)$, can have at most the same amount of elements in $D$.
\item[h)] If $D$ is infinite, then $f(D)$ is infinite. \\
\emph{Counterexample:} Consider $D = \mathbb{R}$ and $f(x) = 0$.
\item[j)] If $D$ is an interval that is not open, then $f(D)$ is an interval that is not open. \\
\emph{Counterexample:} Consider $D = \mathbb{N}$ and $f(x) = 1/x$.
\end{enumerate}

\item[22.6] Show that any polynomial of odd degree has at least one real root.
\begin{enumerate}
\item[] Let $P(x) = a_nx^n + \cdots + a_0$ be our odd degree polynomial.
Now either $P(x) \rightarrow \infty$ or $P(x) \rightarrow -\infty$ as
$x \rightarrow \infty$, depending on whether $a_n$ is positive or negative. 
When $P(x) \rightarrow \infty$ as $x \rightarrow \infty$ then
$P(x) \rightarrow -\infty$ as $x \rightarrow -\infty$. It's just as well, that
when $P(x) \rightarrow -\infty$ as $x \rightarrow \infty$, then $P(x) \to \infty$
as $x \rightarrow -\infty$. This property implies that $P(x)$ takes both
positive and negative values, so we can apply the Intermediate Value Theorem
so that we know there exists at least one $c$ such that $P(c) = 0$, meaning
there is at least one root. 
\end{enumerate}

\item[22.10] Suppose that $f: [a, b] \rightarrow \mathbb{R}$ is two-to-one. That is, for each $y \in \mathbb{R}, f^{-1}(\{y\})$ either is empty of contains exactly two points.
\begin{enumerate}
\item[a)] Find an example of such a function. \\
If we can turn a blind eye to $x = 0$ then $f(x) = x^2$ is a 
two-to-one function for $f: \mathbb{R} \rightarrow \mathbb{R}$.
\end{enumerate}

\item[22.13] Let $f$ be a function defined on an interval $I$. We say that $f$ is {\bf strictly increasing} if $x_1 < x_2$ in $I$ implies that $f(x_1) < f(x_2)$. Similarly, $f$ is {\bf strictly decreasing} if $x_1 < x_2$ in $I$ implies that $f(x_1) > f(x_2)$. Prove the following.
\footnote{
The answer to part(a) and (b) was taken from\\
http://students.csci.unt.edu/\%7Ehgc0005/homeworks/h8ra.pdf
}
\begin{enumerate}
\item[a)] If $f$ is continuous and injective on $I$, then $f$ is strictly increasing or strictly decreasing. \\
If $f$ is not strictly increasing nor decreasing, then there exists
$x \in I$ such that $f(x) < f(z)$ for some $z < x$ in $I$. Now choose any 
$y < z \in I$ such that $f(y) < f(x)$. If there exist no such $y$, then let
$x = c$ such that $f(y) < f(c) < f(z)$ and $c \in (z, x)$, this is guaranteed
by the Intermediate Value Theorem applied to $f(z)$ and $f(x)$. Now, applying
again the Intermediate Value Theorem to $f(y)$ and $f(z)$ there exists 
$d \in (y, z)$ such that $f(d) = f(c)$, clearly $d \neq c$, thus this 
contradicts our assumption that $f$ is injective.
\item[b)] If $f$ is strictly increasing and if $f(I)$ is an interval, then $f$ is continuous. Furthermore, $f^{-1}$ is a strictly increasing continuous function on $f(I)$. \\
Let $V$ be the neighborhood $(y_0 - \varepsilon, y_0 + \varepsilon)$ 
such that $V \subseteq f(I)$, then there exists $x_1, x_2 \in I$ such that
$f(x_1) = x - \varepsilon$ and $f(x_2) = x + \varepsilon$, 
clearly $x_1 < x_2$, let $U$ be the neighborhood $(x_1, x_2)$. Now for every
$y \in V$ there exists an $x \in U$ such that $f(x) = y$ (otherwise if 
$y_0 - \varepsilon < y < y_0 + \varepsilon$ and $x < x_1$ or $x > x_2$ would
contradict the fact that the function is strictly increasing), then $f(U) 
\subseteq V$, then by theorem 21.2(c), $f$ is continuous. Now since $f$
is a strictly increasing function on $I$, then for every $x_1 < x_2 \in I$, 
$f(x_1) < f(x_2)$, this implies that for every $f(x_1) < f(x_2) \in f(I)$,
$f^{-1}(f(x_1)) < f^{-1}(f(x_2))$, in other words for every
$f(x_1) < f(x_2) \in f(I)$, $x_1 < x_2$.
\end{enumerate}

\item[23.2] Let $f: D \rightarrow \mathbb{R}$. Mark each statement True or False. Justify each answer.
\begin{enumerate}
\item[a)] In the definition of uniform continuity, the positive $\delta$ depends only on the function $f$ and the given $\epsilon > 0$.\\
False; $\delta$ only depends on $\varepsilon$ in the definition of
uniform continuity.
\item[b)] If $f$ is continuous and $(x_n)$ is a Cauchy sequence in $D$, then $(f(x_n))$ is a Cauchy sequence. \\
False; $f$ must be uniformly continuous by theorem 23.8.
\item[c)] If $f: (a, b) \rightarrow \mathbb{R}$ can be extended to a function that is continuous on $[a, b]$, then $f$ is uniformly continuous on $(a, b)$. \\
True; by theorem 23.9.
\end{enumerate}

\item[23.3] Determine which of the following continuous functions are uniformly continuous on the given set. Justify your answers.
\begin{enumerate}
\item[a)] $f(x) = e^x/x$ on [2, 5] \\
It is uniformly continuous since it is continuous on a compact set.
\item[b)] $f(x) = e^x/x$ on (0, 2) \\
Not uniformly continuous because it cannot be extended to a function
that is continuous because $e^x/x$ is undefined at $0$ on the compact set $[0, 2]$.
\item[c)] $f(x) = x^2 + 3x - 5$ on [0, 4] \\
It is uniformly continuous since it is continuous on a compact set.
\item[d)] $f(x) = x^2 + 3x - 5$ on (1, 3) \\
It is uniformly continuous since it is continuous on the compact set, [1, 3].
\item[e)] $f(x) = 1/x^2$ on (0, 1) \\
It is uniformly continuous since it is continuous on a compact set.
\item[f)] $f(x) = 1/x^2$ on $(0, \infty)$ \\
It is not uniformly continuous since it cannot be extended to a function
that is continuous on $[0, \infty]$ because $\lim_{x \to 0} f(x) = \infty$.
\item[g)] $f(x) = x \sin (1/x)$ on $(0, 1)$. \\
It is uniformly continuous since it can be extended to a function that
is continuous on [0, 1] by example 21.5.
\end{enumerate}

\item[23.4] Prove that each function is uniformly continuous on the given set by directly verifying the $\epsilon-\delta$ property in Definition 23.1.
\begin{enumerate}
\item[b)] $f(x) = \frac{1}{x}$ on $[2, \infty )$
\[
|f(x) - f(y)| = |\frac{1}{x} - \frac{1}{y}| = \frac{|x - y|}{|xy|}
\]
Hence $x$ and $y$ are both positive so $|xy| = xy$. Also, since both 
$x \geq 2$ and $y \geq 2$ then $\frac{|x - y|}{|xy|} \leq |x - y|$.
Given $\varepsilon$, take $\delta = \varepsilon$.
If $|x - y| < \delta$, then \[
|\frac{1}{x} - \frac{1}{y}| = \frac{|x - y|}{|xy|} \leq |x - y| < \delta =
\varepsilon
\]. Thus $f(x)$ is uniformly continuous on $[2, \infty)$.
\end{enumerate}

\item[23.5] Prove that $f(x) = \sqrt{x}$ is uniformly continuous on $[0, \infty )$.
\begin{enumerate}
\item[] Let $\varepsilon > 0$. Choose $\delta = \varepsilon^2$. Since
$x, y \in [0, \infty)$ then we have two cases. \\
Case 1: $x, y \in [0, \varepsilon^2)$. Then $\sqrt{x}\sqrt{y} \in [0, \varepsilon)$,
thus \[ |f(x) - f(y)| = |\sqrt{x} - \sqrt{y}| < |\varepsilon - 0| = \varepsilon \]
Case 2: At least one of $x, y$ is greater or equal to $\varepsilon^2$. In this
case, $\sqrt{x} + \sqrt{y} \geq \sqrt{\varepsilon^2} = \varepsilon$, then
\[
|\sqrt{x} - \sqrt{y}| = \frac{|x - y|}{|\sqrt{x} + \sqrt{y}|} \leq 
\frac{|x - y|}{\varepsilon} < \delta/\varepsilon = \varepsilon^2/\varepsilon 
= \varepsilon 
\].
\end{enumerate}

\item[23.10] Find two real-valued functions $f$ and $g$ that are uniformly continuous on a set $D$, but such that their product $fg$ is not uniformly continuous on $D$.
\begin{enumerate}
\item[] Consider when $D = \mathbb{R}$ and $f(x) = g(x) = x$. By Example 23.4, we
know that $fg = x^2$, which is not uniformly continuous on $\mathbb{R}$.
\end{enumerate}

\item[23.13] Suppose that $f$ is uniformly continuous on $[a, b]$ and uniformly continuous on $[b, c]$. Prove that $f$ is uniformly continuous on $[a, c]$.
\begin{enumerate}
\item[] Given any $\varepsilon > 0$, find $\delta_1$ so that if $x, y \in [a, b]$ 
with $|x - y| < \delta_1$, then $|f(x) - f(y)| < \varepsilon /2$. Similarly, find
$\delta_2$ so that if $x, y \in [b, c]$, and $|x - y| < \delta_2$, then 
$|f(x) - f(y)| < \varepsilon /2$. Let $\delta = \min \{\delta_1, \delta_2 \}$.
Now, suppose that $x, y \in [a, b]$ with $|x - y| < \delta$. If $x$ and $y$ are both
elements of $[a, b]$, we can conclude that $|f(x) - f(y)| < \varepsilon /2 < \varepsilon$.
Similarly, if $x$ and $y$ are both elements of $[b, c]$, then we conclude that
$|f(x) - f(y)| < \varepsilon$.
Suppose instead that $x \in [a, b]$ and $y \in [b, c]$. We then can conclude
that $|x - b| < \delta$ and $|b - y| < \delta$. Therefore, 
$|f(x) - f(b)| < \varepsilon /2$ and $|f(b) - f(y)| < \varepsilon /2$. So 
$|f(x) - f(y)| = |f(x) - f(b) + f(b) - f(y)| \leq |f(x) - f(b)| - |f(b) - f(y)|
< \varepsilon /2 + \varepsilon /2 = \varepsilon$.
\end{enumerate}
\end{enumerate}
\end{enumerate}
\end{document}